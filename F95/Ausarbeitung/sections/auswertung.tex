\section{Auswertung}
\subsection{Ermittlung der Kamerabewegung aus einer Bildfolge}
\begin{wrapfigure}{r}{0.45\textwidth}
  \vspace{-6pt}
  \vspace{-10pt}
  \caption{Die Translation in Pixeln eines entfernten Objektes im Bild ist linear zur parallelen Verschiebung einer Kamera}
  \label{fig:translrel}
  \vspace{-10pt}
  \resizebox{\linewidth}{!}{%% Creator: Matplotlib, PGF backend
%%
%% To include the figure in your LaTeX document, write
%%   \input{<filename>.pgf}
%%
%% Make sure the required packages are loaded in your preamble
%%   \usepackage{pgf}
%%
%% and, on pdftex
%%   \usepackage[utf8]{inputenc}\DeclareUnicodeCharacter{2212}{-}
%%
%% or, on luatex and xetex
%%   \usepackage{unicode-math}
%%
%% Figures using additional raster images can only be included by \input if
%% they are in the same directory as the main LaTeX file. For loading figures
%% from other directories you can use the `import` package
%%   \usepackage{import}
%%
%% and then include the figures with
%%   \import{<path to file>}{<filename>.pgf}
%%
%% Matplotlib used the following preamble
%%   \usepackage{fontspec}
%%   \setmainfont{DejaVuSerif.ttf}[Path=/usr/share/matplotlib/mpl-data/fonts/ttf/]
%%   \setsansfont{DejaVuSans.ttf}[Path=/usr/share/matplotlib/mpl-data/fonts/ttf/]
%%   \setmonofont{DejaVuSansMono.ttf}[Path=/usr/share/matplotlib/mpl-data/fonts/ttf/]
%%
\begingroup%
\makeatletter%
\begin{pgfpicture}%
\pgfpathrectangle{\pgfpointorigin}{\pgfqpoint{3.000000in}{3.000000in}}%
\pgfusepath{use as bounding box, clip}%
\begin{pgfscope}%
\pgfsetbuttcap%
\pgfsetmiterjoin%
\definecolor{currentfill}{rgb}{1.000000,1.000000,1.000000}%
\pgfsetfillcolor{currentfill}%
\pgfsetlinewidth{0.000000pt}%
\definecolor{currentstroke}{rgb}{1.000000,1.000000,1.000000}%
\pgfsetstrokecolor{currentstroke}%
\pgfsetdash{}{0pt}%
\pgfpathmoveto{\pgfqpoint{0.000000in}{0.000000in}}%
\pgfpathlineto{\pgfqpoint{3.000000in}{0.000000in}}%
\pgfpathlineto{\pgfqpoint{3.000000in}{3.000000in}}%
\pgfpathlineto{\pgfqpoint{0.000000in}{3.000000in}}%
\pgfpathclose%
\pgfusepath{fill}%
\end{pgfscope}%
\begin{pgfscope}%
\pgfsetbuttcap%
\pgfsetmiterjoin%
\definecolor{currentfill}{rgb}{1.000000,1.000000,1.000000}%
\pgfsetfillcolor{currentfill}%
\pgfsetlinewidth{0.000000pt}%
\definecolor{currentstroke}{rgb}{0.000000,0.000000,0.000000}%
\pgfsetstrokecolor{currentstroke}%
\pgfsetstrokeopacity{0.000000}%
\pgfsetdash{}{0pt}%
\pgfpathmoveto{\pgfqpoint{0.702287in}{0.571604in}}%
\pgfpathlineto{\pgfqpoint{2.761635in}{0.571604in}}%
\pgfpathlineto{\pgfqpoint{2.761635in}{2.797238in}}%
\pgfpathlineto{\pgfqpoint{0.702287in}{2.797238in}}%
\pgfpathclose%
\pgfusepath{fill}%
\end{pgfscope}%
\begin{pgfscope}%
\pgfpathrectangle{\pgfqpoint{0.702287in}{0.571604in}}{\pgfqpoint{2.059348in}{2.225635in}}%
\pgfusepath{clip}%
\pgfsetrectcap%
\pgfsetroundjoin%
\pgfsetlinewidth{0.803000pt}%
\definecolor{currentstroke}{rgb}{0.690196,0.690196,0.690196}%
\pgfsetstrokecolor{currentstroke}%
\pgfsetdash{}{0pt}%
\pgfpathmoveto{\pgfqpoint{0.702287in}{0.571604in}}%
\pgfpathlineto{\pgfqpoint{0.702287in}{2.797238in}}%
\pgfusepath{stroke}%
\end{pgfscope}%
\begin{pgfscope}%
\pgfsetbuttcap%
\pgfsetroundjoin%
\definecolor{currentfill}{rgb}{0.000000,0.000000,0.000000}%
\pgfsetfillcolor{currentfill}%
\pgfsetlinewidth{0.803000pt}%
\definecolor{currentstroke}{rgb}{0.000000,0.000000,0.000000}%
\pgfsetstrokecolor{currentstroke}%
\pgfsetdash{}{0pt}%
\pgfsys@defobject{currentmarker}{\pgfqpoint{0.000000in}{-0.048611in}}{\pgfqpoint{0.000000in}{0.000000in}}{%
\pgfpathmoveto{\pgfqpoint{0.000000in}{0.000000in}}%
\pgfpathlineto{\pgfqpoint{0.000000in}{-0.048611in}}%
\pgfusepath{stroke,fill}%
}%
\begin{pgfscope}%
\pgfsys@transformshift{0.702287in}{0.571604in}%
\pgfsys@useobject{currentmarker}{}%
\end{pgfscope}%
\end{pgfscope}%
\begin{pgfscope}%
\definecolor{textcolor}{rgb}{0.000000,0.000000,0.000000}%
\pgfsetstrokecolor{textcolor}%
\pgfsetfillcolor{textcolor}%
\pgftext[x=0.702287in,y=0.474382in,,top]{\color{textcolor}\sffamily\fontsize{10.000000}{12.000000}\selectfont 0}%
\end{pgfscope}%
\begin{pgfscope}%
\pgfpathrectangle{\pgfqpoint{0.702287in}{0.571604in}}{\pgfqpoint{2.059348in}{2.225635in}}%
\pgfusepath{clip}%
\pgfsetrectcap%
\pgfsetroundjoin%
\pgfsetlinewidth{0.803000pt}%
\definecolor{currentstroke}{rgb}{0.690196,0.690196,0.690196}%
\pgfsetstrokecolor{currentstroke}%
\pgfsetdash{}{0pt}%
\pgfpathmoveto{\pgfqpoint{1.217124in}{0.571604in}}%
\pgfpathlineto{\pgfqpoint{1.217124in}{2.797238in}}%
\pgfusepath{stroke}%
\end{pgfscope}%
\begin{pgfscope}%
\pgfsetbuttcap%
\pgfsetroundjoin%
\definecolor{currentfill}{rgb}{0.000000,0.000000,0.000000}%
\pgfsetfillcolor{currentfill}%
\pgfsetlinewidth{0.803000pt}%
\definecolor{currentstroke}{rgb}{0.000000,0.000000,0.000000}%
\pgfsetstrokecolor{currentstroke}%
\pgfsetdash{}{0pt}%
\pgfsys@defobject{currentmarker}{\pgfqpoint{0.000000in}{-0.048611in}}{\pgfqpoint{0.000000in}{0.000000in}}{%
\pgfpathmoveto{\pgfqpoint{0.000000in}{0.000000in}}%
\pgfpathlineto{\pgfqpoint{0.000000in}{-0.048611in}}%
\pgfusepath{stroke,fill}%
}%
\begin{pgfscope}%
\pgfsys@transformshift{1.217124in}{0.571604in}%
\pgfsys@useobject{currentmarker}{}%
\end{pgfscope}%
\end{pgfscope}%
\begin{pgfscope}%
\definecolor{textcolor}{rgb}{0.000000,0.000000,0.000000}%
\pgfsetstrokecolor{textcolor}%
\pgfsetfillcolor{textcolor}%
\pgftext[x=1.217124in,y=0.474382in,,top]{\color{textcolor}\sffamily\fontsize{10.000000}{12.000000}\selectfont 10}%
\end{pgfscope}%
\begin{pgfscope}%
\pgfpathrectangle{\pgfqpoint{0.702287in}{0.571604in}}{\pgfqpoint{2.059348in}{2.225635in}}%
\pgfusepath{clip}%
\pgfsetrectcap%
\pgfsetroundjoin%
\pgfsetlinewidth{0.803000pt}%
\definecolor{currentstroke}{rgb}{0.690196,0.690196,0.690196}%
\pgfsetstrokecolor{currentstroke}%
\pgfsetdash{}{0pt}%
\pgfpathmoveto{\pgfqpoint{1.731961in}{0.571604in}}%
\pgfpathlineto{\pgfqpoint{1.731961in}{2.797238in}}%
\pgfusepath{stroke}%
\end{pgfscope}%
\begin{pgfscope}%
\pgfsetbuttcap%
\pgfsetroundjoin%
\definecolor{currentfill}{rgb}{0.000000,0.000000,0.000000}%
\pgfsetfillcolor{currentfill}%
\pgfsetlinewidth{0.803000pt}%
\definecolor{currentstroke}{rgb}{0.000000,0.000000,0.000000}%
\pgfsetstrokecolor{currentstroke}%
\pgfsetdash{}{0pt}%
\pgfsys@defobject{currentmarker}{\pgfqpoint{0.000000in}{-0.048611in}}{\pgfqpoint{0.000000in}{0.000000in}}{%
\pgfpathmoveto{\pgfqpoint{0.000000in}{0.000000in}}%
\pgfpathlineto{\pgfqpoint{0.000000in}{-0.048611in}}%
\pgfusepath{stroke,fill}%
}%
\begin{pgfscope}%
\pgfsys@transformshift{1.731961in}{0.571604in}%
\pgfsys@useobject{currentmarker}{}%
\end{pgfscope}%
\end{pgfscope}%
\begin{pgfscope}%
\definecolor{textcolor}{rgb}{0.000000,0.000000,0.000000}%
\pgfsetstrokecolor{textcolor}%
\pgfsetfillcolor{textcolor}%
\pgftext[x=1.731961in,y=0.474382in,,top]{\color{textcolor}\sffamily\fontsize{10.000000}{12.000000}\selectfont 20}%
\end{pgfscope}%
\begin{pgfscope}%
\pgfpathrectangle{\pgfqpoint{0.702287in}{0.571604in}}{\pgfqpoint{2.059348in}{2.225635in}}%
\pgfusepath{clip}%
\pgfsetrectcap%
\pgfsetroundjoin%
\pgfsetlinewidth{0.803000pt}%
\definecolor{currentstroke}{rgb}{0.690196,0.690196,0.690196}%
\pgfsetstrokecolor{currentstroke}%
\pgfsetdash{}{0pt}%
\pgfpathmoveto{\pgfqpoint{2.246798in}{0.571604in}}%
\pgfpathlineto{\pgfqpoint{2.246798in}{2.797238in}}%
\pgfusepath{stroke}%
\end{pgfscope}%
\begin{pgfscope}%
\pgfsetbuttcap%
\pgfsetroundjoin%
\definecolor{currentfill}{rgb}{0.000000,0.000000,0.000000}%
\pgfsetfillcolor{currentfill}%
\pgfsetlinewidth{0.803000pt}%
\definecolor{currentstroke}{rgb}{0.000000,0.000000,0.000000}%
\pgfsetstrokecolor{currentstroke}%
\pgfsetdash{}{0pt}%
\pgfsys@defobject{currentmarker}{\pgfqpoint{0.000000in}{-0.048611in}}{\pgfqpoint{0.000000in}{0.000000in}}{%
\pgfpathmoveto{\pgfqpoint{0.000000in}{0.000000in}}%
\pgfpathlineto{\pgfqpoint{0.000000in}{-0.048611in}}%
\pgfusepath{stroke,fill}%
}%
\begin{pgfscope}%
\pgfsys@transformshift{2.246798in}{0.571604in}%
\pgfsys@useobject{currentmarker}{}%
\end{pgfscope}%
\end{pgfscope}%
\begin{pgfscope}%
\definecolor{textcolor}{rgb}{0.000000,0.000000,0.000000}%
\pgfsetstrokecolor{textcolor}%
\pgfsetfillcolor{textcolor}%
\pgftext[x=2.246798in,y=0.474382in,,top]{\color{textcolor}\sffamily\fontsize{10.000000}{12.000000}\selectfont 30}%
\end{pgfscope}%
\begin{pgfscope}%
\pgfpathrectangle{\pgfqpoint{0.702287in}{0.571604in}}{\pgfqpoint{2.059348in}{2.225635in}}%
\pgfusepath{clip}%
\pgfsetrectcap%
\pgfsetroundjoin%
\pgfsetlinewidth{0.803000pt}%
\definecolor{currentstroke}{rgb}{0.690196,0.690196,0.690196}%
\pgfsetstrokecolor{currentstroke}%
\pgfsetdash{}{0pt}%
\pgfpathmoveto{\pgfqpoint{2.761635in}{0.571604in}}%
\pgfpathlineto{\pgfqpoint{2.761635in}{2.797238in}}%
\pgfusepath{stroke}%
\end{pgfscope}%
\begin{pgfscope}%
\pgfsetbuttcap%
\pgfsetroundjoin%
\definecolor{currentfill}{rgb}{0.000000,0.000000,0.000000}%
\pgfsetfillcolor{currentfill}%
\pgfsetlinewidth{0.803000pt}%
\definecolor{currentstroke}{rgb}{0.000000,0.000000,0.000000}%
\pgfsetstrokecolor{currentstroke}%
\pgfsetdash{}{0pt}%
\pgfsys@defobject{currentmarker}{\pgfqpoint{0.000000in}{-0.048611in}}{\pgfqpoint{0.000000in}{0.000000in}}{%
\pgfpathmoveto{\pgfqpoint{0.000000in}{0.000000in}}%
\pgfpathlineto{\pgfqpoint{0.000000in}{-0.048611in}}%
\pgfusepath{stroke,fill}%
}%
\begin{pgfscope}%
\pgfsys@transformshift{2.761635in}{0.571604in}%
\pgfsys@useobject{currentmarker}{}%
\end{pgfscope}%
\end{pgfscope}%
\begin{pgfscope}%
\definecolor{textcolor}{rgb}{0.000000,0.000000,0.000000}%
\pgfsetstrokecolor{textcolor}%
\pgfsetfillcolor{textcolor}%
\pgftext[x=2.761635in,y=0.474382in,,top]{\color{textcolor}\sffamily\fontsize{10.000000}{12.000000}\selectfont 40}%
\end{pgfscope}%
\begin{pgfscope}%
\definecolor{textcolor}{rgb}{0.000000,0.000000,0.000000}%
\pgfsetstrokecolor{textcolor}%
\pgfsetfillcolor{textcolor}%
\pgftext[x=1.731961in,y=0.284413in,,top]{\color{textcolor}\sffamily\fontsize{10.000000}{12.000000}\selectfont Kameraverschiebung [cm]}%
\end{pgfscope}%
\begin{pgfscope}%
\pgfpathrectangle{\pgfqpoint{0.702287in}{0.571604in}}{\pgfqpoint{2.059348in}{2.225635in}}%
\pgfusepath{clip}%
\pgfsetrectcap%
\pgfsetroundjoin%
\pgfsetlinewidth{0.803000pt}%
\definecolor{currentstroke}{rgb}{0.690196,0.690196,0.690196}%
\pgfsetstrokecolor{currentstroke}%
\pgfsetdash{}{0pt}%
\pgfpathmoveto{\pgfqpoint{0.702287in}{0.571604in}}%
\pgfpathlineto{\pgfqpoint{2.761635in}{0.571604in}}%
\pgfusepath{stroke}%
\end{pgfscope}%
\begin{pgfscope}%
\pgfsetbuttcap%
\pgfsetroundjoin%
\definecolor{currentfill}{rgb}{0.000000,0.000000,0.000000}%
\pgfsetfillcolor{currentfill}%
\pgfsetlinewidth{0.803000pt}%
\definecolor{currentstroke}{rgb}{0.000000,0.000000,0.000000}%
\pgfsetstrokecolor{currentstroke}%
\pgfsetdash{}{0pt}%
\pgfsys@defobject{currentmarker}{\pgfqpoint{-0.048611in}{0.000000in}}{\pgfqpoint{0.000000in}{0.000000in}}{%
\pgfpathmoveto{\pgfqpoint{0.000000in}{0.000000in}}%
\pgfpathlineto{\pgfqpoint{-0.048611in}{0.000000in}}%
\pgfusepath{stroke,fill}%
}%
\begin{pgfscope}%
\pgfsys@transformshift{0.702287in}{0.571604in}%
\pgfsys@useobject{currentmarker}{}%
\end{pgfscope}%
\end{pgfscope}%
\begin{pgfscope}%
\definecolor{textcolor}{rgb}{0.000000,0.000000,0.000000}%
\pgfsetstrokecolor{textcolor}%
\pgfsetfillcolor{textcolor}%
\pgftext[x=0.516699in, y=0.518842in, left, base]{\color{textcolor}\sffamily\fontsize{10.000000}{12.000000}\selectfont 0}%
\end{pgfscope}%
\begin{pgfscope}%
\pgfpathrectangle{\pgfqpoint{0.702287in}{0.571604in}}{\pgfqpoint{2.059348in}{2.225635in}}%
\pgfusepath{clip}%
\pgfsetrectcap%
\pgfsetroundjoin%
\pgfsetlinewidth{0.803000pt}%
\definecolor{currentstroke}{rgb}{0.690196,0.690196,0.690196}%
\pgfsetstrokecolor{currentstroke}%
\pgfsetdash{}{0pt}%
\pgfpathmoveto{\pgfqpoint{0.702287in}{0.889552in}}%
\pgfpathlineto{\pgfqpoint{2.761635in}{0.889552in}}%
\pgfusepath{stroke}%
\end{pgfscope}%
\begin{pgfscope}%
\pgfsetbuttcap%
\pgfsetroundjoin%
\definecolor{currentfill}{rgb}{0.000000,0.000000,0.000000}%
\pgfsetfillcolor{currentfill}%
\pgfsetlinewidth{0.803000pt}%
\definecolor{currentstroke}{rgb}{0.000000,0.000000,0.000000}%
\pgfsetstrokecolor{currentstroke}%
\pgfsetdash{}{0pt}%
\pgfsys@defobject{currentmarker}{\pgfqpoint{-0.048611in}{0.000000in}}{\pgfqpoint{0.000000in}{0.000000in}}{%
\pgfpathmoveto{\pgfqpoint{0.000000in}{0.000000in}}%
\pgfpathlineto{\pgfqpoint{-0.048611in}{0.000000in}}%
\pgfusepath{stroke,fill}%
}%
\begin{pgfscope}%
\pgfsys@transformshift{0.702287in}{0.889552in}%
\pgfsys@useobject{currentmarker}{}%
\end{pgfscope}%
\end{pgfscope}%
\begin{pgfscope}%
\definecolor{textcolor}{rgb}{0.000000,0.000000,0.000000}%
\pgfsetstrokecolor{textcolor}%
\pgfsetfillcolor{textcolor}%
\pgftext[x=0.428334in, y=0.836790in, left, base]{\color{textcolor}\sffamily\fontsize{10.000000}{12.000000}\selectfont 20}%
\end{pgfscope}%
\begin{pgfscope}%
\pgfpathrectangle{\pgfqpoint{0.702287in}{0.571604in}}{\pgfqpoint{2.059348in}{2.225635in}}%
\pgfusepath{clip}%
\pgfsetrectcap%
\pgfsetroundjoin%
\pgfsetlinewidth{0.803000pt}%
\definecolor{currentstroke}{rgb}{0.690196,0.690196,0.690196}%
\pgfsetstrokecolor{currentstroke}%
\pgfsetdash{}{0pt}%
\pgfpathmoveto{\pgfqpoint{0.702287in}{1.207499in}}%
\pgfpathlineto{\pgfqpoint{2.761635in}{1.207499in}}%
\pgfusepath{stroke}%
\end{pgfscope}%
\begin{pgfscope}%
\pgfsetbuttcap%
\pgfsetroundjoin%
\definecolor{currentfill}{rgb}{0.000000,0.000000,0.000000}%
\pgfsetfillcolor{currentfill}%
\pgfsetlinewidth{0.803000pt}%
\definecolor{currentstroke}{rgb}{0.000000,0.000000,0.000000}%
\pgfsetstrokecolor{currentstroke}%
\pgfsetdash{}{0pt}%
\pgfsys@defobject{currentmarker}{\pgfqpoint{-0.048611in}{0.000000in}}{\pgfqpoint{0.000000in}{0.000000in}}{%
\pgfpathmoveto{\pgfqpoint{0.000000in}{0.000000in}}%
\pgfpathlineto{\pgfqpoint{-0.048611in}{0.000000in}}%
\pgfusepath{stroke,fill}%
}%
\begin{pgfscope}%
\pgfsys@transformshift{0.702287in}{1.207499in}%
\pgfsys@useobject{currentmarker}{}%
\end{pgfscope}%
\end{pgfscope}%
\begin{pgfscope}%
\definecolor{textcolor}{rgb}{0.000000,0.000000,0.000000}%
\pgfsetstrokecolor{textcolor}%
\pgfsetfillcolor{textcolor}%
\pgftext[x=0.428334in, y=1.154738in, left, base]{\color{textcolor}\sffamily\fontsize{10.000000}{12.000000}\selectfont 40}%
\end{pgfscope}%
\begin{pgfscope}%
\pgfpathrectangle{\pgfqpoint{0.702287in}{0.571604in}}{\pgfqpoint{2.059348in}{2.225635in}}%
\pgfusepath{clip}%
\pgfsetrectcap%
\pgfsetroundjoin%
\pgfsetlinewidth{0.803000pt}%
\definecolor{currentstroke}{rgb}{0.690196,0.690196,0.690196}%
\pgfsetstrokecolor{currentstroke}%
\pgfsetdash{}{0pt}%
\pgfpathmoveto{\pgfqpoint{0.702287in}{1.525447in}}%
\pgfpathlineto{\pgfqpoint{2.761635in}{1.525447in}}%
\pgfusepath{stroke}%
\end{pgfscope}%
\begin{pgfscope}%
\pgfsetbuttcap%
\pgfsetroundjoin%
\definecolor{currentfill}{rgb}{0.000000,0.000000,0.000000}%
\pgfsetfillcolor{currentfill}%
\pgfsetlinewidth{0.803000pt}%
\definecolor{currentstroke}{rgb}{0.000000,0.000000,0.000000}%
\pgfsetstrokecolor{currentstroke}%
\pgfsetdash{}{0pt}%
\pgfsys@defobject{currentmarker}{\pgfqpoint{-0.048611in}{0.000000in}}{\pgfqpoint{0.000000in}{0.000000in}}{%
\pgfpathmoveto{\pgfqpoint{0.000000in}{0.000000in}}%
\pgfpathlineto{\pgfqpoint{-0.048611in}{0.000000in}}%
\pgfusepath{stroke,fill}%
}%
\begin{pgfscope}%
\pgfsys@transformshift{0.702287in}{1.525447in}%
\pgfsys@useobject{currentmarker}{}%
\end{pgfscope}%
\end{pgfscope}%
\begin{pgfscope}%
\definecolor{textcolor}{rgb}{0.000000,0.000000,0.000000}%
\pgfsetstrokecolor{textcolor}%
\pgfsetfillcolor{textcolor}%
\pgftext[x=0.428334in, y=1.472686in, left, base]{\color{textcolor}\sffamily\fontsize{10.000000}{12.000000}\selectfont 60}%
\end{pgfscope}%
\begin{pgfscope}%
\pgfpathrectangle{\pgfqpoint{0.702287in}{0.571604in}}{\pgfqpoint{2.059348in}{2.225635in}}%
\pgfusepath{clip}%
\pgfsetrectcap%
\pgfsetroundjoin%
\pgfsetlinewidth{0.803000pt}%
\definecolor{currentstroke}{rgb}{0.690196,0.690196,0.690196}%
\pgfsetstrokecolor{currentstroke}%
\pgfsetdash{}{0pt}%
\pgfpathmoveto{\pgfqpoint{0.702287in}{1.843395in}}%
\pgfpathlineto{\pgfqpoint{2.761635in}{1.843395in}}%
\pgfusepath{stroke}%
\end{pgfscope}%
\begin{pgfscope}%
\pgfsetbuttcap%
\pgfsetroundjoin%
\definecolor{currentfill}{rgb}{0.000000,0.000000,0.000000}%
\pgfsetfillcolor{currentfill}%
\pgfsetlinewidth{0.803000pt}%
\definecolor{currentstroke}{rgb}{0.000000,0.000000,0.000000}%
\pgfsetstrokecolor{currentstroke}%
\pgfsetdash{}{0pt}%
\pgfsys@defobject{currentmarker}{\pgfqpoint{-0.048611in}{0.000000in}}{\pgfqpoint{0.000000in}{0.000000in}}{%
\pgfpathmoveto{\pgfqpoint{0.000000in}{0.000000in}}%
\pgfpathlineto{\pgfqpoint{-0.048611in}{0.000000in}}%
\pgfusepath{stroke,fill}%
}%
\begin{pgfscope}%
\pgfsys@transformshift{0.702287in}{1.843395in}%
\pgfsys@useobject{currentmarker}{}%
\end{pgfscope}%
\end{pgfscope}%
\begin{pgfscope}%
\definecolor{textcolor}{rgb}{0.000000,0.000000,0.000000}%
\pgfsetstrokecolor{textcolor}%
\pgfsetfillcolor{textcolor}%
\pgftext[x=0.428334in, y=1.790634in, left, base]{\color{textcolor}\sffamily\fontsize{10.000000}{12.000000}\selectfont 80}%
\end{pgfscope}%
\begin{pgfscope}%
\pgfpathrectangle{\pgfqpoint{0.702287in}{0.571604in}}{\pgfqpoint{2.059348in}{2.225635in}}%
\pgfusepath{clip}%
\pgfsetrectcap%
\pgfsetroundjoin%
\pgfsetlinewidth{0.803000pt}%
\definecolor{currentstroke}{rgb}{0.690196,0.690196,0.690196}%
\pgfsetstrokecolor{currentstroke}%
\pgfsetdash{}{0pt}%
\pgfpathmoveto{\pgfqpoint{0.702287in}{2.161343in}}%
\pgfpathlineto{\pgfqpoint{2.761635in}{2.161343in}}%
\pgfusepath{stroke}%
\end{pgfscope}%
\begin{pgfscope}%
\pgfsetbuttcap%
\pgfsetroundjoin%
\definecolor{currentfill}{rgb}{0.000000,0.000000,0.000000}%
\pgfsetfillcolor{currentfill}%
\pgfsetlinewidth{0.803000pt}%
\definecolor{currentstroke}{rgb}{0.000000,0.000000,0.000000}%
\pgfsetstrokecolor{currentstroke}%
\pgfsetdash{}{0pt}%
\pgfsys@defobject{currentmarker}{\pgfqpoint{-0.048611in}{0.000000in}}{\pgfqpoint{0.000000in}{0.000000in}}{%
\pgfpathmoveto{\pgfqpoint{0.000000in}{0.000000in}}%
\pgfpathlineto{\pgfqpoint{-0.048611in}{0.000000in}}%
\pgfusepath{stroke,fill}%
}%
\begin{pgfscope}%
\pgfsys@transformshift{0.702287in}{2.161343in}%
\pgfsys@useobject{currentmarker}{}%
\end{pgfscope}%
\end{pgfscope}%
\begin{pgfscope}%
\definecolor{textcolor}{rgb}{0.000000,0.000000,0.000000}%
\pgfsetstrokecolor{textcolor}%
\pgfsetfillcolor{textcolor}%
\pgftext[x=0.339969in, y=2.108581in, left, base]{\color{textcolor}\sffamily\fontsize{10.000000}{12.000000}\selectfont 100}%
\end{pgfscope}%
\begin{pgfscope}%
\pgfpathrectangle{\pgfqpoint{0.702287in}{0.571604in}}{\pgfqpoint{2.059348in}{2.225635in}}%
\pgfusepath{clip}%
\pgfsetrectcap%
\pgfsetroundjoin%
\pgfsetlinewidth{0.803000pt}%
\definecolor{currentstroke}{rgb}{0.690196,0.690196,0.690196}%
\pgfsetstrokecolor{currentstroke}%
\pgfsetdash{}{0pt}%
\pgfpathmoveto{\pgfqpoint{0.702287in}{2.479291in}}%
\pgfpathlineto{\pgfqpoint{2.761635in}{2.479291in}}%
\pgfusepath{stroke}%
\end{pgfscope}%
\begin{pgfscope}%
\pgfsetbuttcap%
\pgfsetroundjoin%
\definecolor{currentfill}{rgb}{0.000000,0.000000,0.000000}%
\pgfsetfillcolor{currentfill}%
\pgfsetlinewidth{0.803000pt}%
\definecolor{currentstroke}{rgb}{0.000000,0.000000,0.000000}%
\pgfsetstrokecolor{currentstroke}%
\pgfsetdash{}{0pt}%
\pgfsys@defobject{currentmarker}{\pgfqpoint{-0.048611in}{0.000000in}}{\pgfqpoint{0.000000in}{0.000000in}}{%
\pgfpathmoveto{\pgfqpoint{0.000000in}{0.000000in}}%
\pgfpathlineto{\pgfqpoint{-0.048611in}{0.000000in}}%
\pgfusepath{stroke,fill}%
}%
\begin{pgfscope}%
\pgfsys@transformshift{0.702287in}{2.479291in}%
\pgfsys@useobject{currentmarker}{}%
\end{pgfscope}%
\end{pgfscope}%
\begin{pgfscope}%
\definecolor{textcolor}{rgb}{0.000000,0.000000,0.000000}%
\pgfsetstrokecolor{textcolor}%
\pgfsetfillcolor{textcolor}%
\pgftext[x=0.339969in, y=2.426529in, left, base]{\color{textcolor}\sffamily\fontsize{10.000000}{12.000000}\selectfont 120}%
\end{pgfscope}%
\begin{pgfscope}%
\pgfpathrectangle{\pgfqpoint{0.702287in}{0.571604in}}{\pgfqpoint{2.059348in}{2.225635in}}%
\pgfusepath{clip}%
\pgfsetrectcap%
\pgfsetroundjoin%
\pgfsetlinewidth{0.803000pt}%
\definecolor{currentstroke}{rgb}{0.690196,0.690196,0.690196}%
\pgfsetstrokecolor{currentstroke}%
\pgfsetdash{}{0pt}%
\pgfpathmoveto{\pgfqpoint{0.702287in}{2.797238in}}%
\pgfpathlineto{\pgfqpoint{2.761635in}{2.797238in}}%
\pgfusepath{stroke}%
\end{pgfscope}%
\begin{pgfscope}%
\pgfsetbuttcap%
\pgfsetroundjoin%
\definecolor{currentfill}{rgb}{0.000000,0.000000,0.000000}%
\pgfsetfillcolor{currentfill}%
\pgfsetlinewidth{0.803000pt}%
\definecolor{currentstroke}{rgb}{0.000000,0.000000,0.000000}%
\pgfsetstrokecolor{currentstroke}%
\pgfsetdash{}{0pt}%
\pgfsys@defobject{currentmarker}{\pgfqpoint{-0.048611in}{0.000000in}}{\pgfqpoint{0.000000in}{0.000000in}}{%
\pgfpathmoveto{\pgfqpoint{0.000000in}{0.000000in}}%
\pgfpathlineto{\pgfqpoint{-0.048611in}{0.000000in}}%
\pgfusepath{stroke,fill}%
}%
\begin{pgfscope}%
\pgfsys@transformshift{0.702287in}{2.797238in}%
\pgfsys@useobject{currentmarker}{}%
\end{pgfscope}%
\end{pgfscope}%
\begin{pgfscope}%
\definecolor{textcolor}{rgb}{0.000000,0.000000,0.000000}%
\pgfsetstrokecolor{textcolor}%
\pgfsetfillcolor{textcolor}%
\pgftext[x=0.339969in, y=2.744477in, left, base]{\color{textcolor}\sffamily\fontsize{10.000000}{12.000000}\selectfont 140}%
\end{pgfscope}%
\begin{pgfscope}%
\definecolor{textcolor}{rgb}{0.000000,0.000000,0.000000}%
\pgfsetstrokecolor{textcolor}%
\pgfsetfillcolor{textcolor}%
\pgftext[x=0.284413in,y=1.684421in,,bottom,rotate=90.000000]{\color{textcolor}\sffamily\fontsize{10.000000}{12.000000}\selectfont Bildverschiebung [pixel]}%
\end{pgfscope}%
\begin{pgfscope}%
\pgfpathrectangle{\pgfqpoint{0.702287in}{0.571604in}}{\pgfqpoint{2.059348in}{2.225635in}}%
\pgfusepath{clip}%
\pgfsetbuttcap%
\pgfsetroundjoin%
\pgfsetlinewidth{1.505625pt}%
\definecolor{currentstroke}{rgb}{0.121569,0.466667,0.705882}%
\pgfsetstrokecolor{currentstroke}%
\pgfsetdash{}{0pt}%
\pgfpathmoveto{\pgfqpoint{0.933963in}{0.829778in}}%
\pgfpathlineto{\pgfqpoint{0.985447in}{0.829778in}}%
\pgfusepath{stroke}%
\end{pgfscope}%
\begin{pgfscope}%
\pgfpathrectangle{\pgfqpoint{0.702287in}{0.571604in}}{\pgfqpoint{2.059348in}{2.225635in}}%
\pgfusepath{clip}%
\pgfsetbuttcap%
\pgfsetroundjoin%
\pgfsetlinewidth{1.505625pt}%
\definecolor{currentstroke}{rgb}{0.121569,0.466667,0.705882}%
\pgfsetstrokecolor{currentstroke}%
\pgfsetdash{}{0pt}%
\pgfpathmoveto{\pgfqpoint{1.191382in}{1.108459in}}%
\pgfpathlineto{\pgfqpoint{1.242866in}{1.108459in}}%
\pgfusepath{stroke}%
\end{pgfscope}%
\begin{pgfscope}%
\pgfpathrectangle{\pgfqpoint{0.702287in}{0.571604in}}{\pgfqpoint{2.059348in}{2.225635in}}%
\pgfusepath{clip}%
\pgfsetbuttcap%
\pgfsetroundjoin%
\pgfsetlinewidth{1.505625pt}%
\definecolor{currentstroke}{rgb}{0.121569,0.466667,0.705882}%
\pgfsetstrokecolor{currentstroke}%
\pgfsetdash{}{0pt}%
\pgfpathmoveto{\pgfqpoint{1.448800in}{1.366632in}}%
\pgfpathlineto{\pgfqpoint{1.500284in}{1.366632in}}%
\pgfusepath{stroke}%
\end{pgfscope}%
\begin{pgfscope}%
\pgfpathrectangle{\pgfqpoint{0.702287in}{0.571604in}}{\pgfqpoint{2.059348in}{2.225635in}}%
\pgfusepath{clip}%
\pgfsetbuttcap%
\pgfsetroundjoin%
\pgfsetlinewidth{1.505625pt}%
\definecolor{currentstroke}{rgb}{0.121569,0.466667,0.705882}%
\pgfsetstrokecolor{currentstroke}%
\pgfsetdash{}{0pt}%
\pgfpathmoveto{\pgfqpoint{1.706219in}{1.649606in}}%
\pgfpathlineto{\pgfqpoint{1.757703in}{1.649606in}}%
\pgfusepath{stroke}%
\end{pgfscope}%
\begin{pgfscope}%
\pgfpathrectangle{\pgfqpoint{0.702287in}{0.571604in}}{\pgfqpoint{2.059348in}{2.225635in}}%
\pgfusepath{clip}%
\pgfsetbuttcap%
\pgfsetroundjoin%
\pgfsetlinewidth{1.505625pt}%
\definecolor{currentstroke}{rgb}{0.121569,0.466667,0.705882}%
\pgfsetstrokecolor{currentstroke}%
\pgfsetdash{}{0pt}%
\pgfpathmoveto{\pgfqpoint{2.478474in}{2.464347in}}%
\pgfpathlineto{\pgfqpoint{2.529958in}{2.464347in}}%
\pgfusepath{stroke}%
\end{pgfscope}%
\begin{pgfscope}%
\pgfpathrectangle{\pgfqpoint{0.702287in}{0.571604in}}{\pgfqpoint{2.059348in}{2.225635in}}%
\pgfusepath{clip}%
\pgfsetrectcap%
\pgfsetroundjoin%
\pgfsetlinewidth{1.505625pt}%
\definecolor{currentstroke}{rgb}{1.000000,0.498039,0.054902}%
\pgfsetstrokecolor{currentstroke}%
\pgfsetdash{}{0pt}%
\pgfpathmoveto{\pgfqpoint{0.702287in}{0.571604in}}%
\pgfpathlineto{\pgfqpoint{0.744314in}{0.615578in}}%
\pgfpathlineto{\pgfqpoint{0.786342in}{0.659553in}}%
\pgfpathlineto{\pgfqpoint{0.828369in}{0.703528in}}%
\pgfpathlineto{\pgfqpoint{0.870397in}{0.747502in}}%
\pgfpathlineto{\pgfqpoint{0.912424in}{0.791477in}}%
\pgfpathlineto{\pgfqpoint{0.954452in}{0.835451in}}%
\pgfpathlineto{\pgfqpoint{0.996479in}{0.879426in}}%
\pgfpathlineto{\pgfqpoint{1.038507in}{0.923400in}}%
\pgfpathlineto{\pgfqpoint{1.080534in}{0.967375in}}%
\pgfpathlineto{\pgfqpoint{1.122562in}{1.011349in}}%
\pgfpathlineto{\pgfqpoint{1.164589in}{1.055324in}}%
\pgfpathlineto{\pgfqpoint{1.206617in}{1.099298in}}%
\pgfpathlineto{\pgfqpoint{1.248644in}{1.143273in}}%
\pgfpathlineto{\pgfqpoint{1.290672in}{1.187247in}}%
\pgfpathlineto{\pgfqpoint{1.332699in}{1.231222in}}%
\pgfpathlineto{\pgfqpoint{1.374727in}{1.275196in}}%
\pgfpathlineto{\pgfqpoint{1.416754in}{1.319171in}}%
\pgfpathlineto{\pgfqpoint{1.458782in}{1.363146in}}%
\pgfpathlineto{\pgfqpoint{1.500809in}{1.407120in}}%
\pgfpathlineto{\pgfqpoint{1.542837in}{1.451095in}}%
\pgfpathlineto{\pgfqpoint{1.584864in}{1.495069in}}%
\pgfpathlineto{\pgfqpoint{1.626892in}{1.539044in}}%
\pgfpathlineto{\pgfqpoint{1.668919in}{1.583018in}}%
\pgfpathlineto{\pgfqpoint{1.710947in}{1.626993in}}%
\pgfpathlineto{\pgfqpoint{1.752974in}{1.670967in}}%
\pgfpathlineto{\pgfqpoint{1.795002in}{1.714942in}}%
\pgfpathlineto{\pgfqpoint{1.837029in}{1.758916in}}%
\pgfpathlineto{\pgfqpoint{1.879057in}{1.802891in}}%
\pgfpathlineto{\pgfqpoint{1.921085in}{1.846865in}}%
\pgfpathlineto{\pgfqpoint{1.963112in}{1.890840in}}%
\pgfpathlineto{\pgfqpoint{2.005140in}{1.934815in}}%
\pgfpathlineto{\pgfqpoint{2.047167in}{1.978789in}}%
\pgfpathlineto{\pgfqpoint{2.089195in}{2.022764in}}%
\pgfpathlineto{\pgfqpoint{2.131222in}{2.066738in}}%
\pgfpathlineto{\pgfqpoint{2.173250in}{2.110713in}}%
\pgfpathlineto{\pgfqpoint{2.215277in}{2.154687in}}%
\pgfpathlineto{\pgfqpoint{2.257305in}{2.198662in}}%
\pgfpathlineto{\pgfqpoint{2.299332in}{2.242636in}}%
\pgfpathlineto{\pgfqpoint{2.341360in}{2.286611in}}%
\pgfpathlineto{\pgfqpoint{2.383387in}{2.330585in}}%
\pgfpathlineto{\pgfqpoint{2.425415in}{2.374560in}}%
\pgfpathlineto{\pgfqpoint{2.467442in}{2.418534in}}%
\pgfpathlineto{\pgfqpoint{2.509470in}{2.462509in}}%
\pgfpathlineto{\pgfqpoint{2.551497in}{2.506484in}}%
\pgfpathlineto{\pgfqpoint{2.593525in}{2.550458in}}%
\pgfpathlineto{\pgfqpoint{2.635552in}{2.594433in}}%
\pgfpathlineto{\pgfqpoint{2.677580in}{2.638407in}}%
\pgfpathlineto{\pgfqpoint{2.719607in}{2.682382in}}%
\pgfpathlineto{\pgfqpoint{2.761635in}{2.726356in}}%
\pgfusepath{stroke}%
\end{pgfscope}%
\begin{pgfscope}%
\pgfpathrectangle{\pgfqpoint{0.702287in}{0.571604in}}{\pgfqpoint{2.059348in}{2.225635in}}%
\pgfusepath{clip}%
\pgfsetbuttcap%
\pgfsetroundjoin%
\definecolor{currentfill}{rgb}{0.121569,0.466667,0.705882}%
\pgfsetfillcolor{currentfill}%
\pgfsetlinewidth{1.003750pt}%
\definecolor{currentstroke}{rgb}{0.121569,0.466667,0.705882}%
\pgfsetstrokecolor{currentstroke}%
\pgfsetdash{}{0pt}%
\pgfsys@defobject{currentmarker}{\pgfqpoint{-0.020833in}{-0.020833in}}{\pgfqpoint{0.020833in}{0.020833in}}{%
\pgfpathmoveto{\pgfqpoint{0.000000in}{-0.020833in}}%
\pgfpathcurveto{\pgfqpoint{0.005525in}{-0.020833in}}{\pgfqpoint{0.010825in}{-0.018638in}}{\pgfqpoint{0.014731in}{-0.014731in}}%
\pgfpathcurveto{\pgfqpoint{0.018638in}{-0.010825in}}{\pgfqpoint{0.020833in}{-0.005525in}}{\pgfqpoint{0.020833in}{0.000000in}}%
\pgfpathcurveto{\pgfqpoint{0.020833in}{0.005525in}}{\pgfqpoint{0.018638in}{0.010825in}}{\pgfqpoint{0.014731in}{0.014731in}}%
\pgfpathcurveto{\pgfqpoint{0.010825in}{0.018638in}}{\pgfqpoint{0.005525in}{0.020833in}}{\pgfqpoint{0.000000in}{0.020833in}}%
\pgfpathcurveto{\pgfqpoint{-0.005525in}{0.020833in}}{\pgfqpoint{-0.010825in}{0.018638in}}{\pgfqpoint{-0.014731in}{0.014731in}}%
\pgfpathcurveto{\pgfqpoint{-0.018638in}{0.010825in}}{\pgfqpoint{-0.020833in}{0.005525in}}{\pgfqpoint{-0.020833in}{0.000000in}}%
\pgfpathcurveto{\pgfqpoint{-0.020833in}{-0.005525in}}{\pgfqpoint{-0.018638in}{-0.010825in}}{\pgfqpoint{-0.014731in}{-0.014731in}}%
\pgfpathcurveto{\pgfqpoint{-0.010825in}{-0.018638in}}{\pgfqpoint{-0.005525in}{-0.020833in}}{\pgfqpoint{0.000000in}{-0.020833in}}%
\pgfpathclose%
\pgfusepath{stroke,fill}%
}%
\begin{pgfscope}%
\pgfsys@transformshift{0.959705in}{0.829778in}%
\pgfsys@useobject{currentmarker}{}%
\end{pgfscope}%
\begin{pgfscope}%
\pgfsys@transformshift{1.217124in}{1.108459in}%
\pgfsys@useobject{currentmarker}{}%
\end{pgfscope}%
\begin{pgfscope}%
\pgfsys@transformshift{1.474542in}{1.366632in}%
\pgfsys@useobject{currentmarker}{}%
\end{pgfscope}%
\begin{pgfscope}%
\pgfsys@transformshift{1.731961in}{1.649606in}%
\pgfsys@useobject{currentmarker}{}%
\end{pgfscope}%
\begin{pgfscope}%
\pgfsys@transformshift{2.504216in}{2.464347in}%
\pgfsys@useobject{currentmarker}{}%
\end{pgfscope}%
\end{pgfscope}%
\begin{pgfscope}%
\pgfsetrectcap%
\pgfsetmiterjoin%
\pgfsetlinewidth{0.803000pt}%
\definecolor{currentstroke}{rgb}{0.000000,0.000000,0.000000}%
\pgfsetstrokecolor{currentstroke}%
\pgfsetdash{}{0pt}%
\pgfpathmoveto{\pgfqpoint{0.702287in}{0.571604in}}%
\pgfpathlineto{\pgfqpoint{0.702287in}{2.797238in}}%
\pgfusepath{stroke}%
\end{pgfscope}%
\begin{pgfscope}%
\pgfsetrectcap%
\pgfsetmiterjoin%
\pgfsetlinewidth{0.803000pt}%
\definecolor{currentstroke}{rgb}{0.000000,0.000000,0.000000}%
\pgfsetstrokecolor{currentstroke}%
\pgfsetdash{}{0pt}%
\pgfpathmoveto{\pgfqpoint{2.761635in}{0.571604in}}%
\pgfpathlineto{\pgfqpoint{2.761635in}{2.797238in}}%
\pgfusepath{stroke}%
\end{pgfscope}%
\begin{pgfscope}%
\pgfsetrectcap%
\pgfsetmiterjoin%
\pgfsetlinewidth{0.803000pt}%
\definecolor{currentstroke}{rgb}{0.000000,0.000000,0.000000}%
\pgfsetstrokecolor{currentstroke}%
\pgfsetdash{}{0pt}%
\pgfpathmoveto{\pgfqpoint{0.702287in}{0.571604in}}%
\pgfpathlineto{\pgfqpoint{2.761635in}{0.571604in}}%
\pgfusepath{stroke}%
\end{pgfscope}%
\begin{pgfscope}%
\pgfsetrectcap%
\pgfsetmiterjoin%
\pgfsetlinewidth{0.803000pt}%
\definecolor{currentstroke}{rgb}{0.000000,0.000000,0.000000}%
\pgfsetstrokecolor{currentstroke}%
\pgfsetdash{}{0pt}%
\pgfpathmoveto{\pgfqpoint{0.702287in}{2.797238in}}%
\pgfpathlineto{\pgfqpoint{2.761635in}{2.797238in}}%
\pgfusepath{stroke}%
\end{pgfscope}%
\begin{pgfscope}%
\pgfsetbuttcap%
\pgfsetmiterjoin%
\definecolor{currentfill}{rgb}{1.000000,1.000000,1.000000}%
\pgfsetfillcolor{currentfill}%
\pgfsetfillopacity{0.800000}%
\pgfsetlinewidth{1.003750pt}%
\definecolor{currentstroke}{rgb}{0.800000,0.800000,0.800000}%
\pgfsetstrokecolor{currentstroke}%
\pgfsetstrokeopacity{0.800000}%
\pgfsetdash{}{0pt}%
\pgfpathmoveto{\pgfqpoint{0.799509in}{2.278412in}}%
\pgfpathlineto{\pgfqpoint{1.990007in}{2.278412in}}%
\pgfpathquadraticcurveto{\pgfqpoint{2.017784in}{2.278412in}}{\pgfqpoint{2.017784in}{2.306190in}}%
\pgfpathlineto{\pgfqpoint{2.017784in}{2.700016in}}%
\pgfpathquadraticcurveto{\pgfqpoint{2.017784in}{2.727794in}}{\pgfqpoint{1.990007in}{2.727794in}}%
\pgfpathlineto{\pgfqpoint{0.799509in}{2.727794in}}%
\pgfpathquadraticcurveto{\pgfqpoint{0.771731in}{2.727794in}}{\pgfqpoint{0.771731in}{2.700016in}}%
\pgfpathlineto{\pgfqpoint{0.771731in}{2.306190in}}%
\pgfpathquadraticcurveto{\pgfqpoint{0.771731in}{2.278412in}}{\pgfqpoint{0.799509in}{2.278412in}}%
\pgfpathclose%
\pgfusepath{stroke,fill}%
\end{pgfscope}%
\begin{pgfscope}%
\pgfsetrectcap%
\pgfsetroundjoin%
\pgfsetlinewidth{1.505625pt}%
\definecolor{currentstroke}{rgb}{1.000000,0.498039,0.054902}%
\pgfsetstrokecolor{currentstroke}%
\pgfsetdash{}{0pt}%
\pgfpathmoveto{\pgfqpoint{0.827287in}{2.615327in}}%
\pgfpathlineto{\pgfqpoint{1.105065in}{2.615327in}}%
\pgfusepath{stroke}%
\end{pgfscope}%
\begin{pgfscope}%
\definecolor{textcolor}{rgb}{0.000000,0.000000,0.000000}%
\pgfsetstrokecolor{textcolor}%
\pgfsetfillcolor{textcolor}%
\pgftext[x=1.216176in,y=2.566715in,left,base]{\color{textcolor}\sffamily\fontsize{10.000000}{12.000000}\selectfont Fit}%
\end{pgfscope}%
\begin{pgfscope}%
\pgfsetbuttcap%
\pgfsetroundjoin%
\pgfsetlinewidth{1.505625pt}%
\definecolor{currentstroke}{rgb}{0.121569,0.466667,0.705882}%
\pgfsetstrokecolor{currentstroke}%
\pgfsetdash{}{0pt}%
\pgfpathmoveto{\pgfqpoint{0.896731in}{2.411469in}}%
\pgfpathlineto{\pgfqpoint{1.035620in}{2.411469in}}%
\pgfusepath{stroke}%
\end{pgfscope}%
\begin{pgfscope}%
\pgfsetbuttcap%
\pgfsetroundjoin%
\definecolor{currentfill}{rgb}{0.121569,0.466667,0.705882}%
\pgfsetfillcolor{currentfill}%
\pgfsetlinewidth{1.003750pt}%
\definecolor{currentstroke}{rgb}{0.121569,0.466667,0.705882}%
\pgfsetstrokecolor{currentstroke}%
\pgfsetdash{}{0pt}%
\pgfsys@defobject{currentmarker}{\pgfqpoint{-0.020833in}{-0.020833in}}{\pgfqpoint{0.020833in}{0.020833in}}{%
\pgfpathmoveto{\pgfqpoint{0.000000in}{-0.020833in}}%
\pgfpathcurveto{\pgfqpoint{0.005525in}{-0.020833in}}{\pgfqpoint{0.010825in}{-0.018638in}}{\pgfqpoint{0.014731in}{-0.014731in}}%
\pgfpathcurveto{\pgfqpoint{0.018638in}{-0.010825in}}{\pgfqpoint{0.020833in}{-0.005525in}}{\pgfqpoint{0.020833in}{0.000000in}}%
\pgfpathcurveto{\pgfqpoint{0.020833in}{0.005525in}}{\pgfqpoint{0.018638in}{0.010825in}}{\pgfqpoint{0.014731in}{0.014731in}}%
\pgfpathcurveto{\pgfqpoint{0.010825in}{0.018638in}}{\pgfqpoint{0.005525in}{0.020833in}}{\pgfqpoint{0.000000in}{0.020833in}}%
\pgfpathcurveto{\pgfqpoint{-0.005525in}{0.020833in}}{\pgfqpoint{-0.010825in}{0.018638in}}{\pgfqpoint{-0.014731in}{0.014731in}}%
\pgfpathcurveto{\pgfqpoint{-0.018638in}{0.010825in}}{\pgfqpoint{-0.020833in}{0.005525in}}{\pgfqpoint{-0.020833in}{0.000000in}}%
\pgfpathcurveto{\pgfqpoint{-0.020833in}{-0.005525in}}{\pgfqpoint{-0.018638in}{-0.010825in}}{\pgfqpoint{-0.014731in}{-0.014731in}}%
\pgfpathcurveto{\pgfqpoint{-0.010825in}{-0.018638in}}{\pgfqpoint{-0.005525in}{-0.020833in}}{\pgfqpoint{0.000000in}{-0.020833in}}%
\pgfpathclose%
\pgfusepath{stroke,fill}%
}%
\begin{pgfscope}%
\pgfsys@transformshift{0.966176in}{2.411469in}%
\pgfsys@useobject{currentmarker}{}%
\end{pgfscope}%
\end{pgfscope}%
\begin{pgfscope}%
\definecolor{textcolor}{rgb}{0.000000,0.000000,0.000000}%
\pgfsetstrokecolor{textcolor}%
\pgfsetfillcolor{textcolor}%
\pgftext[x=1.216176in,y=2.362858in,left,base]{\color{textcolor}\sffamily\fontsize{10.000000}{12.000000}\selectfont Messwerte}%
\end{pgfscope}%
\end{pgfpicture}%
\makeatother%
\endgroup%
}
  \vspace{-30pt}
\end{wrapfigure}
Das Translationsverhältnis zwischen Bewegung der Kamera und Verschiebung in
Pixeln wurde über die Transformationsmatrix in Matlab ausgelesen. Zusätzlich
haben wir das Ergebnis testweise an einer Bildüberlagerung überprüft. Die
Messergebnisse des Verhältnisses sind in Abbildung \ref{fig:translrel}
dargestellt.\\
Wir haben \SI{3.39(1)}{pixel\per\centi\meter} gemessen, wobei der Fehler rein
aus der Regression stammt, für welche mögliche Fehler der Verschiebungen und
des Transformationsverfahrens nicht in Betracht gezogen wurden.\\
Betrachtet man Abbildung \ref{pic:transl} (Appendix), so ist leicht zu sehen,
dass die Transformation bei \SI{65}{\centi\meter} nicht mehr funktioniert hat.
Dies könnte unter anderem dadurch entstanden sein, dass sich bei einer
Translation von einer 3-dimensionalen Umgebung auch die Perspektive ändert, je
entfernter die Umgebung im Vergleich zur Translation, desto geringer ist der
Effekt. Den Effekt kann man gut sehen wenn man zum Beispiel den Stuhl im Bild
bei \SI{35}{\centi\meter} Translation mit dem Bild bei \SI{5}{\centi\meter}
vergleicht. Im Ersteren erkennt man an der Lehne deutliche Unterschiede in der
Überlagerung (lila beziehungsweise grün), was im Letzteren kaum sichtbar ist.
Der grüne Balken am rechten Rand kommt daher, da dieser Bereich nicht im
Aufnahmebreich des bewegten Bildes liegt.

\subsection{Ähnlichkeit bei monomodalen Bilddaten}
In Abbildung \ref{fig:ct2ct_rotparam} sind die Variationen der drei
verschiedenen Rotationsparameter dargestellt. An der Periodizität von
\num{2\pi} der Parameter \num{0}, \num{1} und \num{2} erkennt man, dass es
sich hierbei um die Rotationsparameter handelt. Ebenfalls gut zu erkennen ist,
dass das Minimum der Metrik bei \num{0} bzw. \num{2\pi} und \num{4\pi} ist,
wobei das Plateau zwischen den Minima vermutlich dadurch entstanden ist, dass
der Algorithmus keine Ähnlichkeit mehr festellen konnte.
\begin{figure}[h]
  \caption{Metrik in Abhängigkeit der Rotationsparameter}
  \label{fig:ct2ct_rotparam}
  \vspace{-10pt}
  \resizebox{0.32\linewidth}{!}{%% Creator: Matplotlib, PGF backend
%%
%% To include the figure in your LaTeX document, write
%%   \input{<filename>.pgf}
%%
%% Make sure the required packages are loaded in your preamble
%%   \usepackage{pgf}
%%
%% Figures using additional raster images can only be included by \input if
%% they are in the same directory as the main LaTeX file. For loading figures
%% from other directories you can use the `import` package
%%   \usepackage{import}
%% and then include the figures with
%%   \import{<path to file>}{<filename>.pgf}
%%
%% Matplotlib used the following preamble
%%   \usepackage{fontspec}
%%   \setmainfont{DejaVuSerif.ttf}[Path=/usr/share/matplotlib/mpl-data/fonts/ttf/]
%%   \setsansfont{DejaVuSans.ttf}[Path=/usr/share/matplotlib/mpl-data/fonts/ttf/]
%%   \setmonofont{DejaVuSansMono.ttf}[Path=/usr/share/matplotlib/mpl-data/fonts/ttf/]
%%
\begingroup%
\makeatletter%
\begin{pgfpicture}%
\pgfpathrectangle{\pgfpointorigin}{\pgfqpoint{3.000000in}{3.000000in}}%
\pgfusepath{use as bounding box, clip}%
\begin{pgfscope}%
\pgfsetbuttcap%
\pgfsetmiterjoin%
\definecolor{currentfill}{rgb}{1.000000,1.000000,1.000000}%
\pgfsetfillcolor{currentfill}%
\pgfsetlinewidth{0.000000pt}%
\definecolor{currentstroke}{rgb}{1.000000,1.000000,1.000000}%
\pgfsetstrokecolor{currentstroke}%
\pgfsetdash{}{0pt}%
\pgfpathmoveto{\pgfqpoint{0.000000in}{0.000000in}}%
\pgfpathlineto{\pgfqpoint{3.000000in}{0.000000in}}%
\pgfpathlineto{\pgfqpoint{3.000000in}{3.000000in}}%
\pgfpathlineto{\pgfqpoint{0.000000in}{3.000000in}}%
\pgfpathclose%
\pgfusepath{fill}%
\end{pgfscope}%
\begin{pgfscope}%
\pgfsetbuttcap%
\pgfsetmiterjoin%
\definecolor{currentfill}{rgb}{1.000000,1.000000,1.000000}%
\pgfsetfillcolor{currentfill}%
\pgfsetlinewidth{0.000000pt}%
\definecolor{currentstroke}{rgb}{0.000000,0.000000,0.000000}%
\pgfsetstrokecolor{currentstroke}%
\pgfsetstrokeopacity{0.000000}%
\pgfsetdash{}{0pt}%
\pgfpathmoveto{\pgfqpoint{0.584475in}{1.826628in}}%
\pgfpathlineto{\pgfqpoint{2.761635in}{1.826628in}}%
\pgfpathlineto{\pgfqpoint{2.761635in}{2.640039in}}%
\pgfpathlineto{\pgfqpoint{0.584475in}{2.640039in}}%
\pgfpathclose%
\pgfusepath{fill}%
\end{pgfscope}%
\begin{pgfscope}%
\pgfpathrectangle{\pgfqpoint{0.584475in}{1.826628in}}{\pgfqpoint{2.177159in}{0.813411in}}%
\pgfusepath{clip}%
\pgfsetrectcap%
\pgfsetroundjoin%
\pgfsetlinewidth{0.803000pt}%
\definecolor{currentstroke}{rgb}{0.690196,0.690196,0.690196}%
\pgfsetstrokecolor{currentstroke}%
\pgfsetdash{}{0pt}%
\pgfpathmoveto{\pgfqpoint{0.584475in}{1.826628in}}%
\pgfpathlineto{\pgfqpoint{0.584475in}{2.640039in}}%
\pgfusepath{stroke}%
\end{pgfscope}%
\begin{pgfscope}%
\pgfsetbuttcap%
\pgfsetroundjoin%
\definecolor{currentfill}{rgb}{0.000000,0.000000,0.000000}%
\pgfsetfillcolor{currentfill}%
\pgfsetlinewidth{0.803000pt}%
\definecolor{currentstroke}{rgb}{0.000000,0.000000,0.000000}%
\pgfsetstrokecolor{currentstroke}%
\pgfsetdash{}{0pt}%
\pgfsys@defobject{currentmarker}{\pgfqpoint{0.000000in}{-0.048611in}}{\pgfqpoint{0.000000in}{0.000000in}}{%
\pgfpathmoveto{\pgfqpoint{0.000000in}{0.000000in}}%
\pgfpathlineto{\pgfqpoint{0.000000in}{-0.048611in}}%
\pgfusepath{stroke,fill}%
}%
\begin{pgfscope}%
\pgfsys@transformshift{0.584475in}{1.826628in}%
\pgfsys@useobject{currentmarker}{}%
\end{pgfscope}%
\end{pgfscope}%
\begin{pgfscope}%
\pgfpathrectangle{\pgfqpoint{0.584475in}{1.826628in}}{\pgfqpoint{2.177159in}{0.813411in}}%
\pgfusepath{clip}%
\pgfsetrectcap%
\pgfsetroundjoin%
\pgfsetlinewidth{0.803000pt}%
\definecolor{currentstroke}{rgb}{0.690196,0.690196,0.690196}%
\pgfsetstrokecolor{currentstroke}%
\pgfsetdash{}{0pt}%
\pgfpathmoveto{\pgfqpoint{1.310195in}{1.826628in}}%
\pgfpathlineto{\pgfqpoint{1.310195in}{2.640039in}}%
\pgfusepath{stroke}%
\end{pgfscope}%
\begin{pgfscope}%
\pgfsetbuttcap%
\pgfsetroundjoin%
\definecolor{currentfill}{rgb}{0.000000,0.000000,0.000000}%
\pgfsetfillcolor{currentfill}%
\pgfsetlinewidth{0.803000pt}%
\definecolor{currentstroke}{rgb}{0.000000,0.000000,0.000000}%
\pgfsetstrokecolor{currentstroke}%
\pgfsetdash{}{0pt}%
\pgfsys@defobject{currentmarker}{\pgfqpoint{0.000000in}{-0.048611in}}{\pgfqpoint{0.000000in}{0.000000in}}{%
\pgfpathmoveto{\pgfqpoint{0.000000in}{0.000000in}}%
\pgfpathlineto{\pgfqpoint{0.000000in}{-0.048611in}}%
\pgfusepath{stroke,fill}%
}%
\begin{pgfscope}%
\pgfsys@transformshift{1.310195in}{1.826628in}%
\pgfsys@useobject{currentmarker}{}%
\end{pgfscope}%
\end{pgfscope}%
\begin{pgfscope}%
\pgfpathrectangle{\pgfqpoint{0.584475in}{1.826628in}}{\pgfqpoint{2.177159in}{0.813411in}}%
\pgfusepath{clip}%
\pgfsetrectcap%
\pgfsetroundjoin%
\pgfsetlinewidth{0.803000pt}%
\definecolor{currentstroke}{rgb}{0.690196,0.690196,0.690196}%
\pgfsetstrokecolor{currentstroke}%
\pgfsetdash{}{0pt}%
\pgfpathmoveto{\pgfqpoint{2.035915in}{1.826628in}}%
\pgfpathlineto{\pgfqpoint{2.035915in}{2.640039in}}%
\pgfusepath{stroke}%
\end{pgfscope}%
\begin{pgfscope}%
\pgfsetbuttcap%
\pgfsetroundjoin%
\definecolor{currentfill}{rgb}{0.000000,0.000000,0.000000}%
\pgfsetfillcolor{currentfill}%
\pgfsetlinewidth{0.803000pt}%
\definecolor{currentstroke}{rgb}{0.000000,0.000000,0.000000}%
\pgfsetstrokecolor{currentstroke}%
\pgfsetdash{}{0pt}%
\pgfsys@defobject{currentmarker}{\pgfqpoint{0.000000in}{-0.048611in}}{\pgfqpoint{0.000000in}{0.000000in}}{%
\pgfpathmoveto{\pgfqpoint{0.000000in}{0.000000in}}%
\pgfpathlineto{\pgfqpoint{0.000000in}{-0.048611in}}%
\pgfusepath{stroke,fill}%
}%
\begin{pgfscope}%
\pgfsys@transformshift{2.035915in}{1.826628in}%
\pgfsys@useobject{currentmarker}{}%
\end{pgfscope}%
\end{pgfscope}%
\begin{pgfscope}%
\pgfpathrectangle{\pgfqpoint{0.584475in}{1.826628in}}{\pgfqpoint{2.177159in}{0.813411in}}%
\pgfusepath{clip}%
\pgfsetrectcap%
\pgfsetroundjoin%
\pgfsetlinewidth{0.803000pt}%
\definecolor{currentstroke}{rgb}{0.690196,0.690196,0.690196}%
\pgfsetstrokecolor{currentstroke}%
\pgfsetdash{}{0pt}%
\pgfpathmoveto{\pgfqpoint{2.761635in}{1.826628in}}%
\pgfpathlineto{\pgfqpoint{2.761635in}{2.640039in}}%
\pgfusepath{stroke}%
\end{pgfscope}%
\begin{pgfscope}%
\pgfsetbuttcap%
\pgfsetroundjoin%
\definecolor{currentfill}{rgb}{0.000000,0.000000,0.000000}%
\pgfsetfillcolor{currentfill}%
\pgfsetlinewidth{0.803000pt}%
\definecolor{currentstroke}{rgb}{0.000000,0.000000,0.000000}%
\pgfsetstrokecolor{currentstroke}%
\pgfsetdash{}{0pt}%
\pgfsys@defobject{currentmarker}{\pgfqpoint{0.000000in}{-0.048611in}}{\pgfqpoint{0.000000in}{0.000000in}}{%
\pgfpathmoveto{\pgfqpoint{0.000000in}{0.000000in}}%
\pgfpathlineto{\pgfqpoint{0.000000in}{-0.048611in}}%
\pgfusepath{stroke,fill}%
}%
\begin{pgfscope}%
\pgfsys@transformshift{2.761635in}{1.826628in}%
\pgfsys@useobject{currentmarker}{}%
\end{pgfscope}%
\end{pgfscope}%
\begin{pgfscope}%
\pgfpathrectangle{\pgfqpoint{0.584475in}{1.826628in}}{\pgfqpoint{2.177159in}{0.813411in}}%
\pgfusepath{clip}%
\pgfsetrectcap%
\pgfsetroundjoin%
\pgfsetlinewidth{0.803000pt}%
\definecolor{currentstroke}{rgb}{0.690196,0.690196,0.690196}%
\pgfsetstrokecolor{currentstroke}%
\pgfsetdash{}{0pt}%
\pgfpathmoveto{\pgfqpoint{0.584475in}{1.826628in}}%
\pgfpathlineto{\pgfqpoint{2.761635in}{1.826628in}}%
\pgfusepath{stroke}%
\end{pgfscope}%
\begin{pgfscope}%
\pgfsetbuttcap%
\pgfsetroundjoin%
\definecolor{currentfill}{rgb}{0.000000,0.000000,0.000000}%
\pgfsetfillcolor{currentfill}%
\pgfsetlinewidth{0.803000pt}%
\definecolor{currentstroke}{rgb}{0.000000,0.000000,0.000000}%
\pgfsetstrokecolor{currentstroke}%
\pgfsetdash{}{0pt}%
\pgfsys@defobject{currentmarker}{\pgfqpoint{-0.048611in}{0.000000in}}{\pgfqpoint{0.000000in}{0.000000in}}{%
\pgfpathmoveto{\pgfqpoint{0.000000in}{0.000000in}}%
\pgfpathlineto{\pgfqpoint{-0.048611in}{0.000000in}}%
\pgfusepath{stroke,fill}%
}%
\begin{pgfscope}%
\pgfsys@transformshift{0.584475in}{1.826628in}%
\pgfsys@useobject{currentmarker}{}%
\end{pgfscope}%
\end{pgfscope}%
\begin{pgfscope}%
\definecolor{textcolor}{rgb}{0.000000,0.000000,0.000000}%
\pgfsetstrokecolor{textcolor}%
\pgfsetfillcolor{textcolor}%
\pgftext[x=0.150000in,y=1.773866in,left,base]{\color{textcolor}\sffamily\fontsize{10.000000}{12.000000}\selectfont −1.0}%
\end{pgfscope}%
\begin{pgfscope}%
\pgfpathrectangle{\pgfqpoint{0.584475in}{1.826628in}}{\pgfqpoint{2.177159in}{0.813411in}}%
\pgfusepath{clip}%
\pgfsetrectcap%
\pgfsetroundjoin%
\pgfsetlinewidth{0.803000pt}%
\definecolor{currentstroke}{rgb}{0.690196,0.690196,0.690196}%
\pgfsetstrokecolor{currentstroke}%
\pgfsetdash{}{0pt}%
\pgfpathmoveto{\pgfqpoint{0.584475in}{2.233333in}}%
\pgfpathlineto{\pgfqpoint{2.761635in}{2.233333in}}%
\pgfusepath{stroke}%
\end{pgfscope}%
\begin{pgfscope}%
\pgfsetbuttcap%
\pgfsetroundjoin%
\definecolor{currentfill}{rgb}{0.000000,0.000000,0.000000}%
\pgfsetfillcolor{currentfill}%
\pgfsetlinewidth{0.803000pt}%
\definecolor{currentstroke}{rgb}{0.000000,0.000000,0.000000}%
\pgfsetstrokecolor{currentstroke}%
\pgfsetdash{}{0pt}%
\pgfsys@defobject{currentmarker}{\pgfqpoint{-0.048611in}{0.000000in}}{\pgfqpoint{0.000000in}{0.000000in}}{%
\pgfpathmoveto{\pgfqpoint{0.000000in}{0.000000in}}%
\pgfpathlineto{\pgfqpoint{-0.048611in}{0.000000in}}%
\pgfusepath{stroke,fill}%
}%
\begin{pgfscope}%
\pgfsys@transformshift{0.584475in}{2.233333in}%
\pgfsys@useobject{currentmarker}{}%
\end{pgfscope}%
\end{pgfscope}%
\begin{pgfscope}%
\definecolor{textcolor}{rgb}{0.000000,0.000000,0.000000}%
\pgfsetstrokecolor{textcolor}%
\pgfsetfillcolor{textcolor}%
\pgftext[x=0.150000in,y=2.180572in,left,base]{\color{textcolor}\sffamily\fontsize{10.000000}{12.000000}\selectfont −0.5}%
\end{pgfscope}%
\begin{pgfscope}%
\pgfpathrectangle{\pgfqpoint{0.584475in}{1.826628in}}{\pgfqpoint{2.177159in}{0.813411in}}%
\pgfusepath{clip}%
\pgfsetrectcap%
\pgfsetroundjoin%
\pgfsetlinewidth{0.803000pt}%
\definecolor{currentstroke}{rgb}{0.690196,0.690196,0.690196}%
\pgfsetstrokecolor{currentstroke}%
\pgfsetdash{}{0pt}%
\pgfpathmoveto{\pgfqpoint{0.584475in}{2.640039in}}%
\pgfpathlineto{\pgfqpoint{2.761635in}{2.640039in}}%
\pgfusepath{stroke}%
\end{pgfscope}%
\begin{pgfscope}%
\pgfsetbuttcap%
\pgfsetroundjoin%
\definecolor{currentfill}{rgb}{0.000000,0.000000,0.000000}%
\pgfsetfillcolor{currentfill}%
\pgfsetlinewidth{0.803000pt}%
\definecolor{currentstroke}{rgb}{0.000000,0.000000,0.000000}%
\pgfsetstrokecolor{currentstroke}%
\pgfsetdash{}{0pt}%
\pgfsys@defobject{currentmarker}{\pgfqpoint{-0.048611in}{0.000000in}}{\pgfqpoint{0.000000in}{0.000000in}}{%
\pgfpathmoveto{\pgfqpoint{0.000000in}{0.000000in}}%
\pgfpathlineto{\pgfqpoint{-0.048611in}{0.000000in}}%
\pgfusepath{stroke,fill}%
}%
\begin{pgfscope}%
\pgfsys@transformshift{0.584475in}{2.640039in}%
\pgfsys@useobject{currentmarker}{}%
\end{pgfscope}%
\end{pgfscope}%
\begin{pgfscope}%
\definecolor{textcolor}{rgb}{0.000000,0.000000,0.000000}%
\pgfsetstrokecolor{textcolor}%
\pgfsetfillcolor{textcolor}%
\pgftext[x=0.266374in,y=2.587277in,left,base]{\color{textcolor}\sffamily\fontsize{10.000000}{12.000000}\selectfont 0.0}%
\end{pgfscope}%
\begin{pgfscope}%
\pgfpathrectangle{\pgfqpoint{0.584475in}{1.826628in}}{\pgfqpoint{2.177159in}{0.813411in}}%
\pgfusepath{clip}%
\pgfsetrectcap%
\pgfsetroundjoin%
\pgfsetlinewidth{1.505625pt}%
\definecolor{currentstroke}{rgb}{0.000000,0.000000,1.000000}%
\pgfsetstrokecolor{currentstroke}%
\pgfsetdash{}{0pt}%
\pgfpathmoveto{\pgfqpoint{0.584475in}{2.212543in}}%
\pgfpathlineto{\pgfqpoint{0.598990in}{2.283953in}}%
\pgfpathlineto{\pgfqpoint{0.613504in}{2.332518in}}%
\pgfpathlineto{\pgfqpoint{0.628019in}{2.387633in}}%
\pgfpathlineto{\pgfqpoint{0.642533in}{2.463488in}}%
\pgfpathlineto{\pgfqpoint{0.657047in}{2.533326in}}%
\pgfpathlineto{\pgfqpoint{0.671562in}{2.589481in}}%
\pgfpathlineto{\pgfqpoint{0.686076in}{2.625824in}}%
\pgfpathlineto{\pgfqpoint{0.700591in}{2.634909in}}%
\pgfpathlineto{\pgfqpoint{0.715105in}{2.633261in}}%
\pgfpathlineto{\pgfqpoint{0.744134in}{2.629248in}}%
\pgfpathlineto{\pgfqpoint{0.758648in}{2.627669in}}%
\pgfpathlineto{\pgfqpoint{0.773163in}{2.625151in}}%
\pgfpathlineto{\pgfqpoint{0.787677in}{2.620953in}}%
\pgfpathlineto{\pgfqpoint{0.802191in}{2.501893in}}%
\pgfpathlineto{\pgfqpoint{0.816706in}{2.418550in}}%
\pgfpathlineto{\pgfqpoint{0.831220in}{2.409551in}}%
\pgfpathlineto{\pgfqpoint{0.845735in}{2.371886in}}%
\pgfpathlineto{\pgfqpoint{0.860249in}{2.371949in}}%
\pgfpathlineto{\pgfqpoint{0.874763in}{2.261466in}}%
\pgfpathlineto{\pgfqpoint{1.165051in}{2.261466in}}%
\pgfpathlineto{\pgfqpoint{1.179566in}{2.422963in}}%
\pgfpathlineto{\pgfqpoint{1.194080in}{2.424819in}}%
\pgfpathlineto{\pgfqpoint{1.208594in}{2.439825in}}%
\pgfpathlineto{\pgfqpoint{1.223109in}{2.468682in}}%
\pgfpathlineto{\pgfqpoint{1.237623in}{2.508654in}}%
\pgfpathlineto{\pgfqpoint{1.252138in}{2.610240in}}%
\pgfpathlineto{\pgfqpoint{1.266652in}{2.607065in}}%
\pgfpathlineto{\pgfqpoint{1.281166in}{2.611352in}}%
\pgfpathlineto{\pgfqpoint{1.310195in}{2.615921in}}%
\pgfpathlineto{\pgfqpoint{1.324710in}{2.620333in}}%
\pgfpathlineto{\pgfqpoint{1.339224in}{2.625563in}}%
\pgfpathlineto{\pgfqpoint{1.353738in}{2.630277in}}%
\pgfpathlineto{\pgfqpoint{1.368253in}{2.618960in}}%
\pgfpathlineto{\pgfqpoint{1.382767in}{2.578152in}}%
\pgfpathlineto{\pgfqpoint{1.397282in}{2.517609in}}%
\pgfpathlineto{\pgfqpoint{1.426310in}{2.369376in}}%
\pgfpathlineto{\pgfqpoint{1.440825in}{2.318794in}}%
\pgfpathlineto{\pgfqpoint{1.455339in}{2.269788in}}%
\pgfpathlineto{\pgfqpoint{1.469854in}{2.192319in}}%
\pgfpathlineto{\pgfqpoint{1.484368in}{2.010898in}}%
\pgfpathlineto{\pgfqpoint{1.498882in}{2.228151in}}%
\pgfpathlineto{\pgfqpoint{1.513397in}{2.293185in}}%
\pgfpathlineto{\pgfqpoint{1.527911in}{2.340655in}}%
\pgfpathlineto{\pgfqpoint{1.542425in}{2.399898in}}%
\pgfpathlineto{\pgfqpoint{1.556940in}{2.476226in}}%
\pgfpathlineto{\pgfqpoint{1.571454in}{2.543886in}}%
\pgfpathlineto{\pgfqpoint{1.585969in}{2.596752in}}%
\pgfpathlineto{\pgfqpoint{1.600483in}{2.630359in}}%
\pgfpathlineto{\pgfqpoint{1.614997in}{2.634702in}}%
\pgfpathlineto{\pgfqpoint{1.673055in}{2.627390in}}%
\pgfpathlineto{\pgfqpoint{1.687569in}{2.624539in}}%
\pgfpathlineto{\pgfqpoint{1.702084in}{2.619489in}}%
\pgfpathlineto{\pgfqpoint{1.716598in}{2.464293in}}%
\pgfpathlineto{\pgfqpoint{1.731113in}{2.418545in}}%
\pgfpathlineto{\pgfqpoint{1.745627in}{2.409555in}}%
\pgfpathlineto{\pgfqpoint{1.760141in}{2.371896in}}%
\pgfpathlineto{\pgfqpoint{1.774656in}{2.371961in}}%
\pgfpathlineto{\pgfqpoint{1.789170in}{2.261466in}}%
\pgfpathlineto{\pgfqpoint{2.079458in}{2.261466in}}%
\pgfpathlineto{\pgfqpoint{2.093972in}{2.423297in}}%
\pgfpathlineto{\pgfqpoint{2.108487in}{2.425033in}}%
\pgfpathlineto{\pgfqpoint{2.123001in}{2.440383in}}%
\pgfpathlineto{\pgfqpoint{2.137516in}{2.468818in}}%
\pgfpathlineto{\pgfqpoint{2.152030in}{2.522726in}}%
\pgfpathlineto{\pgfqpoint{2.166544in}{2.610334in}}%
\pgfpathlineto{\pgfqpoint{2.181059in}{2.607921in}}%
\pgfpathlineto{\pgfqpoint{2.195573in}{2.611709in}}%
\pgfpathlineto{\pgfqpoint{2.210088in}{2.613871in}}%
\pgfpathlineto{\pgfqpoint{2.224602in}{2.616650in}}%
\pgfpathlineto{\pgfqpoint{2.239116in}{2.621184in}}%
\pgfpathlineto{\pgfqpoint{2.253631in}{2.626375in}}%
\pgfpathlineto{\pgfqpoint{2.268145in}{2.631035in}}%
\pgfpathlineto{\pgfqpoint{2.282660in}{2.613117in}}%
\pgfpathlineto{\pgfqpoint{2.297174in}{2.569519in}}%
\pgfpathlineto{\pgfqpoint{2.311688in}{2.506248in}}%
\pgfpathlineto{\pgfqpoint{2.326203in}{2.430328in}}%
\pgfpathlineto{\pgfqpoint{2.340717in}{2.359615in}}%
\pgfpathlineto{\pgfqpoint{2.355232in}{2.311281in}}%
\pgfpathlineto{\pgfqpoint{2.369746in}{2.259550in}}%
\pgfpathlineto{\pgfqpoint{2.384260in}{2.168097in}}%
\pgfpathlineto{\pgfqpoint{2.398775in}{2.096057in}}%
\pgfpathlineto{\pgfqpoint{2.413289in}{2.241845in}}%
\pgfpathlineto{\pgfqpoint{2.427804in}{2.301906in}}%
\pgfpathlineto{\pgfqpoint{2.442318in}{2.348667in}}%
\pgfpathlineto{\pgfqpoint{2.456832in}{2.412691in}}%
\pgfpathlineto{\pgfqpoint{2.471347in}{2.488762in}}%
\pgfpathlineto{\pgfqpoint{2.485861in}{2.553938in}}%
\pgfpathlineto{\pgfqpoint{2.500376in}{2.602868in}}%
\pgfpathlineto{\pgfqpoint{2.514890in}{2.634440in}}%
\pgfpathlineto{\pgfqpoint{2.529404in}{2.634379in}}%
\pgfpathlineto{\pgfqpoint{2.587462in}{2.626921in}}%
\pgfpathlineto{\pgfqpoint{2.601976in}{2.623889in}}%
\pgfpathlineto{\pgfqpoint{2.616491in}{2.617237in}}%
\pgfpathlineto{\pgfqpoint{2.631005in}{2.445282in}}%
\pgfpathlineto{\pgfqpoint{2.645519in}{2.418539in}}%
\pgfpathlineto{\pgfqpoint{2.660034in}{2.371853in}}%
\pgfpathlineto{\pgfqpoint{2.689063in}{2.371974in}}%
\pgfpathlineto{\pgfqpoint{2.703577in}{2.261466in}}%
\pgfpathlineto{\pgfqpoint{2.764968in}{2.261466in}}%
\pgfpathlineto{\pgfqpoint{2.764968in}{2.261466in}}%
\pgfusepath{stroke}%
\end{pgfscope}%
\begin{pgfscope}%
\pgfsetrectcap%
\pgfsetmiterjoin%
\pgfsetlinewidth{0.803000pt}%
\definecolor{currentstroke}{rgb}{0.000000,0.000000,0.000000}%
\pgfsetstrokecolor{currentstroke}%
\pgfsetdash{}{0pt}%
\pgfpathmoveto{\pgfqpoint{0.584475in}{1.826628in}}%
\pgfpathlineto{\pgfqpoint{0.584475in}{2.640039in}}%
\pgfusepath{stroke}%
\end{pgfscope}%
\begin{pgfscope}%
\pgfsetrectcap%
\pgfsetmiterjoin%
\pgfsetlinewidth{0.803000pt}%
\definecolor{currentstroke}{rgb}{0.000000,0.000000,0.000000}%
\pgfsetstrokecolor{currentstroke}%
\pgfsetdash{}{0pt}%
\pgfpathmoveto{\pgfqpoint{2.761635in}{1.826628in}}%
\pgfpathlineto{\pgfqpoint{2.761635in}{2.640039in}}%
\pgfusepath{stroke}%
\end{pgfscope}%
\begin{pgfscope}%
\pgfsetrectcap%
\pgfsetmiterjoin%
\pgfsetlinewidth{0.803000pt}%
\definecolor{currentstroke}{rgb}{0.000000,0.000000,0.000000}%
\pgfsetstrokecolor{currentstroke}%
\pgfsetdash{}{0pt}%
\pgfpathmoveto{\pgfqpoint{0.584475in}{1.826628in}}%
\pgfpathlineto{\pgfqpoint{2.761635in}{1.826628in}}%
\pgfusepath{stroke}%
\end{pgfscope}%
\begin{pgfscope}%
\pgfsetrectcap%
\pgfsetmiterjoin%
\pgfsetlinewidth{0.803000pt}%
\definecolor{currentstroke}{rgb}{0.000000,0.000000,0.000000}%
\pgfsetstrokecolor{currentstroke}%
\pgfsetdash{}{0pt}%
\pgfpathmoveto{\pgfqpoint{0.584475in}{2.640039in}}%
\pgfpathlineto{\pgfqpoint{2.761635in}{2.640039in}}%
\pgfusepath{stroke}%
\end{pgfscope}%
\begin{pgfscope}%
\definecolor{textcolor}{rgb}{0.000000,0.000000,1.000000}%
\pgfsetstrokecolor{textcolor}%
\pgfsetfillcolor{textcolor}%
\pgftext[x=1.673055in,y=2.723372in,,base]{\color{textcolor}\sffamily\fontsize{12.000000}{14.400000}\selectfont MI metric value}%
\end{pgfscope}%
\begin{pgfscope}%
\pgfsetbuttcap%
\pgfsetmiterjoin%
\definecolor{currentfill}{rgb}{1.000000,1.000000,1.000000}%
\pgfsetfillcolor{currentfill}%
\pgfsetlinewidth{0.000000pt}%
\definecolor{currentstroke}{rgb}{0.000000,0.000000,0.000000}%
\pgfsetstrokecolor{currentstroke}%
\pgfsetstrokeopacity{0.000000}%
\pgfsetdash{}{0pt}%
\pgfpathmoveto{\pgfqpoint{0.584475in}{0.571604in}}%
\pgfpathlineto{\pgfqpoint{2.761635in}{0.571604in}}%
\pgfpathlineto{\pgfqpoint{2.761635in}{1.385015in}}%
\pgfpathlineto{\pgfqpoint{0.584475in}{1.385015in}}%
\pgfpathclose%
\pgfusepath{fill}%
\end{pgfscope}%
\begin{pgfscope}%
\pgfpathrectangle{\pgfqpoint{0.584475in}{0.571604in}}{\pgfqpoint{2.177159in}{0.813411in}}%
\pgfusepath{clip}%
\pgfsetrectcap%
\pgfsetroundjoin%
\pgfsetlinewidth{0.803000pt}%
\definecolor{currentstroke}{rgb}{0.690196,0.690196,0.690196}%
\pgfsetstrokecolor{currentstroke}%
\pgfsetdash{}{0pt}%
\pgfpathmoveto{\pgfqpoint{0.584475in}{0.571604in}}%
\pgfpathlineto{\pgfqpoint{0.584475in}{1.385015in}}%
\pgfusepath{stroke}%
\end{pgfscope}%
\begin{pgfscope}%
\pgfsetbuttcap%
\pgfsetroundjoin%
\definecolor{currentfill}{rgb}{0.000000,0.000000,0.000000}%
\pgfsetfillcolor{currentfill}%
\pgfsetlinewidth{0.803000pt}%
\definecolor{currentstroke}{rgb}{0.000000,0.000000,0.000000}%
\pgfsetstrokecolor{currentstroke}%
\pgfsetdash{}{0pt}%
\pgfsys@defobject{currentmarker}{\pgfqpoint{0.000000in}{-0.048611in}}{\pgfqpoint{0.000000in}{0.000000in}}{%
\pgfpathmoveto{\pgfqpoint{0.000000in}{0.000000in}}%
\pgfpathlineto{\pgfqpoint{0.000000in}{-0.048611in}}%
\pgfusepath{stroke,fill}%
}%
\begin{pgfscope}%
\pgfsys@transformshift{0.584475in}{0.571604in}%
\pgfsys@useobject{currentmarker}{}%
\end{pgfscope}%
\end{pgfscope}%
\begin{pgfscope}%
\definecolor{textcolor}{rgb}{0.000000,0.000000,0.000000}%
\pgfsetstrokecolor{textcolor}%
\pgfsetfillcolor{textcolor}%
\pgftext[x=0.584475in,y=0.474382in,,top]{\color{textcolor}\sffamily\fontsize{10.000000}{12.000000}\selectfont 0}%
\end{pgfscope}%
\begin{pgfscope}%
\pgfpathrectangle{\pgfqpoint{0.584475in}{0.571604in}}{\pgfqpoint{2.177159in}{0.813411in}}%
\pgfusepath{clip}%
\pgfsetrectcap%
\pgfsetroundjoin%
\pgfsetlinewidth{0.803000pt}%
\definecolor{currentstroke}{rgb}{0.690196,0.690196,0.690196}%
\pgfsetstrokecolor{currentstroke}%
\pgfsetdash{}{0pt}%
\pgfpathmoveto{\pgfqpoint{1.310195in}{0.571604in}}%
\pgfpathlineto{\pgfqpoint{1.310195in}{1.385015in}}%
\pgfusepath{stroke}%
\end{pgfscope}%
\begin{pgfscope}%
\pgfsetbuttcap%
\pgfsetroundjoin%
\definecolor{currentfill}{rgb}{0.000000,0.000000,0.000000}%
\pgfsetfillcolor{currentfill}%
\pgfsetlinewidth{0.803000pt}%
\definecolor{currentstroke}{rgb}{0.000000,0.000000,0.000000}%
\pgfsetstrokecolor{currentstroke}%
\pgfsetdash{}{0pt}%
\pgfsys@defobject{currentmarker}{\pgfqpoint{0.000000in}{-0.048611in}}{\pgfqpoint{0.000000in}{0.000000in}}{%
\pgfpathmoveto{\pgfqpoint{0.000000in}{0.000000in}}%
\pgfpathlineto{\pgfqpoint{0.000000in}{-0.048611in}}%
\pgfusepath{stroke,fill}%
}%
\begin{pgfscope}%
\pgfsys@transformshift{1.310195in}{0.571604in}%
\pgfsys@useobject{currentmarker}{}%
\end{pgfscope}%
\end{pgfscope}%
\begin{pgfscope}%
\definecolor{textcolor}{rgb}{0.000000,0.000000,0.000000}%
\pgfsetstrokecolor{textcolor}%
\pgfsetfillcolor{textcolor}%
\pgftext[x=1.310195in,y=0.474382in,,top]{\color{textcolor}\sffamily\fontsize{10.000000}{12.000000}\selectfont 5}%
\end{pgfscope}%
\begin{pgfscope}%
\pgfpathrectangle{\pgfqpoint{0.584475in}{0.571604in}}{\pgfqpoint{2.177159in}{0.813411in}}%
\pgfusepath{clip}%
\pgfsetrectcap%
\pgfsetroundjoin%
\pgfsetlinewidth{0.803000pt}%
\definecolor{currentstroke}{rgb}{0.690196,0.690196,0.690196}%
\pgfsetstrokecolor{currentstroke}%
\pgfsetdash{}{0pt}%
\pgfpathmoveto{\pgfqpoint{2.035915in}{0.571604in}}%
\pgfpathlineto{\pgfqpoint{2.035915in}{1.385015in}}%
\pgfusepath{stroke}%
\end{pgfscope}%
\begin{pgfscope}%
\pgfsetbuttcap%
\pgfsetroundjoin%
\definecolor{currentfill}{rgb}{0.000000,0.000000,0.000000}%
\pgfsetfillcolor{currentfill}%
\pgfsetlinewidth{0.803000pt}%
\definecolor{currentstroke}{rgb}{0.000000,0.000000,0.000000}%
\pgfsetstrokecolor{currentstroke}%
\pgfsetdash{}{0pt}%
\pgfsys@defobject{currentmarker}{\pgfqpoint{0.000000in}{-0.048611in}}{\pgfqpoint{0.000000in}{0.000000in}}{%
\pgfpathmoveto{\pgfqpoint{0.000000in}{0.000000in}}%
\pgfpathlineto{\pgfqpoint{0.000000in}{-0.048611in}}%
\pgfusepath{stroke,fill}%
}%
\begin{pgfscope}%
\pgfsys@transformshift{2.035915in}{0.571604in}%
\pgfsys@useobject{currentmarker}{}%
\end{pgfscope}%
\end{pgfscope}%
\begin{pgfscope}%
\definecolor{textcolor}{rgb}{0.000000,0.000000,0.000000}%
\pgfsetstrokecolor{textcolor}%
\pgfsetfillcolor{textcolor}%
\pgftext[x=2.035915in,y=0.474382in,,top]{\color{textcolor}\sffamily\fontsize{10.000000}{12.000000}\selectfont 10}%
\end{pgfscope}%
\begin{pgfscope}%
\pgfpathrectangle{\pgfqpoint{0.584475in}{0.571604in}}{\pgfqpoint{2.177159in}{0.813411in}}%
\pgfusepath{clip}%
\pgfsetrectcap%
\pgfsetroundjoin%
\pgfsetlinewidth{0.803000pt}%
\definecolor{currentstroke}{rgb}{0.690196,0.690196,0.690196}%
\pgfsetstrokecolor{currentstroke}%
\pgfsetdash{}{0pt}%
\pgfpathmoveto{\pgfqpoint{2.761635in}{0.571604in}}%
\pgfpathlineto{\pgfqpoint{2.761635in}{1.385015in}}%
\pgfusepath{stroke}%
\end{pgfscope}%
\begin{pgfscope}%
\pgfsetbuttcap%
\pgfsetroundjoin%
\definecolor{currentfill}{rgb}{0.000000,0.000000,0.000000}%
\pgfsetfillcolor{currentfill}%
\pgfsetlinewidth{0.803000pt}%
\definecolor{currentstroke}{rgb}{0.000000,0.000000,0.000000}%
\pgfsetstrokecolor{currentstroke}%
\pgfsetdash{}{0pt}%
\pgfsys@defobject{currentmarker}{\pgfqpoint{0.000000in}{-0.048611in}}{\pgfqpoint{0.000000in}{0.000000in}}{%
\pgfpathmoveto{\pgfqpoint{0.000000in}{0.000000in}}%
\pgfpathlineto{\pgfqpoint{0.000000in}{-0.048611in}}%
\pgfusepath{stroke,fill}%
}%
\begin{pgfscope}%
\pgfsys@transformshift{2.761635in}{0.571604in}%
\pgfsys@useobject{currentmarker}{}%
\end{pgfscope}%
\end{pgfscope}%
\begin{pgfscope}%
\definecolor{textcolor}{rgb}{0.000000,0.000000,0.000000}%
\pgfsetstrokecolor{textcolor}%
\pgfsetfillcolor{textcolor}%
\pgftext[x=2.761635in,y=0.474382in,,top]{\color{textcolor}\sffamily\fontsize{10.000000}{12.000000}\selectfont 15}%
\end{pgfscope}%
\begin{pgfscope}%
\definecolor{textcolor}{rgb}{0.000000,0.000000,0.000000}%
\pgfsetstrokecolor{textcolor}%
\pgfsetfillcolor{textcolor}%
\pgftext[x=1.673055in,y=0.284413in,,top]{\color{textcolor}\sffamily\fontsize{10.000000}{12.000000}\selectfont Shift of paramter 0}%
\end{pgfscope}%
\begin{pgfscope}%
\pgfpathrectangle{\pgfqpoint{0.584475in}{0.571604in}}{\pgfqpoint{2.177159in}{0.813411in}}%
\pgfusepath{clip}%
\pgfsetrectcap%
\pgfsetroundjoin%
\pgfsetlinewidth{0.803000pt}%
\definecolor{currentstroke}{rgb}{0.690196,0.690196,0.690196}%
\pgfsetstrokecolor{currentstroke}%
\pgfsetdash{}{0pt}%
\pgfpathmoveto{\pgfqpoint{0.584475in}{0.571604in}}%
\pgfpathlineto{\pgfqpoint{2.761635in}{0.571604in}}%
\pgfusepath{stroke}%
\end{pgfscope}%
\begin{pgfscope}%
\pgfsetbuttcap%
\pgfsetroundjoin%
\definecolor{currentfill}{rgb}{0.000000,0.000000,0.000000}%
\pgfsetfillcolor{currentfill}%
\pgfsetlinewidth{0.803000pt}%
\definecolor{currentstroke}{rgb}{0.000000,0.000000,0.000000}%
\pgfsetstrokecolor{currentstroke}%
\pgfsetdash{}{0pt}%
\pgfsys@defobject{currentmarker}{\pgfqpoint{-0.048611in}{0.000000in}}{\pgfqpoint{0.000000in}{0.000000in}}{%
\pgfpathmoveto{\pgfqpoint{0.000000in}{0.000000in}}%
\pgfpathlineto{\pgfqpoint{-0.048611in}{0.000000in}}%
\pgfusepath{stroke,fill}%
}%
\begin{pgfscope}%
\pgfsys@transformshift{0.584475in}{0.571604in}%
\pgfsys@useobject{currentmarker}{}%
\end{pgfscope}%
\end{pgfscope}%
\begin{pgfscope}%
\definecolor{textcolor}{rgb}{0.000000,0.000000,0.000000}%
\pgfsetstrokecolor{textcolor}%
\pgfsetfillcolor{textcolor}%
\pgftext[x=0.398888in,y=0.518842in,left,base]{\color{textcolor}\sffamily\fontsize{10.000000}{12.000000}\selectfont 0}%
\end{pgfscope}%
\begin{pgfscope}%
\pgfpathrectangle{\pgfqpoint{0.584475in}{0.571604in}}{\pgfqpoint{2.177159in}{0.813411in}}%
\pgfusepath{clip}%
\pgfsetrectcap%
\pgfsetroundjoin%
\pgfsetlinewidth{0.803000pt}%
\definecolor{currentstroke}{rgb}{0.690196,0.690196,0.690196}%
\pgfsetstrokecolor{currentstroke}%
\pgfsetdash{}{0pt}%
\pgfpathmoveto{\pgfqpoint{0.584475in}{1.113878in}}%
\pgfpathlineto{\pgfqpoint{2.761635in}{1.113878in}}%
\pgfusepath{stroke}%
\end{pgfscope}%
\begin{pgfscope}%
\pgfsetbuttcap%
\pgfsetroundjoin%
\definecolor{currentfill}{rgb}{0.000000,0.000000,0.000000}%
\pgfsetfillcolor{currentfill}%
\pgfsetlinewidth{0.803000pt}%
\definecolor{currentstroke}{rgb}{0.000000,0.000000,0.000000}%
\pgfsetstrokecolor{currentstroke}%
\pgfsetdash{}{0pt}%
\pgfsys@defobject{currentmarker}{\pgfqpoint{-0.048611in}{0.000000in}}{\pgfqpoint{0.000000in}{0.000000in}}{%
\pgfpathmoveto{\pgfqpoint{0.000000in}{0.000000in}}%
\pgfpathlineto{\pgfqpoint{-0.048611in}{0.000000in}}%
\pgfusepath{stroke,fill}%
}%
\begin{pgfscope}%
\pgfsys@transformshift{0.584475in}{1.113878in}%
\pgfsys@useobject{currentmarker}{}%
\end{pgfscope}%
\end{pgfscope}%
\begin{pgfscope}%
\definecolor{textcolor}{rgb}{0.000000,0.000000,0.000000}%
\pgfsetstrokecolor{textcolor}%
\pgfsetfillcolor{textcolor}%
\pgftext[x=0.398888in,y=1.061117in,left,base]{\color{textcolor}\sffamily\fontsize{10.000000}{12.000000}\selectfont 2}%
\end{pgfscope}%
\begin{pgfscope}%
\definecolor{textcolor}{rgb}{0.000000,0.000000,0.000000}%
\pgfsetstrokecolor{textcolor}%
\pgfsetfillcolor{textcolor}%
\pgftext[x=0.584475in,y=1.426682in,left,base]{\color{textcolor}\sffamily\fontsize{10.000000}{12.000000}\selectfont 1e5}%
\end{pgfscope}%
\begin{pgfscope}%
\pgfpathrectangle{\pgfqpoint{0.584475in}{0.571604in}}{\pgfqpoint{2.177159in}{0.813411in}}%
\pgfusepath{clip}%
\pgfsetrectcap%
\pgfsetroundjoin%
\pgfsetlinewidth{1.505625pt}%
\definecolor{currentstroke}{rgb}{1.000000,0.000000,0.000000}%
\pgfsetstrokecolor{currentstroke}%
\pgfsetdash{}{0pt}%
\pgfpathmoveto{\pgfqpoint{0.584475in}{0.876980in}}%
\pgfpathlineto{\pgfqpoint{0.598990in}{1.003289in}}%
\pgfpathlineto{\pgfqpoint{0.613504in}{1.054252in}}%
\pgfpathlineto{\pgfqpoint{0.642533in}{1.236961in}}%
\pgfpathlineto{\pgfqpoint{0.657047in}{1.289765in}}%
\pgfpathlineto{\pgfqpoint{0.671562in}{1.313674in}}%
\pgfpathlineto{\pgfqpoint{0.686076in}{1.299843in}}%
\pgfpathlineto{\pgfqpoint{0.715105in}{1.236573in}}%
\pgfpathlineto{\pgfqpoint{0.729619in}{1.200420in}}%
\pgfpathlineto{\pgfqpoint{0.744134in}{1.177951in}}%
\pgfpathlineto{\pgfqpoint{0.758648in}{1.163149in}}%
\pgfpathlineto{\pgfqpoint{0.773163in}{1.173908in}}%
\pgfpathlineto{\pgfqpoint{0.787677in}{1.179970in}}%
\pgfpathlineto{\pgfqpoint{0.802191in}{0.571835in}}%
\pgfpathlineto{\pgfqpoint{0.889278in}{0.571604in}}%
\pgfpathlineto{\pgfqpoint{1.237623in}{0.571825in}}%
\pgfpathlineto{\pgfqpoint{1.252138in}{0.905859in}}%
\pgfpathlineto{\pgfqpoint{1.266652in}{1.183693in}}%
\pgfpathlineto{\pgfqpoint{1.281166in}{1.159036in}}%
\pgfpathlineto{\pgfqpoint{1.295681in}{1.149375in}}%
\pgfpathlineto{\pgfqpoint{1.310195in}{1.166159in}}%
\pgfpathlineto{\pgfqpoint{1.324710in}{1.190604in}}%
\pgfpathlineto{\pgfqpoint{1.339224in}{1.226167in}}%
\pgfpathlineto{\pgfqpoint{1.353738in}{1.253389in}}%
\pgfpathlineto{\pgfqpoint{1.368253in}{1.284662in}}%
\pgfpathlineto{\pgfqpoint{1.382767in}{1.305827in}}%
\pgfpathlineto{\pgfqpoint{1.397282in}{1.275148in}}%
\pgfpathlineto{\pgfqpoint{1.411796in}{1.224329in}}%
\pgfpathlineto{\pgfqpoint{1.426310in}{1.127088in}}%
\pgfpathlineto{\pgfqpoint{1.440825in}{1.048057in}}%
\pgfpathlineto{\pgfqpoint{1.455339in}{0.993238in}}%
\pgfpathlineto{\pgfqpoint{1.469854in}{0.835331in}}%
\pgfpathlineto{\pgfqpoint{1.484368in}{0.616184in}}%
\pgfpathlineto{\pgfqpoint{1.498882in}{0.911545in}}%
\pgfpathlineto{\pgfqpoint{1.513397in}{1.013357in}}%
\pgfpathlineto{\pgfqpoint{1.527911in}{1.064499in}}%
\pgfpathlineto{\pgfqpoint{1.542425in}{1.166297in}}%
\pgfpathlineto{\pgfqpoint{1.556940in}{1.246356in}}%
\pgfpathlineto{\pgfqpoint{1.571454in}{1.299415in}}%
\pgfpathlineto{\pgfqpoint{1.585969in}{1.313291in}}%
\pgfpathlineto{\pgfqpoint{1.600483in}{1.296332in}}%
\pgfpathlineto{\pgfqpoint{1.614997in}{1.267079in}}%
\pgfpathlineto{\pgfqpoint{1.629512in}{1.228325in}}%
\pgfpathlineto{\pgfqpoint{1.644026in}{1.200281in}}%
\pgfpathlineto{\pgfqpoint{1.658541in}{1.173691in}}%
\pgfpathlineto{\pgfqpoint{1.673055in}{1.162523in}}%
\pgfpathlineto{\pgfqpoint{1.687569in}{1.186985in}}%
\pgfpathlineto{\pgfqpoint{1.702084in}{1.184078in}}%
\pgfpathlineto{\pgfqpoint{1.716598in}{0.571793in}}%
\pgfpathlineto{\pgfqpoint{1.861742in}{0.571604in}}%
\pgfpathlineto{\pgfqpoint{2.152030in}{0.571836in}}%
\pgfpathlineto{\pgfqpoint{2.166544in}{1.004374in}}%
\pgfpathlineto{\pgfqpoint{2.181059in}{1.186128in}}%
\pgfpathlineto{\pgfqpoint{2.195573in}{1.155639in}}%
\pgfpathlineto{\pgfqpoint{2.210088in}{1.153000in}}%
\pgfpathlineto{\pgfqpoint{2.224602in}{1.168580in}}%
\pgfpathlineto{\pgfqpoint{2.239116in}{1.193939in}}%
\pgfpathlineto{\pgfqpoint{2.253631in}{1.231934in}}%
\pgfpathlineto{\pgfqpoint{2.268145in}{1.258863in}}%
\pgfpathlineto{\pgfqpoint{2.282660in}{1.288111in}}%
\pgfpathlineto{\pgfqpoint{2.297174in}{1.305355in}}%
\pgfpathlineto{\pgfqpoint{2.311688in}{1.266219in}}%
\pgfpathlineto{\pgfqpoint{2.326203in}{1.213055in}}%
\pgfpathlineto{\pgfqpoint{2.340717in}{1.108485in}}%
\pgfpathlineto{\pgfqpoint{2.355232in}{1.040891in}}%
\pgfpathlineto{\pgfqpoint{2.369746in}{0.977089in}}%
\pgfpathlineto{\pgfqpoint{2.384260in}{0.785844in}}%
\pgfpathlineto{\pgfqpoint{2.398775in}{0.683405in}}%
\pgfpathlineto{\pgfqpoint{2.413289in}{0.938520in}}%
\pgfpathlineto{\pgfqpoint{2.427804in}{1.022207in}}%
\pgfpathlineto{\pgfqpoint{2.442318in}{1.076686in}}%
\pgfpathlineto{\pgfqpoint{2.456832in}{1.183975in}}%
\pgfpathlineto{\pgfqpoint{2.471347in}{1.254723in}}%
\pgfpathlineto{\pgfqpoint{2.485861in}{1.305979in}}%
\pgfpathlineto{\pgfqpoint{2.500376in}{1.307790in}}%
\pgfpathlineto{\pgfqpoint{2.514890in}{1.290841in}}%
\pgfpathlineto{\pgfqpoint{2.529404in}{1.258164in}}%
\pgfpathlineto{\pgfqpoint{2.543919in}{1.222382in}}%
\pgfpathlineto{\pgfqpoint{2.558433in}{1.190859in}}%
\pgfpathlineto{\pgfqpoint{2.572948in}{1.172411in}}%
\pgfpathlineto{\pgfqpoint{2.587462in}{1.163079in}}%
\pgfpathlineto{\pgfqpoint{2.601976in}{1.191616in}}%
\pgfpathlineto{\pgfqpoint{2.616491in}{1.150736in}}%
\pgfpathlineto{\pgfqpoint{2.631005in}{0.571770in}}%
\pgfpathlineto{\pgfqpoint{2.764968in}{0.571604in}}%
\pgfpathlineto{\pgfqpoint{2.764968in}{0.571604in}}%
\pgfusepath{stroke}%
\end{pgfscope}%
\begin{pgfscope}%
\pgfsetrectcap%
\pgfsetmiterjoin%
\pgfsetlinewidth{0.803000pt}%
\definecolor{currentstroke}{rgb}{0.000000,0.000000,0.000000}%
\pgfsetstrokecolor{currentstroke}%
\pgfsetdash{}{0pt}%
\pgfpathmoveto{\pgfqpoint{0.584475in}{0.571604in}}%
\pgfpathlineto{\pgfqpoint{0.584475in}{1.385015in}}%
\pgfusepath{stroke}%
\end{pgfscope}%
\begin{pgfscope}%
\pgfsetrectcap%
\pgfsetmiterjoin%
\pgfsetlinewidth{0.803000pt}%
\definecolor{currentstroke}{rgb}{0.000000,0.000000,0.000000}%
\pgfsetstrokecolor{currentstroke}%
\pgfsetdash{}{0pt}%
\pgfpathmoveto{\pgfqpoint{2.761635in}{0.571604in}}%
\pgfpathlineto{\pgfqpoint{2.761635in}{1.385015in}}%
\pgfusepath{stroke}%
\end{pgfscope}%
\begin{pgfscope}%
\pgfsetrectcap%
\pgfsetmiterjoin%
\pgfsetlinewidth{0.803000pt}%
\definecolor{currentstroke}{rgb}{0.000000,0.000000,0.000000}%
\pgfsetstrokecolor{currentstroke}%
\pgfsetdash{}{0pt}%
\pgfpathmoveto{\pgfqpoint{0.584475in}{0.571604in}}%
\pgfpathlineto{\pgfqpoint{2.761635in}{0.571604in}}%
\pgfusepath{stroke}%
\end{pgfscope}%
\begin{pgfscope}%
\pgfsetrectcap%
\pgfsetmiterjoin%
\pgfsetlinewidth{0.803000pt}%
\definecolor{currentstroke}{rgb}{0.000000,0.000000,0.000000}%
\pgfsetstrokecolor{currentstroke}%
\pgfsetdash{}{0pt}%
\pgfpathmoveto{\pgfqpoint{0.584475in}{1.385015in}}%
\pgfpathlineto{\pgfqpoint{2.761635in}{1.385015in}}%
\pgfusepath{stroke}%
\end{pgfscope}%
\begin{pgfscope}%
\definecolor{textcolor}{rgb}{1.000000,0.000000,0.000000}%
\pgfsetstrokecolor{textcolor}%
\pgfsetfillcolor{textcolor}%
\pgftext[x=1.673055in,y=1.468349in,,base]{\color{textcolor}\sffamily\fontsize{12.000000}{14.400000}\selectfont MSD metric value}%
\end{pgfscope}%
\end{pgfpicture}%
\makeatother%
\endgroup%
}
  \hfill
  \resizebox{0.32\linewidth}{!}{%% Creator: Matplotlib, PGF backend
%%
%% To include the figure in your LaTeX document, write
%%   \input{<filename>.pgf}
%%
%% Make sure the required packages are loaded in your preamble
%%   \usepackage{pgf}
%%
%% and, on pdftex
%%   \usepackage[utf8]{inputenc}\DeclareUnicodeCharacter{2212}{-}
%%
%% or, on luatex and xetex
%%   \usepackage{unicode-math}
%%
%% Figures using additional raster images can only be included by \input if
%% they are in the same directory as the main LaTeX file. For loading figures
%% from other directories you can use the `import` package
%%   \usepackage{import}
%%
%% and then include the figures with
%%   \import{<path to file>}{<filename>.pgf}
%%
%% Matplotlib used the following preamble
%%   \usepackage{fontspec}
%%   \setmainfont{DejaVuSerif.ttf}[Path=/usr/share/matplotlib/mpl-data/fonts/ttf/]
%%   \setsansfont{DejaVuSans.ttf}[Path=/usr/share/matplotlib/mpl-data/fonts/ttf/]
%%   \setmonofont{DejaVuSansMono.ttf}[Path=/usr/share/matplotlib/mpl-data/fonts/ttf/]
%%
\begingroup%
\makeatletter%
\begin{pgfpicture}%
\pgfpathrectangle{\pgfpointorigin}{\pgfqpoint{3.000000in}{3.000000in}}%
\pgfusepath{use as bounding box, clip}%
\begin{pgfscope}%
\pgfsetbuttcap%
\pgfsetmiterjoin%
\definecolor{currentfill}{rgb}{1.000000,1.000000,1.000000}%
\pgfsetfillcolor{currentfill}%
\pgfsetlinewidth{0.000000pt}%
\definecolor{currentstroke}{rgb}{1.000000,1.000000,1.000000}%
\pgfsetstrokecolor{currentstroke}%
\pgfsetdash{}{0pt}%
\pgfpathmoveto{\pgfqpoint{0.000000in}{0.000000in}}%
\pgfpathlineto{\pgfqpoint{3.000000in}{0.000000in}}%
\pgfpathlineto{\pgfqpoint{3.000000in}{3.000000in}}%
\pgfpathlineto{\pgfqpoint{0.000000in}{3.000000in}}%
\pgfpathclose%
\pgfusepath{fill}%
\end{pgfscope}%
\begin{pgfscope}%
\pgfsetbuttcap%
\pgfsetmiterjoin%
\definecolor{currentfill}{rgb}{1.000000,1.000000,1.000000}%
\pgfsetfillcolor{currentfill}%
\pgfsetlinewidth{0.000000pt}%
\definecolor{currentstroke}{rgb}{0.000000,0.000000,0.000000}%
\pgfsetstrokecolor{currentstroke}%
\pgfsetstrokeopacity{0.000000}%
\pgfsetdash{}{0pt}%
\pgfpathmoveto{\pgfqpoint{0.584475in}{1.826628in}}%
\pgfpathlineto{\pgfqpoint{2.761635in}{1.826628in}}%
\pgfpathlineto{\pgfqpoint{2.761635in}{2.640039in}}%
\pgfpathlineto{\pgfqpoint{0.584475in}{2.640039in}}%
\pgfpathclose%
\pgfusepath{fill}%
\end{pgfscope}%
\begin{pgfscope}%
\pgfpathrectangle{\pgfqpoint{0.584475in}{1.826628in}}{\pgfqpoint{2.177159in}{0.813411in}}%
\pgfusepath{clip}%
\pgfsetrectcap%
\pgfsetroundjoin%
\pgfsetlinewidth{0.803000pt}%
\definecolor{currentstroke}{rgb}{0.690196,0.690196,0.690196}%
\pgfsetstrokecolor{currentstroke}%
\pgfsetdash{}{0pt}%
\pgfpathmoveto{\pgfqpoint{0.584475in}{1.826628in}}%
\pgfpathlineto{\pgfqpoint{0.584475in}{2.640039in}}%
\pgfusepath{stroke}%
\end{pgfscope}%
\begin{pgfscope}%
\pgfsetbuttcap%
\pgfsetroundjoin%
\definecolor{currentfill}{rgb}{0.000000,0.000000,0.000000}%
\pgfsetfillcolor{currentfill}%
\pgfsetlinewidth{0.803000pt}%
\definecolor{currentstroke}{rgb}{0.000000,0.000000,0.000000}%
\pgfsetstrokecolor{currentstroke}%
\pgfsetdash{}{0pt}%
\pgfsys@defobject{currentmarker}{\pgfqpoint{0.000000in}{-0.048611in}}{\pgfqpoint{0.000000in}{0.000000in}}{%
\pgfpathmoveto{\pgfqpoint{0.000000in}{0.000000in}}%
\pgfpathlineto{\pgfqpoint{0.000000in}{-0.048611in}}%
\pgfusepath{stroke,fill}%
}%
\begin{pgfscope}%
\pgfsys@transformshift{0.584475in}{1.826628in}%
\pgfsys@useobject{currentmarker}{}%
\end{pgfscope}%
\end{pgfscope}%
\begin{pgfscope}%
\pgfpathrectangle{\pgfqpoint{0.584475in}{1.826628in}}{\pgfqpoint{2.177159in}{0.813411in}}%
\pgfusepath{clip}%
\pgfsetrectcap%
\pgfsetroundjoin%
\pgfsetlinewidth{0.803000pt}%
\definecolor{currentstroke}{rgb}{0.690196,0.690196,0.690196}%
\pgfsetstrokecolor{currentstroke}%
\pgfsetdash{}{0pt}%
\pgfpathmoveto{\pgfqpoint{1.310195in}{1.826628in}}%
\pgfpathlineto{\pgfqpoint{1.310195in}{2.640039in}}%
\pgfusepath{stroke}%
\end{pgfscope}%
\begin{pgfscope}%
\pgfsetbuttcap%
\pgfsetroundjoin%
\definecolor{currentfill}{rgb}{0.000000,0.000000,0.000000}%
\pgfsetfillcolor{currentfill}%
\pgfsetlinewidth{0.803000pt}%
\definecolor{currentstroke}{rgb}{0.000000,0.000000,0.000000}%
\pgfsetstrokecolor{currentstroke}%
\pgfsetdash{}{0pt}%
\pgfsys@defobject{currentmarker}{\pgfqpoint{0.000000in}{-0.048611in}}{\pgfqpoint{0.000000in}{0.000000in}}{%
\pgfpathmoveto{\pgfqpoint{0.000000in}{0.000000in}}%
\pgfpathlineto{\pgfqpoint{0.000000in}{-0.048611in}}%
\pgfusepath{stroke,fill}%
}%
\begin{pgfscope}%
\pgfsys@transformshift{1.310195in}{1.826628in}%
\pgfsys@useobject{currentmarker}{}%
\end{pgfscope}%
\end{pgfscope}%
\begin{pgfscope}%
\pgfpathrectangle{\pgfqpoint{0.584475in}{1.826628in}}{\pgfqpoint{2.177159in}{0.813411in}}%
\pgfusepath{clip}%
\pgfsetrectcap%
\pgfsetroundjoin%
\pgfsetlinewidth{0.803000pt}%
\definecolor{currentstroke}{rgb}{0.690196,0.690196,0.690196}%
\pgfsetstrokecolor{currentstroke}%
\pgfsetdash{}{0pt}%
\pgfpathmoveto{\pgfqpoint{2.035915in}{1.826628in}}%
\pgfpathlineto{\pgfqpoint{2.035915in}{2.640039in}}%
\pgfusepath{stroke}%
\end{pgfscope}%
\begin{pgfscope}%
\pgfsetbuttcap%
\pgfsetroundjoin%
\definecolor{currentfill}{rgb}{0.000000,0.000000,0.000000}%
\pgfsetfillcolor{currentfill}%
\pgfsetlinewidth{0.803000pt}%
\definecolor{currentstroke}{rgb}{0.000000,0.000000,0.000000}%
\pgfsetstrokecolor{currentstroke}%
\pgfsetdash{}{0pt}%
\pgfsys@defobject{currentmarker}{\pgfqpoint{0.000000in}{-0.048611in}}{\pgfqpoint{0.000000in}{0.000000in}}{%
\pgfpathmoveto{\pgfqpoint{0.000000in}{0.000000in}}%
\pgfpathlineto{\pgfqpoint{0.000000in}{-0.048611in}}%
\pgfusepath{stroke,fill}%
}%
\begin{pgfscope}%
\pgfsys@transformshift{2.035915in}{1.826628in}%
\pgfsys@useobject{currentmarker}{}%
\end{pgfscope}%
\end{pgfscope}%
\begin{pgfscope}%
\pgfpathrectangle{\pgfqpoint{0.584475in}{1.826628in}}{\pgfqpoint{2.177159in}{0.813411in}}%
\pgfusepath{clip}%
\pgfsetrectcap%
\pgfsetroundjoin%
\pgfsetlinewidth{0.803000pt}%
\definecolor{currentstroke}{rgb}{0.690196,0.690196,0.690196}%
\pgfsetstrokecolor{currentstroke}%
\pgfsetdash{}{0pt}%
\pgfpathmoveto{\pgfqpoint{2.761635in}{1.826628in}}%
\pgfpathlineto{\pgfqpoint{2.761635in}{2.640039in}}%
\pgfusepath{stroke}%
\end{pgfscope}%
\begin{pgfscope}%
\pgfsetbuttcap%
\pgfsetroundjoin%
\definecolor{currentfill}{rgb}{0.000000,0.000000,0.000000}%
\pgfsetfillcolor{currentfill}%
\pgfsetlinewidth{0.803000pt}%
\definecolor{currentstroke}{rgb}{0.000000,0.000000,0.000000}%
\pgfsetstrokecolor{currentstroke}%
\pgfsetdash{}{0pt}%
\pgfsys@defobject{currentmarker}{\pgfqpoint{0.000000in}{-0.048611in}}{\pgfqpoint{0.000000in}{0.000000in}}{%
\pgfpathmoveto{\pgfqpoint{0.000000in}{0.000000in}}%
\pgfpathlineto{\pgfqpoint{0.000000in}{-0.048611in}}%
\pgfusepath{stroke,fill}%
}%
\begin{pgfscope}%
\pgfsys@transformshift{2.761635in}{1.826628in}%
\pgfsys@useobject{currentmarker}{}%
\end{pgfscope}%
\end{pgfscope}%
\begin{pgfscope}%
\pgfpathrectangle{\pgfqpoint{0.584475in}{1.826628in}}{\pgfqpoint{2.177159in}{0.813411in}}%
\pgfusepath{clip}%
\pgfsetrectcap%
\pgfsetroundjoin%
\pgfsetlinewidth{0.803000pt}%
\definecolor{currentstroke}{rgb}{0.690196,0.690196,0.690196}%
\pgfsetstrokecolor{currentstroke}%
\pgfsetdash{}{0pt}%
\pgfpathmoveto{\pgfqpoint{0.584475in}{1.826628in}}%
\pgfpathlineto{\pgfqpoint{2.761635in}{1.826628in}}%
\pgfusepath{stroke}%
\end{pgfscope}%
\begin{pgfscope}%
\pgfsetbuttcap%
\pgfsetroundjoin%
\definecolor{currentfill}{rgb}{0.000000,0.000000,0.000000}%
\pgfsetfillcolor{currentfill}%
\pgfsetlinewidth{0.803000pt}%
\definecolor{currentstroke}{rgb}{0.000000,0.000000,0.000000}%
\pgfsetstrokecolor{currentstroke}%
\pgfsetdash{}{0pt}%
\pgfsys@defobject{currentmarker}{\pgfqpoint{-0.048611in}{0.000000in}}{\pgfqpoint{0.000000in}{0.000000in}}{%
\pgfpathmoveto{\pgfqpoint{0.000000in}{0.000000in}}%
\pgfpathlineto{\pgfqpoint{-0.048611in}{0.000000in}}%
\pgfusepath{stroke,fill}%
}%
\begin{pgfscope}%
\pgfsys@transformshift{0.584475in}{1.826628in}%
\pgfsys@useobject{currentmarker}{}%
\end{pgfscope}%
\end{pgfscope}%
\begin{pgfscope}%
\definecolor{textcolor}{rgb}{0.000000,0.000000,0.000000}%
\pgfsetstrokecolor{textcolor}%
\pgfsetfillcolor{textcolor}%
\pgftext[x=0.150000in, y=1.773866in, left, base]{\color{textcolor}\sffamily\fontsize{10.000000}{12.000000}\selectfont −1.0}%
\end{pgfscope}%
\begin{pgfscope}%
\pgfpathrectangle{\pgfqpoint{0.584475in}{1.826628in}}{\pgfqpoint{2.177159in}{0.813411in}}%
\pgfusepath{clip}%
\pgfsetrectcap%
\pgfsetroundjoin%
\pgfsetlinewidth{0.803000pt}%
\definecolor{currentstroke}{rgb}{0.690196,0.690196,0.690196}%
\pgfsetstrokecolor{currentstroke}%
\pgfsetdash{}{0pt}%
\pgfpathmoveto{\pgfqpoint{0.584475in}{2.233333in}}%
\pgfpathlineto{\pgfqpoint{2.761635in}{2.233333in}}%
\pgfusepath{stroke}%
\end{pgfscope}%
\begin{pgfscope}%
\pgfsetbuttcap%
\pgfsetroundjoin%
\definecolor{currentfill}{rgb}{0.000000,0.000000,0.000000}%
\pgfsetfillcolor{currentfill}%
\pgfsetlinewidth{0.803000pt}%
\definecolor{currentstroke}{rgb}{0.000000,0.000000,0.000000}%
\pgfsetstrokecolor{currentstroke}%
\pgfsetdash{}{0pt}%
\pgfsys@defobject{currentmarker}{\pgfqpoint{-0.048611in}{0.000000in}}{\pgfqpoint{0.000000in}{0.000000in}}{%
\pgfpathmoveto{\pgfqpoint{0.000000in}{0.000000in}}%
\pgfpathlineto{\pgfqpoint{-0.048611in}{0.000000in}}%
\pgfusepath{stroke,fill}%
}%
\begin{pgfscope}%
\pgfsys@transformshift{0.584475in}{2.233333in}%
\pgfsys@useobject{currentmarker}{}%
\end{pgfscope}%
\end{pgfscope}%
\begin{pgfscope}%
\definecolor{textcolor}{rgb}{0.000000,0.000000,0.000000}%
\pgfsetstrokecolor{textcolor}%
\pgfsetfillcolor{textcolor}%
\pgftext[x=0.150000in, y=2.180572in, left, base]{\color{textcolor}\sffamily\fontsize{10.000000}{12.000000}\selectfont −0.5}%
\end{pgfscope}%
\begin{pgfscope}%
\pgfpathrectangle{\pgfqpoint{0.584475in}{1.826628in}}{\pgfqpoint{2.177159in}{0.813411in}}%
\pgfusepath{clip}%
\pgfsetrectcap%
\pgfsetroundjoin%
\pgfsetlinewidth{0.803000pt}%
\definecolor{currentstroke}{rgb}{0.690196,0.690196,0.690196}%
\pgfsetstrokecolor{currentstroke}%
\pgfsetdash{}{0pt}%
\pgfpathmoveto{\pgfqpoint{0.584475in}{2.640039in}}%
\pgfpathlineto{\pgfqpoint{2.761635in}{2.640039in}}%
\pgfusepath{stroke}%
\end{pgfscope}%
\begin{pgfscope}%
\pgfsetbuttcap%
\pgfsetroundjoin%
\definecolor{currentfill}{rgb}{0.000000,0.000000,0.000000}%
\pgfsetfillcolor{currentfill}%
\pgfsetlinewidth{0.803000pt}%
\definecolor{currentstroke}{rgb}{0.000000,0.000000,0.000000}%
\pgfsetstrokecolor{currentstroke}%
\pgfsetdash{}{0pt}%
\pgfsys@defobject{currentmarker}{\pgfqpoint{-0.048611in}{0.000000in}}{\pgfqpoint{0.000000in}{0.000000in}}{%
\pgfpathmoveto{\pgfqpoint{0.000000in}{0.000000in}}%
\pgfpathlineto{\pgfqpoint{-0.048611in}{0.000000in}}%
\pgfusepath{stroke,fill}%
}%
\begin{pgfscope}%
\pgfsys@transformshift{0.584475in}{2.640039in}%
\pgfsys@useobject{currentmarker}{}%
\end{pgfscope}%
\end{pgfscope}%
\begin{pgfscope}%
\definecolor{textcolor}{rgb}{0.000000,0.000000,0.000000}%
\pgfsetstrokecolor{textcolor}%
\pgfsetfillcolor{textcolor}%
\pgftext[x=0.266374in, y=2.587277in, left, base]{\color{textcolor}\sffamily\fontsize{10.000000}{12.000000}\selectfont 0.0}%
\end{pgfscope}%
\begin{pgfscope}%
\pgfpathrectangle{\pgfqpoint{0.584475in}{1.826628in}}{\pgfqpoint{2.177159in}{0.813411in}}%
\pgfusepath{clip}%
\pgfsetrectcap%
\pgfsetroundjoin%
\pgfsetlinewidth{1.505625pt}%
\definecolor{currentstroke}{rgb}{0.000000,0.000000,1.000000}%
\pgfsetstrokecolor{currentstroke}%
\pgfsetdash{}{0pt}%
\pgfpathmoveto{\pgfqpoint{0.584475in}{2.209571in}}%
\pgfpathlineto{\pgfqpoint{0.598990in}{2.284189in}}%
\pgfpathlineto{\pgfqpoint{0.613504in}{2.335766in}}%
\pgfpathlineto{\pgfqpoint{0.628019in}{2.377533in}}%
\pgfpathlineto{\pgfqpoint{0.642533in}{2.438309in}}%
\pgfpathlineto{\pgfqpoint{0.657047in}{2.514552in}}%
\pgfpathlineto{\pgfqpoint{0.671562in}{2.576508in}}%
\pgfpathlineto{\pgfqpoint{0.686076in}{2.622047in}}%
\pgfpathlineto{\pgfqpoint{0.700591in}{2.626516in}}%
\pgfpathlineto{\pgfqpoint{0.715105in}{2.627393in}}%
\pgfpathlineto{\pgfqpoint{0.729619in}{2.628935in}}%
\pgfpathlineto{\pgfqpoint{0.744134in}{2.628862in}}%
\pgfpathlineto{\pgfqpoint{0.758648in}{2.626875in}}%
\pgfpathlineto{\pgfqpoint{0.773163in}{2.624137in}}%
\pgfpathlineto{\pgfqpoint{0.787677in}{2.622673in}}%
\pgfpathlineto{\pgfqpoint{0.802191in}{2.608942in}}%
\pgfpathlineto{\pgfqpoint{0.816706in}{2.544423in}}%
\pgfpathlineto{\pgfqpoint{0.831220in}{2.534396in}}%
\pgfpathlineto{\pgfqpoint{0.845735in}{2.519688in}}%
\pgfpathlineto{\pgfqpoint{0.860249in}{2.506151in}}%
\pgfpathlineto{\pgfqpoint{0.874763in}{2.334796in}}%
\pgfpathlineto{\pgfqpoint{1.165051in}{2.334796in}}%
\pgfpathlineto{\pgfqpoint{1.179566in}{2.470150in}}%
\pgfpathlineto{\pgfqpoint{1.208594in}{2.470195in}}%
\pgfpathlineto{\pgfqpoint{1.223109in}{2.500293in}}%
\pgfpathlineto{\pgfqpoint{1.237623in}{2.512058in}}%
\pgfpathlineto{\pgfqpoint{1.252138in}{2.580012in}}%
\pgfpathlineto{\pgfqpoint{1.266652in}{2.630871in}}%
\pgfpathlineto{\pgfqpoint{1.295681in}{2.633922in}}%
\pgfpathlineto{\pgfqpoint{1.310195in}{2.634543in}}%
\pgfpathlineto{\pgfqpoint{1.324710in}{2.634143in}}%
\pgfpathlineto{\pgfqpoint{1.353738in}{2.631577in}}%
\pgfpathlineto{\pgfqpoint{1.368253in}{2.616337in}}%
\pgfpathlineto{\pgfqpoint{1.382767in}{2.568385in}}%
\pgfpathlineto{\pgfqpoint{1.397282in}{2.506055in}}%
\pgfpathlineto{\pgfqpoint{1.411796in}{2.431211in}}%
\pgfpathlineto{\pgfqpoint{1.426310in}{2.373696in}}%
\pgfpathlineto{\pgfqpoint{1.440825in}{2.331433in}}%
\pgfpathlineto{\pgfqpoint{1.455339in}{2.277286in}}%
\pgfpathlineto{\pgfqpoint{1.469854in}{2.194386in}}%
\pgfpathlineto{\pgfqpoint{1.484368in}{2.008589in}}%
\pgfpathlineto{\pgfqpoint{1.498882in}{2.225578in}}%
\pgfpathlineto{\pgfqpoint{1.513397in}{2.294377in}}%
\pgfpathlineto{\pgfqpoint{1.527911in}{2.342457in}}%
\pgfpathlineto{\pgfqpoint{1.542425in}{2.385734in}}%
\pgfpathlineto{\pgfqpoint{1.556940in}{2.450691in}}%
\pgfpathlineto{\pgfqpoint{1.571454in}{2.526686in}}%
\pgfpathlineto{\pgfqpoint{1.585969in}{2.584997in}}%
\pgfpathlineto{\pgfqpoint{1.600483in}{2.625502in}}%
\pgfpathlineto{\pgfqpoint{1.614997in}{2.626384in}}%
\pgfpathlineto{\pgfqpoint{1.644026in}{2.629046in}}%
\pgfpathlineto{\pgfqpoint{1.658541in}{2.628583in}}%
\pgfpathlineto{\pgfqpoint{1.687569in}{2.623948in}}%
\pgfpathlineto{\pgfqpoint{1.702084in}{2.622688in}}%
\pgfpathlineto{\pgfqpoint{1.716598in}{2.590141in}}%
\pgfpathlineto{\pgfqpoint{1.731113in}{2.544166in}}%
\pgfpathlineto{\pgfqpoint{1.745627in}{2.534391in}}%
\pgfpathlineto{\pgfqpoint{1.774656in}{2.506920in}}%
\pgfpathlineto{\pgfqpoint{1.789170in}{2.334796in}}%
\pgfpathlineto{\pgfqpoint{2.079458in}{2.334796in}}%
\pgfpathlineto{\pgfqpoint{2.093972in}{2.470155in}}%
\pgfpathlineto{\pgfqpoint{2.123001in}{2.470199in}}%
\pgfpathlineto{\pgfqpoint{2.137516in}{2.500295in}}%
\pgfpathlineto{\pgfqpoint{2.152030in}{2.520232in}}%
\pgfpathlineto{\pgfqpoint{2.166544in}{2.628393in}}%
\pgfpathlineto{\pgfqpoint{2.181059in}{2.631591in}}%
\pgfpathlineto{\pgfqpoint{2.210088in}{2.634068in}}%
\pgfpathlineto{\pgfqpoint{2.224602in}{2.634605in}}%
\pgfpathlineto{\pgfqpoint{2.239116in}{2.634015in}}%
\pgfpathlineto{\pgfqpoint{2.268145in}{2.631369in}}%
\pgfpathlineto{\pgfqpoint{2.282660in}{2.608717in}}%
\pgfpathlineto{\pgfqpoint{2.297174in}{2.559249in}}%
\pgfpathlineto{\pgfqpoint{2.311688in}{2.493607in}}%
\pgfpathlineto{\pgfqpoint{2.326203in}{2.419090in}}%
\pgfpathlineto{\pgfqpoint{2.340717in}{2.366648in}}%
\pgfpathlineto{\pgfqpoint{2.355232in}{2.323359in}}%
\pgfpathlineto{\pgfqpoint{2.369746in}{2.266314in}}%
\pgfpathlineto{\pgfqpoint{2.384260in}{2.171034in}}%
\pgfpathlineto{\pgfqpoint{2.398775in}{2.095412in}}%
\pgfpathlineto{\pgfqpoint{2.413289in}{2.239217in}}%
\pgfpathlineto{\pgfqpoint{2.427804in}{2.303672in}}%
\pgfpathlineto{\pgfqpoint{2.456832in}{2.394723in}}%
\pgfpathlineto{\pgfqpoint{2.471347in}{2.463514in}}%
\pgfpathlineto{\pgfqpoint{2.485861in}{2.538010in}}%
\pgfpathlineto{\pgfqpoint{2.500376in}{2.593153in}}%
\pgfpathlineto{\pgfqpoint{2.514890in}{2.628012in}}%
\pgfpathlineto{\pgfqpoint{2.529404in}{2.626283in}}%
\pgfpathlineto{\pgfqpoint{2.543919in}{2.628141in}}%
\pgfpathlineto{\pgfqpoint{2.558433in}{2.629187in}}%
\pgfpathlineto{\pgfqpoint{2.572948in}{2.628275in}}%
\pgfpathlineto{\pgfqpoint{2.601976in}{2.623668in}}%
\pgfpathlineto{\pgfqpoint{2.616491in}{2.621909in}}%
\pgfpathlineto{\pgfqpoint{2.631005in}{2.576172in}}%
\pgfpathlineto{\pgfqpoint{2.645519in}{2.543897in}}%
\pgfpathlineto{\pgfqpoint{2.660034in}{2.517055in}}%
\pgfpathlineto{\pgfqpoint{2.674548in}{2.502052in}}%
\pgfpathlineto{\pgfqpoint{2.689063in}{2.507497in}}%
\pgfpathlineto{\pgfqpoint{2.703577in}{2.334796in}}%
\pgfpathlineto{\pgfqpoint{2.764968in}{2.334796in}}%
\pgfpathlineto{\pgfqpoint{2.764968in}{2.334796in}}%
\pgfusepath{stroke}%
\end{pgfscope}%
\begin{pgfscope}%
\pgfsetrectcap%
\pgfsetmiterjoin%
\pgfsetlinewidth{0.803000pt}%
\definecolor{currentstroke}{rgb}{0.000000,0.000000,0.000000}%
\pgfsetstrokecolor{currentstroke}%
\pgfsetdash{}{0pt}%
\pgfpathmoveto{\pgfqpoint{0.584475in}{1.826628in}}%
\pgfpathlineto{\pgfqpoint{0.584475in}{2.640039in}}%
\pgfusepath{stroke}%
\end{pgfscope}%
\begin{pgfscope}%
\pgfsetrectcap%
\pgfsetmiterjoin%
\pgfsetlinewidth{0.803000pt}%
\definecolor{currentstroke}{rgb}{0.000000,0.000000,0.000000}%
\pgfsetstrokecolor{currentstroke}%
\pgfsetdash{}{0pt}%
\pgfpathmoveto{\pgfqpoint{2.761635in}{1.826628in}}%
\pgfpathlineto{\pgfqpoint{2.761635in}{2.640039in}}%
\pgfusepath{stroke}%
\end{pgfscope}%
\begin{pgfscope}%
\pgfsetrectcap%
\pgfsetmiterjoin%
\pgfsetlinewidth{0.803000pt}%
\definecolor{currentstroke}{rgb}{0.000000,0.000000,0.000000}%
\pgfsetstrokecolor{currentstroke}%
\pgfsetdash{}{0pt}%
\pgfpathmoveto{\pgfqpoint{0.584475in}{1.826628in}}%
\pgfpathlineto{\pgfqpoint{2.761635in}{1.826628in}}%
\pgfusepath{stroke}%
\end{pgfscope}%
\begin{pgfscope}%
\pgfsetrectcap%
\pgfsetmiterjoin%
\pgfsetlinewidth{0.803000pt}%
\definecolor{currentstroke}{rgb}{0.000000,0.000000,0.000000}%
\pgfsetstrokecolor{currentstroke}%
\pgfsetdash{}{0pt}%
\pgfpathmoveto{\pgfqpoint{0.584475in}{2.640039in}}%
\pgfpathlineto{\pgfqpoint{2.761635in}{2.640039in}}%
\pgfusepath{stroke}%
\end{pgfscope}%
\begin{pgfscope}%
\definecolor{textcolor}{rgb}{0.000000,0.000000,1.000000}%
\pgfsetstrokecolor{textcolor}%
\pgfsetfillcolor{textcolor}%
\pgftext[x=1.673055in,y=2.723372in,,base]{\color{textcolor}\sffamily\fontsize{12.000000}{14.400000}\selectfont MI Metrik}%
\end{pgfscope}%
\begin{pgfscope}%
\pgfsetbuttcap%
\pgfsetmiterjoin%
\definecolor{currentfill}{rgb}{1.000000,1.000000,1.000000}%
\pgfsetfillcolor{currentfill}%
\pgfsetlinewidth{0.000000pt}%
\definecolor{currentstroke}{rgb}{0.000000,0.000000,0.000000}%
\pgfsetstrokecolor{currentstroke}%
\pgfsetstrokeopacity{0.000000}%
\pgfsetdash{}{0pt}%
\pgfpathmoveto{\pgfqpoint{0.584475in}{0.571604in}}%
\pgfpathlineto{\pgfqpoint{2.761635in}{0.571604in}}%
\pgfpathlineto{\pgfqpoint{2.761635in}{1.385015in}}%
\pgfpathlineto{\pgfqpoint{0.584475in}{1.385015in}}%
\pgfpathclose%
\pgfusepath{fill}%
\end{pgfscope}%
\begin{pgfscope}%
\pgfpathrectangle{\pgfqpoint{0.584475in}{0.571604in}}{\pgfqpoint{2.177159in}{0.813411in}}%
\pgfusepath{clip}%
\pgfsetrectcap%
\pgfsetroundjoin%
\pgfsetlinewidth{0.803000pt}%
\definecolor{currentstroke}{rgb}{0.690196,0.690196,0.690196}%
\pgfsetstrokecolor{currentstroke}%
\pgfsetdash{}{0pt}%
\pgfpathmoveto{\pgfqpoint{0.584475in}{0.571604in}}%
\pgfpathlineto{\pgfqpoint{0.584475in}{1.385015in}}%
\pgfusepath{stroke}%
\end{pgfscope}%
\begin{pgfscope}%
\pgfsetbuttcap%
\pgfsetroundjoin%
\definecolor{currentfill}{rgb}{0.000000,0.000000,0.000000}%
\pgfsetfillcolor{currentfill}%
\pgfsetlinewidth{0.803000pt}%
\definecolor{currentstroke}{rgb}{0.000000,0.000000,0.000000}%
\pgfsetstrokecolor{currentstroke}%
\pgfsetdash{}{0pt}%
\pgfsys@defobject{currentmarker}{\pgfqpoint{0.000000in}{-0.048611in}}{\pgfqpoint{0.000000in}{0.000000in}}{%
\pgfpathmoveto{\pgfqpoint{0.000000in}{0.000000in}}%
\pgfpathlineto{\pgfqpoint{0.000000in}{-0.048611in}}%
\pgfusepath{stroke,fill}%
}%
\begin{pgfscope}%
\pgfsys@transformshift{0.584475in}{0.571604in}%
\pgfsys@useobject{currentmarker}{}%
\end{pgfscope}%
\end{pgfscope}%
\begin{pgfscope}%
\definecolor{textcolor}{rgb}{0.000000,0.000000,0.000000}%
\pgfsetstrokecolor{textcolor}%
\pgfsetfillcolor{textcolor}%
\pgftext[x=0.584475in,y=0.474382in,,top]{\color{textcolor}\sffamily\fontsize{10.000000}{12.000000}\selectfont 0}%
\end{pgfscope}%
\begin{pgfscope}%
\pgfpathrectangle{\pgfqpoint{0.584475in}{0.571604in}}{\pgfqpoint{2.177159in}{0.813411in}}%
\pgfusepath{clip}%
\pgfsetrectcap%
\pgfsetroundjoin%
\pgfsetlinewidth{0.803000pt}%
\definecolor{currentstroke}{rgb}{0.690196,0.690196,0.690196}%
\pgfsetstrokecolor{currentstroke}%
\pgfsetdash{}{0pt}%
\pgfpathmoveto{\pgfqpoint{1.310195in}{0.571604in}}%
\pgfpathlineto{\pgfqpoint{1.310195in}{1.385015in}}%
\pgfusepath{stroke}%
\end{pgfscope}%
\begin{pgfscope}%
\pgfsetbuttcap%
\pgfsetroundjoin%
\definecolor{currentfill}{rgb}{0.000000,0.000000,0.000000}%
\pgfsetfillcolor{currentfill}%
\pgfsetlinewidth{0.803000pt}%
\definecolor{currentstroke}{rgb}{0.000000,0.000000,0.000000}%
\pgfsetstrokecolor{currentstroke}%
\pgfsetdash{}{0pt}%
\pgfsys@defobject{currentmarker}{\pgfqpoint{0.000000in}{-0.048611in}}{\pgfqpoint{0.000000in}{0.000000in}}{%
\pgfpathmoveto{\pgfqpoint{0.000000in}{0.000000in}}%
\pgfpathlineto{\pgfqpoint{0.000000in}{-0.048611in}}%
\pgfusepath{stroke,fill}%
}%
\begin{pgfscope}%
\pgfsys@transformshift{1.310195in}{0.571604in}%
\pgfsys@useobject{currentmarker}{}%
\end{pgfscope}%
\end{pgfscope}%
\begin{pgfscope}%
\definecolor{textcolor}{rgb}{0.000000,0.000000,0.000000}%
\pgfsetstrokecolor{textcolor}%
\pgfsetfillcolor{textcolor}%
\pgftext[x=1.310195in,y=0.474382in,,top]{\color{textcolor}\sffamily\fontsize{10.000000}{12.000000}\selectfont 5}%
\end{pgfscope}%
\begin{pgfscope}%
\pgfpathrectangle{\pgfqpoint{0.584475in}{0.571604in}}{\pgfqpoint{2.177159in}{0.813411in}}%
\pgfusepath{clip}%
\pgfsetrectcap%
\pgfsetroundjoin%
\pgfsetlinewidth{0.803000pt}%
\definecolor{currentstroke}{rgb}{0.690196,0.690196,0.690196}%
\pgfsetstrokecolor{currentstroke}%
\pgfsetdash{}{0pt}%
\pgfpathmoveto{\pgfqpoint{2.035915in}{0.571604in}}%
\pgfpathlineto{\pgfqpoint{2.035915in}{1.385015in}}%
\pgfusepath{stroke}%
\end{pgfscope}%
\begin{pgfscope}%
\pgfsetbuttcap%
\pgfsetroundjoin%
\definecolor{currentfill}{rgb}{0.000000,0.000000,0.000000}%
\pgfsetfillcolor{currentfill}%
\pgfsetlinewidth{0.803000pt}%
\definecolor{currentstroke}{rgb}{0.000000,0.000000,0.000000}%
\pgfsetstrokecolor{currentstroke}%
\pgfsetdash{}{0pt}%
\pgfsys@defobject{currentmarker}{\pgfqpoint{0.000000in}{-0.048611in}}{\pgfqpoint{0.000000in}{0.000000in}}{%
\pgfpathmoveto{\pgfqpoint{0.000000in}{0.000000in}}%
\pgfpathlineto{\pgfqpoint{0.000000in}{-0.048611in}}%
\pgfusepath{stroke,fill}%
}%
\begin{pgfscope}%
\pgfsys@transformshift{2.035915in}{0.571604in}%
\pgfsys@useobject{currentmarker}{}%
\end{pgfscope}%
\end{pgfscope}%
\begin{pgfscope}%
\definecolor{textcolor}{rgb}{0.000000,0.000000,0.000000}%
\pgfsetstrokecolor{textcolor}%
\pgfsetfillcolor{textcolor}%
\pgftext[x=2.035915in,y=0.474382in,,top]{\color{textcolor}\sffamily\fontsize{10.000000}{12.000000}\selectfont 10}%
\end{pgfscope}%
\begin{pgfscope}%
\pgfpathrectangle{\pgfqpoint{0.584475in}{0.571604in}}{\pgfqpoint{2.177159in}{0.813411in}}%
\pgfusepath{clip}%
\pgfsetrectcap%
\pgfsetroundjoin%
\pgfsetlinewidth{0.803000pt}%
\definecolor{currentstroke}{rgb}{0.690196,0.690196,0.690196}%
\pgfsetstrokecolor{currentstroke}%
\pgfsetdash{}{0pt}%
\pgfpathmoveto{\pgfqpoint{2.761635in}{0.571604in}}%
\pgfpathlineto{\pgfqpoint{2.761635in}{1.385015in}}%
\pgfusepath{stroke}%
\end{pgfscope}%
\begin{pgfscope}%
\pgfsetbuttcap%
\pgfsetroundjoin%
\definecolor{currentfill}{rgb}{0.000000,0.000000,0.000000}%
\pgfsetfillcolor{currentfill}%
\pgfsetlinewidth{0.803000pt}%
\definecolor{currentstroke}{rgb}{0.000000,0.000000,0.000000}%
\pgfsetstrokecolor{currentstroke}%
\pgfsetdash{}{0pt}%
\pgfsys@defobject{currentmarker}{\pgfqpoint{0.000000in}{-0.048611in}}{\pgfqpoint{0.000000in}{0.000000in}}{%
\pgfpathmoveto{\pgfqpoint{0.000000in}{0.000000in}}%
\pgfpathlineto{\pgfqpoint{0.000000in}{-0.048611in}}%
\pgfusepath{stroke,fill}%
}%
\begin{pgfscope}%
\pgfsys@transformshift{2.761635in}{0.571604in}%
\pgfsys@useobject{currentmarker}{}%
\end{pgfscope}%
\end{pgfscope}%
\begin{pgfscope}%
\definecolor{textcolor}{rgb}{0.000000,0.000000,0.000000}%
\pgfsetstrokecolor{textcolor}%
\pgfsetfillcolor{textcolor}%
\pgftext[x=2.761635in,y=0.474382in,,top]{\color{textcolor}\sffamily\fontsize{10.000000}{12.000000}\selectfont 15}%
\end{pgfscope}%
\begin{pgfscope}%
\definecolor{textcolor}{rgb}{0.000000,0.000000,0.000000}%
\pgfsetstrokecolor{textcolor}%
\pgfsetfillcolor{textcolor}%
\pgftext[x=1.673055in,y=0.284413in,,top]{\color{textcolor}\sffamily\fontsize{10.000000}{12.000000}\selectfont Shift von Paramter 1}%
\end{pgfscope}%
\begin{pgfscope}%
\pgfpathrectangle{\pgfqpoint{0.584475in}{0.571604in}}{\pgfqpoint{2.177159in}{0.813411in}}%
\pgfusepath{clip}%
\pgfsetrectcap%
\pgfsetroundjoin%
\pgfsetlinewidth{0.803000pt}%
\definecolor{currentstroke}{rgb}{0.690196,0.690196,0.690196}%
\pgfsetstrokecolor{currentstroke}%
\pgfsetdash{}{0pt}%
\pgfpathmoveto{\pgfqpoint{0.584475in}{0.571604in}}%
\pgfpathlineto{\pgfqpoint{2.761635in}{0.571604in}}%
\pgfusepath{stroke}%
\end{pgfscope}%
\begin{pgfscope}%
\pgfsetbuttcap%
\pgfsetroundjoin%
\definecolor{currentfill}{rgb}{0.000000,0.000000,0.000000}%
\pgfsetfillcolor{currentfill}%
\pgfsetlinewidth{0.803000pt}%
\definecolor{currentstroke}{rgb}{0.000000,0.000000,0.000000}%
\pgfsetstrokecolor{currentstroke}%
\pgfsetdash{}{0pt}%
\pgfsys@defobject{currentmarker}{\pgfqpoint{-0.048611in}{0.000000in}}{\pgfqpoint{0.000000in}{0.000000in}}{%
\pgfpathmoveto{\pgfqpoint{0.000000in}{0.000000in}}%
\pgfpathlineto{\pgfqpoint{-0.048611in}{0.000000in}}%
\pgfusepath{stroke,fill}%
}%
\begin{pgfscope}%
\pgfsys@transformshift{0.584475in}{0.571604in}%
\pgfsys@useobject{currentmarker}{}%
\end{pgfscope}%
\end{pgfscope}%
\begin{pgfscope}%
\definecolor{textcolor}{rgb}{0.000000,0.000000,0.000000}%
\pgfsetstrokecolor{textcolor}%
\pgfsetfillcolor{textcolor}%
\pgftext[x=0.266374in, y=0.518842in, left, base]{\color{textcolor}\sffamily\fontsize{10.000000}{12.000000}\selectfont 0.0}%
\end{pgfscope}%
\begin{pgfscope}%
\pgfpathrectangle{\pgfqpoint{0.584475in}{0.571604in}}{\pgfqpoint{2.177159in}{0.813411in}}%
\pgfusepath{clip}%
\pgfsetrectcap%
\pgfsetroundjoin%
\pgfsetlinewidth{0.803000pt}%
\definecolor{currentstroke}{rgb}{0.690196,0.690196,0.690196}%
\pgfsetstrokecolor{currentstroke}%
\pgfsetdash{}{0pt}%
\pgfpathmoveto{\pgfqpoint{0.584475in}{0.978310in}}%
\pgfpathlineto{\pgfqpoint{2.761635in}{0.978310in}}%
\pgfusepath{stroke}%
\end{pgfscope}%
\begin{pgfscope}%
\pgfsetbuttcap%
\pgfsetroundjoin%
\definecolor{currentfill}{rgb}{0.000000,0.000000,0.000000}%
\pgfsetfillcolor{currentfill}%
\pgfsetlinewidth{0.803000pt}%
\definecolor{currentstroke}{rgb}{0.000000,0.000000,0.000000}%
\pgfsetstrokecolor{currentstroke}%
\pgfsetdash{}{0pt}%
\pgfsys@defobject{currentmarker}{\pgfqpoint{-0.048611in}{0.000000in}}{\pgfqpoint{0.000000in}{0.000000in}}{%
\pgfpathmoveto{\pgfqpoint{0.000000in}{0.000000in}}%
\pgfpathlineto{\pgfqpoint{-0.048611in}{0.000000in}}%
\pgfusepath{stroke,fill}%
}%
\begin{pgfscope}%
\pgfsys@transformshift{0.584475in}{0.978310in}%
\pgfsys@useobject{currentmarker}{}%
\end{pgfscope}%
\end{pgfscope}%
\begin{pgfscope}%
\definecolor{textcolor}{rgb}{0.000000,0.000000,0.000000}%
\pgfsetstrokecolor{textcolor}%
\pgfsetfillcolor{textcolor}%
\pgftext[x=0.266374in, y=0.925548in, left, base]{\color{textcolor}\sffamily\fontsize{10.000000}{12.000000}\selectfont 2.5}%
\end{pgfscope}%
\begin{pgfscope}%
\pgfpathrectangle{\pgfqpoint{0.584475in}{0.571604in}}{\pgfqpoint{2.177159in}{0.813411in}}%
\pgfusepath{clip}%
\pgfsetrectcap%
\pgfsetroundjoin%
\pgfsetlinewidth{0.803000pt}%
\definecolor{currentstroke}{rgb}{0.690196,0.690196,0.690196}%
\pgfsetstrokecolor{currentstroke}%
\pgfsetdash{}{0pt}%
\pgfpathmoveto{\pgfqpoint{0.584475in}{1.385015in}}%
\pgfpathlineto{\pgfqpoint{2.761635in}{1.385015in}}%
\pgfusepath{stroke}%
\end{pgfscope}%
\begin{pgfscope}%
\pgfsetbuttcap%
\pgfsetroundjoin%
\definecolor{currentfill}{rgb}{0.000000,0.000000,0.000000}%
\pgfsetfillcolor{currentfill}%
\pgfsetlinewidth{0.803000pt}%
\definecolor{currentstroke}{rgb}{0.000000,0.000000,0.000000}%
\pgfsetstrokecolor{currentstroke}%
\pgfsetdash{}{0pt}%
\pgfsys@defobject{currentmarker}{\pgfqpoint{-0.048611in}{0.000000in}}{\pgfqpoint{0.000000in}{0.000000in}}{%
\pgfpathmoveto{\pgfqpoint{0.000000in}{0.000000in}}%
\pgfpathlineto{\pgfqpoint{-0.048611in}{0.000000in}}%
\pgfusepath{stroke,fill}%
}%
\begin{pgfscope}%
\pgfsys@transformshift{0.584475in}{1.385015in}%
\pgfsys@useobject{currentmarker}{}%
\end{pgfscope}%
\end{pgfscope}%
\begin{pgfscope}%
\definecolor{textcolor}{rgb}{0.000000,0.000000,0.000000}%
\pgfsetstrokecolor{textcolor}%
\pgfsetfillcolor{textcolor}%
\pgftext[x=0.266374in, y=1.332254in, left, base]{\color{textcolor}\sffamily\fontsize{10.000000}{12.000000}\selectfont 5.0}%
\end{pgfscope}%
\begin{pgfscope}%
\definecolor{textcolor}{rgb}{0.000000,0.000000,0.000000}%
\pgfsetstrokecolor{textcolor}%
\pgfsetfillcolor{textcolor}%
\pgftext[x=0.584475in,y=1.426682in,left,base]{\color{textcolor}\sffamily\fontsize{10.000000}{12.000000}\selectfont 1e5}%
\end{pgfscope}%
\begin{pgfscope}%
\pgfpathrectangle{\pgfqpoint{0.584475in}{0.571604in}}{\pgfqpoint{2.177159in}{0.813411in}}%
\pgfusepath{clip}%
\pgfsetrectcap%
\pgfsetroundjoin%
\pgfsetlinewidth{1.505625pt}%
\definecolor{currentstroke}{rgb}{1.000000,0.000000,0.000000}%
\pgfsetstrokecolor{currentstroke}%
\pgfsetdash{}{0pt}%
\pgfpathmoveto{\pgfqpoint{0.584475in}{0.758939in}}%
\pgfpathlineto{\pgfqpoint{0.598990in}{0.861943in}}%
\pgfpathlineto{\pgfqpoint{0.628019in}{1.001337in}}%
\pgfpathlineto{\pgfqpoint{0.642533in}{1.085176in}}%
\pgfpathlineto{\pgfqpoint{0.657047in}{1.163084in}}%
\pgfpathlineto{\pgfqpoint{0.671562in}{1.181352in}}%
\pgfpathlineto{\pgfqpoint{0.686076in}{1.206823in}}%
\pgfpathlineto{\pgfqpoint{0.700591in}{1.213675in}}%
\pgfpathlineto{\pgfqpoint{0.715105in}{1.219473in}}%
\pgfpathlineto{\pgfqpoint{0.729619in}{1.216675in}}%
\pgfpathlineto{\pgfqpoint{0.744134in}{1.206522in}}%
\pgfpathlineto{\pgfqpoint{0.758648in}{1.199226in}}%
\pgfpathlineto{\pgfqpoint{0.773163in}{1.196405in}}%
\pgfpathlineto{\pgfqpoint{0.787677in}{1.171690in}}%
\pgfpathlineto{\pgfqpoint{0.802191in}{0.574783in}}%
\pgfpathlineto{\pgfqpoint{0.816706in}{0.571720in}}%
\pgfpathlineto{\pgfqpoint{1.252138in}{0.571730in}}%
\pgfpathlineto{\pgfqpoint{1.266652in}{1.206499in}}%
\pgfpathlineto{\pgfqpoint{1.281166in}{1.208952in}}%
\pgfpathlineto{\pgfqpoint{1.295681in}{1.210203in}}%
\pgfpathlineto{\pgfqpoint{1.310195in}{1.216680in}}%
\pgfpathlineto{\pgfqpoint{1.324710in}{1.227270in}}%
\pgfpathlineto{\pgfqpoint{1.339224in}{1.227609in}}%
\pgfpathlineto{\pgfqpoint{1.353738in}{1.221035in}}%
\pgfpathlineto{\pgfqpoint{1.368253in}{1.207314in}}%
\pgfpathlineto{\pgfqpoint{1.382767in}{1.184408in}}%
\pgfpathlineto{\pgfqpoint{1.397282in}{1.157545in}}%
\pgfpathlineto{\pgfqpoint{1.411796in}{1.071422in}}%
\pgfpathlineto{\pgfqpoint{1.426310in}{0.988407in}}%
\pgfpathlineto{\pgfqpoint{1.440825in}{0.919594in}}%
\pgfpathlineto{\pgfqpoint{1.455339in}{0.848746in}}%
\pgfpathlineto{\pgfqpoint{1.469854in}{0.734240in}}%
\pgfpathlineto{\pgfqpoint{1.484368in}{0.597245in}}%
\pgfpathlineto{\pgfqpoint{1.498882in}{0.782025in}}%
\pgfpathlineto{\pgfqpoint{1.513397in}{0.874917in}}%
\pgfpathlineto{\pgfqpoint{1.527911in}{0.942087in}}%
\pgfpathlineto{\pgfqpoint{1.542425in}{1.014692in}}%
\pgfpathlineto{\pgfqpoint{1.556940in}{1.100879in}}%
\pgfpathlineto{\pgfqpoint{1.571454in}{1.169456in}}%
\pgfpathlineto{\pgfqpoint{1.585969in}{1.184181in}}%
\pgfpathlineto{\pgfqpoint{1.600483in}{1.210607in}}%
\pgfpathlineto{\pgfqpoint{1.629512in}{1.220155in}}%
\pgfpathlineto{\pgfqpoint{1.644026in}{1.216748in}}%
\pgfpathlineto{\pgfqpoint{1.658541in}{1.205725in}}%
\pgfpathlineto{\pgfqpoint{1.673055in}{1.198615in}}%
\pgfpathlineto{\pgfqpoint{1.687569in}{1.190525in}}%
\pgfpathlineto{\pgfqpoint{1.702084in}{1.152825in}}%
\pgfpathlineto{\pgfqpoint{1.716598in}{0.571713in}}%
\pgfpathlineto{\pgfqpoint{2.152030in}{0.571722in}}%
\pgfpathlineto{\pgfqpoint{2.166544in}{0.988876in}}%
\pgfpathlineto{\pgfqpoint{2.181059in}{1.175653in}}%
\pgfpathlineto{\pgfqpoint{2.195573in}{1.202426in}}%
\pgfpathlineto{\pgfqpoint{2.210088in}{1.213673in}}%
\pgfpathlineto{\pgfqpoint{2.224602in}{1.219100in}}%
\pgfpathlineto{\pgfqpoint{2.239116in}{1.226442in}}%
\pgfpathlineto{\pgfqpoint{2.253631in}{1.227169in}}%
\pgfpathlineto{\pgfqpoint{2.268145in}{1.220239in}}%
\pgfpathlineto{\pgfqpoint{2.282660in}{1.201981in}}%
\pgfpathlineto{\pgfqpoint{2.297174in}{1.181690in}}%
\pgfpathlineto{\pgfqpoint{2.311688in}{1.146632in}}%
\pgfpathlineto{\pgfqpoint{2.326203in}{1.056334in}}%
\pgfpathlineto{\pgfqpoint{2.340717in}{0.976152in}}%
\pgfpathlineto{\pgfqpoint{2.355232in}{0.908494in}}%
\pgfpathlineto{\pgfqpoint{2.369746in}{0.834022in}}%
\pgfpathlineto{\pgfqpoint{2.384260in}{0.704797in}}%
\pgfpathlineto{\pgfqpoint{2.398775in}{0.636538in}}%
\pgfpathlineto{\pgfqpoint{2.413289in}{0.802126in}}%
\pgfpathlineto{\pgfqpoint{2.427804in}{0.887095in}}%
\pgfpathlineto{\pgfqpoint{2.442318in}{0.953094in}}%
\pgfpathlineto{\pgfqpoint{2.456832in}{1.028403in}}%
\pgfpathlineto{\pgfqpoint{2.471347in}{1.116863in}}%
\pgfpathlineto{\pgfqpoint{2.485861in}{1.173312in}}%
\pgfpathlineto{\pgfqpoint{2.500376in}{1.186517in}}%
\pgfpathlineto{\pgfqpoint{2.514890in}{1.212387in}}%
\pgfpathlineto{\pgfqpoint{2.529404in}{1.217176in}}%
\pgfpathlineto{\pgfqpoint{2.543919in}{1.218783in}}%
\pgfpathlineto{\pgfqpoint{2.558433in}{1.214540in}}%
\pgfpathlineto{\pgfqpoint{2.572948in}{1.205155in}}%
\pgfpathlineto{\pgfqpoint{2.587462in}{1.198350in}}%
\pgfpathlineto{\pgfqpoint{2.601976in}{1.192487in}}%
\pgfpathlineto{\pgfqpoint{2.616491in}{1.130367in}}%
\pgfpathlineto{\pgfqpoint{2.631005in}{0.571715in}}%
\pgfpathlineto{\pgfqpoint{2.764968in}{0.571604in}}%
\pgfpathlineto{\pgfqpoint{2.764968in}{0.571604in}}%
\pgfusepath{stroke}%
\end{pgfscope}%
\begin{pgfscope}%
\pgfsetrectcap%
\pgfsetmiterjoin%
\pgfsetlinewidth{0.803000pt}%
\definecolor{currentstroke}{rgb}{0.000000,0.000000,0.000000}%
\pgfsetstrokecolor{currentstroke}%
\pgfsetdash{}{0pt}%
\pgfpathmoveto{\pgfqpoint{0.584475in}{0.571604in}}%
\pgfpathlineto{\pgfqpoint{0.584475in}{1.385015in}}%
\pgfusepath{stroke}%
\end{pgfscope}%
\begin{pgfscope}%
\pgfsetrectcap%
\pgfsetmiterjoin%
\pgfsetlinewidth{0.803000pt}%
\definecolor{currentstroke}{rgb}{0.000000,0.000000,0.000000}%
\pgfsetstrokecolor{currentstroke}%
\pgfsetdash{}{0pt}%
\pgfpathmoveto{\pgfqpoint{2.761635in}{0.571604in}}%
\pgfpathlineto{\pgfqpoint{2.761635in}{1.385015in}}%
\pgfusepath{stroke}%
\end{pgfscope}%
\begin{pgfscope}%
\pgfsetrectcap%
\pgfsetmiterjoin%
\pgfsetlinewidth{0.803000pt}%
\definecolor{currentstroke}{rgb}{0.000000,0.000000,0.000000}%
\pgfsetstrokecolor{currentstroke}%
\pgfsetdash{}{0pt}%
\pgfpathmoveto{\pgfqpoint{0.584475in}{0.571604in}}%
\pgfpathlineto{\pgfqpoint{2.761635in}{0.571604in}}%
\pgfusepath{stroke}%
\end{pgfscope}%
\begin{pgfscope}%
\pgfsetrectcap%
\pgfsetmiterjoin%
\pgfsetlinewidth{0.803000pt}%
\definecolor{currentstroke}{rgb}{0.000000,0.000000,0.000000}%
\pgfsetstrokecolor{currentstroke}%
\pgfsetdash{}{0pt}%
\pgfpathmoveto{\pgfqpoint{0.584475in}{1.385015in}}%
\pgfpathlineto{\pgfqpoint{2.761635in}{1.385015in}}%
\pgfusepath{stroke}%
\end{pgfscope}%
\begin{pgfscope}%
\definecolor{textcolor}{rgb}{1.000000,0.000000,0.000000}%
\pgfsetstrokecolor{textcolor}%
\pgfsetfillcolor{textcolor}%
\pgftext[x=1.673055in,y=1.468349in,,base]{\color{textcolor}\sffamily\fontsize{12.000000}{14.400000}\selectfont MSD Metrik}%
\end{pgfscope}%
\end{pgfpicture}%
\makeatother%
\endgroup%
}
  \hfill
  \resizebox{0.32\linewidth}{!}{%% Creator: Matplotlib, PGF backend
%%
%% To include the figure in your LaTeX document, write
%%   \input{<filename>.pgf}
%%
%% Make sure the required packages are loaded in your preamble
%%   \usepackage{pgf}
%%
%% Figures using additional raster images can only be included by \input if
%% they are in the same directory as the main LaTeX file. For loading figures
%% from other directories you can use the `import` package
%%   \usepackage{import}
%% and then include the figures with
%%   \import{<path to file>}{<filename>.pgf}
%%
%% Matplotlib used the following preamble
%%   \usepackage{fontspec}
%%   \setmainfont{DejaVuSerif.ttf}[Path=/usr/share/matplotlib/mpl-data/fonts/ttf/]
%%   \setsansfont{DejaVuSans.ttf}[Path=/usr/share/matplotlib/mpl-data/fonts/ttf/]
%%   \setmonofont{DejaVuSansMono.ttf}[Path=/usr/share/matplotlib/mpl-data/fonts/ttf/]
%%
\begingroup%
\makeatletter%
\begin{pgfpicture}%
\pgfpathrectangle{\pgfpointorigin}{\pgfqpoint{3.000000in}{3.000000in}}%
\pgfusepath{use as bounding box, clip}%
\begin{pgfscope}%
\pgfsetbuttcap%
\pgfsetmiterjoin%
\definecolor{currentfill}{rgb}{1.000000,1.000000,1.000000}%
\pgfsetfillcolor{currentfill}%
\pgfsetlinewidth{0.000000pt}%
\definecolor{currentstroke}{rgb}{1.000000,1.000000,1.000000}%
\pgfsetstrokecolor{currentstroke}%
\pgfsetdash{}{0pt}%
\pgfpathmoveto{\pgfqpoint{0.000000in}{0.000000in}}%
\pgfpathlineto{\pgfqpoint{3.000000in}{0.000000in}}%
\pgfpathlineto{\pgfqpoint{3.000000in}{3.000000in}}%
\pgfpathlineto{\pgfqpoint{0.000000in}{3.000000in}}%
\pgfpathclose%
\pgfusepath{fill}%
\end{pgfscope}%
\begin{pgfscope}%
\pgfsetbuttcap%
\pgfsetmiterjoin%
\definecolor{currentfill}{rgb}{1.000000,1.000000,1.000000}%
\pgfsetfillcolor{currentfill}%
\pgfsetlinewidth{0.000000pt}%
\definecolor{currentstroke}{rgb}{0.000000,0.000000,0.000000}%
\pgfsetstrokecolor{currentstroke}%
\pgfsetstrokeopacity{0.000000}%
\pgfsetdash{}{0pt}%
\pgfpathmoveto{\pgfqpoint{0.584475in}{1.826628in}}%
\pgfpathlineto{\pgfqpoint{2.761635in}{1.826628in}}%
\pgfpathlineto{\pgfqpoint{2.761635in}{2.640039in}}%
\pgfpathlineto{\pgfqpoint{0.584475in}{2.640039in}}%
\pgfpathclose%
\pgfusepath{fill}%
\end{pgfscope}%
\begin{pgfscope}%
\pgfpathrectangle{\pgfqpoint{0.584475in}{1.826628in}}{\pgfqpoint{2.177159in}{0.813411in}}%
\pgfusepath{clip}%
\pgfsetrectcap%
\pgfsetroundjoin%
\pgfsetlinewidth{0.803000pt}%
\definecolor{currentstroke}{rgb}{0.690196,0.690196,0.690196}%
\pgfsetstrokecolor{currentstroke}%
\pgfsetdash{}{0pt}%
\pgfpathmoveto{\pgfqpoint{0.584475in}{1.826628in}}%
\pgfpathlineto{\pgfqpoint{0.584475in}{2.640039in}}%
\pgfusepath{stroke}%
\end{pgfscope}%
\begin{pgfscope}%
\pgfsetbuttcap%
\pgfsetroundjoin%
\definecolor{currentfill}{rgb}{0.000000,0.000000,0.000000}%
\pgfsetfillcolor{currentfill}%
\pgfsetlinewidth{0.803000pt}%
\definecolor{currentstroke}{rgb}{0.000000,0.000000,0.000000}%
\pgfsetstrokecolor{currentstroke}%
\pgfsetdash{}{0pt}%
\pgfsys@defobject{currentmarker}{\pgfqpoint{0.000000in}{-0.048611in}}{\pgfqpoint{0.000000in}{0.000000in}}{%
\pgfpathmoveto{\pgfqpoint{0.000000in}{0.000000in}}%
\pgfpathlineto{\pgfqpoint{0.000000in}{-0.048611in}}%
\pgfusepath{stroke,fill}%
}%
\begin{pgfscope}%
\pgfsys@transformshift{0.584475in}{1.826628in}%
\pgfsys@useobject{currentmarker}{}%
\end{pgfscope}%
\end{pgfscope}%
\begin{pgfscope}%
\pgfpathrectangle{\pgfqpoint{0.584475in}{1.826628in}}{\pgfqpoint{2.177159in}{0.813411in}}%
\pgfusepath{clip}%
\pgfsetrectcap%
\pgfsetroundjoin%
\pgfsetlinewidth{0.803000pt}%
\definecolor{currentstroke}{rgb}{0.690196,0.690196,0.690196}%
\pgfsetstrokecolor{currentstroke}%
\pgfsetdash{}{0pt}%
\pgfpathmoveto{\pgfqpoint{1.310195in}{1.826628in}}%
\pgfpathlineto{\pgfqpoint{1.310195in}{2.640039in}}%
\pgfusepath{stroke}%
\end{pgfscope}%
\begin{pgfscope}%
\pgfsetbuttcap%
\pgfsetroundjoin%
\definecolor{currentfill}{rgb}{0.000000,0.000000,0.000000}%
\pgfsetfillcolor{currentfill}%
\pgfsetlinewidth{0.803000pt}%
\definecolor{currentstroke}{rgb}{0.000000,0.000000,0.000000}%
\pgfsetstrokecolor{currentstroke}%
\pgfsetdash{}{0pt}%
\pgfsys@defobject{currentmarker}{\pgfqpoint{0.000000in}{-0.048611in}}{\pgfqpoint{0.000000in}{0.000000in}}{%
\pgfpathmoveto{\pgfqpoint{0.000000in}{0.000000in}}%
\pgfpathlineto{\pgfqpoint{0.000000in}{-0.048611in}}%
\pgfusepath{stroke,fill}%
}%
\begin{pgfscope}%
\pgfsys@transformshift{1.310195in}{1.826628in}%
\pgfsys@useobject{currentmarker}{}%
\end{pgfscope}%
\end{pgfscope}%
\begin{pgfscope}%
\pgfpathrectangle{\pgfqpoint{0.584475in}{1.826628in}}{\pgfqpoint{2.177159in}{0.813411in}}%
\pgfusepath{clip}%
\pgfsetrectcap%
\pgfsetroundjoin%
\pgfsetlinewidth{0.803000pt}%
\definecolor{currentstroke}{rgb}{0.690196,0.690196,0.690196}%
\pgfsetstrokecolor{currentstroke}%
\pgfsetdash{}{0pt}%
\pgfpathmoveto{\pgfqpoint{2.035915in}{1.826628in}}%
\pgfpathlineto{\pgfqpoint{2.035915in}{2.640039in}}%
\pgfusepath{stroke}%
\end{pgfscope}%
\begin{pgfscope}%
\pgfsetbuttcap%
\pgfsetroundjoin%
\definecolor{currentfill}{rgb}{0.000000,0.000000,0.000000}%
\pgfsetfillcolor{currentfill}%
\pgfsetlinewidth{0.803000pt}%
\definecolor{currentstroke}{rgb}{0.000000,0.000000,0.000000}%
\pgfsetstrokecolor{currentstroke}%
\pgfsetdash{}{0pt}%
\pgfsys@defobject{currentmarker}{\pgfqpoint{0.000000in}{-0.048611in}}{\pgfqpoint{0.000000in}{0.000000in}}{%
\pgfpathmoveto{\pgfqpoint{0.000000in}{0.000000in}}%
\pgfpathlineto{\pgfqpoint{0.000000in}{-0.048611in}}%
\pgfusepath{stroke,fill}%
}%
\begin{pgfscope}%
\pgfsys@transformshift{2.035915in}{1.826628in}%
\pgfsys@useobject{currentmarker}{}%
\end{pgfscope}%
\end{pgfscope}%
\begin{pgfscope}%
\pgfpathrectangle{\pgfqpoint{0.584475in}{1.826628in}}{\pgfqpoint{2.177159in}{0.813411in}}%
\pgfusepath{clip}%
\pgfsetrectcap%
\pgfsetroundjoin%
\pgfsetlinewidth{0.803000pt}%
\definecolor{currentstroke}{rgb}{0.690196,0.690196,0.690196}%
\pgfsetstrokecolor{currentstroke}%
\pgfsetdash{}{0pt}%
\pgfpathmoveto{\pgfqpoint{2.761635in}{1.826628in}}%
\pgfpathlineto{\pgfqpoint{2.761635in}{2.640039in}}%
\pgfusepath{stroke}%
\end{pgfscope}%
\begin{pgfscope}%
\pgfsetbuttcap%
\pgfsetroundjoin%
\definecolor{currentfill}{rgb}{0.000000,0.000000,0.000000}%
\pgfsetfillcolor{currentfill}%
\pgfsetlinewidth{0.803000pt}%
\definecolor{currentstroke}{rgb}{0.000000,0.000000,0.000000}%
\pgfsetstrokecolor{currentstroke}%
\pgfsetdash{}{0pt}%
\pgfsys@defobject{currentmarker}{\pgfqpoint{0.000000in}{-0.048611in}}{\pgfqpoint{0.000000in}{0.000000in}}{%
\pgfpathmoveto{\pgfqpoint{0.000000in}{0.000000in}}%
\pgfpathlineto{\pgfqpoint{0.000000in}{-0.048611in}}%
\pgfusepath{stroke,fill}%
}%
\begin{pgfscope}%
\pgfsys@transformshift{2.761635in}{1.826628in}%
\pgfsys@useobject{currentmarker}{}%
\end{pgfscope}%
\end{pgfscope}%
\begin{pgfscope}%
\pgfpathrectangle{\pgfqpoint{0.584475in}{1.826628in}}{\pgfqpoint{2.177159in}{0.813411in}}%
\pgfusepath{clip}%
\pgfsetrectcap%
\pgfsetroundjoin%
\pgfsetlinewidth{0.803000pt}%
\definecolor{currentstroke}{rgb}{0.690196,0.690196,0.690196}%
\pgfsetstrokecolor{currentstroke}%
\pgfsetdash{}{0pt}%
\pgfpathmoveto{\pgfqpoint{0.584475in}{1.826628in}}%
\pgfpathlineto{\pgfqpoint{2.761635in}{1.826628in}}%
\pgfusepath{stroke}%
\end{pgfscope}%
\begin{pgfscope}%
\pgfsetbuttcap%
\pgfsetroundjoin%
\definecolor{currentfill}{rgb}{0.000000,0.000000,0.000000}%
\pgfsetfillcolor{currentfill}%
\pgfsetlinewidth{0.803000pt}%
\definecolor{currentstroke}{rgb}{0.000000,0.000000,0.000000}%
\pgfsetstrokecolor{currentstroke}%
\pgfsetdash{}{0pt}%
\pgfsys@defobject{currentmarker}{\pgfqpoint{-0.048611in}{0.000000in}}{\pgfqpoint{0.000000in}{0.000000in}}{%
\pgfpathmoveto{\pgfqpoint{0.000000in}{0.000000in}}%
\pgfpathlineto{\pgfqpoint{-0.048611in}{0.000000in}}%
\pgfusepath{stroke,fill}%
}%
\begin{pgfscope}%
\pgfsys@transformshift{0.584475in}{1.826628in}%
\pgfsys@useobject{currentmarker}{}%
\end{pgfscope}%
\end{pgfscope}%
\begin{pgfscope}%
\definecolor{textcolor}{rgb}{0.000000,0.000000,0.000000}%
\pgfsetstrokecolor{textcolor}%
\pgfsetfillcolor{textcolor}%
\pgftext[x=0.150000in,y=1.773866in,left,base]{\color{textcolor}\sffamily\fontsize{10.000000}{12.000000}\selectfont −1.0}%
\end{pgfscope}%
\begin{pgfscope}%
\pgfpathrectangle{\pgfqpoint{0.584475in}{1.826628in}}{\pgfqpoint{2.177159in}{0.813411in}}%
\pgfusepath{clip}%
\pgfsetrectcap%
\pgfsetroundjoin%
\pgfsetlinewidth{0.803000pt}%
\definecolor{currentstroke}{rgb}{0.690196,0.690196,0.690196}%
\pgfsetstrokecolor{currentstroke}%
\pgfsetdash{}{0pt}%
\pgfpathmoveto{\pgfqpoint{0.584475in}{2.233333in}}%
\pgfpathlineto{\pgfqpoint{2.761635in}{2.233333in}}%
\pgfusepath{stroke}%
\end{pgfscope}%
\begin{pgfscope}%
\pgfsetbuttcap%
\pgfsetroundjoin%
\definecolor{currentfill}{rgb}{0.000000,0.000000,0.000000}%
\pgfsetfillcolor{currentfill}%
\pgfsetlinewidth{0.803000pt}%
\definecolor{currentstroke}{rgb}{0.000000,0.000000,0.000000}%
\pgfsetstrokecolor{currentstroke}%
\pgfsetdash{}{0pt}%
\pgfsys@defobject{currentmarker}{\pgfqpoint{-0.048611in}{0.000000in}}{\pgfqpoint{0.000000in}{0.000000in}}{%
\pgfpathmoveto{\pgfqpoint{0.000000in}{0.000000in}}%
\pgfpathlineto{\pgfqpoint{-0.048611in}{0.000000in}}%
\pgfusepath{stroke,fill}%
}%
\begin{pgfscope}%
\pgfsys@transformshift{0.584475in}{2.233333in}%
\pgfsys@useobject{currentmarker}{}%
\end{pgfscope}%
\end{pgfscope}%
\begin{pgfscope}%
\definecolor{textcolor}{rgb}{0.000000,0.000000,0.000000}%
\pgfsetstrokecolor{textcolor}%
\pgfsetfillcolor{textcolor}%
\pgftext[x=0.150000in,y=2.180572in,left,base]{\color{textcolor}\sffamily\fontsize{10.000000}{12.000000}\selectfont −0.5}%
\end{pgfscope}%
\begin{pgfscope}%
\pgfpathrectangle{\pgfqpoint{0.584475in}{1.826628in}}{\pgfqpoint{2.177159in}{0.813411in}}%
\pgfusepath{clip}%
\pgfsetrectcap%
\pgfsetroundjoin%
\pgfsetlinewidth{0.803000pt}%
\definecolor{currentstroke}{rgb}{0.690196,0.690196,0.690196}%
\pgfsetstrokecolor{currentstroke}%
\pgfsetdash{}{0pt}%
\pgfpathmoveto{\pgfqpoint{0.584475in}{2.640039in}}%
\pgfpathlineto{\pgfqpoint{2.761635in}{2.640039in}}%
\pgfusepath{stroke}%
\end{pgfscope}%
\begin{pgfscope}%
\pgfsetbuttcap%
\pgfsetroundjoin%
\definecolor{currentfill}{rgb}{0.000000,0.000000,0.000000}%
\pgfsetfillcolor{currentfill}%
\pgfsetlinewidth{0.803000pt}%
\definecolor{currentstroke}{rgb}{0.000000,0.000000,0.000000}%
\pgfsetstrokecolor{currentstroke}%
\pgfsetdash{}{0pt}%
\pgfsys@defobject{currentmarker}{\pgfqpoint{-0.048611in}{0.000000in}}{\pgfqpoint{0.000000in}{0.000000in}}{%
\pgfpathmoveto{\pgfqpoint{0.000000in}{0.000000in}}%
\pgfpathlineto{\pgfqpoint{-0.048611in}{0.000000in}}%
\pgfusepath{stroke,fill}%
}%
\begin{pgfscope}%
\pgfsys@transformshift{0.584475in}{2.640039in}%
\pgfsys@useobject{currentmarker}{}%
\end{pgfscope}%
\end{pgfscope}%
\begin{pgfscope}%
\definecolor{textcolor}{rgb}{0.000000,0.000000,0.000000}%
\pgfsetstrokecolor{textcolor}%
\pgfsetfillcolor{textcolor}%
\pgftext[x=0.266374in,y=2.587277in,left,base]{\color{textcolor}\sffamily\fontsize{10.000000}{12.000000}\selectfont 0.0}%
\end{pgfscope}%
\begin{pgfscope}%
\pgfpathrectangle{\pgfqpoint{0.584475in}{1.826628in}}{\pgfqpoint{2.177159in}{0.813411in}}%
\pgfusepath{clip}%
\pgfsetrectcap%
\pgfsetroundjoin%
\pgfsetlinewidth{1.505625pt}%
\definecolor{currentstroke}{rgb}{0.000000,0.000000,1.000000}%
\pgfsetstrokecolor{currentstroke}%
\pgfsetdash{}{0pt}%
\pgfpathmoveto{\pgfqpoint{0.584475in}{2.350259in}}%
\pgfpathlineto{\pgfqpoint{0.598990in}{2.507329in}}%
\pgfpathlineto{\pgfqpoint{0.613504in}{2.599105in}}%
\pgfpathlineto{\pgfqpoint{0.628019in}{2.626373in}}%
\pgfpathlineto{\pgfqpoint{0.642533in}{2.609954in}}%
\pgfpathlineto{\pgfqpoint{0.657047in}{2.576337in}}%
\pgfpathlineto{\pgfqpoint{0.671562in}{2.549259in}}%
\pgfpathlineto{\pgfqpoint{0.686076in}{2.550521in}}%
\pgfpathlineto{\pgfqpoint{0.700591in}{2.589721in}}%
\pgfpathlineto{\pgfqpoint{0.715105in}{2.619864in}}%
\pgfpathlineto{\pgfqpoint{0.729619in}{2.633302in}}%
\pgfpathlineto{\pgfqpoint{0.744134in}{2.636657in}}%
\pgfpathlineto{\pgfqpoint{0.758648in}{2.633319in}}%
\pgfpathlineto{\pgfqpoint{0.773163in}{2.632228in}}%
\pgfpathlineto{\pgfqpoint{0.787677in}{2.630343in}}%
\pgfpathlineto{\pgfqpoint{0.802191in}{2.640039in}}%
\pgfpathlineto{\pgfqpoint{1.237623in}{2.640039in}}%
\pgfpathlineto{\pgfqpoint{1.252138in}{2.541512in}}%
\pgfpathlineto{\pgfqpoint{1.266652in}{2.630497in}}%
\pgfpathlineto{\pgfqpoint{1.281166in}{2.632771in}}%
\pgfpathlineto{\pgfqpoint{1.295681in}{2.633869in}}%
\pgfpathlineto{\pgfqpoint{1.310195in}{2.637165in}}%
\pgfpathlineto{\pgfqpoint{1.324710in}{2.631009in}}%
\pgfpathlineto{\pgfqpoint{1.339224in}{2.612737in}}%
\pgfpathlineto{\pgfqpoint{1.353738in}{2.578457in}}%
\pgfpathlineto{\pgfqpoint{1.368253in}{2.542956in}}%
\pgfpathlineto{\pgfqpoint{1.382767in}{2.550915in}}%
\pgfpathlineto{\pgfqpoint{1.397282in}{2.580793in}}%
\pgfpathlineto{\pgfqpoint{1.411796in}{2.613633in}}%
\pgfpathlineto{\pgfqpoint{1.426310in}{2.625456in}}%
\pgfpathlineto{\pgfqpoint{1.440825in}{2.588991in}}%
\pgfpathlineto{\pgfqpoint{1.455339in}{2.486624in}}%
\pgfpathlineto{\pgfqpoint{1.469854in}{2.315345in}}%
\pgfpathlineto{\pgfqpoint{1.484368in}{2.104568in}}%
\pgfpathlineto{\pgfqpoint{1.498882in}{2.382058in}}%
\pgfpathlineto{\pgfqpoint{1.513397in}{2.527129in}}%
\pgfpathlineto{\pgfqpoint{1.527911in}{2.608211in}}%
\pgfpathlineto{\pgfqpoint{1.542425in}{2.625948in}}%
\pgfpathlineto{\pgfqpoint{1.556940in}{2.604820in}}%
\pgfpathlineto{\pgfqpoint{1.571454in}{2.570813in}}%
\pgfpathlineto{\pgfqpoint{1.585969in}{2.546455in}}%
\pgfpathlineto{\pgfqpoint{1.600483in}{2.556379in}}%
\pgfpathlineto{\pgfqpoint{1.614997in}{2.595867in}}%
\pgfpathlineto{\pgfqpoint{1.629512in}{2.623312in}}%
\pgfpathlineto{\pgfqpoint{1.644026in}{2.634540in}}%
\pgfpathlineto{\pgfqpoint{1.658541in}{2.636185in}}%
\pgfpathlineto{\pgfqpoint{1.673055in}{2.632953in}}%
\pgfpathlineto{\pgfqpoint{1.687569in}{2.631922in}}%
\pgfpathlineto{\pgfqpoint{1.702084in}{2.630338in}}%
\pgfpathlineto{\pgfqpoint{1.716598in}{2.640039in}}%
\pgfpathlineto{\pgfqpoint{2.152030in}{2.640039in}}%
\pgfpathlineto{\pgfqpoint{2.166544in}{2.602306in}}%
\pgfpathlineto{\pgfqpoint{2.181059in}{2.631296in}}%
\pgfpathlineto{\pgfqpoint{2.210088in}{2.634196in}}%
\pgfpathlineto{\pgfqpoint{2.224602in}{2.637098in}}%
\pgfpathlineto{\pgfqpoint{2.239116in}{2.628953in}}%
\pgfpathlineto{\pgfqpoint{2.253631in}{2.608024in}}%
\pgfpathlineto{\pgfqpoint{2.268145in}{2.571896in}}%
\pgfpathlineto{\pgfqpoint{2.282660in}{2.541219in}}%
\pgfpathlineto{\pgfqpoint{2.297174in}{2.555151in}}%
\pgfpathlineto{\pgfqpoint{2.326203in}{2.617829in}}%
\pgfpathlineto{\pgfqpoint{2.340717in}{2.623853in}}%
\pgfpathlineto{\pgfqpoint{2.355232in}{2.576118in}}%
\pgfpathlineto{\pgfqpoint{2.369746in}{2.463135in}}%
\pgfpathlineto{\pgfqpoint{2.384260in}{2.277116in}}%
\pgfpathlineto{\pgfqpoint{2.398775in}{2.184413in}}%
\pgfpathlineto{\pgfqpoint{2.413289in}{2.411476in}}%
\pgfpathlineto{\pgfqpoint{2.427804in}{2.545110in}}%
\pgfpathlineto{\pgfqpoint{2.442318in}{2.615527in}}%
\pgfpathlineto{\pgfqpoint{2.456832in}{2.624169in}}%
\pgfpathlineto{\pgfqpoint{2.471347in}{2.599343in}}%
\pgfpathlineto{\pgfqpoint{2.485861in}{2.565820in}}%
\pgfpathlineto{\pgfqpoint{2.500376in}{2.544394in}}%
\pgfpathlineto{\pgfqpoint{2.514890in}{2.563299in}}%
\pgfpathlineto{\pgfqpoint{2.529404in}{2.601619in}}%
\pgfpathlineto{\pgfqpoint{2.543919in}{2.626269in}}%
\pgfpathlineto{\pgfqpoint{2.558433in}{2.635601in}}%
\pgfpathlineto{\pgfqpoint{2.572948in}{2.635252in}}%
\pgfpathlineto{\pgfqpoint{2.587462in}{2.632842in}}%
\pgfpathlineto{\pgfqpoint{2.601976in}{2.631799in}}%
\pgfpathlineto{\pgfqpoint{2.616491in}{2.628579in}}%
\pgfpathlineto{\pgfqpoint{2.631005in}{2.640039in}}%
\pgfpathlineto{\pgfqpoint{2.764968in}{2.640039in}}%
\pgfpathlineto{\pgfqpoint{2.764968in}{2.640039in}}%
\pgfusepath{stroke}%
\end{pgfscope}%
\begin{pgfscope}%
\pgfsetrectcap%
\pgfsetmiterjoin%
\pgfsetlinewidth{0.803000pt}%
\definecolor{currentstroke}{rgb}{0.000000,0.000000,0.000000}%
\pgfsetstrokecolor{currentstroke}%
\pgfsetdash{}{0pt}%
\pgfpathmoveto{\pgfqpoint{0.584475in}{1.826628in}}%
\pgfpathlineto{\pgfqpoint{0.584475in}{2.640039in}}%
\pgfusepath{stroke}%
\end{pgfscope}%
\begin{pgfscope}%
\pgfsetrectcap%
\pgfsetmiterjoin%
\pgfsetlinewidth{0.803000pt}%
\definecolor{currentstroke}{rgb}{0.000000,0.000000,0.000000}%
\pgfsetstrokecolor{currentstroke}%
\pgfsetdash{}{0pt}%
\pgfpathmoveto{\pgfqpoint{2.761635in}{1.826628in}}%
\pgfpathlineto{\pgfqpoint{2.761635in}{2.640039in}}%
\pgfusepath{stroke}%
\end{pgfscope}%
\begin{pgfscope}%
\pgfsetrectcap%
\pgfsetmiterjoin%
\pgfsetlinewidth{0.803000pt}%
\definecolor{currentstroke}{rgb}{0.000000,0.000000,0.000000}%
\pgfsetstrokecolor{currentstroke}%
\pgfsetdash{}{0pt}%
\pgfpathmoveto{\pgfqpoint{0.584475in}{1.826628in}}%
\pgfpathlineto{\pgfqpoint{2.761635in}{1.826628in}}%
\pgfusepath{stroke}%
\end{pgfscope}%
\begin{pgfscope}%
\pgfsetrectcap%
\pgfsetmiterjoin%
\pgfsetlinewidth{0.803000pt}%
\definecolor{currentstroke}{rgb}{0.000000,0.000000,0.000000}%
\pgfsetstrokecolor{currentstroke}%
\pgfsetdash{}{0pt}%
\pgfpathmoveto{\pgfqpoint{0.584475in}{2.640039in}}%
\pgfpathlineto{\pgfqpoint{2.761635in}{2.640039in}}%
\pgfusepath{stroke}%
\end{pgfscope}%
\begin{pgfscope}%
\definecolor{textcolor}{rgb}{0.000000,0.000000,1.000000}%
\pgfsetstrokecolor{textcolor}%
\pgfsetfillcolor{textcolor}%
\pgftext[x=1.673055in,y=2.723372in,,base]{\color{textcolor}\sffamily\fontsize{12.000000}{14.400000}\selectfont MI metric value}%
\end{pgfscope}%
\begin{pgfscope}%
\pgfsetbuttcap%
\pgfsetmiterjoin%
\definecolor{currentfill}{rgb}{1.000000,1.000000,1.000000}%
\pgfsetfillcolor{currentfill}%
\pgfsetlinewidth{0.000000pt}%
\definecolor{currentstroke}{rgb}{0.000000,0.000000,0.000000}%
\pgfsetstrokecolor{currentstroke}%
\pgfsetstrokeopacity{0.000000}%
\pgfsetdash{}{0pt}%
\pgfpathmoveto{\pgfqpoint{0.584475in}{0.571604in}}%
\pgfpathlineto{\pgfqpoint{2.761635in}{0.571604in}}%
\pgfpathlineto{\pgfqpoint{2.761635in}{1.385015in}}%
\pgfpathlineto{\pgfqpoint{0.584475in}{1.385015in}}%
\pgfpathclose%
\pgfusepath{fill}%
\end{pgfscope}%
\begin{pgfscope}%
\pgfpathrectangle{\pgfqpoint{0.584475in}{0.571604in}}{\pgfqpoint{2.177159in}{0.813411in}}%
\pgfusepath{clip}%
\pgfsetrectcap%
\pgfsetroundjoin%
\pgfsetlinewidth{0.803000pt}%
\definecolor{currentstroke}{rgb}{0.690196,0.690196,0.690196}%
\pgfsetstrokecolor{currentstroke}%
\pgfsetdash{}{0pt}%
\pgfpathmoveto{\pgfqpoint{0.584475in}{0.571604in}}%
\pgfpathlineto{\pgfqpoint{0.584475in}{1.385015in}}%
\pgfusepath{stroke}%
\end{pgfscope}%
\begin{pgfscope}%
\pgfsetbuttcap%
\pgfsetroundjoin%
\definecolor{currentfill}{rgb}{0.000000,0.000000,0.000000}%
\pgfsetfillcolor{currentfill}%
\pgfsetlinewidth{0.803000pt}%
\definecolor{currentstroke}{rgb}{0.000000,0.000000,0.000000}%
\pgfsetstrokecolor{currentstroke}%
\pgfsetdash{}{0pt}%
\pgfsys@defobject{currentmarker}{\pgfqpoint{0.000000in}{-0.048611in}}{\pgfqpoint{0.000000in}{0.000000in}}{%
\pgfpathmoveto{\pgfqpoint{0.000000in}{0.000000in}}%
\pgfpathlineto{\pgfqpoint{0.000000in}{-0.048611in}}%
\pgfusepath{stroke,fill}%
}%
\begin{pgfscope}%
\pgfsys@transformshift{0.584475in}{0.571604in}%
\pgfsys@useobject{currentmarker}{}%
\end{pgfscope}%
\end{pgfscope}%
\begin{pgfscope}%
\definecolor{textcolor}{rgb}{0.000000,0.000000,0.000000}%
\pgfsetstrokecolor{textcolor}%
\pgfsetfillcolor{textcolor}%
\pgftext[x=0.584475in,y=0.474382in,,top]{\color{textcolor}\sffamily\fontsize{10.000000}{12.000000}\selectfont 0}%
\end{pgfscope}%
\begin{pgfscope}%
\pgfpathrectangle{\pgfqpoint{0.584475in}{0.571604in}}{\pgfqpoint{2.177159in}{0.813411in}}%
\pgfusepath{clip}%
\pgfsetrectcap%
\pgfsetroundjoin%
\pgfsetlinewidth{0.803000pt}%
\definecolor{currentstroke}{rgb}{0.690196,0.690196,0.690196}%
\pgfsetstrokecolor{currentstroke}%
\pgfsetdash{}{0pt}%
\pgfpathmoveto{\pgfqpoint{1.310195in}{0.571604in}}%
\pgfpathlineto{\pgfqpoint{1.310195in}{1.385015in}}%
\pgfusepath{stroke}%
\end{pgfscope}%
\begin{pgfscope}%
\pgfsetbuttcap%
\pgfsetroundjoin%
\definecolor{currentfill}{rgb}{0.000000,0.000000,0.000000}%
\pgfsetfillcolor{currentfill}%
\pgfsetlinewidth{0.803000pt}%
\definecolor{currentstroke}{rgb}{0.000000,0.000000,0.000000}%
\pgfsetstrokecolor{currentstroke}%
\pgfsetdash{}{0pt}%
\pgfsys@defobject{currentmarker}{\pgfqpoint{0.000000in}{-0.048611in}}{\pgfqpoint{0.000000in}{0.000000in}}{%
\pgfpathmoveto{\pgfqpoint{0.000000in}{0.000000in}}%
\pgfpathlineto{\pgfqpoint{0.000000in}{-0.048611in}}%
\pgfusepath{stroke,fill}%
}%
\begin{pgfscope}%
\pgfsys@transformshift{1.310195in}{0.571604in}%
\pgfsys@useobject{currentmarker}{}%
\end{pgfscope}%
\end{pgfscope}%
\begin{pgfscope}%
\definecolor{textcolor}{rgb}{0.000000,0.000000,0.000000}%
\pgfsetstrokecolor{textcolor}%
\pgfsetfillcolor{textcolor}%
\pgftext[x=1.310195in,y=0.474382in,,top]{\color{textcolor}\sffamily\fontsize{10.000000}{12.000000}\selectfont 5}%
\end{pgfscope}%
\begin{pgfscope}%
\pgfpathrectangle{\pgfqpoint{0.584475in}{0.571604in}}{\pgfqpoint{2.177159in}{0.813411in}}%
\pgfusepath{clip}%
\pgfsetrectcap%
\pgfsetroundjoin%
\pgfsetlinewidth{0.803000pt}%
\definecolor{currentstroke}{rgb}{0.690196,0.690196,0.690196}%
\pgfsetstrokecolor{currentstroke}%
\pgfsetdash{}{0pt}%
\pgfpathmoveto{\pgfqpoint{2.035915in}{0.571604in}}%
\pgfpathlineto{\pgfqpoint{2.035915in}{1.385015in}}%
\pgfusepath{stroke}%
\end{pgfscope}%
\begin{pgfscope}%
\pgfsetbuttcap%
\pgfsetroundjoin%
\definecolor{currentfill}{rgb}{0.000000,0.000000,0.000000}%
\pgfsetfillcolor{currentfill}%
\pgfsetlinewidth{0.803000pt}%
\definecolor{currentstroke}{rgb}{0.000000,0.000000,0.000000}%
\pgfsetstrokecolor{currentstroke}%
\pgfsetdash{}{0pt}%
\pgfsys@defobject{currentmarker}{\pgfqpoint{0.000000in}{-0.048611in}}{\pgfqpoint{0.000000in}{0.000000in}}{%
\pgfpathmoveto{\pgfqpoint{0.000000in}{0.000000in}}%
\pgfpathlineto{\pgfqpoint{0.000000in}{-0.048611in}}%
\pgfusepath{stroke,fill}%
}%
\begin{pgfscope}%
\pgfsys@transformshift{2.035915in}{0.571604in}%
\pgfsys@useobject{currentmarker}{}%
\end{pgfscope}%
\end{pgfscope}%
\begin{pgfscope}%
\definecolor{textcolor}{rgb}{0.000000,0.000000,0.000000}%
\pgfsetstrokecolor{textcolor}%
\pgfsetfillcolor{textcolor}%
\pgftext[x=2.035915in,y=0.474382in,,top]{\color{textcolor}\sffamily\fontsize{10.000000}{12.000000}\selectfont 10}%
\end{pgfscope}%
\begin{pgfscope}%
\pgfpathrectangle{\pgfqpoint{0.584475in}{0.571604in}}{\pgfqpoint{2.177159in}{0.813411in}}%
\pgfusepath{clip}%
\pgfsetrectcap%
\pgfsetroundjoin%
\pgfsetlinewidth{0.803000pt}%
\definecolor{currentstroke}{rgb}{0.690196,0.690196,0.690196}%
\pgfsetstrokecolor{currentstroke}%
\pgfsetdash{}{0pt}%
\pgfpathmoveto{\pgfqpoint{2.761635in}{0.571604in}}%
\pgfpathlineto{\pgfqpoint{2.761635in}{1.385015in}}%
\pgfusepath{stroke}%
\end{pgfscope}%
\begin{pgfscope}%
\pgfsetbuttcap%
\pgfsetroundjoin%
\definecolor{currentfill}{rgb}{0.000000,0.000000,0.000000}%
\pgfsetfillcolor{currentfill}%
\pgfsetlinewidth{0.803000pt}%
\definecolor{currentstroke}{rgb}{0.000000,0.000000,0.000000}%
\pgfsetstrokecolor{currentstroke}%
\pgfsetdash{}{0pt}%
\pgfsys@defobject{currentmarker}{\pgfqpoint{0.000000in}{-0.048611in}}{\pgfqpoint{0.000000in}{0.000000in}}{%
\pgfpathmoveto{\pgfqpoint{0.000000in}{0.000000in}}%
\pgfpathlineto{\pgfqpoint{0.000000in}{-0.048611in}}%
\pgfusepath{stroke,fill}%
}%
\begin{pgfscope}%
\pgfsys@transformshift{2.761635in}{0.571604in}%
\pgfsys@useobject{currentmarker}{}%
\end{pgfscope}%
\end{pgfscope}%
\begin{pgfscope}%
\definecolor{textcolor}{rgb}{0.000000,0.000000,0.000000}%
\pgfsetstrokecolor{textcolor}%
\pgfsetfillcolor{textcolor}%
\pgftext[x=2.761635in,y=0.474382in,,top]{\color{textcolor}\sffamily\fontsize{10.000000}{12.000000}\selectfont 15}%
\end{pgfscope}%
\begin{pgfscope}%
\definecolor{textcolor}{rgb}{0.000000,0.000000,0.000000}%
\pgfsetstrokecolor{textcolor}%
\pgfsetfillcolor{textcolor}%
\pgftext[x=1.673055in,y=0.284413in,,top]{\color{textcolor}\sffamily\fontsize{10.000000}{12.000000}\selectfont Shift of paramter 2}%
\end{pgfscope}%
\begin{pgfscope}%
\pgfpathrectangle{\pgfqpoint{0.584475in}{0.571604in}}{\pgfqpoint{2.177159in}{0.813411in}}%
\pgfusepath{clip}%
\pgfsetrectcap%
\pgfsetroundjoin%
\pgfsetlinewidth{0.803000pt}%
\definecolor{currentstroke}{rgb}{0.690196,0.690196,0.690196}%
\pgfsetstrokecolor{currentstroke}%
\pgfsetdash{}{0pt}%
\pgfpathmoveto{\pgfqpoint{0.584475in}{0.571604in}}%
\pgfpathlineto{\pgfqpoint{2.761635in}{0.571604in}}%
\pgfusepath{stroke}%
\end{pgfscope}%
\begin{pgfscope}%
\pgfsetbuttcap%
\pgfsetroundjoin%
\definecolor{currentfill}{rgb}{0.000000,0.000000,0.000000}%
\pgfsetfillcolor{currentfill}%
\pgfsetlinewidth{0.803000pt}%
\definecolor{currentstroke}{rgb}{0.000000,0.000000,0.000000}%
\pgfsetstrokecolor{currentstroke}%
\pgfsetdash{}{0pt}%
\pgfsys@defobject{currentmarker}{\pgfqpoint{-0.048611in}{0.000000in}}{\pgfqpoint{0.000000in}{0.000000in}}{%
\pgfpathmoveto{\pgfqpoint{0.000000in}{0.000000in}}%
\pgfpathlineto{\pgfqpoint{-0.048611in}{0.000000in}}%
\pgfusepath{stroke,fill}%
}%
\begin{pgfscope}%
\pgfsys@transformshift{0.584475in}{0.571604in}%
\pgfsys@useobject{currentmarker}{}%
\end{pgfscope}%
\end{pgfscope}%
\begin{pgfscope}%
\definecolor{textcolor}{rgb}{0.000000,0.000000,0.000000}%
\pgfsetstrokecolor{textcolor}%
\pgfsetfillcolor{textcolor}%
\pgftext[x=0.266374in,y=0.518842in,left,base]{\color{textcolor}\sffamily\fontsize{10.000000}{12.000000}\selectfont 0.0}%
\end{pgfscope}%
\begin{pgfscope}%
\pgfpathrectangle{\pgfqpoint{0.584475in}{0.571604in}}{\pgfqpoint{2.177159in}{0.813411in}}%
\pgfusepath{clip}%
\pgfsetrectcap%
\pgfsetroundjoin%
\pgfsetlinewidth{0.803000pt}%
\definecolor{currentstroke}{rgb}{0.690196,0.690196,0.690196}%
\pgfsetstrokecolor{currentstroke}%
\pgfsetdash{}{0pt}%
\pgfpathmoveto{\pgfqpoint{0.584475in}{0.978310in}}%
\pgfpathlineto{\pgfqpoint{2.761635in}{0.978310in}}%
\pgfusepath{stroke}%
\end{pgfscope}%
\begin{pgfscope}%
\pgfsetbuttcap%
\pgfsetroundjoin%
\definecolor{currentfill}{rgb}{0.000000,0.000000,0.000000}%
\pgfsetfillcolor{currentfill}%
\pgfsetlinewidth{0.803000pt}%
\definecolor{currentstroke}{rgb}{0.000000,0.000000,0.000000}%
\pgfsetstrokecolor{currentstroke}%
\pgfsetdash{}{0pt}%
\pgfsys@defobject{currentmarker}{\pgfqpoint{-0.048611in}{0.000000in}}{\pgfqpoint{0.000000in}{0.000000in}}{%
\pgfpathmoveto{\pgfqpoint{0.000000in}{0.000000in}}%
\pgfpathlineto{\pgfqpoint{-0.048611in}{0.000000in}}%
\pgfusepath{stroke,fill}%
}%
\begin{pgfscope}%
\pgfsys@transformshift{0.584475in}{0.978310in}%
\pgfsys@useobject{currentmarker}{}%
\end{pgfscope}%
\end{pgfscope}%
\begin{pgfscope}%
\definecolor{textcolor}{rgb}{0.000000,0.000000,0.000000}%
\pgfsetstrokecolor{textcolor}%
\pgfsetfillcolor{textcolor}%
\pgftext[x=0.266374in,y=0.925548in,left,base]{\color{textcolor}\sffamily\fontsize{10.000000}{12.000000}\selectfont 0.5}%
\end{pgfscope}%
\begin{pgfscope}%
\pgfpathrectangle{\pgfqpoint{0.584475in}{0.571604in}}{\pgfqpoint{2.177159in}{0.813411in}}%
\pgfusepath{clip}%
\pgfsetrectcap%
\pgfsetroundjoin%
\pgfsetlinewidth{0.803000pt}%
\definecolor{currentstroke}{rgb}{0.690196,0.690196,0.690196}%
\pgfsetstrokecolor{currentstroke}%
\pgfsetdash{}{0pt}%
\pgfpathmoveto{\pgfqpoint{0.584475in}{1.385015in}}%
\pgfpathlineto{\pgfqpoint{2.761635in}{1.385015in}}%
\pgfusepath{stroke}%
\end{pgfscope}%
\begin{pgfscope}%
\pgfsetbuttcap%
\pgfsetroundjoin%
\definecolor{currentfill}{rgb}{0.000000,0.000000,0.000000}%
\pgfsetfillcolor{currentfill}%
\pgfsetlinewidth{0.803000pt}%
\definecolor{currentstroke}{rgb}{0.000000,0.000000,0.000000}%
\pgfsetstrokecolor{currentstroke}%
\pgfsetdash{}{0pt}%
\pgfsys@defobject{currentmarker}{\pgfqpoint{-0.048611in}{0.000000in}}{\pgfqpoint{0.000000in}{0.000000in}}{%
\pgfpathmoveto{\pgfqpoint{0.000000in}{0.000000in}}%
\pgfpathlineto{\pgfqpoint{-0.048611in}{0.000000in}}%
\pgfusepath{stroke,fill}%
}%
\begin{pgfscope}%
\pgfsys@transformshift{0.584475in}{1.385015in}%
\pgfsys@useobject{currentmarker}{}%
\end{pgfscope}%
\end{pgfscope}%
\begin{pgfscope}%
\definecolor{textcolor}{rgb}{0.000000,0.000000,0.000000}%
\pgfsetstrokecolor{textcolor}%
\pgfsetfillcolor{textcolor}%
\pgftext[x=0.266374in,y=1.332254in,left,base]{\color{textcolor}\sffamily\fontsize{10.000000}{12.000000}\selectfont 1.0}%
\end{pgfscope}%
\begin{pgfscope}%
\definecolor{textcolor}{rgb}{0.000000,0.000000,0.000000}%
\pgfsetstrokecolor{textcolor}%
\pgfsetfillcolor{textcolor}%
\pgftext[x=0.584475in,y=1.426682in,left,base]{\color{textcolor}\sffamily\fontsize{10.000000}{12.000000}\selectfont 1e6}%
\end{pgfscope}%
\begin{pgfscope}%
\pgfpathrectangle{\pgfqpoint{0.584475in}{0.571604in}}{\pgfqpoint{2.177159in}{0.813411in}}%
\pgfusepath{clip}%
\pgfsetrectcap%
\pgfsetroundjoin%
\pgfsetlinewidth{1.505625pt}%
\definecolor{currentstroke}{rgb}{1.000000,0.000000,0.000000}%
\pgfsetstrokecolor{currentstroke}%
\pgfsetdash{}{0pt}%
\pgfpathmoveto{\pgfqpoint{0.584475in}{0.776985in}}%
\pgfpathlineto{\pgfqpoint{0.598990in}{0.916616in}}%
\pgfpathlineto{\pgfqpoint{0.628019in}{1.209972in}}%
\pgfpathlineto{\pgfqpoint{0.642533in}{1.289351in}}%
\pgfpathlineto{\pgfqpoint{0.657047in}{1.333905in}}%
\pgfpathlineto{\pgfqpoint{0.671562in}{1.337722in}}%
\pgfpathlineto{\pgfqpoint{0.686076in}{1.273849in}}%
\pgfpathlineto{\pgfqpoint{0.700591in}{1.139225in}}%
\pgfpathlineto{\pgfqpoint{0.715105in}{0.970992in}}%
\pgfpathlineto{\pgfqpoint{0.729619in}{0.765681in}}%
\pgfpathlineto{\pgfqpoint{0.744134in}{0.579024in}}%
\pgfpathlineto{\pgfqpoint{0.758648in}{0.571639in}}%
\pgfpathlineto{\pgfqpoint{1.295681in}{0.571713in}}%
\pgfpathlineto{\pgfqpoint{1.310195in}{0.596250in}}%
\pgfpathlineto{\pgfqpoint{1.324710in}{0.804252in}}%
\pgfpathlineto{\pgfqpoint{1.339224in}{1.001800in}}%
\pgfpathlineto{\pgfqpoint{1.353738in}{1.164107in}}%
\pgfpathlineto{\pgfqpoint{1.368253in}{1.296998in}}%
\pgfpathlineto{\pgfqpoint{1.382767in}{1.338365in}}%
\pgfpathlineto{\pgfqpoint{1.397282in}{1.328668in}}%
\pgfpathlineto{\pgfqpoint{1.411796in}{1.279807in}}%
\pgfpathlineto{\pgfqpoint{1.426310in}{1.189974in}}%
\pgfpathlineto{\pgfqpoint{1.440825in}{1.037911in}}%
\pgfpathlineto{\pgfqpoint{1.469854in}{0.751628in}}%
\pgfpathlineto{\pgfqpoint{1.484368in}{0.612869in}}%
\pgfpathlineto{\pgfqpoint{1.498882in}{0.800855in}}%
\pgfpathlineto{\pgfqpoint{1.513397in}{0.940326in}}%
\pgfpathlineto{\pgfqpoint{1.527911in}{1.088977in}}%
\pgfpathlineto{\pgfqpoint{1.542425in}{1.226906in}}%
\pgfpathlineto{\pgfqpoint{1.556940in}{1.299035in}}%
\pgfpathlineto{\pgfqpoint{1.571454in}{1.337532in}}%
\pgfpathlineto{\pgfqpoint{1.585969in}{1.337669in}}%
\pgfpathlineto{\pgfqpoint{1.600483in}{1.251810in}}%
\pgfpathlineto{\pgfqpoint{1.614997in}{1.113565in}}%
\pgfpathlineto{\pgfqpoint{1.629512in}{0.939510in}}%
\pgfpathlineto{\pgfqpoint{1.644026in}{0.728923in}}%
\pgfpathlineto{\pgfqpoint{1.658541in}{0.574096in}}%
\pgfpathlineto{\pgfqpoint{1.673055in}{0.571638in}}%
\pgfpathlineto{\pgfqpoint{2.210088in}{0.571943in}}%
\pgfpathlineto{\pgfqpoint{2.224602in}{0.627897in}}%
\pgfpathlineto{\pgfqpoint{2.239116in}{0.840559in}}%
\pgfpathlineto{\pgfqpoint{2.253631in}{1.031744in}}%
\pgfpathlineto{\pgfqpoint{2.268145in}{1.187617in}}%
\pgfpathlineto{\pgfqpoint{2.282660in}{1.316256in}}%
\pgfpathlineto{\pgfqpoint{2.297174in}{1.339537in}}%
\pgfpathlineto{\pgfqpoint{2.311688in}{1.322040in}}%
\pgfpathlineto{\pgfqpoint{2.326203in}{1.269002in}}%
\pgfpathlineto{\pgfqpoint{2.340717in}{1.165570in}}%
\pgfpathlineto{\pgfqpoint{2.355232in}{1.012736in}}%
\pgfpathlineto{\pgfqpoint{2.369746in}{0.869909in}}%
\pgfpathlineto{\pgfqpoint{2.384260in}{0.722945in}}%
\pgfpathlineto{\pgfqpoint{2.398775in}{0.649681in}}%
\pgfpathlineto{\pgfqpoint{2.413289in}{0.824434in}}%
\pgfpathlineto{\pgfqpoint{2.427804in}{0.964539in}}%
\pgfpathlineto{\pgfqpoint{2.442318in}{1.114955in}}%
\pgfpathlineto{\pgfqpoint{2.456832in}{1.242297in}}%
\pgfpathlineto{\pgfqpoint{2.471347in}{1.308158in}}%
\pgfpathlineto{\pgfqpoint{2.485861in}{1.339987in}}%
\pgfpathlineto{\pgfqpoint{2.500376in}{1.335227in}}%
\pgfpathlineto{\pgfqpoint{2.514890in}{1.230270in}}%
\pgfpathlineto{\pgfqpoint{2.529404in}{1.086894in}}%
\pgfpathlineto{\pgfqpoint{2.543919in}{0.906227in}}%
\pgfpathlineto{\pgfqpoint{2.558433in}{0.693960in}}%
\pgfpathlineto{\pgfqpoint{2.572948in}{0.572686in}}%
\pgfpathlineto{\pgfqpoint{2.587462in}{0.571639in}}%
\pgfpathlineto{\pgfqpoint{2.764968in}{0.571604in}}%
\pgfpathlineto{\pgfqpoint{2.764968in}{0.571604in}}%
\pgfusepath{stroke}%
\end{pgfscope}%
\begin{pgfscope}%
\pgfsetrectcap%
\pgfsetmiterjoin%
\pgfsetlinewidth{0.803000pt}%
\definecolor{currentstroke}{rgb}{0.000000,0.000000,0.000000}%
\pgfsetstrokecolor{currentstroke}%
\pgfsetdash{}{0pt}%
\pgfpathmoveto{\pgfqpoint{0.584475in}{0.571604in}}%
\pgfpathlineto{\pgfqpoint{0.584475in}{1.385015in}}%
\pgfusepath{stroke}%
\end{pgfscope}%
\begin{pgfscope}%
\pgfsetrectcap%
\pgfsetmiterjoin%
\pgfsetlinewidth{0.803000pt}%
\definecolor{currentstroke}{rgb}{0.000000,0.000000,0.000000}%
\pgfsetstrokecolor{currentstroke}%
\pgfsetdash{}{0pt}%
\pgfpathmoveto{\pgfqpoint{2.761635in}{0.571604in}}%
\pgfpathlineto{\pgfqpoint{2.761635in}{1.385015in}}%
\pgfusepath{stroke}%
\end{pgfscope}%
\begin{pgfscope}%
\pgfsetrectcap%
\pgfsetmiterjoin%
\pgfsetlinewidth{0.803000pt}%
\definecolor{currentstroke}{rgb}{0.000000,0.000000,0.000000}%
\pgfsetstrokecolor{currentstroke}%
\pgfsetdash{}{0pt}%
\pgfpathmoveto{\pgfqpoint{0.584475in}{0.571604in}}%
\pgfpathlineto{\pgfqpoint{2.761635in}{0.571604in}}%
\pgfusepath{stroke}%
\end{pgfscope}%
\begin{pgfscope}%
\pgfsetrectcap%
\pgfsetmiterjoin%
\pgfsetlinewidth{0.803000pt}%
\definecolor{currentstroke}{rgb}{0.000000,0.000000,0.000000}%
\pgfsetstrokecolor{currentstroke}%
\pgfsetdash{}{0pt}%
\pgfpathmoveto{\pgfqpoint{0.584475in}{1.385015in}}%
\pgfpathlineto{\pgfqpoint{2.761635in}{1.385015in}}%
\pgfusepath{stroke}%
\end{pgfscope}%
\begin{pgfscope}%
\definecolor{textcolor}{rgb}{1.000000,0.000000,0.000000}%
\pgfsetstrokecolor{textcolor}%
\pgfsetfillcolor{textcolor}%
\pgftext[x=1.673055in,y=1.468349in,,base]{\color{textcolor}\sffamily\fontsize{12.000000}{14.400000}\selectfont MSD metric value}%
\end{pgfscope}%
\end{pgfpicture}%
\makeatother%
\endgroup%
}
  \vspace{-10pt}
\end{figure}
\newpage
In Abbildung \ref{fig:ct2ct_translparam} sind die Variationen der drei
Translationsparameter dargestellt. Die Parameter \num{3} und \num{4} sind die
Translationsparameter für die x- und y-Achse, während Parameter \num{5} der
Translationsparameter für die z-Achse ist. Dies erkennt man daran, dass die
z-Achse kleine Plateaus aufweist, die auf die geringere Auflösung im Vergleich
zur x- bzw. y-Achse zurückzuführen sind. Hier bleibt eine kleine Verschiebung
der Metrik im gleichen Voxel.
\begin{figure}[h]
  \caption{Metrik in Abhängigkeit der Translationsparameter}
  \label{fig:ct2ct_translparam}
  \vspace{-10pt}
  \resizebox{0.32\linewidth}{!}{%% Creator: Matplotlib, PGF backend
%%
%% To include the figure in your LaTeX document, write
%%   \input{<filename>.pgf}
%%
%% Make sure the required packages are loaded in your preamble
%%   \usepackage{pgf}
%%
%% Figures using additional raster images can only be included by \input if
%% they are in the same directory as the main LaTeX file. For loading figures
%% from other directories you can use the `import` package
%%   \usepackage{import}
%% and then include the figures with
%%   \import{<path to file>}{<filename>.pgf}
%%
%% Matplotlib used the following preamble
%%   \usepackage{fontspec}
%%   \setmainfont{DejaVuSerif.ttf}[Path=/usr/share/matplotlib/mpl-data/fonts/ttf/]
%%   \setsansfont{DejaVuSans.ttf}[Path=/usr/share/matplotlib/mpl-data/fonts/ttf/]
%%   \setmonofont{DejaVuSansMono.ttf}[Path=/usr/share/matplotlib/mpl-data/fonts/ttf/]
%%
\begingroup%
\makeatletter%
\begin{pgfpicture}%
\pgfpathrectangle{\pgfpointorigin}{\pgfqpoint{3.000000in}{3.000000in}}%
\pgfusepath{use as bounding box, clip}%
\begin{pgfscope}%
\pgfsetbuttcap%
\pgfsetmiterjoin%
\definecolor{currentfill}{rgb}{1.000000,1.000000,1.000000}%
\pgfsetfillcolor{currentfill}%
\pgfsetlinewidth{0.000000pt}%
\definecolor{currentstroke}{rgb}{1.000000,1.000000,1.000000}%
\pgfsetstrokecolor{currentstroke}%
\pgfsetdash{}{0pt}%
\pgfpathmoveto{\pgfqpoint{0.000000in}{0.000000in}}%
\pgfpathlineto{\pgfqpoint{3.000000in}{0.000000in}}%
\pgfpathlineto{\pgfqpoint{3.000000in}{3.000000in}}%
\pgfpathlineto{\pgfqpoint{0.000000in}{3.000000in}}%
\pgfpathclose%
\pgfusepath{fill}%
\end{pgfscope}%
\begin{pgfscope}%
\pgfsetbuttcap%
\pgfsetmiterjoin%
\definecolor{currentfill}{rgb}{1.000000,1.000000,1.000000}%
\pgfsetfillcolor{currentfill}%
\pgfsetlinewidth{0.000000pt}%
\definecolor{currentstroke}{rgb}{0.000000,0.000000,0.000000}%
\pgfsetstrokecolor{currentstroke}%
\pgfsetstrokeopacity{0.000000}%
\pgfsetdash{}{0pt}%
\pgfpathmoveto{\pgfqpoint{0.584475in}{1.826628in}}%
\pgfpathlineto{\pgfqpoint{2.850000in}{1.826628in}}%
\pgfpathlineto{\pgfqpoint{2.850000in}{2.640039in}}%
\pgfpathlineto{\pgfqpoint{0.584475in}{2.640039in}}%
\pgfpathclose%
\pgfusepath{fill}%
\end{pgfscope}%
\begin{pgfscope}%
\pgfpathrectangle{\pgfqpoint{0.584475in}{1.826628in}}{\pgfqpoint{2.265525in}{0.813411in}}%
\pgfusepath{clip}%
\pgfsetrectcap%
\pgfsetroundjoin%
\pgfsetlinewidth{0.803000pt}%
\definecolor{currentstroke}{rgb}{0.690196,0.690196,0.690196}%
\pgfsetstrokecolor{currentstroke}%
\pgfsetdash{}{0pt}%
\pgfpathmoveto{\pgfqpoint{0.811028in}{1.826628in}}%
\pgfpathlineto{\pgfqpoint{0.811028in}{2.640039in}}%
\pgfusepath{stroke}%
\end{pgfscope}%
\begin{pgfscope}%
\pgfsetbuttcap%
\pgfsetroundjoin%
\definecolor{currentfill}{rgb}{0.000000,0.000000,0.000000}%
\pgfsetfillcolor{currentfill}%
\pgfsetlinewidth{0.803000pt}%
\definecolor{currentstroke}{rgb}{0.000000,0.000000,0.000000}%
\pgfsetstrokecolor{currentstroke}%
\pgfsetdash{}{0pt}%
\pgfsys@defobject{currentmarker}{\pgfqpoint{0.000000in}{-0.048611in}}{\pgfqpoint{0.000000in}{0.000000in}}{%
\pgfpathmoveto{\pgfqpoint{0.000000in}{0.000000in}}%
\pgfpathlineto{\pgfqpoint{0.000000in}{-0.048611in}}%
\pgfusepath{stroke,fill}%
}%
\begin{pgfscope}%
\pgfsys@transformshift{0.811028in}{1.826628in}%
\pgfsys@useobject{currentmarker}{}%
\end{pgfscope}%
\end{pgfscope}%
\begin{pgfscope}%
\pgfpathrectangle{\pgfqpoint{0.584475in}{1.826628in}}{\pgfqpoint{2.265525in}{0.813411in}}%
\pgfusepath{clip}%
\pgfsetrectcap%
\pgfsetroundjoin%
\pgfsetlinewidth{0.803000pt}%
\definecolor{currentstroke}{rgb}{0.690196,0.690196,0.690196}%
\pgfsetstrokecolor{currentstroke}%
\pgfsetdash{}{0pt}%
\pgfpathmoveto{\pgfqpoint{1.264133in}{1.826628in}}%
\pgfpathlineto{\pgfqpoint{1.264133in}{2.640039in}}%
\pgfusepath{stroke}%
\end{pgfscope}%
\begin{pgfscope}%
\pgfsetbuttcap%
\pgfsetroundjoin%
\definecolor{currentfill}{rgb}{0.000000,0.000000,0.000000}%
\pgfsetfillcolor{currentfill}%
\pgfsetlinewidth{0.803000pt}%
\definecolor{currentstroke}{rgb}{0.000000,0.000000,0.000000}%
\pgfsetstrokecolor{currentstroke}%
\pgfsetdash{}{0pt}%
\pgfsys@defobject{currentmarker}{\pgfqpoint{0.000000in}{-0.048611in}}{\pgfqpoint{0.000000in}{0.000000in}}{%
\pgfpathmoveto{\pgfqpoint{0.000000in}{0.000000in}}%
\pgfpathlineto{\pgfqpoint{0.000000in}{-0.048611in}}%
\pgfusepath{stroke,fill}%
}%
\begin{pgfscope}%
\pgfsys@transformshift{1.264133in}{1.826628in}%
\pgfsys@useobject{currentmarker}{}%
\end{pgfscope}%
\end{pgfscope}%
\begin{pgfscope}%
\pgfpathrectangle{\pgfqpoint{0.584475in}{1.826628in}}{\pgfqpoint{2.265525in}{0.813411in}}%
\pgfusepath{clip}%
\pgfsetrectcap%
\pgfsetroundjoin%
\pgfsetlinewidth{0.803000pt}%
\definecolor{currentstroke}{rgb}{0.690196,0.690196,0.690196}%
\pgfsetstrokecolor{currentstroke}%
\pgfsetdash{}{0pt}%
\pgfpathmoveto{\pgfqpoint{1.717238in}{1.826628in}}%
\pgfpathlineto{\pgfqpoint{1.717238in}{2.640039in}}%
\pgfusepath{stroke}%
\end{pgfscope}%
\begin{pgfscope}%
\pgfsetbuttcap%
\pgfsetroundjoin%
\definecolor{currentfill}{rgb}{0.000000,0.000000,0.000000}%
\pgfsetfillcolor{currentfill}%
\pgfsetlinewidth{0.803000pt}%
\definecolor{currentstroke}{rgb}{0.000000,0.000000,0.000000}%
\pgfsetstrokecolor{currentstroke}%
\pgfsetdash{}{0pt}%
\pgfsys@defobject{currentmarker}{\pgfqpoint{0.000000in}{-0.048611in}}{\pgfqpoint{0.000000in}{0.000000in}}{%
\pgfpathmoveto{\pgfqpoint{0.000000in}{0.000000in}}%
\pgfpathlineto{\pgfqpoint{0.000000in}{-0.048611in}}%
\pgfusepath{stroke,fill}%
}%
\begin{pgfscope}%
\pgfsys@transformshift{1.717238in}{1.826628in}%
\pgfsys@useobject{currentmarker}{}%
\end{pgfscope}%
\end{pgfscope}%
\begin{pgfscope}%
\pgfpathrectangle{\pgfqpoint{0.584475in}{1.826628in}}{\pgfqpoint{2.265525in}{0.813411in}}%
\pgfusepath{clip}%
\pgfsetrectcap%
\pgfsetroundjoin%
\pgfsetlinewidth{0.803000pt}%
\definecolor{currentstroke}{rgb}{0.690196,0.690196,0.690196}%
\pgfsetstrokecolor{currentstroke}%
\pgfsetdash{}{0pt}%
\pgfpathmoveto{\pgfqpoint{2.170343in}{1.826628in}}%
\pgfpathlineto{\pgfqpoint{2.170343in}{2.640039in}}%
\pgfusepath{stroke}%
\end{pgfscope}%
\begin{pgfscope}%
\pgfsetbuttcap%
\pgfsetroundjoin%
\definecolor{currentfill}{rgb}{0.000000,0.000000,0.000000}%
\pgfsetfillcolor{currentfill}%
\pgfsetlinewidth{0.803000pt}%
\definecolor{currentstroke}{rgb}{0.000000,0.000000,0.000000}%
\pgfsetstrokecolor{currentstroke}%
\pgfsetdash{}{0pt}%
\pgfsys@defobject{currentmarker}{\pgfqpoint{0.000000in}{-0.048611in}}{\pgfqpoint{0.000000in}{0.000000in}}{%
\pgfpathmoveto{\pgfqpoint{0.000000in}{0.000000in}}%
\pgfpathlineto{\pgfqpoint{0.000000in}{-0.048611in}}%
\pgfusepath{stroke,fill}%
}%
\begin{pgfscope}%
\pgfsys@transformshift{2.170343in}{1.826628in}%
\pgfsys@useobject{currentmarker}{}%
\end{pgfscope}%
\end{pgfscope}%
\begin{pgfscope}%
\pgfpathrectangle{\pgfqpoint{0.584475in}{1.826628in}}{\pgfqpoint{2.265525in}{0.813411in}}%
\pgfusepath{clip}%
\pgfsetrectcap%
\pgfsetroundjoin%
\pgfsetlinewidth{0.803000pt}%
\definecolor{currentstroke}{rgb}{0.690196,0.690196,0.690196}%
\pgfsetstrokecolor{currentstroke}%
\pgfsetdash{}{0pt}%
\pgfpathmoveto{\pgfqpoint{2.623448in}{1.826628in}}%
\pgfpathlineto{\pgfqpoint{2.623448in}{2.640039in}}%
\pgfusepath{stroke}%
\end{pgfscope}%
\begin{pgfscope}%
\pgfsetbuttcap%
\pgfsetroundjoin%
\definecolor{currentfill}{rgb}{0.000000,0.000000,0.000000}%
\pgfsetfillcolor{currentfill}%
\pgfsetlinewidth{0.803000pt}%
\definecolor{currentstroke}{rgb}{0.000000,0.000000,0.000000}%
\pgfsetstrokecolor{currentstroke}%
\pgfsetdash{}{0pt}%
\pgfsys@defobject{currentmarker}{\pgfqpoint{0.000000in}{-0.048611in}}{\pgfqpoint{0.000000in}{0.000000in}}{%
\pgfpathmoveto{\pgfqpoint{0.000000in}{0.000000in}}%
\pgfpathlineto{\pgfqpoint{0.000000in}{-0.048611in}}%
\pgfusepath{stroke,fill}%
}%
\begin{pgfscope}%
\pgfsys@transformshift{2.623448in}{1.826628in}%
\pgfsys@useobject{currentmarker}{}%
\end{pgfscope}%
\end{pgfscope}%
\begin{pgfscope}%
\pgfpathrectangle{\pgfqpoint{0.584475in}{1.826628in}}{\pgfqpoint{2.265525in}{0.813411in}}%
\pgfusepath{clip}%
\pgfsetrectcap%
\pgfsetroundjoin%
\pgfsetlinewidth{0.803000pt}%
\definecolor{currentstroke}{rgb}{0.690196,0.690196,0.690196}%
\pgfsetstrokecolor{currentstroke}%
\pgfsetdash{}{0pt}%
\pgfpathmoveto{\pgfqpoint{0.584475in}{2.097765in}}%
\pgfpathlineto{\pgfqpoint{2.850000in}{2.097765in}}%
\pgfusepath{stroke}%
\end{pgfscope}%
\begin{pgfscope}%
\pgfsetbuttcap%
\pgfsetroundjoin%
\definecolor{currentfill}{rgb}{0.000000,0.000000,0.000000}%
\pgfsetfillcolor{currentfill}%
\pgfsetlinewidth{0.803000pt}%
\definecolor{currentstroke}{rgb}{0.000000,0.000000,0.000000}%
\pgfsetstrokecolor{currentstroke}%
\pgfsetdash{}{0pt}%
\pgfsys@defobject{currentmarker}{\pgfqpoint{-0.048611in}{0.000000in}}{\pgfqpoint{0.000000in}{0.000000in}}{%
\pgfpathmoveto{\pgfqpoint{0.000000in}{0.000000in}}%
\pgfpathlineto{\pgfqpoint{-0.048611in}{0.000000in}}%
\pgfusepath{stroke,fill}%
}%
\begin{pgfscope}%
\pgfsys@transformshift{0.584475in}{2.097765in}%
\pgfsys@useobject{currentmarker}{}%
\end{pgfscope}%
\end{pgfscope}%
\begin{pgfscope}%
\definecolor{textcolor}{rgb}{0.000000,0.000000,0.000000}%
\pgfsetstrokecolor{textcolor}%
\pgfsetfillcolor{textcolor}%
\pgftext[x=0.282514in,y=2.045003in,left,base]{\color{textcolor}\sffamily\fontsize{10.000000}{12.000000}\selectfont −1}%
\end{pgfscope}%
\begin{pgfscope}%
\pgfpathrectangle{\pgfqpoint{0.584475in}{1.826628in}}{\pgfqpoint{2.265525in}{0.813411in}}%
\pgfusepath{clip}%
\pgfsetrectcap%
\pgfsetroundjoin%
\pgfsetlinewidth{0.803000pt}%
\definecolor{currentstroke}{rgb}{0.690196,0.690196,0.690196}%
\pgfsetstrokecolor{currentstroke}%
\pgfsetdash{}{0pt}%
\pgfpathmoveto{\pgfqpoint{0.584475in}{2.640039in}}%
\pgfpathlineto{\pgfqpoint{2.850000in}{2.640039in}}%
\pgfusepath{stroke}%
\end{pgfscope}%
\begin{pgfscope}%
\pgfsetbuttcap%
\pgfsetroundjoin%
\definecolor{currentfill}{rgb}{0.000000,0.000000,0.000000}%
\pgfsetfillcolor{currentfill}%
\pgfsetlinewidth{0.803000pt}%
\definecolor{currentstroke}{rgb}{0.000000,0.000000,0.000000}%
\pgfsetstrokecolor{currentstroke}%
\pgfsetdash{}{0pt}%
\pgfsys@defobject{currentmarker}{\pgfqpoint{-0.048611in}{0.000000in}}{\pgfqpoint{0.000000in}{0.000000in}}{%
\pgfpathmoveto{\pgfqpoint{0.000000in}{0.000000in}}%
\pgfpathlineto{\pgfqpoint{-0.048611in}{0.000000in}}%
\pgfusepath{stroke,fill}%
}%
\begin{pgfscope}%
\pgfsys@transformshift{0.584475in}{2.640039in}%
\pgfsys@useobject{currentmarker}{}%
\end{pgfscope}%
\end{pgfscope}%
\begin{pgfscope}%
\definecolor{textcolor}{rgb}{0.000000,0.000000,0.000000}%
\pgfsetstrokecolor{textcolor}%
\pgfsetfillcolor{textcolor}%
\pgftext[x=0.398888in,y=2.587277in,left,base]{\color{textcolor}\sffamily\fontsize{10.000000}{12.000000}\selectfont 0}%
\end{pgfscope}%
\begin{pgfscope}%
\pgfpathrectangle{\pgfqpoint{0.584475in}{1.826628in}}{\pgfqpoint{2.265525in}{0.813411in}}%
\pgfusepath{clip}%
\pgfsetrectcap%
\pgfsetroundjoin%
\pgfsetlinewidth{1.505625pt}%
\definecolor{currentstroke}{rgb}{0.000000,0.000000,1.000000}%
\pgfsetstrokecolor{currentstroke}%
\pgfsetdash{}{0pt}%
\pgfpathmoveto{\pgfqpoint{0.584475in}{2.468668in}}%
\pgfpathlineto{\pgfqpoint{0.607131in}{2.465943in}}%
\pgfpathlineto{\pgfqpoint{0.629786in}{2.463363in}}%
\pgfpathlineto{\pgfqpoint{0.652441in}{2.460804in}}%
\pgfpathlineto{\pgfqpoint{0.675096in}{2.457517in}}%
\pgfpathlineto{\pgfqpoint{0.697752in}{2.454571in}}%
\pgfpathlineto{\pgfqpoint{0.720407in}{2.451773in}}%
\pgfpathlineto{\pgfqpoint{0.743062in}{2.449181in}}%
\pgfpathlineto{\pgfqpoint{0.765717in}{2.445816in}}%
\pgfpathlineto{\pgfqpoint{0.788373in}{2.442602in}}%
\pgfpathlineto{\pgfqpoint{0.811028in}{2.439565in}}%
\pgfpathlineto{\pgfqpoint{0.833683in}{2.436910in}}%
\pgfpathlineto{\pgfqpoint{0.856338in}{2.433540in}}%
\pgfpathlineto{\pgfqpoint{0.878994in}{2.430063in}}%
\pgfpathlineto{\pgfqpoint{0.901649in}{2.426767in}}%
\pgfpathlineto{\pgfqpoint{0.924304in}{2.423912in}}%
\pgfpathlineto{\pgfqpoint{0.946959in}{2.420621in}}%
\pgfpathlineto{\pgfqpoint{0.969615in}{2.416760in}}%
\pgfpathlineto{\pgfqpoint{0.992270in}{2.413377in}}%
\pgfpathlineto{\pgfqpoint{1.014925in}{2.410061in}}%
\pgfpathlineto{\pgfqpoint{1.037580in}{2.406765in}}%
\pgfpathlineto{\pgfqpoint{1.060236in}{2.402627in}}%
\pgfpathlineto{\pgfqpoint{1.082891in}{2.398896in}}%
\pgfpathlineto{\pgfqpoint{1.105546in}{2.395322in}}%
\pgfpathlineto{\pgfqpoint{1.128201in}{2.392042in}}%
\pgfpathlineto{\pgfqpoint{1.150857in}{2.387672in}}%
\pgfpathlineto{\pgfqpoint{1.173512in}{2.383604in}}%
\pgfpathlineto{\pgfqpoint{1.196167in}{2.379695in}}%
\pgfpathlineto{\pgfqpoint{1.218822in}{2.376174in}}%
\pgfpathlineto{\pgfqpoint{1.241478in}{2.371502in}}%
\pgfpathlineto{\pgfqpoint{1.264133in}{2.366747in}}%
\pgfpathlineto{\pgfqpoint{1.286788in}{2.362265in}}%
\pgfpathlineto{\pgfqpoint{1.309443in}{2.358162in}}%
\pgfpathlineto{\pgfqpoint{1.332099in}{2.353083in}}%
\pgfpathlineto{\pgfqpoint{1.354754in}{2.347084in}}%
\pgfpathlineto{\pgfqpoint{1.377409in}{2.341896in}}%
\pgfpathlineto{\pgfqpoint{1.400064in}{2.336557in}}%
\pgfpathlineto{\pgfqpoint{1.422720in}{2.330699in}}%
\pgfpathlineto{\pgfqpoint{1.445375in}{2.323225in}}%
\pgfpathlineto{\pgfqpoint{1.468030in}{2.317000in}}%
\pgfpathlineto{\pgfqpoint{1.490685in}{2.310199in}}%
\pgfpathlineto{\pgfqpoint{1.513340in}{2.302671in}}%
\pgfpathlineto{\pgfqpoint{1.535996in}{2.292073in}}%
\pgfpathlineto{\pgfqpoint{1.558651in}{2.283274in}}%
\pgfpathlineto{\pgfqpoint{1.581306in}{2.272295in}}%
\pgfpathlineto{\pgfqpoint{1.603961in}{2.259491in}}%
\pgfpathlineto{\pgfqpoint{1.626617in}{2.241631in}}%
\pgfpathlineto{\pgfqpoint{1.649272in}{2.222358in}}%
\pgfpathlineto{\pgfqpoint{1.671927in}{2.186803in}}%
\pgfpathlineto{\pgfqpoint{1.694582in}{2.122989in}}%
\pgfpathlineto{\pgfqpoint{1.717238in}{1.926948in}}%
\pgfpathlineto{\pgfqpoint{1.739893in}{2.122932in}}%
\pgfpathlineto{\pgfqpoint{1.762548in}{2.186696in}}%
\pgfpathlineto{\pgfqpoint{1.785203in}{2.222157in}}%
\pgfpathlineto{\pgfqpoint{1.807859in}{2.241516in}}%
\pgfpathlineto{\pgfqpoint{1.830514in}{2.259321in}}%
\pgfpathlineto{\pgfqpoint{1.853169in}{2.272036in}}%
\pgfpathlineto{\pgfqpoint{1.875824in}{2.283021in}}%
\pgfpathlineto{\pgfqpoint{1.898480in}{2.291837in}}%
\pgfpathlineto{\pgfqpoint{1.921135in}{2.302358in}}%
\pgfpathlineto{\pgfqpoint{1.943790in}{2.309853in}}%
\pgfpathlineto{\pgfqpoint{1.966445in}{2.316693in}}%
\pgfpathlineto{\pgfqpoint{1.989101in}{2.322957in}}%
\pgfpathlineto{\pgfqpoint{2.011756in}{2.330434in}}%
\pgfpathlineto{\pgfqpoint{2.034411in}{2.336306in}}%
\pgfpathlineto{\pgfqpoint{2.057066in}{2.341675in}}%
\pgfpathlineto{\pgfqpoint{2.079722in}{2.346876in}}%
\pgfpathlineto{\pgfqpoint{2.102377in}{2.352908in}}%
\pgfpathlineto{\pgfqpoint{2.125032in}{2.357939in}}%
\pgfpathlineto{\pgfqpoint{2.147687in}{2.362091in}}%
\pgfpathlineto{\pgfqpoint{2.170343in}{2.366593in}}%
\pgfpathlineto{\pgfqpoint{2.192998in}{2.371406in}}%
\pgfpathlineto{\pgfqpoint{2.215653in}{2.376013in}}%
\pgfpathlineto{\pgfqpoint{2.238308in}{2.379594in}}%
\pgfpathlineto{\pgfqpoint{2.260964in}{2.383553in}}%
\pgfpathlineto{\pgfqpoint{2.283619in}{2.387694in}}%
\pgfpathlineto{\pgfqpoint{2.306274in}{2.392053in}}%
\pgfpathlineto{\pgfqpoint{2.328929in}{2.395356in}}%
\pgfpathlineto{\pgfqpoint{2.351585in}{2.398946in}}%
\pgfpathlineto{\pgfqpoint{2.374240in}{2.402733in}}%
\pgfpathlineto{\pgfqpoint{2.396895in}{2.406883in}}%
\pgfpathlineto{\pgfqpoint{2.419550in}{2.410151in}}%
\pgfpathlineto{\pgfqpoint{2.442206in}{2.413456in}}%
\pgfpathlineto{\pgfqpoint{2.464861in}{2.416940in}}%
\pgfpathlineto{\pgfqpoint{2.487516in}{2.420829in}}%
\pgfpathlineto{\pgfqpoint{2.510171in}{2.424096in}}%
\pgfpathlineto{\pgfqpoint{2.532827in}{2.426958in}}%
\pgfpathlineto{\pgfqpoint{2.555482in}{2.430265in}}%
\pgfpathlineto{\pgfqpoint{2.578137in}{2.433789in}}%
\pgfpathlineto{\pgfqpoint{2.600792in}{2.437113in}}%
\pgfpathlineto{\pgfqpoint{2.623448in}{2.439788in}}%
\pgfpathlineto{\pgfqpoint{2.646103in}{2.442829in}}%
\pgfpathlineto{\pgfqpoint{2.668758in}{2.446090in}}%
\pgfpathlineto{\pgfqpoint{2.691413in}{2.449436in}}%
\pgfpathlineto{\pgfqpoint{2.714069in}{2.451986in}}%
\pgfpathlineto{\pgfqpoint{2.736724in}{2.454799in}}%
\pgfpathlineto{\pgfqpoint{2.759379in}{2.457774in}}%
\pgfpathlineto{\pgfqpoint{2.782034in}{2.461073in}}%
\pgfpathlineto{\pgfqpoint{2.804690in}{2.463611in}}%
\pgfpathlineto{\pgfqpoint{2.827345in}{2.466188in}}%
\pgfpathlineto{\pgfqpoint{2.850000in}{2.468938in}}%
\pgfusepath{stroke}%
\end{pgfscope}%
\begin{pgfscope}%
\pgfsetrectcap%
\pgfsetmiterjoin%
\pgfsetlinewidth{0.803000pt}%
\definecolor{currentstroke}{rgb}{0.000000,0.000000,0.000000}%
\pgfsetstrokecolor{currentstroke}%
\pgfsetdash{}{0pt}%
\pgfpathmoveto{\pgfqpoint{0.584475in}{1.826628in}}%
\pgfpathlineto{\pgfqpoint{0.584475in}{2.640039in}}%
\pgfusepath{stroke}%
\end{pgfscope}%
\begin{pgfscope}%
\pgfsetrectcap%
\pgfsetmiterjoin%
\pgfsetlinewidth{0.803000pt}%
\definecolor{currentstroke}{rgb}{0.000000,0.000000,0.000000}%
\pgfsetstrokecolor{currentstroke}%
\pgfsetdash{}{0pt}%
\pgfpathmoveto{\pgfqpoint{2.850000in}{1.826628in}}%
\pgfpathlineto{\pgfqpoint{2.850000in}{2.640039in}}%
\pgfusepath{stroke}%
\end{pgfscope}%
\begin{pgfscope}%
\pgfsetrectcap%
\pgfsetmiterjoin%
\pgfsetlinewidth{0.803000pt}%
\definecolor{currentstroke}{rgb}{0.000000,0.000000,0.000000}%
\pgfsetstrokecolor{currentstroke}%
\pgfsetdash{}{0pt}%
\pgfpathmoveto{\pgfqpoint{0.584475in}{1.826628in}}%
\pgfpathlineto{\pgfqpoint{2.850000in}{1.826628in}}%
\pgfusepath{stroke}%
\end{pgfscope}%
\begin{pgfscope}%
\pgfsetrectcap%
\pgfsetmiterjoin%
\pgfsetlinewidth{0.803000pt}%
\definecolor{currentstroke}{rgb}{0.000000,0.000000,0.000000}%
\pgfsetstrokecolor{currentstroke}%
\pgfsetdash{}{0pt}%
\pgfpathmoveto{\pgfqpoint{0.584475in}{2.640039in}}%
\pgfpathlineto{\pgfqpoint{2.850000in}{2.640039in}}%
\pgfusepath{stroke}%
\end{pgfscope}%
\begin{pgfscope}%
\definecolor{textcolor}{rgb}{0.000000,0.000000,1.000000}%
\pgfsetstrokecolor{textcolor}%
\pgfsetfillcolor{textcolor}%
\pgftext[x=1.717238in,y=2.723372in,,base]{\color{textcolor}\sffamily\fontsize{12.000000}{14.400000}\selectfont MI metric value}%
\end{pgfscope}%
\begin{pgfscope}%
\pgfsetbuttcap%
\pgfsetmiterjoin%
\definecolor{currentfill}{rgb}{1.000000,1.000000,1.000000}%
\pgfsetfillcolor{currentfill}%
\pgfsetlinewidth{0.000000pt}%
\definecolor{currentstroke}{rgb}{0.000000,0.000000,0.000000}%
\pgfsetstrokecolor{currentstroke}%
\pgfsetstrokeopacity{0.000000}%
\pgfsetdash{}{0pt}%
\pgfpathmoveto{\pgfqpoint{0.584475in}{0.571604in}}%
\pgfpathlineto{\pgfqpoint{2.850000in}{0.571604in}}%
\pgfpathlineto{\pgfqpoint{2.850000in}{1.385015in}}%
\pgfpathlineto{\pgfqpoint{0.584475in}{1.385015in}}%
\pgfpathclose%
\pgfusepath{fill}%
\end{pgfscope}%
\begin{pgfscope}%
\pgfpathrectangle{\pgfqpoint{0.584475in}{0.571604in}}{\pgfqpoint{2.265525in}{0.813411in}}%
\pgfusepath{clip}%
\pgfsetrectcap%
\pgfsetroundjoin%
\pgfsetlinewidth{0.803000pt}%
\definecolor{currentstroke}{rgb}{0.690196,0.690196,0.690196}%
\pgfsetstrokecolor{currentstroke}%
\pgfsetdash{}{0pt}%
\pgfpathmoveto{\pgfqpoint{0.811028in}{0.571604in}}%
\pgfpathlineto{\pgfqpoint{0.811028in}{1.385015in}}%
\pgfusepath{stroke}%
\end{pgfscope}%
\begin{pgfscope}%
\pgfsetbuttcap%
\pgfsetroundjoin%
\definecolor{currentfill}{rgb}{0.000000,0.000000,0.000000}%
\pgfsetfillcolor{currentfill}%
\pgfsetlinewidth{0.803000pt}%
\definecolor{currentstroke}{rgb}{0.000000,0.000000,0.000000}%
\pgfsetstrokecolor{currentstroke}%
\pgfsetdash{}{0pt}%
\pgfsys@defobject{currentmarker}{\pgfqpoint{0.000000in}{-0.048611in}}{\pgfqpoint{0.000000in}{0.000000in}}{%
\pgfpathmoveto{\pgfqpoint{0.000000in}{0.000000in}}%
\pgfpathlineto{\pgfqpoint{0.000000in}{-0.048611in}}%
\pgfusepath{stroke,fill}%
}%
\begin{pgfscope}%
\pgfsys@transformshift{0.811028in}{0.571604in}%
\pgfsys@useobject{currentmarker}{}%
\end{pgfscope}%
\end{pgfscope}%
\begin{pgfscope}%
\definecolor{textcolor}{rgb}{0.000000,0.000000,0.000000}%
\pgfsetstrokecolor{textcolor}%
\pgfsetfillcolor{textcolor}%
\pgftext[x=0.811028in,y=0.474382in,,top]{\color{textcolor}\sffamily\fontsize{10.000000}{12.000000}\selectfont −20}%
\end{pgfscope}%
\begin{pgfscope}%
\pgfpathrectangle{\pgfqpoint{0.584475in}{0.571604in}}{\pgfqpoint{2.265525in}{0.813411in}}%
\pgfusepath{clip}%
\pgfsetrectcap%
\pgfsetroundjoin%
\pgfsetlinewidth{0.803000pt}%
\definecolor{currentstroke}{rgb}{0.690196,0.690196,0.690196}%
\pgfsetstrokecolor{currentstroke}%
\pgfsetdash{}{0pt}%
\pgfpathmoveto{\pgfqpoint{1.264133in}{0.571604in}}%
\pgfpathlineto{\pgfqpoint{1.264133in}{1.385015in}}%
\pgfusepath{stroke}%
\end{pgfscope}%
\begin{pgfscope}%
\pgfsetbuttcap%
\pgfsetroundjoin%
\definecolor{currentfill}{rgb}{0.000000,0.000000,0.000000}%
\pgfsetfillcolor{currentfill}%
\pgfsetlinewidth{0.803000pt}%
\definecolor{currentstroke}{rgb}{0.000000,0.000000,0.000000}%
\pgfsetstrokecolor{currentstroke}%
\pgfsetdash{}{0pt}%
\pgfsys@defobject{currentmarker}{\pgfqpoint{0.000000in}{-0.048611in}}{\pgfqpoint{0.000000in}{0.000000in}}{%
\pgfpathmoveto{\pgfqpoint{0.000000in}{0.000000in}}%
\pgfpathlineto{\pgfqpoint{0.000000in}{-0.048611in}}%
\pgfusepath{stroke,fill}%
}%
\begin{pgfscope}%
\pgfsys@transformshift{1.264133in}{0.571604in}%
\pgfsys@useobject{currentmarker}{}%
\end{pgfscope}%
\end{pgfscope}%
\begin{pgfscope}%
\definecolor{textcolor}{rgb}{0.000000,0.000000,0.000000}%
\pgfsetstrokecolor{textcolor}%
\pgfsetfillcolor{textcolor}%
\pgftext[x=1.264133in,y=0.474382in,,top]{\color{textcolor}\sffamily\fontsize{10.000000}{12.000000}\selectfont −10}%
\end{pgfscope}%
\begin{pgfscope}%
\pgfpathrectangle{\pgfqpoint{0.584475in}{0.571604in}}{\pgfqpoint{2.265525in}{0.813411in}}%
\pgfusepath{clip}%
\pgfsetrectcap%
\pgfsetroundjoin%
\pgfsetlinewidth{0.803000pt}%
\definecolor{currentstroke}{rgb}{0.690196,0.690196,0.690196}%
\pgfsetstrokecolor{currentstroke}%
\pgfsetdash{}{0pt}%
\pgfpathmoveto{\pgfqpoint{1.717238in}{0.571604in}}%
\pgfpathlineto{\pgfqpoint{1.717238in}{1.385015in}}%
\pgfusepath{stroke}%
\end{pgfscope}%
\begin{pgfscope}%
\pgfsetbuttcap%
\pgfsetroundjoin%
\definecolor{currentfill}{rgb}{0.000000,0.000000,0.000000}%
\pgfsetfillcolor{currentfill}%
\pgfsetlinewidth{0.803000pt}%
\definecolor{currentstroke}{rgb}{0.000000,0.000000,0.000000}%
\pgfsetstrokecolor{currentstroke}%
\pgfsetdash{}{0pt}%
\pgfsys@defobject{currentmarker}{\pgfqpoint{0.000000in}{-0.048611in}}{\pgfqpoint{0.000000in}{0.000000in}}{%
\pgfpathmoveto{\pgfqpoint{0.000000in}{0.000000in}}%
\pgfpathlineto{\pgfqpoint{0.000000in}{-0.048611in}}%
\pgfusepath{stroke,fill}%
}%
\begin{pgfscope}%
\pgfsys@transformshift{1.717238in}{0.571604in}%
\pgfsys@useobject{currentmarker}{}%
\end{pgfscope}%
\end{pgfscope}%
\begin{pgfscope}%
\definecolor{textcolor}{rgb}{0.000000,0.000000,0.000000}%
\pgfsetstrokecolor{textcolor}%
\pgfsetfillcolor{textcolor}%
\pgftext[x=1.717238in,y=0.474382in,,top]{\color{textcolor}\sffamily\fontsize{10.000000}{12.000000}\selectfont 0}%
\end{pgfscope}%
\begin{pgfscope}%
\pgfpathrectangle{\pgfqpoint{0.584475in}{0.571604in}}{\pgfqpoint{2.265525in}{0.813411in}}%
\pgfusepath{clip}%
\pgfsetrectcap%
\pgfsetroundjoin%
\pgfsetlinewidth{0.803000pt}%
\definecolor{currentstroke}{rgb}{0.690196,0.690196,0.690196}%
\pgfsetstrokecolor{currentstroke}%
\pgfsetdash{}{0pt}%
\pgfpathmoveto{\pgfqpoint{2.170343in}{0.571604in}}%
\pgfpathlineto{\pgfqpoint{2.170343in}{1.385015in}}%
\pgfusepath{stroke}%
\end{pgfscope}%
\begin{pgfscope}%
\pgfsetbuttcap%
\pgfsetroundjoin%
\definecolor{currentfill}{rgb}{0.000000,0.000000,0.000000}%
\pgfsetfillcolor{currentfill}%
\pgfsetlinewidth{0.803000pt}%
\definecolor{currentstroke}{rgb}{0.000000,0.000000,0.000000}%
\pgfsetstrokecolor{currentstroke}%
\pgfsetdash{}{0pt}%
\pgfsys@defobject{currentmarker}{\pgfqpoint{0.000000in}{-0.048611in}}{\pgfqpoint{0.000000in}{0.000000in}}{%
\pgfpathmoveto{\pgfqpoint{0.000000in}{0.000000in}}%
\pgfpathlineto{\pgfqpoint{0.000000in}{-0.048611in}}%
\pgfusepath{stroke,fill}%
}%
\begin{pgfscope}%
\pgfsys@transformshift{2.170343in}{0.571604in}%
\pgfsys@useobject{currentmarker}{}%
\end{pgfscope}%
\end{pgfscope}%
\begin{pgfscope}%
\definecolor{textcolor}{rgb}{0.000000,0.000000,0.000000}%
\pgfsetstrokecolor{textcolor}%
\pgfsetfillcolor{textcolor}%
\pgftext[x=2.170343in,y=0.474382in,,top]{\color{textcolor}\sffamily\fontsize{10.000000}{12.000000}\selectfont 10}%
\end{pgfscope}%
\begin{pgfscope}%
\pgfpathrectangle{\pgfqpoint{0.584475in}{0.571604in}}{\pgfqpoint{2.265525in}{0.813411in}}%
\pgfusepath{clip}%
\pgfsetrectcap%
\pgfsetroundjoin%
\pgfsetlinewidth{0.803000pt}%
\definecolor{currentstroke}{rgb}{0.690196,0.690196,0.690196}%
\pgfsetstrokecolor{currentstroke}%
\pgfsetdash{}{0pt}%
\pgfpathmoveto{\pgfqpoint{2.623448in}{0.571604in}}%
\pgfpathlineto{\pgfqpoint{2.623448in}{1.385015in}}%
\pgfusepath{stroke}%
\end{pgfscope}%
\begin{pgfscope}%
\pgfsetbuttcap%
\pgfsetroundjoin%
\definecolor{currentfill}{rgb}{0.000000,0.000000,0.000000}%
\pgfsetfillcolor{currentfill}%
\pgfsetlinewidth{0.803000pt}%
\definecolor{currentstroke}{rgb}{0.000000,0.000000,0.000000}%
\pgfsetstrokecolor{currentstroke}%
\pgfsetdash{}{0pt}%
\pgfsys@defobject{currentmarker}{\pgfqpoint{0.000000in}{-0.048611in}}{\pgfqpoint{0.000000in}{0.000000in}}{%
\pgfpathmoveto{\pgfqpoint{0.000000in}{0.000000in}}%
\pgfpathlineto{\pgfqpoint{0.000000in}{-0.048611in}}%
\pgfusepath{stroke,fill}%
}%
\begin{pgfscope}%
\pgfsys@transformshift{2.623448in}{0.571604in}%
\pgfsys@useobject{currentmarker}{}%
\end{pgfscope}%
\end{pgfscope}%
\begin{pgfscope}%
\definecolor{textcolor}{rgb}{0.000000,0.000000,0.000000}%
\pgfsetstrokecolor{textcolor}%
\pgfsetfillcolor{textcolor}%
\pgftext[x=2.623448in,y=0.474382in,,top]{\color{textcolor}\sffamily\fontsize{10.000000}{12.000000}\selectfont 20}%
\end{pgfscope}%
\begin{pgfscope}%
\definecolor{textcolor}{rgb}{0.000000,0.000000,0.000000}%
\pgfsetstrokecolor{textcolor}%
\pgfsetfillcolor{textcolor}%
\pgftext[x=1.717238in,y=0.284413in,,top]{\color{textcolor}\sffamily\fontsize{10.000000}{12.000000}\selectfont Shift of paramter 3}%
\end{pgfscope}%
\begin{pgfscope}%
\pgfpathrectangle{\pgfqpoint{0.584475in}{0.571604in}}{\pgfqpoint{2.265525in}{0.813411in}}%
\pgfusepath{clip}%
\pgfsetrectcap%
\pgfsetroundjoin%
\pgfsetlinewidth{0.803000pt}%
\definecolor{currentstroke}{rgb}{0.690196,0.690196,0.690196}%
\pgfsetstrokecolor{currentstroke}%
\pgfsetdash{}{0pt}%
\pgfpathmoveto{\pgfqpoint{0.584475in}{0.571604in}}%
\pgfpathlineto{\pgfqpoint{2.850000in}{0.571604in}}%
\pgfusepath{stroke}%
\end{pgfscope}%
\begin{pgfscope}%
\pgfsetbuttcap%
\pgfsetroundjoin%
\definecolor{currentfill}{rgb}{0.000000,0.000000,0.000000}%
\pgfsetfillcolor{currentfill}%
\pgfsetlinewidth{0.803000pt}%
\definecolor{currentstroke}{rgb}{0.000000,0.000000,0.000000}%
\pgfsetstrokecolor{currentstroke}%
\pgfsetdash{}{0pt}%
\pgfsys@defobject{currentmarker}{\pgfqpoint{-0.048611in}{0.000000in}}{\pgfqpoint{0.000000in}{0.000000in}}{%
\pgfpathmoveto{\pgfqpoint{0.000000in}{0.000000in}}%
\pgfpathlineto{\pgfqpoint{-0.048611in}{0.000000in}}%
\pgfusepath{stroke,fill}%
}%
\begin{pgfscope}%
\pgfsys@transformshift{0.584475in}{0.571604in}%
\pgfsys@useobject{currentmarker}{}%
\end{pgfscope}%
\end{pgfscope}%
\begin{pgfscope}%
\definecolor{textcolor}{rgb}{0.000000,0.000000,0.000000}%
\pgfsetstrokecolor{textcolor}%
\pgfsetfillcolor{textcolor}%
\pgftext[x=0.398888in,y=0.518842in,left,base]{\color{textcolor}\sffamily\fontsize{10.000000}{12.000000}\selectfont 0}%
\end{pgfscope}%
\begin{pgfscope}%
\pgfpathrectangle{\pgfqpoint{0.584475in}{0.571604in}}{\pgfqpoint{2.265525in}{0.813411in}}%
\pgfusepath{clip}%
\pgfsetrectcap%
\pgfsetroundjoin%
\pgfsetlinewidth{0.803000pt}%
\definecolor{currentstroke}{rgb}{0.690196,0.690196,0.690196}%
\pgfsetstrokecolor{currentstroke}%
\pgfsetdash{}{0pt}%
\pgfpathmoveto{\pgfqpoint{0.584475in}{1.113878in}}%
\pgfpathlineto{\pgfqpoint{2.850000in}{1.113878in}}%
\pgfusepath{stroke}%
\end{pgfscope}%
\begin{pgfscope}%
\pgfsetbuttcap%
\pgfsetroundjoin%
\definecolor{currentfill}{rgb}{0.000000,0.000000,0.000000}%
\pgfsetfillcolor{currentfill}%
\pgfsetlinewidth{0.803000pt}%
\definecolor{currentstroke}{rgb}{0.000000,0.000000,0.000000}%
\pgfsetstrokecolor{currentstroke}%
\pgfsetdash{}{0pt}%
\pgfsys@defobject{currentmarker}{\pgfqpoint{-0.048611in}{0.000000in}}{\pgfqpoint{0.000000in}{0.000000in}}{%
\pgfpathmoveto{\pgfqpoint{0.000000in}{0.000000in}}%
\pgfpathlineto{\pgfqpoint{-0.048611in}{0.000000in}}%
\pgfusepath{stroke,fill}%
}%
\begin{pgfscope}%
\pgfsys@transformshift{0.584475in}{1.113878in}%
\pgfsys@useobject{currentmarker}{}%
\end{pgfscope}%
\end{pgfscope}%
\begin{pgfscope}%
\definecolor{textcolor}{rgb}{0.000000,0.000000,0.000000}%
\pgfsetstrokecolor{textcolor}%
\pgfsetfillcolor{textcolor}%
\pgftext[x=0.398888in,y=1.061117in,left,base]{\color{textcolor}\sffamily\fontsize{10.000000}{12.000000}\selectfont 2}%
\end{pgfscope}%
\begin{pgfscope}%
\definecolor{textcolor}{rgb}{0.000000,0.000000,0.000000}%
\pgfsetstrokecolor{textcolor}%
\pgfsetfillcolor{textcolor}%
\pgftext[x=0.584475in,y=1.426682in,left,base]{\color{textcolor}\sffamily\fontsize{10.000000}{12.000000}\selectfont 1e5}%
\end{pgfscope}%
\begin{pgfscope}%
\pgfpathrectangle{\pgfqpoint{0.584475in}{0.571604in}}{\pgfqpoint{2.265525in}{0.813411in}}%
\pgfusepath{clip}%
\pgfsetrectcap%
\pgfsetroundjoin%
\pgfsetlinewidth{1.505625pt}%
\definecolor{currentstroke}{rgb}{1.000000,0.000000,0.000000}%
\pgfsetstrokecolor{currentstroke}%
\pgfsetdash{}{0pt}%
\pgfpathmoveto{\pgfqpoint{0.584475in}{1.300296in}}%
\pgfpathlineto{\pgfqpoint{0.607131in}{1.289315in}}%
\pgfpathlineto{\pgfqpoint{0.629786in}{1.280969in}}%
\pgfpathlineto{\pgfqpoint{0.652441in}{1.272252in}}%
\pgfpathlineto{\pgfqpoint{0.675096in}{1.260533in}}%
\pgfpathlineto{\pgfqpoint{0.697752in}{1.248983in}}%
\pgfpathlineto{\pgfqpoint{0.720407in}{1.239960in}}%
\pgfpathlineto{\pgfqpoint{0.743062in}{1.230638in}}%
\pgfpathlineto{\pgfqpoint{0.765717in}{1.219586in}}%
\pgfpathlineto{\pgfqpoint{0.788373in}{1.207301in}}%
\pgfpathlineto{\pgfqpoint{0.811028in}{1.197329in}}%
\pgfpathlineto{\pgfqpoint{0.833683in}{1.187012in}}%
\pgfpathlineto{\pgfqpoint{0.856338in}{1.176123in}}%
\pgfpathlineto{\pgfqpoint{0.878994in}{1.162555in}}%
\pgfpathlineto{\pgfqpoint{0.901649in}{1.151108in}}%
\pgfpathlineto{\pgfqpoint{0.924304in}{1.139357in}}%
\pgfpathlineto{\pgfqpoint{0.946959in}{1.128308in}}%
\pgfpathlineto{\pgfqpoint{0.969615in}{1.113317in}}%
\pgfpathlineto{\pgfqpoint{0.992270in}{1.099302in}}%
\pgfpathlineto{\pgfqpoint{1.014925in}{1.087315in}}%
\pgfpathlineto{\pgfqpoint{1.037580in}{1.075219in}}%
\pgfpathlineto{\pgfqpoint{1.060236in}{1.060090in}}%
\pgfpathlineto{\pgfqpoint{1.082891in}{1.044787in}}%
\pgfpathlineto{\pgfqpoint{1.105546in}{1.031276in}}%
\pgfpathlineto{\pgfqpoint{1.128201in}{1.017627in}}%
\pgfpathlineto{\pgfqpoint{1.150857in}{1.002058in}}%
\pgfpathlineto{\pgfqpoint{1.173512in}{0.984928in}}%
\pgfpathlineto{\pgfqpoint{1.196167in}{0.969479in}}%
\pgfpathlineto{\pgfqpoint{1.218822in}{0.954130in}}%
\pgfpathlineto{\pgfqpoint{1.241478in}{0.938425in}}%
\pgfpathlineto{\pgfqpoint{1.264133in}{0.920313in}}%
\pgfpathlineto{\pgfqpoint{1.286788in}{0.904037in}}%
\pgfpathlineto{\pgfqpoint{1.309443in}{0.888200in}}%
\pgfpathlineto{\pgfqpoint{1.332099in}{0.873358in}}%
\pgfpathlineto{\pgfqpoint{1.354754in}{0.854766in}}%
\pgfpathlineto{\pgfqpoint{1.377409in}{0.837295in}}%
\pgfpathlineto{\pgfqpoint{1.400064in}{0.821506in}}%
\pgfpathlineto{\pgfqpoint{1.422720in}{0.806422in}}%
\pgfpathlineto{\pgfqpoint{1.445375in}{0.788926in}}%
\pgfpathlineto{\pgfqpoint{1.468030in}{0.771344in}}%
\pgfpathlineto{\pgfqpoint{1.490685in}{0.754950in}}%
\pgfpathlineto{\pgfqpoint{1.513340in}{0.738812in}}%
\pgfpathlineto{\pgfqpoint{1.535996in}{0.720521in}}%
\pgfpathlineto{\pgfqpoint{1.558651in}{0.700040in}}%
\pgfpathlineto{\pgfqpoint{1.581306in}{0.679952in}}%
\pgfpathlineto{\pgfqpoint{1.603961in}{0.659678in}}%
\pgfpathlineto{\pgfqpoint{1.626617in}{0.637990in}}%
\pgfpathlineto{\pgfqpoint{1.649272in}{0.613357in}}%
\pgfpathlineto{\pgfqpoint{1.671927in}{0.592369in}}%
\pgfpathlineto{\pgfqpoint{1.694582in}{0.577499in}}%
\pgfpathlineto{\pgfqpoint{1.717238in}{0.571604in}}%
\pgfpathlineto{\pgfqpoint{1.739893in}{0.577499in}}%
\pgfpathlineto{\pgfqpoint{1.762548in}{0.592369in}}%
\pgfpathlineto{\pgfqpoint{1.785203in}{0.613357in}}%
\pgfpathlineto{\pgfqpoint{1.807859in}{0.637990in}}%
\pgfpathlineto{\pgfqpoint{1.830514in}{0.659678in}}%
\pgfpathlineto{\pgfqpoint{1.853169in}{0.679952in}}%
\pgfpathlineto{\pgfqpoint{1.875824in}{0.700040in}}%
\pgfpathlineto{\pgfqpoint{1.898480in}{0.720521in}}%
\pgfpathlineto{\pgfqpoint{1.921135in}{0.738812in}}%
\pgfpathlineto{\pgfqpoint{1.943790in}{0.754950in}}%
\pgfpathlineto{\pgfqpoint{1.966445in}{0.771344in}}%
\pgfpathlineto{\pgfqpoint{1.989101in}{0.788926in}}%
\pgfpathlineto{\pgfqpoint{2.011756in}{0.806422in}}%
\pgfpathlineto{\pgfqpoint{2.034411in}{0.821506in}}%
\pgfpathlineto{\pgfqpoint{2.057066in}{0.837295in}}%
\pgfpathlineto{\pgfqpoint{2.079722in}{0.854766in}}%
\pgfpathlineto{\pgfqpoint{2.102377in}{0.873358in}}%
\pgfpathlineto{\pgfqpoint{2.125032in}{0.888200in}}%
\pgfpathlineto{\pgfqpoint{2.147687in}{0.904037in}}%
\pgfpathlineto{\pgfqpoint{2.170343in}{0.920313in}}%
\pgfpathlineto{\pgfqpoint{2.192998in}{0.938425in}}%
\pgfpathlineto{\pgfqpoint{2.215653in}{0.954130in}}%
\pgfpathlineto{\pgfqpoint{2.238308in}{0.969479in}}%
\pgfpathlineto{\pgfqpoint{2.260964in}{0.984928in}}%
\pgfpathlineto{\pgfqpoint{2.283619in}{1.002058in}}%
\pgfpathlineto{\pgfqpoint{2.306274in}{1.017627in}}%
\pgfpathlineto{\pgfqpoint{2.328929in}{1.031276in}}%
\pgfpathlineto{\pgfqpoint{2.351585in}{1.044787in}}%
\pgfpathlineto{\pgfqpoint{2.374240in}{1.060090in}}%
\pgfpathlineto{\pgfqpoint{2.396895in}{1.075219in}}%
\pgfpathlineto{\pgfqpoint{2.419550in}{1.087315in}}%
\pgfpathlineto{\pgfqpoint{2.442206in}{1.099302in}}%
\pgfpathlineto{\pgfqpoint{2.464861in}{1.113317in}}%
\pgfpathlineto{\pgfqpoint{2.487516in}{1.128308in}}%
\pgfpathlineto{\pgfqpoint{2.510171in}{1.139357in}}%
\pgfpathlineto{\pgfqpoint{2.532827in}{1.151108in}}%
\pgfpathlineto{\pgfqpoint{2.555482in}{1.162555in}}%
\pgfpathlineto{\pgfqpoint{2.578137in}{1.176123in}}%
\pgfpathlineto{\pgfqpoint{2.600792in}{1.187012in}}%
\pgfpathlineto{\pgfqpoint{2.623448in}{1.197329in}}%
\pgfpathlineto{\pgfqpoint{2.646103in}{1.207301in}}%
\pgfpathlineto{\pgfqpoint{2.668758in}{1.219586in}}%
\pgfpathlineto{\pgfqpoint{2.691413in}{1.230638in}}%
\pgfpathlineto{\pgfqpoint{2.714069in}{1.239960in}}%
\pgfpathlineto{\pgfqpoint{2.736724in}{1.248983in}}%
\pgfpathlineto{\pgfqpoint{2.759379in}{1.260533in}}%
\pgfpathlineto{\pgfqpoint{2.782034in}{1.272252in}}%
\pgfpathlineto{\pgfqpoint{2.804690in}{1.280969in}}%
\pgfpathlineto{\pgfqpoint{2.827345in}{1.289315in}}%
\pgfpathlineto{\pgfqpoint{2.850000in}{1.300296in}}%
\pgfusepath{stroke}%
\end{pgfscope}%
\begin{pgfscope}%
\pgfsetrectcap%
\pgfsetmiterjoin%
\pgfsetlinewidth{0.803000pt}%
\definecolor{currentstroke}{rgb}{0.000000,0.000000,0.000000}%
\pgfsetstrokecolor{currentstroke}%
\pgfsetdash{}{0pt}%
\pgfpathmoveto{\pgfqpoint{0.584475in}{0.571604in}}%
\pgfpathlineto{\pgfqpoint{0.584475in}{1.385015in}}%
\pgfusepath{stroke}%
\end{pgfscope}%
\begin{pgfscope}%
\pgfsetrectcap%
\pgfsetmiterjoin%
\pgfsetlinewidth{0.803000pt}%
\definecolor{currentstroke}{rgb}{0.000000,0.000000,0.000000}%
\pgfsetstrokecolor{currentstroke}%
\pgfsetdash{}{0pt}%
\pgfpathmoveto{\pgfqpoint{2.850000in}{0.571604in}}%
\pgfpathlineto{\pgfqpoint{2.850000in}{1.385015in}}%
\pgfusepath{stroke}%
\end{pgfscope}%
\begin{pgfscope}%
\pgfsetrectcap%
\pgfsetmiterjoin%
\pgfsetlinewidth{0.803000pt}%
\definecolor{currentstroke}{rgb}{0.000000,0.000000,0.000000}%
\pgfsetstrokecolor{currentstroke}%
\pgfsetdash{}{0pt}%
\pgfpathmoveto{\pgfqpoint{0.584475in}{0.571604in}}%
\pgfpathlineto{\pgfqpoint{2.850000in}{0.571604in}}%
\pgfusepath{stroke}%
\end{pgfscope}%
\begin{pgfscope}%
\pgfsetrectcap%
\pgfsetmiterjoin%
\pgfsetlinewidth{0.803000pt}%
\definecolor{currentstroke}{rgb}{0.000000,0.000000,0.000000}%
\pgfsetstrokecolor{currentstroke}%
\pgfsetdash{}{0pt}%
\pgfpathmoveto{\pgfqpoint{0.584475in}{1.385015in}}%
\pgfpathlineto{\pgfqpoint{2.850000in}{1.385015in}}%
\pgfusepath{stroke}%
\end{pgfscope}%
\begin{pgfscope}%
\definecolor{textcolor}{rgb}{1.000000,0.000000,0.000000}%
\pgfsetstrokecolor{textcolor}%
\pgfsetfillcolor{textcolor}%
\pgftext[x=1.717238in,y=1.468349in,,base]{\color{textcolor}\sffamily\fontsize{12.000000}{14.400000}\selectfont MSD metric value}%
\end{pgfscope}%
\end{pgfpicture}%
\makeatother%
\endgroup%
}
  \hfill
  \resizebox{0.32\linewidth}{!}{%% Creator: Matplotlib, PGF backend
%%
%% To include the figure in your LaTeX document, write
%%   \input{<filename>.pgf}
%%
%% Make sure the required packages are loaded in your preamble
%%   \usepackage{pgf}
%%
%% Figures using additional raster images can only be included by \input if
%% they are in the same directory as the main LaTeX file. For loading figures
%% from other directories you can use the `import` package
%%   \usepackage{import}
%% and then include the figures with
%%   \import{<path to file>}{<filename>.pgf}
%%
%% Matplotlib used the following preamble
%%   \usepackage{fontspec}
%%   \setmainfont{DejaVuSerif.ttf}[Path=/usr/share/matplotlib/mpl-data/fonts/ttf/]
%%   \setsansfont{DejaVuSans.ttf}[Path=/usr/share/matplotlib/mpl-data/fonts/ttf/]
%%   \setmonofont{DejaVuSansMono.ttf}[Path=/usr/share/matplotlib/mpl-data/fonts/ttf/]
%%
\begingroup%
\makeatletter%
\begin{pgfpicture}%
\pgfpathrectangle{\pgfpointorigin}{\pgfqpoint{3.000000in}{3.000000in}}%
\pgfusepath{use as bounding box, clip}%
\begin{pgfscope}%
\pgfsetbuttcap%
\pgfsetmiterjoin%
\definecolor{currentfill}{rgb}{1.000000,1.000000,1.000000}%
\pgfsetfillcolor{currentfill}%
\pgfsetlinewidth{0.000000pt}%
\definecolor{currentstroke}{rgb}{1.000000,1.000000,1.000000}%
\pgfsetstrokecolor{currentstroke}%
\pgfsetdash{}{0pt}%
\pgfpathmoveto{\pgfqpoint{0.000000in}{0.000000in}}%
\pgfpathlineto{\pgfqpoint{3.000000in}{0.000000in}}%
\pgfpathlineto{\pgfqpoint{3.000000in}{3.000000in}}%
\pgfpathlineto{\pgfqpoint{0.000000in}{3.000000in}}%
\pgfpathclose%
\pgfusepath{fill}%
\end{pgfscope}%
\begin{pgfscope}%
\pgfsetbuttcap%
\pgfsetmiterjoin%
\definecolor{currentfill}{rgb}{1.000000,1.000000,1.000000}%
\pgfsetfillcolor{currentfill}%
\pgfsetlinewidth{0.000000pt}%
\definecolor{currentstroke}{rgb}{0.000000,0.000000,0.000000}%
\pgfsetstrokecolor{currentstroke}%
\pgfsetstrokeopacity{0.000000}%
\pgfsetdash{}{0pt}%
\pgfpathmoveto{\pgfqpoint{0.584475in}{1.826628in}}%
\pgfpathlineto{\pgfqpoint{2.850000in}{1.826628in}}%
\pgfpathlineto{\pgfqpoint{2.850000in}{2.640039in}}%
\pgfpathlineto{\pgfqpoint{0.584475in}{2.640039in}}%
\pgfpathclose%
\pgfusepath{fill}%
\end{pgfscope}%
\begin{pgfscope}%
\pgfpathrectangle{\pgfqpoint{0.584475in}{1.826628in}}{\pgfqpoint{2.265525in}{0.813411in}}%
\pgfusepath{clip}%
\pgfsetrectcap%
\pgfsetroundjoin%
\pgfsetlinewidth{0.803000pt}%
\definecolor{currentstroke}{rgb}{0.690196,0.690196,0.690196}%
\pgfsetstrokecolor{currentstroke}%
\pgfsetdash{}{0pt}%
\pgfpathmoveto{\pgfqpoint{0.811028in}{1.826628in}}%
\pgfpathlineto{\pgfqpoint{0.811028in}{2.640039in}}%
\pgfusepath{stroke}%
\end{pgfscope}%
\begin{pgfscope}%
\pgfsetbuttcap%
\pgfsetroundjoin%
\definecolor{currentfill}{rgb}{0.000000,0.000000,0.000000}%
\pgfsetfillcolor{currentfill}%
\pgfsetlinewidth{0.803000pt}%
\definecolor{currentstroke}{rgb}{0.000000,0.000000,0.000000}%
\pgfsetstrokecolor{currentstroke}%
\pgfsetdash{}{0pt}%
\pgfsys@defobject{currentmarker}{\pgfqpoint{0.000000in}{-0.048611in}}{\pgfqpoint{0.000000in}{0.000000in}}{%
\pgfpathmoveto{\pgfqpoint{0.000000in}{0.000000in}}%
\pgfpathlineto{\pgfqpoint{0.000000in}{-0.048611in}}%
\pgfusepath{stroke,fill}%
}%
\begin{pgfscope}%
\pgfsys@transformshift{0.811028in}{1.826628in}%
\pgfsys@useobject{currentmarker}{}%
\end{pgfscope}%
\end{pgfscope}%
\begin{pgfscope}%
\pgfpathrectangle{\pgfqpoint{0.584475in}{1.826628in}}{\pgfqpoint{2.265525in}{0.813411in}}%
\pgfusepath{clip}%
\pgfsetrectcap%
\pgfsetroundjoin%
\pgfsetlinewidth{0.803000pt}%
\definecolor{currentstroke}{rgb}{0.690196,0.690196,0.690196}%
\pgfsetstrokecolor{currentstroke}%
\pgfsetdash{}{0pt}%
\pgfpathmoveto{\pgfqpoint{1.264133in}{1.826628in}}%
\pgfpathlineto{\pgfqpoint{1.264133in}{2.640039in}}%
\pgfusepath{stroke}%
\end{pgfscope}%
\begin{pgfscope}%
\pgfsetbuttcap%
\pgfsetroundjoin%
\definecolor{currentfill}{rgb}{0.000000,0.000000,0.000000}%
\pgfsetfillcolor{currentfill}%
\pgfsetlinewidth{0.803000pt}%
\definecolor{currentstroke}{rgb}{0.000000,0.000000,0.000000}%
\pgfsetstrokecolor{currentstroke}%
\pgfsetdash{}{0pt}%
\pgfsys@defobject{currentmarker}{\pgfqpoint{0.000000in}{-0.048611in}}{\pgfqpoint{0.000000in}{0.000000in}}{%
\pgfpathmoveto{\pgfqpoint{0.000000in}{0.000000in}}%
\pgfpathlineto{\pgfqpoint{0.000000in}{-0.048611in}}%
\pgfusepath{stroke,fill}%
}%
\begin{pgfscope}%
\pgfsys@transformshift{1.264133in}{1.826628in}%
\pgfsys@useobject{currentmarker}{}%
\end{pgfscope}%
\end{pgfscope}%
\begin{pgfscope}%
\pgfpathrectangle{\pgfqpoint{0.584475in}{1.826628in}}{\pgfqpoint{2.265525in}{0.813411in}}%
\pgfusepath{clip}%
\pgfsetrectcap%
\pgfsetroundjoin%
\pgfsetlinewidth{0.803000pt}%
\definecolor{currentstroke}{rgb}{0.690196,0.690196,0.690196}%
\pgfsetstrokecolor{currentstroke}%
\pgfsetdash{}{0pt}%
\pgfpathmoveto{\pgfqpoint{1.717238in}{1.826628in}}%
\pgfpathlineto{\pgfqpoint{1.717238in}{2.640039in}}%
\pgfusepath{stroke}%
\end{pgfscope}%
\begin{pgfscope}%
\pgfsetbuttcap%
\pgfsetroundjoin%
\definecolor{currentfill}{rgb}{0.000000,0.000000,0.000000}%
\pgfsetfillcolor{currentfill}%
\pgfsetlinewidth{0.803000pt}%
\definecolor{currentstroke}{rgb}{0.000000,0.000000,0.000000}%
\pgfsetstrokecolor{currentstroke}%
\pgfsetdash{}{0pt}%
\pgfsys@defobject{currentmarker}{\pgfqpoint{0.000000in}{-0.048611in}}{\pgfqpoint{0.000000in}{0.000000in}}{%
\pgfpathmoveto{\pgfqpoint{0.000000in}{0.000000in}}%
\pgfpathlineto{\pgfqpoint{0.000000in}{-0.048611in}}%
\pgfusepath{stroke,fill}%
}%
\begin{pgfscope}%
\pgfsys@transformshift{1.717238in}{1.826628in}%
\pgfsys@useobject{currentmarker}{}%
\end{pgfscope}%
\end{pgfscope}%
\begin{pgfscope}%
\pgfpathrectangle{\pgfqpoint{0.584475in}{1.826628in}}{\pgfqpoint{2.265525in}{0.813411in}}%
\pgfusepath{clip}%
\pgfsetrectcap%
\pgfsetroundjoin%
\pgfsetlinewidth{0.803000pt}%
\definecolor{currentstroke}{rgb}{0.690196,0.690196,0.690196}%
\pgfsetstrokecolor{currentstroke}%
\pgfsetdash{}{0pt}%
\pgfpathmoveto{\pgfqpoint{2.170343in}{1.826628in}}%
\pgfpathlineto{\pgfqpoint{2.170343in}{2.640039in}}%
\pgfusepath{stroke}%
\end{pgfscope}%
\begin{pgfscope}%
\pgfsetbuttcap%
\pgfsetroundjoin%
\definecolor{currentfill}{rgb}{0.000000,0.000000,0.000000}%
\pgfsetfillcolor{currentfill}%
\pgfsetlinewidth{0.803000pt}%
\definecolor{currentstroke}{rgb}{0.000000,0.000000,0.000000}%
\pgfsetstrokecolor{currentstroke}%
\pgfsetdash{}{0pt}%
\pgfsys@defobject{currentmarker}{\pgfqpoint{0.000000in}{-0.048611in}}{\pgfqpoint{0.000000in}{0.000000in}}{%
\pgfpathmoveto{\pgfqpoint{0.000000in}{0.000000in}}%
\pgfpathlineto{\pgfqpoint{0.000000in}{-0.048611in}}%
\pgfusepath{stroke,fill}%
}%
\begin{pgfscope}%
\pgfsys@transformshift{2.170343in}{1.826628in}%
\pgfsys@useobject{currentmarker}{}%
\end{pgfscope}%
\end{pgfscope}%
\begin{pgfscope}%
\pgfpathrectangle{\pgfqpoint{0.584475in}{1.826628in}}{\pgfqpoint{2.265525in}{0.813411in}}%
\pgfusepath{clip}%
\pgfsetrectcap%
\pgfsetroundjoin%
\pgfsetlinewidth{0.803000pt}%
\definecolor{currentstroke}{rgb}{0.690196,0.690196,0.690196}%
\pgfsetstrokecolor{currentstroke}%
\pgfsetdash{}{0pt}%
\pgfpathmoveto{\pgfqpoint{2.623448in}{1.826628in}}%
\pgfpathlineto{\pgfqpoint{2.623448in}{2.640039in}}%
\pgfusepath{stroke}%
\end{pgfscope}%
\begin{pgfscope}%
\pgfsetbuttcap%
\pgfsetroundjoin%
\definecolor{currentfill}{rgb}{0.000000,0.000000,0.000000}%
\pgfsetfillcolor{currentfill}%
\pgfsetlinewidth{0.803000pt}%
\definecolor{currentstroke}{rgb}{0.000000,0.000000,0.000000}%
\pgfsetstrokecolor{currentstroke}%
\pgfsetdash{}{0pt}%
\pgfsys@defobject{currentmarker}{\pgfqpoint{0.000000in}{-0.048611in}}{\pgfqpoint{0.000000in}{0.000000in}}{%
\pgfpathmoveto{\pgfqpoint{0.000000in}{0.000000in}}%
\pgfpathlineto{\pgfqpoint{0.000000in}{-0.048611in}}%
\pgfusepath{stroke,fill}%
}%
\begin{pgfscope}%
\pgfsys@transformshift{2.623448in}{1.826628in}%
\pgfsys@useobject{currentmarker}{}%
\end{pgfscope}%
\end{pgfscope}%
\begin{pgfscope}%
\pgfpathrectangle{\pgfqpoint{0.584475in}{1.826628in}}{\pgfqpoint{2.265525in}{0.813411in}}%
\pgfusepath{clip}%
\pgfsetrectcap%
\pgfsetroundjoin%
\pgfsetlinewidth{0.803000pt}%
\definecolor{currentstroke}{rgb}{0.690196,0.690196,0.690196}%
\pgfsetstrokecolor{currentstroke}%
\pgfsetdash{}{0pt}%
\pgfpathmoveto{\pgfqpoint{0.584475in}{2.097765in}}%
\pgfpathlineto{\pgfqpoint{2.850000in}{2.097765in}}%
\pgfusepath{stroke}%
\end{pgfscope}%
\begin{pgfscope}%
\pgfsetbuttcap%
\pgfsetroundjoin%
\definecolor{currentfill}{rgb}{0.000000,0.000000,0.000000}%
\pgfsetfillcolor{currentfill}%
\pgfsetlinewidth{0.803000pt}%
\definecolor{currentstroke}{rgb}{0.000000,0.000000,0.000000}%
\pgfsetstrokecolor{currentstroke}%
\pgfsetdash{}{0pt}%
\pgfsys@defobject{currentmarker}{\pgfqpoint{-0.048611in}{0.000000in}}{\pgfqpoint{0.000000in}{0.000000in}}{%
\pgfpathmoveto{\pgfqpoint{0.000000in}{0.000000in}}%
\pgfpathlineto{\pgfqpoint{-0.048611in}{0.000000in}}%
\pgfusepath{stroke,fill}%
}%
\begin{pgfscope}%
\pgfsys@transformshift{0.584475in}{2.097765in}%
\pgfsys@useobject{currentmarker}{}%
\end{pgfscope}%
\end{pgfscope}%
\begin{pgfscope}%
\definecolor{textcolor}{rgb}{0.000000,0.000000,0.000000}%
\pgfsetstrokecolor{textcolor}%
\pgfsetfillcolor{textcolor}%
\pgftext[x=0.282514in,y=2.045003in,left,base]{\color{textcolor}\sffamily\fontsize{10.000000}{12.000000}\selectfont −1}%
\end{pgfscope}%
\begin{pgfscope}%
\pgfpathrectangle{\pgfqpoint{0.584475in}{1.826628in}}{\pgfqpoint{2.265525in}{0.813411in}}%
\pgfusepath{clip}%
\pgfsetrectcap%
\pgfsetroundjoin%
\pgfsetlinewidth{0.803000pt}%
\definecolor{currentstroke}{rgb}{0.690196,0.690196,0.690196}%
\pgfsetstrokecolor{currentstroke}%
\pgfsetdash{}{0pt}%
\pgfpathmoveto{\pgfqpoint{0.584475in}{2.640039in}}%
\pgfpathlineto{\pgfqpoint{2.850000in}{2.640039in}}%
\pgfusepath{stroke}%
\end{pgfscope}%
\begin{pgfscope}%
\pgfsetbuttcap%
\pgfsetroundjoin%
\definecolor{currentfill}{rgb}{0.000000,0.000000,0.000000}%
\pgfsetfillcolor{currentfill}%
\pgfsetlinewidth{0.803000pt}%
\definecolor{currentstroke}{rgb}{0.000000,0.000000,0.000000}%
\pgfsetstrokecolor{currentstroke}%
\pgfsetdash{}{0pt}%
\pgfsys@defobject{currentmarker}{\pgfqpoint{-0.048611in}{0.000000in}}{\pgfqpoint{0.000000in}{0.000000in}}{%
\pgfpathmoveto{\pgfqpoint{0.000000in}{0.000000in}}%
\pgfpathlineto{\pgfqpoint{-0.048611in}{0.000000in}}%
\pgfusepath{stroke,fill}%
}%
\begin{pgfscope}%
\pgfsys@transformshift{0.584475in}{2.640039in}%
\pgfsys@useobject{currentmarker}{}%
\end{pgfscope}%
\end{pgfscope}%
\begin{pgfscope}%
\definecolor{textcolor}{rgb}{0.000000,0.000000,0.000000}%
\pgfsetstrokecolor{textcolor}%
\pgfsetfillcolor{textcolor}%
\pgftext[x=0.398888in,y=2.587277in,left,base]{\color{textcolor}\sffamily\fontsize{10.000000}{12.000000}\selectfont 0}%
\end{pgfscope}%
\begin{pgfscope}%
\pgfpathrectangle{\pgfqpoint{0.584475in}{1.826628in}}{\pgfqpoint{2.265525in}{0.813411in}}%
\pgfusepath{clip}%
\pgfsetrectcap%
\pgfsetroundjoin%
\pgfsetlinewidth{1.505625pt}%
\definecolor{currentstroke}{rgb}{0.000000,0.000000,1.000000}%
\pgfsetstrokecolor{currentstroke}%
\pgfsetdash{}{0pt}%
\pgfpathmoveto{\pgfqpoint{0.584475in}{2.419972in}}%
\pgfpathlineto{\pgfqpoint{0.607131in}{2.417676in}}%
\pgfpathlineto{\pgfqpoint{0.629786in}{2.415340in}}%
\pgfpathlineto{\pgfqpoint{0.652441in}{2.413262in}}%
\pgfpathlineto{\pgfqpoint{0.675096in}{2.410661in}}%
\pgfpathlineto{\pgfqpoint{0.697752in}{2.408215in}}%
\pgfpathlineto{\pgfqpoint{0.720407in}{2.405786in}}%
\pgfpathlineto{\pgfqpoint{0.743062in}{2.403693in}}%
\pgfpathlineto{\pgfqpoint{0.765717in}{2.401136in}}%
\pgfpathlineto{\pgfqpoint{0.788373in}{2.398553in}}%
\pgfpathlineto{\pgfqpoint{0.811028in}{2.395959in}}%
\pgfpathlineto{\pgfqpoint{0.833683in}{2.393763in}}%
\pgfpathlineto{\pgfqpoint{0.856338in}{2.391257in}}%
\pgfpathlineto{\pgfqpoint{0.878994in}{2.388476in}}%
\pgfpathlineto{\pgfqpoint{0.901649in}{2.385727in}}%
\pgfpathlineto{\pgfqpoint{0.924304in}{2.383408in}}%
\pgfpathlineto{\pgfqpoint{0.946959in}{2.380911in}}%
\pgfpathlineto{\pgfqpoint{0.969615in}{2.377884in}}%
\pgfpathlineto{\pgfqpoint{0.992270in}{2.375148in}}%
\pgfpathlineto{\pgfqpoint{1.014925in}{2.372327in}}%
\pgfpathlineto{\pgfqpoint{1.037580in}{2.369643in}}%
\pgfpathlineto{\pgfqpoint{1.060236in}{2.366381in}}%
\pgfpathlineto{\pgfqpoint{1.082891in}{2.363384in}}%
\pgfpathlineto{\pgfqpoint{1.105546in}{2.360234in}}%
\pgfpathlineto{\pgfqpoint{1.128201in}{2.357271in}}%
\pgfpathlineto{\pgfqpoint{1.150857in}{2.353772in}}%
\pgfpathlineto{\pgfqpoint{1.173512in}{2.350388in}}%
\pgfpathlineto{\pgfqpoint{1.196167in}{2.346752in}}%
\pgfpathlineto{\pgfqpoint{1.218822in}{2.343243in}}%
\pgfpathlineto{\pgfqpoint{1.241478in}{2.339279in}}%
\pgfpathlineto{\pgfqpoint{1.264133in}{2.335227in}}%
\pgfpathlineto{\pgfqpoint{1.286788in}{2.330911in}}%
\pgfpathlineto{\pgfqpoint{1.309443in}{2.326832in}}%
\pgfpathlineto{\pgfqpoint{1.332099in}{2.322431in}}%
\pgfpathlineto{\pgfqpoint{1.354754in}{2.317681in}}%
\pgfpathlineto{\pgfqpoint{1.377409in}{2.312983in}}%
\pgfpathlineto{\pgfqpoint{1.400064in}{2.307960in}}%
\pgfpathlineto{\pgfqpoint{1.422720in}{2.302582in}}%
\pgfpathlineto{\pgfqpoint{1.445375in}{2.296434in}}%
\pgfpathlineto{\pgfqpoint{1.468030in}{2.290505in}}%
\pgfpathlineto{\pgfqpoint{1.490685in}{2.284143in}}%
\pgfpathlineto{\pgfqpoint{1.513340in}{2.277142in}}%
\pgfpathlineto{\pgfqpoint{1.535996in}{2.268153in}}%
\pgfpathlineto{\pgfqpoint{1.558651in}{2.260057in}}%
\pgfpathlineto{\pgfqpoint{1.581306in}{2.249953in}}%
\pgfpathlineto{\pgfqpoint{1.603961in}{2.237612in}}%
\pgfpathlineto{\pgfqpoint{1.626617in}{2.219731in}}%
\pgfpathlineto{\pgfqpoint{1.649272in}{2.199340in}}%
\pgfpathlineto{\pgfqpoint{1.671927in}{2.163405in}}%
\pgfpathlineto{\pgfqpoint{1.694582in}{2.101265in}}%
\pgfpathlineto{\pgfqpoint{1.717238in}{1.926948in}}%
\pgfpathlineto{\pgfqpoint{1.739893in}{2.101212in}}%
\pgfpathlineto{\pgfqpoint{1.762548in}{2.163456in}}%
\pgfpathlineto{\pgfqpoint{1.785203in}{2.199581in}}%
\pgfpathlineto{\pgfqpoint{1.807859in}{2.219934in}}%
\pgfpathlineto{\pgfqpoint{1.830514in}{2.237952in}}%
\pgfpathlineto{\pgfqpoint{1.853169in}{2.250166in}}%
\pgfpathlineto{\pgfqpoint{1.875824in}{2.260302in}}%
\pgfpathlineto{\pgfqpoint{1.898480in}{2.268439in}}%
\pgfpathlineto{\pgfqpoint{1.921135in}{2.277726in}}%
\pgfpathlineto{\pgfqpoint{1.943790in}{2.284692in}}%
\pgfpathlineto{\pgfqpoint{1.966445in}{2.291106in}}%
\pgfpathlineto{\pgfqpoint{1.989101in}{2.297029in}}%
\pgfpathlineto{\pgfqpoint{2.011756in}{2.303333in}}%
\pgfpathlineto{\pgfqpoint{2.034411in}{2.308744in}}%
\pgfpathlineto{\pgfqpoint{2.057066in}{2.313808in}}%
\pgfpathlineto{\pgfqpoint{2.079722in}{2.318441in}}%
\pgfpathlineto{\pgfqpoint{2.102377in}{2.323197in}}%
\pgfpathlineto{\pgfqpoint{2.125032in}{2.327713in}}%
\pgfpathlineto{\pgfqpoint{2.147687in}{2.331667in}}%
\pgfpathlineto{\pgfqpoint{2.170343in}{2.335979in}}%
\pgfpathlineto{\pgfqpoint{2.192998in}{2.340005in}}%
\pgfpathlineto{\pgfqpoint{2.215653in}{2.344105in}}%
\pgfpathlineto{\pgfqpoint{2.238308in}{2.347527in}}%
\pgfpathlineto{\pgfqpoint{2.260964in}{2.351154in}}%
\pgfpathlineto{\pgfqpoint{2.283619in}{2.354517in}}%
\pgfpathlineto{\pgfqpoint{2.306274in}{2.358146in}}%
\pgfpathlineto{\pgfqpoint{2.328929in}{2.361121in}}%
\pgfpathlineto{\pgfqpoint{2.351585in}{2.364195in}}%
\pgfpathlineto{\pgfqpoint{2.374240in}{2.367173in}}%
\pgfpathlineto{\pgfqpoint{2.396895in}{2.370496in}}%
\pgfpathlineto{\pgfqpoint{2.419550in}{2.373302in}}%
\pgfpathlineto{\pgfqpoint{2.442206in}{2.376052in}}%
\pgfpathlineto{\pgfqpoint{2.464861in}{2.378744in}}%
\pgfpathlineto{\pgfqpoint{2.487516in}{2.381785in}}%
\pgfpathlineto{\pgfqpoint{2.510171in}{2.384499in}}%
\pgfpathlineto{\pgfqpoint{2.532827in}{2.386813in}}%
\pgfpathlineto{\pgfqpoint{2.555482in}{2.389479in}}%
\pgfpathlineto{\pgfqpoint{2.578137in}{2.392225in}}%
\pgfpathlineto{\pgfqpoint{2.600792in}{2.394923in}}%
\pgfpathlineto{\pgfqpoint{2.623448in}{2.397154in}}%
\pgfpathlineto{\pgfqpoint{2.646103in}{2.399670in}}%
\pgfpathlineto{\pgfqpoint{2.668758in}{2.402227in}}%
\pgfpathlineto{\pgfqpoint{2.691413in}{2.404888in}}%
\pgfpathlineto{\pgfqpoint{2.714069in}{2.407096in}}%
\pgfpathlineto{\pgfqpoint{2.736724in}{2.409462in}}%
\pgfpathlineto{\pgfqpoint{2.759379in}{2.411857in}}%
\pgfpathlineto{\pgfqpoint{2.782034in}{2.414516in}}%
\pgfpathlineto{\pgfqpoint{2.804690in}{2.416713in}}%
\pgfpathlineto{\pgfqpoint{2.827345in}{2.418958in}}%
\pgfpathlineto{\pgfqpoint{2.850000in}{2.421256in}}%
\pgfusepath{stroke}%
\end{pgfscope}%
\begin{pgfscope}%
\pgfsetrectcap%
\pgfsetmiterjoin%
\pgfsetlinewidth{0.803000pt}%
\definecolor{currentstroke}{rgb}{0.000000,0.000000,0.000000}%
\pgfsetstrokecolor{currentstroke}%
\pgfsetdash{}{0pt}%
\pgfpathmoveto{\pgfqpoint{0.584475in}{1.826628in}}%
\pgfpathlineto{\pgfqpoint{0.584475in}{2.640039in}}%
\pgfusepath{stroke}%
\end{pgfscope}%
\begin{pgfscope}%
\pgfsetrectcap%
\pgfsetmiterjoin%
\pgfsetlinewidth{0.803000pt}%
\definecolor{currentstroke}{rgb}{0.000000,0.000000,0.000000}%
\pgfsetstrokecolor{currentstroke}%
\pgfsetdash{}{0pt}%
\pgfpathmoveto{\pgfqpoint{2.850000in}{1.826628in}}%
\pgfpathlineto{\pgfqpoint{2.850000in}{2.640039in}}%
\pgfusepath{stroke}%
\end{pgfscope}%
\begin{pgfscope}%
\pgfsetrectcap%
\pgfsetmiterjoin%
\pgfsetlinewidth{0.803000pt}%
\definecolor{currentstroke}{rgb}{0.000000,0.000000,0.000000}%
\pgfsetstrokecolor{currentstroke}%
\pgfsetdash{}{0pt}%
\pgfpathmoveto{\pgfqpoint{0.584475in}{1.826628in}}%
\pgfpathlineto{\pgfqpoint{2.850000in}{1.826628in}}%
\pgfusepath{stroke}%
\end{pgfscope}%
\begin{pgfscope}%
\pgfsetrectcap%
\pgfsetmiterjoin%
\pgfsetlinewidth{0.803000pt}%
\definecolor{currentstroke}{rgb}{0.000000,0.000000,0.000000}%
\pgfsetstrokecolor{currentstroke}%
\pgfsetdash{}{0pt}%
\pgfpathmoveto{\pgfqpoint{0.584475in}{2.640039in}}%
\pgfpathlineto{\pgfqpoint{2.850000in}{2.640039in}}%
\pgfusepath{stroke}%
\end{pgfscope}%
\begin{pgfscope}%
\definecolor{textcolor}{rgb}{0.000000,0.000000,1.000000}%
\pgfsetstrokecolor{textcolor}%
\pgfsetfillcolor{textcolor}%
\pgftext[x=1.717238in,y=2.723372in,,base]{\color{textcolor}\sffamily\fontsize{12.000000}{14.400000}\selectfont MI metric value}%
\end{pgfscope}%
\begin{pgfscope}%
\pgfsetbuttcap%
\pgfsetmiterjoin%
\definecolor{currentfill}{rgb}{1.000000,1.000000,1.000000}%
\pgfsetfillcolor{currentfill}%
\pgfsetlinewidth{0.000000pt}%
\definecolor{currentstroke}{rgb}{0.000000,0.000000,0.000000}%
\pgfsetstrokecolor{currentstroke}%
\pgfsetstrokeopacity{0.000000}%
\pgfsetdash{}{0pt}%
\pgfpathmoveto{\pgfqpoint{0.584475in}{0.571604in}}%
\pgfpathlineto{\pgfqpoint{2.850000in}{0.571604in}}%
\pgfpathlineto{\pgfqpoint{2.850000in}{1.385015in}}%
\pgfpathlineto{\pgfqpoint{0.584475in}{1.385015in}}%
\pgfpathclose%
\pgfusepath{fill}%
\end{pgfscope}%
\begin{pgfscope}%
\pgfpathrectangle{\pgfqpoint{0.584475in}{0.571604in}}{\pgfqpoint{2.265525in}{0.813411in}}%
\pgfusepath{clip}%
\pgfsetrectcap%
\pgfsetroundjoin%
\pgfsetlinewidth{0.803000pt}%
\definecolor{currentstroke}{rgb}{0.690196,0.690196,0.690196}%
\pgfsetstrokecolor{currentstroke}%
\pgfsetdash{}{0pt}%
\pgfpathmoveto{\pgfqpoint{0.811028in}{0.571604in}}%
\pgfpathlineto{\pgfqpoint{0.811028in}{1.385015in}}%
\pgfusepath{stroke}%
\end{pgfscope}%
\begin{pgfscope}%
\pgfsetbuttcap%
\pgfsetroundjoin%
\definecolor{currentfill}{rgb}{0.000000,0.000000,0.000000}%
\pgfsetfillcolor{currentfill}%
\pgfsetlinewidth{0.803000pt}%
\definecolor{currentstroke}{rgb}{0.000000,0.000000,0.000000}%
\pgfsetstrokecolor{currentstroke}%
\pgfsetdash{}{0pt}%
\pgfsys@defobject{currentmarker}{\pgfqpoint{0.000000in}{-0.048611in}}{\pgfqpoint{0.000000in}{0.000000in}}{%
\pgfpathmoveto{\pgfqpoint{0.000000in}{0.000000in}}%
\pgfpathlineto{\pgfqpoint{0.000000in}{-0.048611in}}%
\pgfusepath{stroke,fill}%
}%
\begin{pgfscope}%
\pgfsys@transformshift{0.811028in}{0.571604in}%
\pgfsys@useobject{currentmarker}{}%
\end{pgfscope}%
\end{pgfscope}%
\begin{pgfscope}%
\definecolor{textcolor}{rgb}{0.000000,0.000000,0.000000}%
\pgfsetstrokecolor{textcolor}%
\pgfsetfillcolor{textcolor}%
\pgftext[x=0.811028in,y=0.474382in,,top]{\color{textcolor}\sffamily\fontsize{10.000000}{12.000000}\selectfont −20}%
\end{pgfscope}%
\begin{pgfscope}%
\pgfpathrectangle{\pgfqpoint{0.584475in}{0.571604in}}{\pgfqpoint{2.265525in}{0.813411in}}%
\pgfusepath{clip}%
\pgfsetrectcap%
\pgfsetroundjoin%
\pgfsetlinewidth{0.803000pt}%
\definecolor{currentstroke}{rgb}{0.690196,0.690196,0.690196}%
\pgfsetstrokecolor{currentstroke}%
\pgfsetdash{}{0pt}%
\pgfpathmoveto{\pgfqpoint{1.264133in}{0.571604in}}%
\pgfpathlineto{\pgfqpoint{1.264133in}{1.385015in}}%
\pgfusepath{stroke}%
\end{pgfscope}%
\begin{pgfscope}%
\pgfsetbuttcap%
\pgfsetroundjoin%
\definecolor{currentfill}{rgb}{0.000000,0.000000,0.000000}%
\pgfsetfillcolor{currentfill}%
\pgfsetlinewidth{0.803000pt}%
\definecolor{currentstroke}{rgb}{0.000000,0.000000,0.000000}%
\pgfsetstrokecolor{currentstroke}%
\pgfsetdash{}{0pt}%
\pgfsys@defobject{currentmarker}{\pgfqpoint{0.000000in}{-0.048611in}}{\pgfqpoint{0.000000in}{0.000000in}}{%
\pgfpathmoveto{\pgfqpoint{0.000000in}{0.000000in}}%
\pgfpathlineto{\pgfqpoint{0.000000in}{-0.048611in}}%
\pgfusepath{stroke,fill}%
}%
\begin{pgfscope}%
\pgfsys@transformshift{1.264133in}{0.571604in}%
\pgfsys@useobject{currentmarker}{}%
\end{pgfscope}%
\end{pgfscope}%
\begin{pgfscope}%
\definecolor{textcolor}{rgb}{0.000000,0.000000,0.000000}%
\pgfsetstrokecolor{textcolor}%
\pgfsetfillcolor{textcolor}%
\pgftext[x=1.264133in,y=0.474382in,,top]{\color{textcolor}\sffamily\fontsize{10.000000}{12.000000}\selectfont −10}%
\end{pgfscope}%
\begin{pgfscope}%
\pgfpathrectangle{\pgfqpoint{0.584475in}{0.571604in}}{\pgfqpoint{2.265525in}{0.813411in}}%
\pgfusepath{clip}%
\pgfsetrectcap%
\pgfsetroundjoin%
\pgfsetlinewidth{0.803000pt}%
\definecolor{currentstroke}{rgb}{0.690196,0.690196,0.690196}%
\pgfsetstrokecolor{currentstroke}%
\pgfsetdash{}{0pt}%
\pgfpathmoveto{\pgfqpoint{1.717238in}{0.571604in}}%
\pgfpathlineto{\pgfqpoint{1.717238in}{1.385015in}}%
\pgfusepath{stroke}%
\end{pgfscope}%
\begin{pgfscope}%
\pgfsetbuttcap%
\pgfsetroundjoin%
\definecolor{currentfill}{rgb}{0.000000,0.000000,0.000000}%
\pgfsetfillcolor{currentfill}%
\pgfsetlinewidth{0.803000pt}%
\definecolor{currentstroke}{rgb}{0.000000,0.000000,0.000000}%
\pgfsetstrokecolor{currentstroke}%
\pgfsetdash{}{0pt}%
\pgfsys@defobject{currentmarker}{\pgfqpoint{0.000000in}{-0.048611in}}{\pgfqpoint{0.000000in}{0.000000in}}{%
\pgfpathmoveto{\pgfqpoint{0.000000in}{0.000000in}}%
\pgfpathlineto{\pgfqpoint{0.000000in}{-0.048611in}}%
\pgfusepath{stroke,fill}%
}%
\begin{pgfscope}%
\pgfsys@transformshift{1.717238in}{0.571604in}%
\pgfsys@useobject{currentmarker}{}%
\end{pgfscope}%
\end{pgfscope}%
\begin{pgfscope}%
\definecolor{textcolor}{rgb}{0.000000,0.000000,0.000000}%
\pgfsetstrokecolor{textcolor}%
\pgfsetfillcolor{textcolor}%
\pgftext[x=1.717238in,y=0.474382in,,top]{\color{textcolor}\sffamily\fontsize{10.000000}{12.000000}\selectfont 0}%
\end{pgfscope}%
\begin{pgfscope}%
\pgfpathrectangle{\pgfqpoint{0.584475in}{0.571604in}}{\pgfqpoint{2.265525in}{0.813411in}}%
\pgfusepath{clip}%
\pgfsetrectcap%
\pgfsetroundjoin%
\pgfsetlinewidth{0.803000pt}%
\definecolor{currentstroke}{rgb}{0.690196,0.690196,0.690196}%
\pgfsetstrokecolor{currentstroke}%
\pgfsetdash{}{0pt}%
\pgfpathmoveto{\pgfqpoint{2.170343in}{0.571604in}}%
\pgfpathlineto{\pgfqpoint{2.170343in}{1.385015in}}%
\pgfusepath{stroke}%
\end{pgfscope}%
\begin{pgfscope}%
\pgfsetbuttcap%
\pgfsetroundjoin%
\definecolor{currentfill}{rgb}{0.000000,0.000000,0.000000}%
\pgfsetfillcolor{currentfill}%
\pgfsetlinewidth{0.803000pt}%
\definecolor{currentstroke}{rgb}{0.000000,0.000000,0.000000}%
\pgfsetstrokecolor{currentstroke}%
\pgfsetdash{}{0pt}%
\pgfsys@defobject{currentmarker}{\pgfqpoint{0.000000in}{-0.048611in}}{\pgfqpoint{0.000000in}{0.000000in}}{%
\pgfpathmoveto{\pgfqpoint{0.000000in}{0.000000in}}%
\pgfpathlineto{\pgfqpoint{0.000000in}{-0.048611in}}%
\pgfusepath{stroke,fill}%
}%
\begin{pgfscope}%
\pgfsys@transformshift{2.170343in}{0.571604in}%
\pgfsys@useobject{currentmarker}{}%
\end{pgfscope}%
\end{pgfscope}%
\begin{pgfscope}%
\definecolor{textcolor}{rgb}{0.000000,0.000000,0.000000}%
\pgfsetstrokecolor{textcolor}%
\pgfsetfillcolor{textcolor}%
\pgftext[x=2.170343in,y=0.474382in,,top]{\color{textcolor}\sffamily\fontsize{10.000000}{12.000000}\selectfont 10}%
\end{pgfscope}%
\begin{pgfscope}%
\pgfpathrectangle{\pgfqpoint{0.584475in}{0.571604in}}{\pgfqpoint{2.265525in}{0.813411in}}%
\pgfusepath{clip}%
\pgfsetrectcap%
\pgfsetroundjoin%
\pgfsetlinewidth{0.803000pt}%
\definecolor{currentstroke}{rgb}{0.690196,0.690196,0.690196}%
\pgfsetstrokecolor{currentstroke}%
\pgfsetdash{}{0pt}%
\pgfpathmoveto{\pgfqpoint{2.623448in}{0.571604in}}%
\pgfpathlineto{\pgfqpoint{2.623448in}{1.385015in}}%
\pgfusepath{stroke}%
\end{pgfscope}%
\begin{pgfscope}%
\pgfsetbuttcap%
\pgfsetroundjoin%
\definecolor{currentfill}{rgb}{0.000000,0.000000,0.000000}%
\pgfsetfillcolor{currentfill}%
\pgfsetlinewidth{0.803000pt}%
\definecolor{currentstroke}{rgb}{0.000000,0.000000,0.000000}%
\pgfsetstrokecolor{currentstroke}%
\pgfsetdash{}{0pt}%
\pgfsys@defobject{currentmarker}{\pgfqpoint{0.000000in}{-0.048611in}}{\pgfqpoint{0.000000in}{0.000000in}}{%
\pgfpathmoveto{\pgfqpoint{0.000000in}{0.000000in}}%
\pgfpathlineto{\pgfqpoint{0.000000in}{-0.048611in}}%
\pgfusepath{stroke,fill}%
}%
\begin{pgfscope}%
\pgfsys@transformshift{2.623448in}{0.571604in}%
\pgfsys@useobject{currentmarker}{}%
\end{pgfscope}%
\end{pgfscope}%
\begin{pgfscope}%
\definecolor{textcolor}{rgb}{0.000000,0.000000,0.000000}%
\pgfsetstrokecolor{textcolor}%
\pgfsetfillcolor{textcolor}%
\pgftext[x=2.623448in,y=0.474382in,,top]{\color{textcolor}\sffamily\fontsize{10.000000}{12.000000}\selectfont 20}%
\end{pgfscope}%
\begin{pgfscope}%
\definecolor{textcolor}{rgb}{0.000000,0.000000,0.000000}%
\pgfsetstrokecolor{textcolor}%
\pgfsetfillcolor{textcolor}%
\pgftext[x=1.717238in,y=0.284413in,,top]{\color{textcolor}\sffamily\fontsize{10.000000}{12.000000}\selectfont Shift of paramter 4}%
\end{pgfscope}%
\begin{pgfscope}%
\pgfpathrectangle{\pgfqpoint{0.584475in}{0.571604in}}{\pgfqpoint{2.265525in}{0.813411in}}%
\pgfusepath{clip}%
\pgfsetrectcap%
\pgfsetroundjoin%
\pgfsetlinewidth{0.803000pt}%
\definecolor{currentstroke}{rgb}{0.690196,0.690196,0.690196}%
\pgfsetstrokecolor{currentstroke}%
\pgfsetdash{}{0pt}%
\pgfpathmoveto{\pgfqpoint{0.584475in}{0.571604in}}%
\pgfpathlineto{\pgfqpoint{2.850000in}{0.571604in}}%
\pgfusepath{stroke}%
\end{pgfscope}%
\begin{pgfscope}%
\pgfsetbuttcap%
\pgfsetroundjoin%
\definecolor{currentfill}{rgb}{0.000000,0.000000,0.000000}%
\pgfsetfillcolor{currentfill}%
\pgfsetlinewidth{0.803000pt}%
\definecolor{currentstroke}{rgb}{0.000000,0.000000,0.000000}%
\pgfsetstrokecolor{currentstroke}%
\pgfsetdash{}{0pt}%
\pgfsys@defobject{currentmarker}{\pgfqpoint{-0.048611in}{0.000000in}}{\pgfqpoint{0.000000in}{0.000000in}}{%
\pgfpathmoveto{\pgfqpoint{0.000000in}{0.000000in}}%
\pgfpathlineto{\pgfqpoint{-0.048611in}{0.000000in}}%
\pgfusepath{stroke,fill}%
}%
\begin{pgfscope}%
\pgfsys@transformshift{0.584475in}{0.571604in}%
\pgfsys@useobject{currentmarker}{}%
\end{pgfscope}%
\end{pgfscope}%
\begin{pgfscope}%
\definecolor{textcolor}{rgb}{0.000000,0.000000,0.000000}%
\pgfsetstrokecolor{textcolor}%
\pgfsetfillcolor{textcolor}%
\pgftext[x=0.398888in,y=0.518842in,left,base]{\color{textcolor}\sffamily\fontsize{10.000000}{12.000000}\selectfont 0}%
\end{pgfscope}%
\begin{pgfscope}%
\pgfpathrectangle{\pgfqpoint{0.584475in}{0.571604in}}{\pgfqpoint{2.265525in}{0.813411in}}%
\pgfusepath{clip}%
\pgfsetrectcap%
\pgfsetroundjoin%
\pgfsetlinewidth{0.803000pt}%
\definecolor{currentstroke}{rgb}{0.690196,0.690196,0.690196}%
\pgfsetstrokecolor{currentstroke}%
\pgfsetdash{}{0pt}%
\pgfpathmoveto{\pgfqpoint{0.584475in}{1.113878in}}%
\pgfpathlineto{\pgfqpoint{2.850000in}{1.113878in}}%
\pgfusepath{stroke}%
\end{pgfscope}%
\begin{pgfscope}%
\pgfsetbuttcap%
\pgfsetroundjoin%
\definecolor{currentfill}{rgb}{0.000000,0.000000,0.000000}%
\pgfsetfillcolor{currentfill}%
\pgfsetlinewidth{0.803000pt}%
\definecolor{currentstroke}{rgb}{0.000000,0.000000,0.000000}%
\pgfsetstrokecolor{currentstroke}%
\pgfsetdash{}{0pt}%
\pgfsys@defobject{currentmarker}{\pgfqpoint{-0.048611in}{0.000000in}}{\pgfqpoint{0.000000in}{0.000000in}}{%
\pgfpathmoveto{\pgfqpoint{0.000000in}{0.000000in}}%
\pgfpathlineto{\pgfqpoint{-0.048611in}{0.000000in}}%
\pgfusepath{stroke,fill}%
}%
\begin{pgfscope}%
\pgfsys@transformshift{0.584475in}{1.113878in}%
\pgfsys@useobject{currentmarker}{}%
\end{pgfscope}%
\end{pgfscope}%
\begin{pgfscope}%
\definecolor{textcolor}{rgb}{0.000000,0.000000,0.000000}%
\pgfsetstrokecolor{textcolor}%
\pgfsetfillcolor{textcolor}%
\pgftext[x=0.398888in,y=1.061117in,left,base]{\color{textcolor}\sffamily\fontsize{10.000000}{12.000000}\selectfont 2}%
\end{pgfscope}%
\begin{pgfscope}%
\definecolor{textcolor}{rgb}{0.000000,0.000000,0.000000}%
\pgfsetstrokecolor{textcolor}%
\pgfsetfillcolor{textcolor}%
\pgftext[x=0.584475in,y=1.426682in,left,base]{\color{textcolor}\sffamily\fontsize{10.000000}{12.000000}\selectfont 1e5}%
\end{pgfscope}%
\begin{pgfscope}%
\pgfpathrectangle{\pgfqpoint{0.584475in}{0.571604in}}{\pgfqpoint{2.265525in}{0.813411in}}%
\pgfusepath{clip}%
\pgfsetrectcap%
\pgfsetroundjoin%
\pgfsetlinewidth{1.505625pt}%
\definecolor{currentstroke}{rgb}{1.000000,0.000000,0.000000}%
\pgfsetstrokecolor{currentstroke}%
\pgfsetdash{}{0pt}%
\pgfpathmoveto{\pgfqpoint{0.584475in}{1.173531in}}%
\pgfpathlineto{\pgfqpoint{0.607131in}{1.164166in}}%
\pgfpathlineto{\pgfqpoint{0.629786in}{1.156710in}}%
\pgfpathlineto{\pgfqpoint{0.652441in}{1.148668in}}%
\pgfpathlineto{\pgfqpoint{0.675096in}{1.138487in}}%
\pgfpathlineto{\pgfqpoint{0.697752in}{1.128281in}}%
\pgfpathlineto{\pgfqpoint{0.720407in}{1.119840in}}%
\pgfpathlineto{\pgfqpoint{0.743062in}{1.110847in}}%
\pgfpathlineto{\pgfqpoint{0.765717in}{1.100571in}}%
\pgfpathlineto{\pgfqpoint{0.788373in}{1.089389in}}%
\pgfpathlineto{\pgfqpoint{0.811028in}{1.079894in}}%
\pgfpathlineto{\pgfqpoint{0.833683in}{1.069954in}}%
\pgfpathlineto{\pgfqpoint{0.856338in}{1.059637in}}%
\pgfpathlineto{\pgfqpoint{0.878994in}{1.047577in}}%
\pgfpathlineto{\pgfqpoint{0.901649in}{1.037108in}}%
\pgfpathlineto{\pgfqpoint{0.924304in}{1.026285in}}%
\pgfpathlineto{\pgfqpoint{0.946959in}{1.015943in}}%
\pgfpathlineto{\pgfqpoint{0.969615in}{1.003016in}}%
\pgfpathlineto{\pgfqpoint{0.992270in}{0.990749in}}%
\pgfpathlineto{\pgfqpoint{1.014925in}{0.979982in}}%
\pgfpathlineto{\pgfqpoint{1.037580in}{0.969034in}}%
\pgfpathlineto{\pgfqpoint{1.060236in}{0.956158in}}%
\pgfpathlineto{\pgfqpoint{1.082891in}{0.943295in}}%
\pgfpathlineto{\pgfqpoint{1.105546in}{0.931880in}}%
\pgfpathlineto{\pgfqpoint{1.128201in}{0.920419in}}%
\pgfpathlineto{\pgfqpoint{1.150857in}{0.907912in}}%
\pgfpathlineto{\pgfqpoint{1.173512in}{0.894542in}}%
\pgfpathlineto{\pgfqpoint{1.196167in}{0.882479in}}%
\pgfpathlineto{\pgfqpoint{1.218822in}{0.870438in}}%
\pgfpathlineto{\pgfqpoint{1.241478in}{0.858177in}}%
\pgfpathlineto{\pgfqpoint{1.264133in}{0.844270in}}%
\pgfpathlineto{\pgfqpoint{1.286788in}{0.831622in}}%
\pgfpathlineto{\pgfqpoint{1.309443in}{0.819170in}}%
\pgfpathlineto{\pgfqpoint{1.332099in}{0.807400in}}%
\pgfpathlineto{\pgfqpoint{1.354754in}{0.793196in}}%
\pgfpathlineto{\pgfqpoint{1.377409in}{0.779669in}}%
\pgfpathlineto{\pgfqpoint{1.400064in}{0.767184in}}%
\pgfpathlineto{\pgfqpoint{1.422720in}{0.754920in}}%
\pgfpathlineto{\pgfqpoint{1.445375in}{0.740826in}}%
\pgfpathlineto{\pgfqpoint{1.468030in}{0.726639in}}%
\pgfpathlineto{\pgfqpoint{1.490685in}{0.713315in}}%
\pgfpathlineto{\pgfqpoint{1.513340in}{0.700163in}}%
\pgfpathlineto{\pgfqpoint{1.535996in}{0.685558in}}%
\pgfpathlineto{\pgfqpoint{1.558651in}{0.669217in}}%
\pgfpathlineto{\pgfqpoint{1.581306in}{0.652899in}}%
\pgfpathlineto{\pgfqpoint{1.603961in}{0.636310in}}%
\pgfpathlineto{\pgfqpoint{1.626617in}{0.618953in}}%
\pgfpathlineto{\pgfqpoint{1.649272in}{0.600677in}}%
\pgfpathlineto{\pgfqpoint{1.671927in}{0.585723in}}%
\pgfpathlineto{\pgfqpoint{1.694582in}{0.575501in}}%
\pgfpathlineto{\pgfqpoint{1.717238in}{0.571604in}}%
\pgfpathlineto{\pgfqpoint{1.739893in}{0.575501in}}%
\pgfpathlineto{\pgfqpoint{1.762548in}{0.585723in}}%
\pgfpathlineto{\pgfqpoint{1.785203in}{0.600677in}}%
\pgfpathlineto{\pgfqpoint{1.807859in}{0.618953in}}%
\pgfpathlineto{\pgfqpoint{1.830514in}{0.636311in}}%
\pgfpathlineto{\pgfqpoint{1.853169in}{0.652899in}}%
\pgfpathlineto{\pgfqpoint{1.875824in}{0.669217in}}%
\pgfpathlineto{\pgfqpoint{1.898480in}{0.685558in}}%
\pgfpathlineto{\pgfqpoint{1.921135in}{0.700163in}}%
\pgfpathlineto{\pgfqpoint{1.943790in}{0.713315in}}%
\pgfpathlineto{\pgfqpoint{1.966445in}{0.726639in}}%
\pgfpathlineto{\pgfqpoint{1.989101in}{0.740826in}}%
\pgfpathlineto{\pgfqpoint{2.011756in}{0.754920in}}%
\pgfpathlineto{\pgfqpoint{2.034411in}{0.767185in}}%
\pgfpathlineto{\pgfqpoint{2.057066in}{0.779669in}}%
\pgfpathlineto{\pgfqpoint{2.079722in}{0.793196in}}%
\pgfpathlineto{\pgfqpoint{2.102377in}{0.807400in}}%
\pgfpathlineto{\pgfqpoint{2.125032in}{0.819171in}}%
\pgfpathlineto{\pgfqpoint{2.147687in}{0.831622in}}%
\pgfpathlineto{\pgfqpoint{2.170343in}{0.844270in}}%
\pgfpathlineto{\pgfqpoint{2.192998in}{0.858177in}}%
\pgfpathlineto{\pgfqpoint{2.215653in}{0.870438in}}%
\pgfpathlineto{\pgfqpoint{2.238308in}{0.882479in}}%
\pgfpathlineto{\pgfqpoint{2.260964in}{0.894542in}}%
\pgfpathlineto{\pgfqpoint{2.283619in}{0.907912in}}%
\pgfpathlineto{\pgfqpoint{2.306274in}{0.920419in}}%
\pgfpathlineto{\pgfqpoint{2.328929in}{0.931880in}}%
\pgfpathlineto{\pgfqpoint{2.351585in}{0.943295in}}%
\pgfpathlineto{\pgfqpoint{2.374240in}{0.956158in}}%
\pgfpathlineto{\pgfqpoint{2.396895in}{0.969034in}}%
\pgfpathlineto{\pgfqpoint{2.419550in}{0.979985in}}%
\pgfpathlineto{\pgfqpoint{2.442206in}{0.990749in}}%
\pgfpathlineto{\pgfqpoint{2.464861in}{1.003016in}}%
\pgfpathlineto{\pgfqpoint{2.487516in}{1.015943in}}%
\pgfpathlineto{\pgfqpoint{2.510171in}{1.026285in}}%
\pgfpathlineto{\pgfqpoint{2.532827in}{1.037108in}}%
\pgfpathlineto{\pgfqpoint{2.555482in}{1.047577in}}%
\pgfpathlineto{\pgfqpoint{2.578137in}{1.059637in}}%
\pgfpathlineto{\pgfqpoint{2.600792in}{1.069954in}}%
\pgfpathlineto{\pgfqpoint{2.623448in}{1.079894in}}%
\pgfpathlineto{\pgfqpoint{2.646103in}{1.089389in}}%
\pgfpathlineto{\pgfqpoint{2.668758in}{1.100571in}}%
\pgfpathlineto{\pgfqpoint{2.691413in}{1.110850in}}%
\pgfpathlineto{\pgfqpoint{2.714069in}{1.119840in}}%
\pgfpathlineto{\pgfqpoint{2.736724in}{1.128281in}}%
\pgfpathlineto{\pgfqpoint{2.759379in}{1.138489in}}%
\pgfpathlineto{\pgfqpoint{2.782034in}{1.148668in}}%
\pgfpathlineto{\pgfqpoint{2.804690in}{1.156710in}}%
\pgfpathlineto{\pgfqpoint{2.827345in}{1.164169in}}%
\pgfpathlineto{\pgfqpoint{2.850000in}{1.173531in}}%
\pgfusepath{stroke}%
\end{pgfscope}%
\begin{pgfscope}%
\pgfsetrectcap%
\pgfsetmiterjoin%
\pgfsetlinewidth{0.803000pt}%
\definecolor{currentstroke}{rgb}{0.000000,0.000000,0.000000}%
\pgfsetstrokecolor{currentstroke}%
\pgfsetdash{}{0pt}%
\pgfpathmoveto{\pgfqpoint{0.584475in}{0.571604in}}%
\pgfpathlineto{\pgfqpoint{0.584475in}{1.385015in}}%
\pgfusepath{stroke}%
\end{pgfscope}%
\begin{pgfscope}%
\pgfsetrectcap%
\pgfsetmiterjoin%
\pgfsetlinewidth{0.803000pt}%
\definecolor{currentstroke}{rgb}{0.000000,0.000000,0.000000}%
\pgfsetstrokecolor{currentstroke}%
\pgfsetdash{}{0pt}%
\pgfpathmoveto{\pgfqpoint{2.850000in}{0.571604in}}%
\pgfpathlineto{\pgfqpoint{2.850000in}{1.385015in}}%
\pgfusepath{stroke}%
\end{pgfscope}%
\begin{pgfscope}%
\pgfsetrectcap%
\pgfsetmiterjoin%
\pgfsetlinewidth{0.803000pt}%
\definecolor{currentstroke}{rgb}{0.000000,0.000000,0.000000}%
\pgfsetstrokecolor{currentstroke}%
\pgfsetdash{}{0pt}%
\pgfpathmoveto{\pgfqpoint{0.584475in}{0.571604in}}%
\pgfpathlineto{\pgfqpoint{2.850000in}{0.571604in}}%
\pgfusepath{stroke}%
\end{pgfscope}%
\begin{pgfscope}%
\pgfsetrectcap%
\pgfsetmiterjoin%
\pgfsetlinewidth{0.803000pt}%
\definecolor{currentstroke}{rgb}{0.000000,0.000000,0.000000}%
\pgfsetstrokecolor{currentstroke}%
\pgfsetdash{}{0pt}%
\pgfpathmoveto{\pgfqpoint{0.584475in}{1.385015in}}%
\pgfpathlineto{\pgfqpoint{2.850000in}{1.385015in}}%
\pgfusepath{stroke}%
\end{pgfscope}%
\begin{pgfscope}%
\definecolor{textcolor}{rgb}{1.000000,0.000000,0.000000}%
\pgfsetstrokecolor{textcolor}%
\pgfsetfillcolor{textcolor}%
\pgftext[x=1.717238in,y=1.468349in,,base]{\color{textcolor}\sffamily\fontsize{12.000000}{14.400000}\selectfont MSD metric value}%
\end{pgfscope}%
\end{pgfpicture}%
\makeatother%
\endgroup%
}
  \hfill
  \resizebox{0.32\linewidth}{!}{%% Creator: Matplotlib, PGF backend
%%
%% To include the figure in your LaTeX document, write
%%   \input{<filename>.pgf}
%%
%% Make sure the required packages are loaded in your preamble
%%   \usepackage{pgf}
%%
%% and, on pdftex
%%   \usepackage[utf8]{inputenc}\DeclareUnicodeCharacter{2212}{-}
%%
%% or, on luatex and xetex
%%   \usepackage{unicode-math}
%%
%% Figures using additional raster images can only be included by \input if
%% they are in the same directory as the main LaTeX file. For loading figures
%% from other directories you can use the `import` package
%%   \usepackage{import}
%%
%% and then include the figures with
%%   \import{<path to file>}{<filename>.pgf}
%%
%% Matplotlib used the following preamble
%%   \usepackage{fontspec}
%%   \setmainfont{DejaVuSerif.ttf}[Path=/usr/share/matplotlib/mpl-data/fonts/ttf/]
%%   \setsansfont{DejaVuSans.ttf}[Path=/usr/share/matplotlib/mpl-data/fonts/ttf/]
%%   \setmonofont{DejaVuSansMono.ttf}[Path=/usr/share/matplotlib/mpl-data/fonts/ttf/]
%%
\begingroup%
\makeatletter%
\begin{pgfpicture}%
\pgfpathrectangle{\pgfpointorigin}{\pgfqpoint{3.000000in}{3.000000in}}%
\pgfusepath{use as bounding box, clip}%
\begin{pgfscope}%
\pgfsetbuttcap%
\pgfsetmiterjoin%
\definecolor{currentfill}{rgb}{1.000000,1.000000,1.000000}%
\pgfsetfillcolor{currentfill}%
\pgfsetlinewidth{0.000000pt}%
\definecolor{currentstroke}{rgb}{1.000000,1.000000,1.000000}%
\pgfsetstrokecolor{currentstroke}%
\pgfsetdash{}{0pt}%
\pgfpathmoveto{\pgfqpoint{0.000000in}{0.000000in}}%
\pgfpathlineto{\pgfqpoint{3.000000in}{0.000000in}}%
\pgfpathlineto{\pgfqpoint{3.000000in}{3.000000in}}%
\pgfpathlineto{\pgfqpoint{0.000000in}{3.000000in}}%
\pgfpathclose%
\pgfusepath{fill}%
\end{pgfscope}%
\begin{pgfscope}%
\pgfsetbuttcap%
\pgfsetmiterjoin%
\definecolor{currentfill}{rgb}{1.000000,1.000000,1.000000}%
\pgfsetfillcolor{currentfill}%
\pgfsetlinewidth{0.000000pt}%
\definecolor{currentstroke}{rgb}{0.000000,0.000000,0.000000}%
\pgfsetstrokecolor{currentstroke}%
\pgfsetstrokeopacity{0.000000}%
\pgfsetdash{}{0pt}%
\pgfpathmoveto{\pgfqpoint{0.584475in}{1.810107in}}%
\pgfpathlineto{\pgfqpoint{2.761635in}{1.810107in}}%
\pgfpathlineto{\pgfqpoint{2.761635in}{2.640039in}}%
\pgfpathlineto{\pgfqpoint{0.584475in}{2.640039in}}%
\pgfpathclose%
\pgfusepath{fill}%
\end{pgfscope}%
\begin{pgfscope}%
\pgfpathrectangle{\pgfqpoint{0.584475in}{1.810107in}}{\pgfqpoint{2.177159in}{0.829932in}}%
\pgfusepath{clip}%
\pgfsetrectcap%
\pgfsetroundjoin%
\pgfsetlinewidth{0.803000pt}%
\definecolor{currentstroke}{rgb}{0.690196,0.690196,0.690196}%
\pgfsetstrokecolor{currentstroke}%
\pgfsetdash{}{0pt}%
\pgfpathmoveto{\pgfqpoint{0.584475in}{1.810107in}}%
\pgfpathlineto{\pgfqpoint{0.584475in}{2.640039in}}%
\pgfusepath{stroke}%
\end{pgfscope}%
\begin{pgfscope}%
\pgfsetbuttcap%
\pgfsetroundjoin%
\definecolor{currentfill}{rgb}{0.000000,0.000000,0.000000}%
\pgfsetfillcolor{currentfill}%
\pgfsetlinewidth{0.803000pt}%
\definecolor{currentstroke}{rgb}{0.000000,0.000000,0.000000}%
\pgfsetstrokecolor{currentstroke}%
\pgfsetdash{}{0pt}%
\pgfsys@defobject{currentmarker}{\pgfqpoint{0.000000in}{-0.048611in}}{\pgfqpoint{0.000000in}{0.000000in}}{%
\pgfpathmoveto{\pgfqpoint{0.000000in}{0.000000in}}%
\pgfpathlineto{\pgfqpoint{0.000000in}{-0.048611in}}%
\pgfusepath{stroke,fill}%
}%
\begin{pgfscope}%
\pgfsys@transformshift{0.584475in}{1.810107in}%
\pgfsys@useobject{currentmarker}{}%
\end{pgfscope}%
\end{pgfscope}%
\begin{pgfscope}%
\pgfpathrectangle{\pgfqpoint{0.584475in}{1.810107in}}{\pgfqpoint{2.177159in}{0.829932in}}%
\pgfusepath{clip}%
\pgfsetrectcap%
\pgfsetroundjoin%
\pgfsetlinewidth{0.803000pt}%
\definecolor{currentstroke}{rgb}{0.690196,0.690196,0.690196}%
\pgfsetstrokecolor{currentstroke}%
\pgfsetdash{}{0pt}%
\pgfpathmoveto{\pgfqpoint{1.310195in}{1.810107in}}%
\pgfpathlineto{\pgfqpoint{1.310195in}{2.640039in}}%
\pgfusepath{stroke}%
\end{pgfscope}%
\begin{pgfscope}%
\pgfsetbuttcap%
\pgfsetroundjoin%
\definecolor{currentfill}{rgb}{0.000000,0.000000,0.000000}%
\pgfsetfillcolor{currentfill}%
\pgfsetlinewidth{0.803000pt}%
\definecolor{currentstroke}{rgb}{0.000000,0.000000,0.000000}%
\pgfsetstrokecolor{currentstroke}%
\pgfsetdash{}{0pt}%
\pgfsys@defobject{currentmarker}{\pgfqpoint{0.000000in}{-0.048611in}}{\pgfqpoint{0.000000in}{0.000000in}}{%
\pgfpathmoveto{\pgfqpoint{0.000000in}{0.000000in}}%
\pgfpathlineto{\pgfqpoint{0.000000in}{-0.048611in}}%
\pgfusepath{stroke,fill}%
}%
\begin{pgfscope}%
\pgfsys@transformshift{1.310195in}{1.810107in}%
\pgfsys@useobject{currentmarker}{}%
\end{pgfscope}%
\end{pgfscope}%
\begin{pgfscope}%
\pgfpathrectangle{\pgfqpoint{0.584475in}{1.810107in}}{\pgfqpoint{2.177159in}{0.829932in}}%
\pgfusepath{clip}%
\pgfsetrectcap%
\pgfsetroundjoin%
\pgfsetlinewidth{0.803000pt}%
\definecolor{currentstroke}{rgb}{0.690196,0.690196,0.690196}%
\pgfsetstrokecolor{currentstroke}%
\pgfsetdash{}{0pt}%
\pgfpathmoveto{\pgfqpoint{2.035915in}{1.810107in}}%
\pgfpathlineto{\pgfqpoint{2.035915in}{2.640039in}}%
\pgfusepath{stroke}%
\end{pgfscope}%
\begin{pgfscope}%
\pgfsetbuttcap%
\pgfsetroundjoin%
\definecolor{currentfill}{rgb}{0.000000,0.000000,0.000000}%
\pgfsetfillcolor{currentfill}%
\pgfsetlinewidth{0.803000pt}%
\definecolor{currentstroke}{rgb}{0.000000,0.000000,0.000000}%
\pgfsetstrokecolor{currentstroke}%
\pgfsetdash{}{0pt}%
\pgfsys@defobject{currentmarker}{\pgfqpoint{0.000000in}{-0.048611in}}{\pgfqpoint{0.000000in}{0.000000in}}{%
\pgfpathmoveto{\pgfqpoint{0.000000in}{0.000000in}}%
\pgfpathlineto{\pgfqpoint{0.000000in}{-0.048611in}}%
\pgfusepath{stroke,fill}%
}%
\begin{pgfscope}%
\pgfsys@transformshift{2.035915in}{1.810107in}%
\pgfsys@useobject{currentmarker}{}%
\end{pgfscope}%
\end{pgfscope}%
\begin{pgfscope}%
\pgfpathrectangle{\pgfqpoint{0.584475in}{1.810107in}}{\pgfqpoint{2.177159in}{0.829932in}}%
\pgfusepath{clip}%
\pgfsetrectcap%
\pgfsetroundjoin%
\pgfsetlinewidth{0.803000pt}%
\definecolor{currentstroke}{rgb}{0.690196,0.690196,0.690196}%
\pgfsetstrokecolor{currentstroke}%
\pgfsetdash{}{0pt}%
\pgfpathmoveto{\pgfqpoint{2.761635in}{1.810107in}}%
\pgfpathlineto{\pgfqpoint{2.761635in}{2.640039in}}%
\pgfusepath{stroke}%
\end{pgfscope}%
\begin{pgfscope}%
\pgfsetbuttcap%
\pgfsetroundjoin%
\definecolor{currentfill}{rgb}{0.000000,0.000000,0.000000}%
\pgfsetfillcolor{currentfill}%
\pgfsetlinewidth{0.803000pt}%
\definecolor{currentstroke}{rgb}{0.000000,0.000000,0.000000}%
\pgfsetstrokecolor{currentstroke}%
\pgfsetdash{}{0pt}%
\pgfsys@defobject{currentmarker}{\pgfqpoint{0.000000in}{-0.048611in}}{\pgfqpoint{0.000000in}{0.000000in}}{%
\pgfpathmoveto{\pgfqpoint{0.000000in}{0.000000in}}%
\pgfpathlineto{\pgfqpoint{0.000000in}{-0.048611in}}%
\pgfusepath{stroke,fill}%
}%
\begin{pgfscope}%
\pgfsys@transformshift{2.761635in}{1.810107in}%
\pgfsys@useobject{currentmarker}{}%
\end{pgfscope}%
\end{pgfscope}%
\begin{pgfscope}%
\pgfpathrectangle{\pgfqpoint{0.584475in}{1.810107in}}{\pgfqpoint{2.177159in}{0.829932in}}%
\pgfusepath{clip}%
\pgfsetrectcap%
\pgfsetroundjoin%
\pgfsetlinewidth{0.803000pt}%
\definecolor{currentstroke}{rgb}{0.690196,0.690196,0.690196}%
\pgfsetstrokecolor{currentstroke}%
\pgfsetdash{}{0pt}%
\pgfpathmoveto{\pgfqpoint{0.584475in}{2.017590in}}%
\pgfpathlineto{\pgfqpoint{2.761635in}{2.017590in}}%
\pgfusepath{stroke}%
\end{pgfscope}%
\begin{pgfscope}%
\pgfsetbuttcap%
\pgfsetroundjoin%
\definecolor{currentfill}{rgb}{0.000000,0.000000,0.000000}%
\pgfsetfillcolor{currentfill}%
\pgfsetlinewidth{0.803000pt}%
\definecolor{currentstroke}{rgb}{0.000000,0.000000,0.000000}%
\pgfsetstrokecolor{currentstroke}%
\pgfsetdash{}{0pt}%
\pgfsys@defobject{currentmarker}{\pgfqpoint{-0.048611in}{0.000000in}}{\pgfqpoint{0.000000in}{0.000000in}}{%
\pgfpathmoveto{\pgfqpoint{0.000000in}{0.000000in}}%
\pgfpathlineto{\pgfqpoint{-0.048611in}{0.000000in}}%
\pgfusepath{stroke,fill}%
}%
\begin{pgfscope}%
\pgfsys@transformshift{0.584475in}{2.017590in}%
\pgfsys@useobject{currentmarker}{}%
\end{pgfscope}%
\end{pgfscope}%
\begin{pgfscope}%
\definecolor{textcolor}{rgb}{0.000000,0.000000,0.000000}%
\pgfsetstrokecolor{textcolor}%
\pgfsetfillcolor{textcolor}%
\pgftext[x=0.150000in, y=1.964829in, left, base]{\color{textcolor}\sffamily\fontsize{10.000000}{12.000000}\selectfont −1.0}%
\end{pgfscope}%
\begin{pgfscope}%
\pgfpathrectangle{\pgfqpoint{0.584475in}{1.810107in}}{\pgfqpoint{2.177159in}{0.829932in}}%
\pgfusepath{clip}%
\pgfsetrectcap%
\pgfsetroundjoin%
\pgfsetlinewidth{0.803000pt}%
\definecolor{currentstroke}{rgb}{0.690196,0.690196,0.690196}%
\pgfsetstrokecolor{currentstroke}%
\pgfsetdash{}{0pt}%
\pgfpathmoveto{\pgfqpoint{0.584475in}{2.432556in}}%
\pgfpathlineto{\pgfqpoint{2.761635in}{2.432556in}}%
\pgfusepath{stroke}%
\end{pgfscope}%
\begin{pgfscope}%
\pgfsetbuttcap%
\pgfsetroundjoin%
\definecolor{currentfill}{rgb}{0.000000,0.000000,0.000000}%
\pgfsetfillcolor{currentfill}%
\pgfsetlinewidth{0.803000pt}%
\definecolor{currentstroke}{rgb}{0.000000,0.000000,0.000000}%
\pgfsetstrokecolor{currentstroke}%
\pgfsetdash{}{0pt}%
\pgfsys@defobject{currentmarker}{\pgfqpoint{-0.048611in}{0.000000in}}{\pgfqpoint{0.000000in}{0.000000in}}{%
\pgfpathmoveto{\pgfqpoint{0.000000in}{0.000000in}}%
\pgfpathlineto{\pgfqpoint{-0.048611in}{0.000000in}}%
\pgfusepath{stroke,fill}%
}%
\begin{pgfscope}%
\pgfsys@transformshift{0.584475in}{2.432556in}%
\pgfsys@useobject{currentmarker}{}%
\end{pgfscope}%
\end{pgfscope}%
\begin{pgfscope}%
\definecolor{textcolor}{rgb}{0.000000,0.000000,0.000000}%
\pgfsetstrokecolor{textcolor}%
\pgfsetfillcolor{textcolor}%
\pgftext[x=0.150000in, y=2.379795in, left, base]{\color{textcolor}\sffamily\fontsize{10.000000}{12.000000}\selectfont −0.5}%
\end{pgfscope}%
\begin{pgfscope}%
\pgfpathrectangle{\pgfqpoint{0.584475in}{1.810107in}}{\pgfqpoint{2.177159in}{0.829932in}}%
\pgfusepath{clip}%
\pgfsetrectcap%
\pgfsetroundjoin%
\pgfsetlinewidth{1.505625pt}%
\definecolor{currentstroke}{rgb}{0.000000,0.000000,1.000000}%
\pgfsetstrokecolor{currentstroke}%
\pgfsetdash{}{0pt}%
\pgfpathmoveto{\pgfqpoint{0.584475in}{1.813602in}}%
\pgfpathlineto{\pgfqpoint{0.591733in}{1.859804in}}%
\pgfpathlineto{\pgfqpoint{0.598990in}{1.888362in}}%
\pgfpathlineto{\pgfqpoint{0.606247in}{1.909791in}}%
\pgfpathlineto{\pgfqpoint{0.613504in}{1.927551in}}%
\pgfpathlineto{\pgfqpoint{0.620761in}{1.943187in}}%
\pgfpathlineto{\pgfqpoint{0.628019in}{1.957371in}}%
\pgfpathlineto{\pgfqpoint{0.642533in}{1.982874in}}%
\pgfpathlineto{\pgfqpoint{0.657047in}{2.006021in}}%
\pgfpathlineto{\pgfqpoint{0.671562in}{2.027621in}}%
\pgfpathlineto{\pgfqpoint{0.686076in}{2.047938in}}%
\pgfpathlineto{\pgfqpoint{0.700591in}{2.067142in}}%
\pgfpathlineto{\pgfqpoint{0.715105in}{2.085383in}}%
\pgfpathlineto{\pgfqpoint{0.722362in}{2.091322in}}%
\pgfpathlineto{\pgfqpoint{0.744134in}{2.117013in}}%
\pgfpathlineto{\pgfqpoint{0.765905in}{2.141314in}}%
\pgfpathlineto{\pgfqpoint{0.809449in}{2.189024in}}%
\pgfpathlineto{\pgfqpoint{0.823963in}{2.206157in}}%
\pgfpathlineto{\pgfqpoint{0.831220in}{2.215338in}}%
\pgfpathlineto{\pgfqpoint{0.838477in}{2.225131in}}%
\pgfpathlineto{\pgfqpoint{0.845735in}{2.235777in}}%
\pgfpathlineto{\pgfqpoint{0.852992in}{2.247120in}}%
\pgfpathlineto{\pgfqpoint{0.860249in}{2.256331in}}%
\pgfpathlineto{\pgfqpoint{0.867506in}{2.259993in}}%
\pgfpathlineto{\pgfqpoint{0.874763in}{2.262409in}}%
\pgfpathlineto{\pgfqpoint{0.882021in}{2.264414in}}%
\pgfpathlineto{\pgfqpoint{0.889278in}{2.265476in}}%
\pgfpathlineto{\pgfqpoint{0.918306in}{2.268211in}}%
\pgfpathlineto{\pgfqpoint{0.940078in}{2.271202in}}%
\pgfpathlineto{\pgfqpoint{0.961850in}{2.274888in}}%
\pgfpathlineto{\pgfqpoint{0.998136in}{2.282022in}}%
\pgfpathlineto{\pgfqpoint{1.005393in}{2.283566in}}%
\pgfpathlineto{\pgfqpoint{1.012650in}{2.279269in}}%
\pgfpathlineto{\pgfqpoint{1.048936in}{2.287884in}}%
\pgfpathlineto{\pgfqpoint{1.070708in}{2.293730in}}%
\pgfpathlineto{\pgfqpoint{1.085222in}{2.298165in}}%
\pgfpathlineto{\pgfqpoint{1.099736in}{2.303240in}}%
\pgfpathlineto{\pgfqpoint{1.114251in}{2.309453in}}%
\pgfpathlineto{\pgfqpoint{1.121508in}{2.313175in}}%
\pgfpathlineto{\pgfqpoint{1.128765in}{2.317489in}}%
\pgfpathlineto{\pgfqpoint{1.136022in}{2.322648in}}%
\pgfpathlineto{\pgfqpoint{1.143280in}{2.328572in}}%
\pgfpathlineto{\pgfqpoint{1.150537in}{2.332487in}}%
\pgfpathlineto{\pgfqpoint{1.157794in}{2.331692in}}%
\pgfpathlineto{\pgfqpoint{1.165051in}{2.333460in}}%
\pgfpathlineto{\pgfqpoint{1.172308in}{2.334644in}}%
\pgfpathlineto{\pgfqpoint{1.179566in}{2.335098in}}%
\pgfpathlineto{\pgfqpoint{1.201337in}{2.335523in}}%
\pgfpathlineto{\pgfqpoint{1.230366in}{2.337478in}}%
\pgfpathlineto{\pgfqpoint{1.259395in}{2.340369in}}%
\pgfpathlineto{\pgfqpoint{1.302938in}{2.345443in}}%
\pgfpathlineto{\pgfqpoint{1.310195in}{2.340015in}}%
\pgfpathlineto{\pgfqpoint{1.346481in}{2.344890in}}%
\pgfpathlineto{\pgfqpoint{1.368253in}{2.348435in}}%
\pgfpathlineto{\pgfqpoint{1.382767in}{2.351335in}}%
\pgfpathlineto{\pgfqpoint{1.397282in}{2.354940in}}%
\pgfpathlineto{\pgfqpoint{1.411796in}{2.359664in}}%
\pgfpathlineto{\pgfqpoint{1.419053in}{2.362717in}}%
\pgfpathlineto{\pgfqpoint{1.426310in}{2.366469in}}%
\pgfpathlineto{\pgfqpoint{1.433568in}{2.370840in}}%
\pgfpathlineto{\pgfqpoint{1.440825in}{2.373224in}}%
\pgfpathlineto{\pgfqpoint{1.448082in}{2.371495in}}%
\pgfpathlineto{\pgfqpoint{1.455339in}{2.373144in}}%
\pgfpathlineto{\pgfqpoint{1.462596in}{2.374099in}}%
\pgfpathlineto{\pgfqpoint{1.469854in}{2.374404in}}%
\pgfpathlineto{\pgfqpoint{1.498882in}{2.374331in}}%
\pgfpathlineto{\pgfqpoint{1.527911in}{2.375551in}}%
\pgfpathlineto{\pgfqpoint{1.571454in}{2.378540in}}%
\pgfpathlineto{\pgfqpoint{1.593226in}{2.380273in}}%
\pgfpathlineto{\pgfqpoint{1.600483in}{2.373643in}}%
\pgfpathlineto{\pgfqpoint{1.636769in}{2.377249in}}%
\pgfpathlineto{\pgfqpoint{1.658541in}{2.380057in}}%
\pgfpathlineto{\pgfqpoint{1.673055in}{2.382374in}}%
\pgfpathlineto{\pgfqpoint{1.687569in}{2.385244in}}%
\pgfpathlineto{\pgfqpoint{1.702084in}{2.389079in}}%
\pgfpathlineto{\pgfqpoint{1.709341in}{2.391564in}}%
\pgfpathlineto{\pgfqpoint{1.716598in}{2.394663in}}%
\pgfpathlineto{\pgfqpoint{1.723855in}{2.398276in}}%
\pgfpathlineto{\pgfqpoint{1.731113in}{2.399966in}}%
\pgfpathlineto{\pgfqpoint{1.738370in}{2.397872in}}%
\pgfpathlineto{\pgfqpoint{1.745627in}{2.399316in}}%
\pgfpathlineto{\pgfqpoint{1.752884in}{2.399893in}}%
\pgfpathlineto{\pgfqpoint{1.767399in}{2.399620in}}%
\pgfpathlineto{\pgfqpoint{1.781913in}{2.399220in}}%
\pgfpathlineto{\pgfqpoint{1.810942in}{2.399595in}}%
\pgfpathlineto{\pgfqpoint{1.839971in}{2.400812in}}%
\pgfpathlineto{\pgfqpoint{1.883514in}{2.403361in}}%
\pgfpathlineto{\pgfqpoint{1.890771in}{2.396417in}}%
\pgfpathlineto{\pgfqpoint{1.927057in}{2.399639in}}%
\pgfpathlineto{\pgfqpoint{1.948829in}{2.402107in}}%
\pgfpathlineto{\pgfqpoint{1.970600in}{2.405291in}}%
\pgfpathlineto{\pgfqpoint{1.985115in}{2.408091in}}%
\pgfpathlineto{\pgfqpoint{1.992372in}{2.409879in}}%
\pgfpathlineto{\pgfqpoint{1.999629in}{2.412078in}}%
\pgfpathlineto{\pgfqpoint{2.006886in}{2.414962in}}%
\pgfpathlineto{\pgfqpoint{2.014143in}{2.418488in}}%
\pgfpathlineto{\pgfqpoint{2.021401in}{2.420088in}}%
\pgfpathlineto{\pgfqpoint{2.028658in}{2.417918in}}%
\pgfpathlineto{\pgfqpoint{2.035915in}{2.419116in}}%
\pgfpathlineto{\pgfqpoint{2.043172in}{2.419686in}}%
\pgfpathlineto{\pgfqpoint{2.057686in}{2.419649in}}%
\pgfpathlineto{\pgfqpoint{2.079458in}{2.419222in}}%
\pgfpathlineto{\pgfqpoint{2.108487in}{2.419401in}}%
\pgfpathlineto{\pgfqpoint{2.144773in}{2.420619in}}%
\pgfpathlineto{\pgfqpoint{2.166544in}{2.421687in}}%
\pgfpathlineto{\pgfqpoint{2.173802in}{2.414025in}}%
\pgfpathlineto{\pgfqpoint{2.210088in}{2.416792in}}%
\pgfpathlineto{\pgfqpoint{2.239116in}{2.419752in}}%
\pgfpathlineto{\pgfqpoint{2.260888in}{2.422702in}}%
\pgfpathlineto{\pgfqpoint{2.275402in}{2.425299in}}%
\pgfpathlineto{\pgfqpoint{2.282660in}{2.426977in}}%
\pgfpathlineto{\pgfqpoint{2.289917in}{2.429043in}}%
\pgfpathlineto{\pgfqpoint{2.297174in}{2.431758in}}%
\pgfpathlineto{\pgfqpoint{2.304431in}{2.435043in}}%
\pgfpathlineto{\pgfqpoint{2.311688in}{2.436280in}}%
\pgfpathlineto{\pgfqpoint{2.318946in}{2.434009in}}%
\pgfpathlineto{\pgfqpoint{2.326203in}{2.435098in}}%
\pgfpathlineto{\pgfqpoint{2.340717in}{2.435747in}}%
\pgfpathlineto{\pgfqpoint{2.406032in}{2.435840in}}%
\pgfpathlineto{\pgfqpoint{2.442318in}{2.436975in}}%
\pgfpathlineto{\pgfqpoint{2.456832in}{2.437637in}}%
\pgfpathlineto{\pgfqpoint{2.464090in}{2.429365in}}%
\pgfpathlineto{\pgfqpoint{2.493118in}{2.431369in}}%
\pgfpathlineto{\pgfqpoint{2.522147in}{2.434036in}}%
\pgfpathlineto{\pgfqpoint{2.543919in}{2.436590in}}%
\pgfpathlineto{\pgfqpoint{2.558433in}{2.438764in}}%
\pgfpathlineto{\pgfqpoint{2.572948in}{2.441673in}}%
\pgfpathlineto{\pgfqpoint{2.580205in}{2.443625in}}%
\pgfpathlineto{\pgfqpoint{2.587462in}{2.446199in}}%
\pgfpathlineto{\pgfqpoint{2.594719in}{2.449281in}}%
\pgfpathlineto{\pgfqpoint{2.601976in}{2.450267in}}%
\pgfpathlineto{\pgfqpoint{2.609234in}{2.447915in}}%
\pgfpathlineto{\pgfqpoint{2.616491in}{2.448819in}}%
\pgfpathlineto{\pgfqpoint{2.631005in}{2.449522in}}%
\pgfpathlineto{\pgfqpoint{2.696320in}{2.449462in}}%
\pgfpathlineto{\pgfqpoint{2.739863in}{2.450764in}}%
\pgfpathlineto{\pgfqpoint{2.747120in}{2.451077in}}%
\pgfpathlineto{\pgfqpoint{2.754377in}{2.442192in}}%
\pgfpathlineto{\pgfqpoint{2.761635in}{2.442616in}}%
\pgfpathlineto{\pgfqpoint{2.761635in}{2.442616in}}%
\pgfusepath{stroke}%
\end{pgfscope}%
\begin{pgfscope}%
\pgfsetrectcap%
\pgfsetmiterjoin%
\pgfsetlinewidth{0.803000pt}%
\definecolor{currentstroke}{rgb}{0.000000,0.000000,0.000000}%
\pgfsetstrokecolor{currentstroke}%
\pgfsetdash{}{0pt}%
\pgfpathmoveto{\pgfqpoint{0.584475in}{1.810107in}}%
\pgfpathlineto{\pgfqpoint{0.584475in}{2.640039in}}%
\pgfusepath{stroke}%
\end{pgfscope}%
\begin{pgfscope}%
\pgfsetrectcap%
\pgfsetmiterjoin%
\pgfsetlinewidth{0.803000pt}%
\definecolor{currentstroke}{rgb}{0.000000,0.000000,0.000000}%
\pgfsetstrokecolor{currentstroke}%
\pgfsetdash{}{0pt}%
\pgfpathmoveto{\pgfqpoint{2.761635in}{1.810107in}}%
\pgfpathlineto{\pgfqpoint{2.761635in}{2.640039in}}%
\pgfusepath{stroke}%
\end{pgfscope}%
\begin{pgfscope}%
\pgfsetrectcap%
\pgfsetmiterjoin%
\pgfsetlinewidth{0.803000pt}%
\definecolor{currentstroke}{rgb}{0.000000,0.000000,0.000000}%
\pgfsetstrokecolor{currentstroke}%
\pgfsetdash{}{0pt}%
\pgfpathmoveto{\pgfqpoint{0.584475in}{1.810107in}}%
\pgfpathlineto{\pgfqpoint{2.761635in}{1.810107in}}%
\pgfusepath{stroke}%
\end{pgfscope}%
\begin{pgfscope}%
\pgfsetrectcap%
\pgfsetmiterjoin%
\pgfsetlinewidth{0.803000pt}%
\definecolor{currentstroke}{rgb}{0.000000,0.000000,0.000000}%
\pgfsetstrokecolor{currentstroke}%
\pgfsetdash{}{0pt}%
\pgfpathmoveto{\pgfqpoint{0.584475in}{2.640039in}}%
\pgfpathlineto{\pgfqpoint{2.761635in}{2.640039in}}%
\pgfusepath{stroke}%
\end{pgfscope}%
\begin{pgfscope}%
\definecolor{textcolor}{rgb}{0.000000,0.000000,1.000000}%
\pgfsetstrokecolor{textcolor}%
\pgfsetfillcolor{textcolor}%
\pgftext[x=1.673055in,y=2.723372in,,base]{\color{textcolor}\sffamily\fontsize{12.000000}{14.400000}\selectfont MI Metrik}%
\end{pgfscope}%
\begin{pgfscope}%
\pgfsetbuttcap%
\pgfsetmiterjoin%
\definecolor{currentfill}{rgb}{1.000000,1.000000,1.000000}%
\pgfsetfillcolor{currentfill}%
\pgfsetlinewidth{0.000000pt}%
\definecolor{currentstroke}{rgb}{0.000000,0.000000,0.000000}%
\pgfsetstrokecolor{currentstroke}%
\pgfsetstrokeopacity{0.000000}%
\pgfsetdash{}{0pt}%
\pgfpathmoveto{\pgfqpoint{0.584475in}{0.571604in}}%
\pgfpathlineto{\pgfqpoint{2.761635in}{0.571604in}}%
\pgfpathlineto{\pgfqpoint{2.761635in}{1.401535in}}%
\pgfpathlineto{\pgfqpoint{0.584475in}{1.401535in}}%
\pgfpathclose%
\pgfusepath{fill}%
\end{pgfscope}%
\begin{pgfscope}%
\pgfpathrectangle{\pgfqpoint{0.584475in}{0.571604in}}{\pgfqpoint{2.177159in}{0.829932in}}%
\pgfusepath{clip}%
\pgfsetrectcap%
\pgfsetroundjoin%
\pgfsetlinewidth{0.803000pt}%
\definecolor{currentstroke}{rgb}{0.690196,0.690196,0.690196}%
\pgfsetstrokecolor{currentstroke}%
\pgfsetdash{}{0pt}%
\pgfpathmoveto{\pgfqpoint{0.584475in}{0.571604in}}%
\pgfpathlineto{\pgfqpoint{0.584475in}{1.401535in}}%
\pgfusepath{stroke}%
\end{pgfscope}%
\begin{pgfscope}%
\pgfsetbuttcap%
\pgfsetroundjoin%
\definecolor{currentfill}{rgb}{0.000000,0.000000,0.000000}%
\pgfsetfillcolor{currentfill}%
\pgfsetlinewidth{0.803000pt}%
\definecolor{currentstroke}{rgb}{0.000000,0.000000,0.000000}%
\pgfsetstrokecolor{currentstroke}%
\pgfsetdash{}{0pt}%
\pgfsys@defobject{currentmarker}{\pgfqpoint{0.000000in}{-0.048611in}}{\pgfqpoint{0.000000in}{0.000000in}}{%
\pgfpathmoveto{\pgfqpoint{0.000000in}{0.000000in}}%
\pgfpathlineto{\pgfqpoint{0.000000in}{-0.048611in}}%
\pgfusepath{stroke,fill}%
}%
\begin{pgfscope}%
\pgfsys@transformshift{0.584475in}{0.571604in}%
\pgfsys@useobject{currentmarker}{}%
\end{pgfscope}%
\end{pgfscope}%
\begin{pgfscope}%
\definecolor{textcolor}{rgb}{0.000000,0.000000,0.000000}%
\pgfsetstrokecolor{textcolor}%
\pgfsetfillcolor{textcolor}%
\pgftext[x=0.584475in,y=0.474382in,,top]{\color{textcolor}\sffamily\fontsize{10.000000}{12.000000}\selectfont 0}%
\end{pgfscope}%
\begin{pgfscope}%
\pgfpathrectangle{\pgfqpoint{0.584475in}{0.571604in}}{\pgfqpoint{2.177159in}{0.829932in}}%
\pgfusepath{clip}%
\pgfsetrectcap%
\pgfsetroundjoin%
\pgfsetlinewidth{0.803000pt}%
\definecolor{currentstroke}{rgb}{0.690196,0.690196,0.690196}%
\pgfsetstrokecolor{currentstroke}%
\pgfsetdash{}{0pt}%
\pgfpathmoveto{\pgfqpoint{1.310195in}{0.571604in}}%
\pgfpathlineto{\pgfqpoint{1.310195in}{1.401535in}}%
\pgfusepath{stroke}%
\end{pgfscope}%
\begin{pgfscope}%
\pgfsetbuttcap%
\pgfsetroundjoin%
\definecolor{currentfill}{rgb}{0.000000,0.000000,0.000000}%
\pgfsetfillcolor{currentfill}%
\pgfsetlinewidth{0.803000pt}%
\definecolor{currentstroke}{rgb}{0.000000,0.000000,0.000000}%
\pgfsetstrokecolor{currentstroke}%
\pgfsetdash{}{0pt}%
\pgfsys@defobject{currentmarker}{\pgfqpoint{0.000000in}{-0.048611in}}{\pgfqpoint{0.000000in}{0.000000in}}{%
\pgfpathmoveto{\pgfqpoint{0.000000in}{0.000000in}}%
\pgfpathlineto{\pgfqpoint{0.000000in}{-0.048611in}}%
\pgfusepath{stroke,fill}%
}%
\begin{pgfscope}%
\pgfsys@transformshift{1.310195in}{0.571604in}%
\pgfsys@useobject{currentmarker}{}%
\end{pgfscope}%
\end{pgfscope}%
\begin{pgfscope}%
\definecolor{textcolor}{rgb}{0.000000,0.000000,0.000000}%
\pgfsetstrokecolor{textcolor}%
\pgfsetfillcolor{textcolor}%
\pgftext[x=1.310195in,y=0.474382in,,top]{\color{textcolor}\sffamily\fontsize{10.000000}{12.000000}\selectfont 10}%
\end{pgfscope}%
\begin{pgfscope}%
\pgfpathrectangle{\pgfqpoint{0.584475in}{0.571604in}}{\pgfqpoint{2.177159in}{0.829932in}}%
\pgfusepath{clip}%
\pgfsetrectcap%
\pgfsetroundjoin%
\pgfsetlinewidth{0.803000pt}%
\definecolor{currentstroke}{rgb}{0.690196,0.690196,0.690196}%
\pgfsetstrokecolor{currentstroke}%
\pgfsetdash{}{0pt}%
\pgfpathmoveto{\pgfqpoint{2.035915in}{0.571604in}}%
\pgfpathlineto{\pgfqpoint{2.035915in}{1.401535in}}%
\pgfusepath{stroke}%
\end{pgfscope}%
\begin{pgfscope}%
\pgfsetbuttcap%
\pgfsetroundjoin%
\definecolor{currentfill}{rgb}{0.000000,0.000000,0.000000}%
\pgfsetfillcolor{currentfill}%
\pgfsetlinewidth{0.803000pt}%
\definecolor{currentstroke}{rgb}{0.000000,0.000000,0.000000}%
\pgfsetstrokecolor{currentstroke}%
\pgfsetdash{}{0pt}%
\pgfsys@defobject{currentmarker}{\pgfqpoint{0.000000in}{-0.048611in}}{\pgfqpoint{0.000000in}{0.000000in}}{%
\pgfpathmoveto{\pgfqpoint{0.000000in}{0.000000in}}%
\pgfpathlineto{\pgfqpoint{0.000000in}{-0.048611in}}%
\pgfusepath{stroke,fill}%
}%
\begin{pgfscope}%
\pgfsys@transformshift{2.035915in}{0.571604in}%
\pgfsys@useobject{currentmarker}{}%
\end{pgfscope}%
\end{pgfscope}%
\begin{pgfscope}%
\definecolor{textcolor}{rgb}{0.000000,0.000000,0.000000}%
\pgfsetstrokecolor{textcolor}%
\pgfsetfillcolor{textcolor}%
\pgftext[x=2.035915in,y=0.474382in,,top]{\color{textcolor}\sffamily\fontsize{10.000000}{12.000000}\selectfont 20}%
\end{pgfscope}%
\begin{pgfscope}%
\pgfpathrectangle{\pgfqpoint{0.584475in}{0.571604in}}{\pgfqpoint{2.177159in}{0.829932in}}%
\pgfusepath{clip}%
\pgfsetrectcap%
\pgfsetroundjoin%
\pgfsetlinewidth{0.803000pt}%
\definecolor{currentstroke}{rgb}{0.690196,0.690196,0.690196}%
\pgfsetstrokecolor{currentstroke}%
\pgfsetdash{}{0pt}%
\pgfpathmoveto{\pgfqpoint{2.761635in}{0.571604in}}%
\pgfpathlineto{\pgfqpoint{2.761635in}{1.401535in}}%
\pgfusepath{stroke}%
\end{pgfscope}%
\begin{pgfscope}%
\pgfsetbuttcap%
\pgfsetroundjoin%
\definecolor{currentfill}{rgb}{0.000000,0.000000,0.000000}%
\pgfsetfillcolor{currentfill}%
\pgfsetlinewidth{0.803000pt}%
\definecolor{currentstroke}{rgb}{0.000000,0.000000,0.000000}%
\pgfsetstrokecolor{currentstroke}%
\pgfsetdash{}{0pt}%
\pgfsys@defobject{currentmarker}{\pgfqpoint{0.000000in}{-0.048611in}}{\pgfqpoint{0.000000in}{0.000000in}}{%
\pgfpathmoveto{\pgfqpoint{0.000000in}{0.000000in}}%
\pgfpathlineto{\pgfqpoint{0.000000in}{-0.048611in}}%
\pgfusepath{stroke,fill}%
}%
\begin{pgfscope}%
\pgfsys@transformshift{2.761635in}{0.571604in}%
\pgfsys@useobject{currentmarker}{}%
\end{pgfscope}%
\end{pgfscope}%
\begin{pgfscope}%
\definecolor{textcolor}{rgb}{0.000000,0.000000,0.000000}%
\pgfsetstrokecolor{textcolor}%
\pgfsetfillcolor{textcolor}%
\pgftext[x=2.761635in,y=0.474382in,,top]{\color{textcolor}\sffamily\fontsize{10.000000}{12.000000}\selectfont 30}%
\end{pgfscope}%
\begin{pgfscope}%
\definecolor{textcolor}{rgb}{0.000000,0.000000,0.000000}%
\pgfsetstrokecolor{textcolor}%
\pgfsetfillcolor{textcolor}%
\pgftext[x=1.673055in,y=0.284413in,,top]{\color{textcolor}\sffamily\fontsize{10.000000}{12.000000}\selectfont Shift von Paramter 5}%
\end{pgfscope}%
\begin{pgfscope}%
\pgfpathrectangle{\pgfqpoint{0.584475in}{0.571604in}}{\pgfqpoint{2.177159in}{0.829932in}}%
\pgfusepath{clip}%
\pgfsetrectcap%
\pgfsetroundjoin%
\pgfsetlinewidth{0.803000pt}%
\definecolor{currentstroke}{rgb}{0.690196,0.690196,0.690196}%
\pgfsetstrokecolor{currentstroke}%
\pgfsetdash{}{0pt}%
\pgfpathmoveto{\pgfqpoint{0.584475in}{0.571604in}}%
\pgfpathlineto{\pgfqpoint{2.761635in}{0.571604in}}%
\pgfusepath{stroke}%
\end{pgfscope}%
\begin{pgfscope}%
\pgfsetbuttcap%
\pgfsetroundjoin%
\definecolor{currentfill}{rgb}{0.000000,0.000000,0.000000}%
\pgfsetfillcolor{currentfill}%
\pgfsetlinewidth{0.803000pt}%
\definecolor{currentstroke}{rgb}{0.000000,0.000000,0.000000}%
\pgfsetstrokecolor{currentstroke}%
\pgfsetdash{}{0pt}%
\pgfsys@defobject{currentmarker}{\pgfqpoint{-0.048611in}{0.000000in}}{\pgfqpoint{0.000000in}{0.000000in}}{%
\pgfpathmoveto{\pgfqpoint{0.000000in}{0.000000in}}%
\pgfpathlineto{\pgfqpoint{-0.048611in}{0.000000in}}%
\pgfusepath{stroke,fill}%
}%
\begin{pgfscope}%
\pgfsys@transformshift{0.584475in}{0.571604in}%
\pgfsys@useobject{currentmarker}{}%
\end{pgfscope}%
\end{pgfscope}%
\begin{pgfscope}%
\definecolor{textcolor}{rgb}{0.000000,0.000000,0.000000}%
\pgfsetstrokecolor{textcolor}%
\pgfsetfillcolor{textcolor}%
\pgftext[x=0.398888in, y=0.518842in, left, base]{\color{textcolor}\sffamily\fontsize{10.000000}{12.000000}\selectfont 0}%
\end{pgfscope}%
\begin{pgfscope}%
\pgfpathrectangle{\pgfqpoint{0.584475in}{0.571604in}}{\pgfqpoint{2.177159in}{0.829932in}}%
\pgfusepath{clip}%
\pgfsetrectcap%
\pgfsetroundjoin%
\pgfsetlinewidth{0.803000pt}%
\definecolor{currentstroke}{rgb}{0.690196,0.690196,0.690196}%
\pgfsetstrokecolor{currentstroke}%
\pgfsetdash{}{0pt}%
\pgfpathmoveto{\pgfqpoint{0.584475in}{1.124892in}}%
\pgfpathlineto{\pgfqpoint{2.761635in}{1.124892in}}%
\pgfusepath{stroke}%
\end{pgfscope}%
\begin{pgfscope}%
\pgfsetbuttcap%
\pgfsetroundjoin%
\definecolor{currentfill}{rgb}{0.000000,0.000000,0.000000}%
\pgfsetfillcolor{currentfill}%
\pgfsetlinewidth{0.803000pt}%
\definecolor{currentstroke}{rgb}{0.000000,0.000000,0.000000}%
\pgfsetstrokecolor{currentstroke}%
\pgfsetdash{}{0pt}%
\pgfsys@defobject{currentmarker}{\pgfqpoint{-0.048611in}{0.000000in}}{\pgfqpoint{0.000000in}{0.000000in}}{%
\pgfpathmoveto{\pgfqpoint{0.000000in}{0.000000in}}%
\pgfpathlineto{\pgfqpoint{-0.048611in}{0.000000in}}%
\pgfusepath{stroke,fill}%
}%
\begin{pgfscope}%
\pgfsys@transformshift{0.584475in}{1.124892in}%
\pgfsys@useobject{currentmarker}{}%
\end{pgfscope}%
\end{pgfscope}%
\begin{pgfscope}%
\definecolor{textcolor}{rgb}{0.000000,0.000000,0.000000}%
\pgfsetstrokecolor{textcolor}%
\pgfsetfillcolor{textcolor}%
\pgftext[x=0.398888in, y=1.072130in, left, base]{\color{textcolor}\sffamily\fontsize{10.000000}{12.000000}\selectfont 1}%
\end{pgfscope}%
\begin{pgfscope}%
\definecolor{textcolor}{rgb}{0.000000,0.000000,0.000000}%
\pgfsetstrokecolor{textcolor}%
\pgfsetfillcolor{textcolor}%
\pgftext[x=0.584475in,y=1.443202in,left,base]{\color{textcolor}\sffamily\fontsize{10.000000}{12.000000}\selectfont 1e5}%
\end{pgfscope}%
\begin{pgfscope}%
\pgfpathrectangle{\pgfqpoint{0.584475in}{0.571604in}}{\pgfqpoint{2.177159in}{0.829932in}}%
\pgfusepath{clip}%
\pgfsetrectcap%
\pgfsetroundjoin%
\pgfsetlinewidth{1.505625pt}%
\definecolor{currentstroke}{rgb}{1.000000,0.000000,0.000000}%
\pgfsetstrokecolor{currentstroke}%
\pgfsetdash{}{0pt}%
\pgfpathmoveto{\pgfqpoint{0.584475in}{0.571701in}}%
\pgfpathlineto{\pgfqpoint{0.598990in}{0.572474in}}%
\pgfpathlineto{\pgfqpoint{0.613504in}{0.574021in}}%
\pgfpathlineto{\pgfqpoint{0.628019in}{0.576341in}}%
\pgfpathlineto{\pgfqpoint{0.642533in}{0.579434in}}%
\pgfpathlineto{\pgfqpoint{0.657047in}{0.583301in}}%
\pgfpathlineto{\pgfqpoint{0.671562in}{0.587942in}}%
\pgfpathlineto{\pgfqpoint{0.686076in}{0.593355in}}%
\pgfpathlineto{\pgfqpoint{0.700591in}{0.599543in}}%
\pgfpathlineto{\pgfqpoint{0.715105in}{0.606503in}}%
\pgfpathlineto{\pgfqpoint{0.722362in}{0.611654in}}%
\pgfpathlineto{\pgfqpoint{0.736877in}{0.620065in}}%
\pgfpathlineto{\pgfqpoint{0.751391in}{0.629276in}}%
\pgfpathlineto{\pgfqpoint{0.765905in}{0.639289in}}%
\pgfpathlineto{\pgfqpoint{0.780420in}{0.650103in}}%
\pgfpathlineto{\pgfqpoint{0.794934in}{0.661717in}}%
\pgfpathlineto{\pgfqpoint{0.809449in}{0.674133in}}%
\pgfpathlineto{\pgfqpoint{0.823963in}{0.687349in}}%
\pgfpathlineto{\pgfqpoint{0.838477in}{0.701367in}}%
\pgfpathlineto{\pgfqpoint{0.852992in}{0.716186in}}%
\pgfpathlineto{\pgfqpoint{0.867506in}{0.731806in}}%
\pgfpathlineto{\pgfqpoint{0.882021in}{0.732494in}}%
\pgfpathlineto{\pgfqpoint{0.896535in}{0.733935in}}%
\pgfpathlineto{\pgfqpoint{0.911049in}{0.736127in}}%
\pgfpathlineto{\pgfqpoint{0.925564in}{0.739070in}}%
\pgfpathlineto{\pgfqpoint{0.940078in}{0.742765in}}%
\pgfpathlineto{\pgfqpoint{0.954592in}{0.747210in}}%
\pgfpathlineto{\pgfqpoint{0.969107in}{0.752408in}}%
\pgfpathlineto{\pgfqpoint{0.983621in}{0.758357in}}%
\pgfpathlineto{\pgfqpoint{0.998136in}{0.765057in}}%
\pgfpathlineto{\pgfqpoint{1.005393in}{0.768689in}}%
\pgfpathlineto{\pgfqpoint{1.012650in}{0.769430in}}%
\pgfpathlineto{\pgfqpoint{1.027164in}{0.777936in}}%
\pgfpathlineto{\pgfqpoint{1.041679in}{0.787223in}}%
\pgfpathlineto{\pgfqpoint{1.056193in}{0.797288in}}%
\pgfpathlineto{\pgfqpoint{1.070708in}{0.808133in}}%
\pgfpathlineto{\pgfqpoint{1.085222in}{0.819757in}}%
\pgfpathlineto{\pgfqpoint{1.099736in}{0.832160in}}%
\pgfpathlineto{\pgfqpoint{1.114251in}{0.845342in}}%
\pgfpathlineto{\pgfqpoint{1.128765in}{0.859304in}}%
\pgfpathlineto{\pgfqpoint{1.143280in}{0.874045in}}%
\pgfpathlineto{\pgfqpoint{1.157794in}{0.889564in}}%
\pgfpathlineto{\pgfqpoint{1.172308in}{0.889767in}}%
\pgfpathlineto{\pgfqpoint{1.186823in}{0.890698in}}%
\pgfpathlineto{\pgfqpoint{1.201337in}{0.892356in}}%
\pgfpathlineto{\pgfqpoint{1.215852in}{0.894742in}}%
\pgfpathlineto{\pgfqpoint{1.230366in}{0.897856in}}%
\pgfpathlineto{\pgfqpoint{1.244880in}{0.901698in}}%
\pgfpathlineto{\pgfqpoint{1.259395in}{0.906268in}}%
\pgfpathlineto{\pgfqpoint{1.273909in}{0.911566in}}%
\pgfpathlineto{\pgfqpoint{1.288424in}{0.917591in}}%
\pgfpathlineto{\pgfqpoint{1.302938in}{0.924345in}}%
\pgfpathlineto{\pgfqpoint{1.310195in}{0.919870in}}%
\pgfpathlineto{\pgfqpoint{1.324710in}{0.928017in}}%
\pgfpathlineto{\pgfqpoint{1.339224in}{0.936920in}}%
\pgfpathlineto{\pgfqpoint{1.353738in}{0.946579in}}%
\pgfpathlineto{\pgfqpoint{1.368253in}{0.956994in}}%
\pgfpathlineto{\pgfqpoint{1.382767in}{0.968165in}}%
\pgfpathlineto{\pgfqpoint{1.397282in}{0.980091in}}%
\pgfpathlineto{\pgfqpoint{1.411796in}{0.992774in}}%
\pgfpathlineto{\pgfqpoint{1.426310in}{1.006212in}}%
\pgfpathlineto{\pgfqpoint{1.440825in}{1.020407in}}%
\pgfpathlineto{\pgfqpoint{1.448082in}{1.027787in}}%
\pgfpathlineto{\pgfqpoint{1.462596in}{1.027348in}}%
\pgfpathlineto{\pgfqpoint{1.477111in}{1.027622in}}%
\pgfpathlineto{\pgfqpoint{1.491625in}{1.028610in}}%
\pgfpathlineto{\pgfqpoint{1.506139in}{1.030312in}}%
\pgfpathlineto{\pgfqpoint{1.520654in}{1.032728in}}%
\pgfpathlineto{\pgfqpoint{1.535168in}{1.035858in}}%
\pgfpathlineto{\pgfqpoint{1.549683in}{1.039702in}}%
\pgfpathlineto{\pgfqpoint{1.564197in}{1.044259in}}%
\pgfpathlineto{\pgfqpoint{1.578711in}{1.049531in}}%
\pgfpathlineto{\pgfqpoint{1.593226in}{1.055516in}}%
\pgfpathlineto{\pgfqpoint{1.600483in}{1.039040in}}%
\pgfpathlineto{\pgfqpoint{1.614997in}{1.046378in}}%
\pgfpathlineto{\pgfqpoint{1.629512in}{1.054459in}}%
\pgfpathlineto{\pgfqpoint{1.644026in}{1.063282in}}%
\pgfpathlineto{\pgfqpoint{1.658541in}{1.072848in}}%
\pgfpathlineto{\pgfqpoint{1.673055in}{1.083157in}}%
\pgfpathlineto{\pgfqpoint{1.687569in}{1.094207in}}%
\pgfpathlineto{\pgfqpoint{1.702084in}{1.106001in}}%
\pgfpathlineto{\pgfqpoint{1.716598in}{1.118536in}}%
\pgfpathlineto{\pgfqpoint{1.731113in}{1.131813in}}%
\pgfpathlineto{\pgfqpoint{1.738370in}{1.138729in}}%
\pgfpathlineto{\pgfqpoint{1.752884in}{1.137628in}}%
\pgfpathlineto{\pgfqpoint{1.767399in}{1.137235in}}%
\pgfpathlineto{\pgfqpoint{1.781913in}{1.137545in}}%
\pgfpathlineto{\pgfqpoint{1.796427in}{1.138569in}}%
\pgfpathlineto{\pgfqpoint{1.810942in}{1.140301in}}%
\pgfpathlineto{\pgfqpoint{1.825456in}{1.142735in}}%
\pgfpathlineto{\pgfqpoint{1.839971in}{1.145883in}}%
\pgfpathlineto{\pgfqpoint{1.854485in}{1.149734in}}%
\pgfpathlineto{\pgfqpoint{1.868999in}{1.154299in}}%
\pgfpathlineto{\pgfqpoint{1.883514in}{1.159566in}}%
\pgfpathlineto{\pgfqpoint{1.890771in}{1.135686in}}%
\pgfpathlineto{\pgfqpoint{1.905285in}{1.142281in}}%
\pgfpathlineto{\pgfqpoint{1.919800in}{1.149612in}}%
\pgfpathlineto{\pgfqpoint{1.934314in}{1.157685in}}%
\pgfpathlineto{\pgfqpoint{1.948829in}{1.166493in}}%
\pgfpathlineto{\pgfqpoint{1.963343in}{1.176043in}}%
\pgfpathlineto{\pgfqpoint{1.977857in}{1.186329in}}%
\pgfpathlineto{\pgfqpoint{1.992372in}{1.197350in}}%
\pgfpathlineto{\pgfqpoint{2.006886in}{1.209107in}}%
\pgfpathlineto{\pgfqpoint{2.021401in}{1.221606in}}%
\pgfpathlineto{\pgfqpoint{2.028658in}{1.228130in}}%
\pgfpathlineto{\pgfqpoint{2.043172in}{1.226481in}}%
\pgfpathlineto{\pgfqpoint{2.057686in}{1.225535in}}%
\pgfpathlineto{\pgfqpoint{2.072201in}{1.225291in}}%
\pgfpathlineto{\pgfqpoint{2.086715in}{1.225750in}}%
\pgfpathlineto{\pgfqpoint{2.101230in}{1.226912in}}%
\pgfpathlineto{\pgfqpoint{2.115744in}{1.228777in}}%
\pgfpathlineto{\pgfqpoint{2.130258in}{1.231344in}}%
\pgfpathlineto{\pgfqpoint{2.144773in}{1.234609in}}%
\pgfpathlineto{\pgfqpoint{2.159287in}{1.238581in}}%
\pgfpathlineto{\pgfqpoint{2.166544in}{1.240827in}}%
\pgfpathlineto{\pgfqpoint{2.173802in}{1.210723in}}%
\pgfpathlineto{\pgfqpoint{2.188316in}{1.216328in}}%
\pgfpathlineto{\pgfqpoint{2.202830in}{1.222669in}}%
\pgfpathlineto{\pgfqpoint{2.217345in}{1.229745in}}%
\pgfpathlineto{\pgfqpoint{2.231859in}{1.237552in}}%
\pgfpathlineto{\pgfqpoint{2.246374in}{1.246089in}}%
\pgfpathlineto{\pgfqpoint{2.260888in}{1.255362in}}%
\pgfpathlineto{\pgfqpoint{2.275402in}{1.265366in}}%
\pgfpathlineto{\pgfqpoint{2.289917in}{1.276105in}}%
\pgfpathlineto{\pgfqpoint{2.304431in}{1.287575in}}%
\pgfpathlineto{\pgfqpoint{2.318946in}{1.299780in}}%
\pgfpathlineto{\pgfqpoint{2.333460in}{1.297473in}}%
\pgfpathlineto{\pgfqpoint{2.347974in}{1.295880in}}%
\pgfpathlineto{\pgfqpoint{2.362489in}{1.295000in}}%
\pgfpathlineto{\pgfqpoint{2.377003in}{1.294823in}}%
\pgfpathlineto{\pgfqpoint{2.391518in}{1.295360in}}%
\pgfpathlineto{\pgfqpoint{2.406032in}{1.296604in}}%
\pgfpathlineto{\pgfqpoint{2.420546in}{1.298558in}}%
\pgfpathlineto{\pgfqpoint{2.435061in}{1.301224in}}%
\pgfpathlineto{\pgfqpoint{2.449575in}{1.304599in}}%
\pgfpathlineto{\pgfqpoint{2.456832in}{1.306553in}}%
\pgfpathlineto{\pgfqpoint{2.464090in}{1.271922in}}%
\pgfpathlineto{\pgfqpoint{2.478604in}{1.276935in}}%
\pgfpathlineto{\pgfqpoint{2.493118in}{1.282684in}}%
\pgfpathlineto{\pgfqpoint{2.507633in}{1.289179in}}%
\pgfpathlineto{\pgfqpoint{2.522147in}{1.296416in}}%
\pgfpathlineto{\pgfqpoint{2.536662in}{1.304395in}}%
\pgfpathlineto{\pgfqpoint{2.551176in}{1.313115in}}%
\pgfpathlineto{\pgfqpoint{2.565690in}{1.322576in}}%
\pgfpathlineto{\pgfqpoint{2.580205in}{1.332784in}}%
\pgfpathlineto{\pgfqpoint{2.594719in}{1.343728in}}%
\pgfpathlineto{\pgfqpoint{2.609234in}{1.355413in}}%
\pgfpathlineto{\pgfqpoint{2.623748in}{1.352614in}}%
\pgfpathlineto{\pgfqpoint{2.638262in}{1.350539in}}%
\pgfpathlineto{\pgfqpoint{2.652777in}{1.349178in}}%
\pgfpathlineto{\pgfqpoint{2.667291in}{1.348530in}}%
\pgfpathlineto{\pgfqpoint{2.681805in}{1.348608in}}%
\pgfpathlineto{\pgfqpoint{2.696320in}{1.349405in}}%
\pgfpathlineto{\pgfqpoint{2.710834in}{1.350921in}}%
\pgfpathlineto{\pgfqpoint{2.725349in}{1.353150in}}%
\pgfpathlineto{\pgfqpoint{2.739863in}{1.356105in}}%
\pgfpathlineto{\pgfqpoint{2.747120in}{1.357848in}}%
\pgfpathlineto{\pgfqpoint{2.754377in}{1.321358in}}%
\pgfpathlineto{\pgfqpoint{2.761635in}{1.323566in}}%
\pgfpathlineto{\pgfqpoint{2.761635in}{1.323566in}}%
\pgfusepath{stroke}%
\end{pgfscope}%
\begin{pgfscope}%
\pgfsetrectcap%
\pgfsetmiterjoin%
\pgfsetlinewidth{0.803000pt}%
\definecolor{currentstroke}{rgb}{0.000000,0.000000,0.000000}%
\pgfsetstrokecolor{currentstroke}%
\pgfsetdash{}{0pt}%
\pgfpathmoveto{\pgfqpoint{0.584475in}{0.571604in}}%
\pgfpathlineto{\pgfqpoint{0.584475in}{1.401535in}}%
\pgfusepath{stroke}%
\end{pgfscope}%
\begin{pgfscope}%
\pgfsetrectcap%
\pgfsetmiterjoin%
\pgfsetlinewidth{0.803000pt}%
\definecolor{currentstroke}{rgb}{0.000000,0.000000,0.000000}%
\pgfsetstrokecolor{currentstroke}%
\pgfsetdash{}{0pt}%
\pgfpathmoveto{\pgfqpoint{2.761635in}{0.571604in}}%
\pgfpathlineto{\pgfqpoint{2.761635in}{1.401535in}}%
\pgfusepath{stroke}%
\end{pgfscope}%
\begin{pgfscope}%
\pgfsetrectcap%
\pgfsetmiterjoin%
\pgfsetlinewidth{0.803000pt}%
\definecolor{currentstroke}{rgb}{0.000000,0.000000,0.000000}%
\pgfsetstrokecolor{currentstroke}%
\pgfsetdash{}{0pt}%
\pgfpathmoveto{\pgfqpoint{0.584475in}{0.571604in}}%
\pgfpathlineto{\pgfqpoint{2.761635in}{0.571604in}}%
\pgfusepath{stroke}%
\end{pgfscope}%
\begin{pgfscope}%
\pgfsetrectcap%
\pgfsetmiterjoin%
\pgfsetlinewidth{0.803000pt}%
\definecolor{currentstroke}{rgb}{0.000000,0.000000,0.000000}%
\pgfsetstrokecolor{currentstroke}%
\pgfsetdash{}{0pt}%
\pgfpathmoveto{\pgfqpoint{0.584475in}{1.401535in}}%
\pgfpathlineto{\pgfqpoint{2.761635in}{1.401535in}}%
\pgfusepath{stroke}%
\end{pgfscope}%
\begin{pgfscope}%
\definecolor{textcolor}{rgb}{1.000000,0.000000,0.000000}%
\pgfsetstrokecolor{textcolor}%
\pgfsetfillcolor{textcolor}%
\pgftext[x=1.673055in,y=1.484869in,,base]{\color{textcolor}\sffamily\fontsize{12.000000}{14.400000}\selectfont MSD Metrik}%
\end{pgfscope}%
\end{pgfpicture}%
\makeatother%
\endgroup%
}
  \vspace{-10pt}
\end{figure}

\subsection{Ähnlichkeit bei multimodalen Bilddaten}
\begin{figure}[h]
  \caption{Metrikverhalten bei Überlagerung von CT und MRT Daten}
  \label{fig:ct2mrt_param}
  \vspace{-10pt}
  \resizebox{0.48\linewidth}{!}{%% Creator: Matplotlib, PGF backend
%%
%% To include the figure in your LaTeX document, write
%%   \input{<filename>.pgf}
%%
%% Make sure the required packages are loaded in your preamble
%%   \usepackage{pgf}
%%
%% and, on pdftex
%%   \usepackage[utf8]{inputenc}\DeclareUnicodeCharacter{2212}{-}
%%
%% or, on luatex and xetex
%%   \usepackage{unicode-math}
%%
%% Figures using additional raster images can only be included by \input if
%% they are in the same directory as the main LaTeX file. For loading figures
%% from other directories you can use the `import` package
%%   \usepackage{import}
%%
%% and then include the figures with
%%   \import{<path to file>}{<filename>.pgf}
%%
%% Matplotlib used the following preamble
%%   \usepackage{fontspec}
%%   \setmainfont{DejaVuSerif.ttf}[Path=/usr/share/matplotlib/mpl-data/fonts/ttf/]
%%   \setsansfont{DejaVuSans.ttf}[Path=/usr/share/matplotlib/mpl-data/fonts/ttf/]
%%   \setmonofont{DejaVuSansMono.ttf}[Path=/usr/share/matplotlib/mpl-data/fonts/ttf/]
%%
\begingroup%
\makeatletter%
\begin{pgfpicture}%
\pgfpathrectangle{\pgfpointorigin}{\pgfqpoint{3.000000in}{3.000000in}}%
\pgfusepath{use as bounding box, clip}%
\begin{pgfscope}%
\pgfsetbuttcap%
\pgfsetmiterjoin%
\definecolor{currentfill}{rgb}{1.000000,1.000000,1.000000}%
\pgfsetfillcolor{currentfill}%
\pgfsetlinewidth{0.000000pt}%
\definecolor{currentstroke}{rgb}{1.000000,1.000000,1.000000}%
\pgfsetstrokecolor{currentstroke}%
\pgfsetdash{}{0pt}%
\pgfpathmoveto{\pgfqpoint{0.000000in}{0.000000in}}%
\pgfpathlineto{\pgfqpoint{3.000000in}{0.000000in}}%
\pgfpathlineto{\pgfqpoint{3.000000in}{3.000000in}}%
\pgfpathlineto{\pgfqpoint{0.000000in}{3.000000in}}%
\pgfpathclose%
\pgfusepath{fill}%
\end{pgfscope}%
\begin{pgfscope}%
\pgfsetbuttcap%
\pgfsetmiterjoin%
\definecolor{currentfill}{rgb}{1.000000,1.000000,1.000000}%
\pgfsetfillcolor{currentfill}%
\pgfsetlinewidth{0.000000pt}%
\definecolor{currentstroke}{rgb}{0.000000,0.000000,0.000000}%
\pgfsetstrokecolor{currentstroke}%
\pgfsetstrokeopacity{0.000000}%
\pgfsetdash{}{0pt}%
\pgfpathmoveto{\pgfqpoint{0.774444in}{0.571604in}}%
\pgfpathlineto{\pgfqpoint{2.341930in}{0.571604in}}%
\pgfpathlineto{\pgfqpoint{2.341930in}{2.702810in}}%
\pgfpathlineto{\pgfqpoint{0.774444in}{2.702810in}}%
\pgfpathclose%
\pgfusepath{fill}%
\end{pgfscope}%
\begin{pgfscope}%
\pgfsetbuttcap%
\pgfsetroundjoin%
\definecolor{currentfill}{rgb}{0.000000,0.000000,0.000000}%
\pgfsetfillcolor{currentfill}%
\pgfsetlinewidth{0.803000pt}%
\definecolor{currentstroke}{rgb}{0.000000,0.000000,0.000000}%
\pgfsetstrokecolor{currentstroke}%
\pgfsetdash{}{0pt}%
\pgfsys@defobject{currentmarker}{\pgfqpoint{0.000000in}{-0.048611in}}{\pgfqpoint{0.000000in}{0.000000in}}{%
\pgfpathmoveto{\pgfqpoint{0.000000in}{0.000000in}}%
\pgfpathlineto{\pgfqpoint{0.000000in}{-0.048611in}}%
\pgfusepath{stroke,fill}%
}%
\begin{pgfscope}%
\pgfsys@transformshift{0.774444in}{0.571604in}%
\pgfsys@useobject{currentmarker}{}%
\end{pgfscope}%
\end{pgfscope}%
\begin{pgfscope}%
\definecolor{textcolor}{rgb}{0.000000,0.000000,0.000000}%
\pgfsetstrokecolor{textcolor}%
\pgfsetfillcolor{textcolor}%
\pgftext[x=0.774444in,y=0.474382in,,top]{\color{textcolor}\sffamily\fontsize{10.000000}{12.000000}\selectfont 0.0}%
\end{pgfscope}%
\begin{pgfscope}%
\pgfsetbuttcap%
\pgfsetroundjoin%
\definecolor{currentfill}{rgb}{0.000000,0.000000,0.000000}%
\pgfsetfillcolor{currentfill}%
\pgfsetlinewidth{0.803000pt}%
\definecolor{currentstroke}{rgb}{0.000000,0.000000,0.000000}%
\pgfsetstrokecolor{currentstroke}%
\pgfsetdash{}{0pt}%
\pgfsys@defobject{currentmarker}{\pgfqpoint{0.000000in}{-0.048611in}}{\pgfqpoint{0.000000in}{0.000000in}}{%
\pgfpathmoveto{\pgfqpoint{0.000000in}{0.000000in}}%
\pgfpathlineto{\pgfqpoint{0.000000in}{-0.048611in}}%
\pgfusepath{stroke,fill}%
}%
\begin{pgfscope}%
\pgfsys@transformshift{1.558187in}{0.571604in}%
\pgfsys@useobject{currentmarker}{}%
\end{pgfscope}%
\end{pgfscope}%
\begin{pgfscope}%
\definecolor{textcolor}{rgb}{0.000000,0.000000,0.000000}%
\pgfsetstrokecolor{textcolor}%
\pgfsetfillcolor{textcolor}%
\pgftext[x=1.558187in,y=0.474382in,,top]{\color{textcolor}\sffamily\fontsize{10.000000}{12.000000}\selectfont 0.5}%
\end{pgfscope}%
\begin{pgfscope}%
\pgfsetbuttcap%
\pgfsetroundjoin%
\definecolor{currentfill}{rgb}{0.000000,0.000000,0.000000}%
\pgfsetfillcolor{currentfill}%
\pgfsetlinewidth{0.803000pt}%
\definecolor{currentstroke}{rgb}{0.000000,0.000000,0.000000}%
\pgfsetstrokecolor{currentstroke}%
\pgfsetdash{}{0pt}%
\pgfsys@defobject{currentmarker}{\pgfqpoint{0.000000in}{-0.048611in}}{\pgfqpoint{0.000000in}{0.000000in}}{%
\pgfpathmoveto{\pgfqpoint{0.000000in}{0.000000in}}%
\pgfpathlineto{\pgfqpoint{0.000000in}{-0.048611in}}%
\pgfusepath{stroke,fill}%
}%
\begin{pgfscope}%
\pgfsys@transformshift{2.341930in}{0.571604in}%
\pgfsys@useobject{currentmarker}{}%
\end{pgfscope}%
\end{pgfscope}%
\begin{pgfscope}%
\definecolor{textcolor}{rgb}{0.000000,0.000000,0.000000}%
\pgfsetstrokecolor{textcolor}%
\pgfsetfillcolor{textcolor}%
\pgftext[x=2.341930in,y=0.474382in,,top]{\color{textcolor}\sffamily\fontsize{10.000000}{12.000000}\selectfont 1.0}%
\end{pgfscope}%
\begin{pgfscope}%
\definecolor{textcolor}{rgb}{0.000000,0.000000,0.000000}%
\pgfsetstrokecolor{textcolor}%
\pgfsetfillcolor{textcolor}%
\pgftext[x=1.558187in,y=0.284413in,,top]{\color{textcolor}\sffamily\fontsize{10.000000}{12.000000}\selectfont Shift von Paramter 0}%
\end{pgfscope}%
\begin{pgfscope}%
\pgfsetbuttcap%
\pgfsetroundjoin%
\definecolor{currentfill}{rgb}{0.000000,0.000000,1.000000}%
\pgfsetfillcolor{currentfill}%
\pgfsetlinewidth{0.803000pt}%
\definecolor{currentstroke}{rgb}{0.000000,0.000000,1.000000}%
\pgfsetstrokecolor{currentstroke}%
\pgfsetdash{}{0pt}%
\pgfsys@defobject{currentmarker}{\pgfqpoint{-0.048611in}{0.000000in}}{\pgfqpoint{0.000000in}{0.000000in}}{%
\pgfpathmoveto{\pgfqpoint{0.000000in}{0.000000in}}%
\pgfpathlineto{\pgfqpoint{-0.048611in}{0.000000in}}%
\pgfusepath{stroke,fill}%
}%
\begin{pgfscope}%
\pgfsys@transformshift{0.774444in}{0.571604in}%
\pgfsys@useobject{currentmarker}{}%
\end{pgfscope}%
\end{pgfscope}%
\begin{pgfscope}%
\definecolor{textcolor}{rgb}{0.000000,0.000000,1.000000}%
\pgfsetstrokecolor{textcolor}%
\pgfsetfillcolor{textcolor}%
\pgftext[x=0.339969in, y=0.518842in, left, base]{\color{textcolor}\sffamily\fontsize{10.000000}{12.000000}\selectfont −0.8}%
\end{pgfscope}%
\begin{pgfscope}%
\pgfsetbuttcap%
\pgfsetroundjoin%
\definecolor{currentfill}{rgb}{0.000000,0.000000,1.000000}%
\pgfsetfillcolor{currentfill}%
\pgfsetlinewidth{0.803000pt}%
\definecolor{currentstroke}{rgb}{0.000000,0.000000,1.000000}%
\pgfsetstrokecolor{currentstroke}%
\pgfsetdash{}{0pt}%
\pgfsys@defobject{currentmarker}{\pgfqpoint{-0.048611in}{0.000000in}}{\pgfqpoint{0.000000in}{0.000000in}}{%
\pgfpathmoveto{\pgfqpoint{0.000000in}{0.000000in}}%
\pgfpathlineto{\pgfqpoint{-0.048611in}{0.000000in}}%
\pgfusepath{stroke,fill}%
}%
\begin{pgfscope}%
\pgfsys@transformshift{0.774444in}{1.104405in}%
\pgfsys@useobject{currentmarker}{}%
\end{pgfscope}%
\end{pgfscope}%
\begin{pgfscope}%
\definecolor{textcolor}{rgb}{0.000000,0.000000,1.000000}%
\pgfsetstrokecolor{textcolor}%
\pgfsetfillcolor{textcolor}%
\pgftext[x=0.339969in, y=1.051644in, left, base]{\color{textcolor}\sffamily\fontsize{10.000000}{12.000000}\selectfont −0.6}%
\end{pgfscope}%
\begin{pgfscope}%
\pgfsetbuttcap%
\pgfsetroundjoin%
\definecolor{currentfill}{rgb}{0.000000,0.000000,1.000000}%
\pgfsetfillcolor{currentfill}%
\pgfsetlinewidth{0.803000pt}%
\definecolor{currentstroke}{rgb}{0.000000,0.000000,1.000000}%
\pgfsetstrokecolor{currentstroke}%
\pgfsetdash{}{0pt}%
\pgfsys@defobject{currentmarker}{\pgfqpoint{-0.048611in}{0.000000in}}{\pgfqpoint{0.000000in}{0.000000in}}{%
\pgfpathmoveto{\pgfqpoint{0.000000in}{0.000000in}}%
\pgfpathlineto{\pgfqpoint{-0.048611in}{0.000000in}}%
\pgfusepath{stroke,fill}%
}%
\begin{pgfscope}%
\pgfsys@transformshift{0.774444in}{1.637207in}%
\pgfsys@useobject{currentmarker}{}%
\end{pgfscope}%
\end{pgfscope}%
\begin{pgfscope}%
\definecolor{textcolor}{rgb}{0.000000,0.000000,1.000000}%
\pgfsetstrokecolor{textcolor}%
\pgfsetfillcolor{textcolor}%
\pgftext[x=0.339969in, y=1.584446in, left, base]{\color{textcolor}\sffamily\fontsize{10.000000}{12.000000}\selectfont −0.4}%
\end{pgfscope}%
\begin{pgfscope}%
\pgfsetbuttcap%
\pgfsetroundjoin%
\definecolor{currentfill}{rgb}{0.000000,0.000000,1.000000}%
\pgfsetfillcolor{currentfill}%
\pgfsetlinewidth{0.803000pt}%
\definecolor{currentstroke}{rgb}{0.000000,0.000000,1.000000}%
\pgfsetstrokecolor{currentstroke}%
\pgfsetdash{}{0pt}%
\pgfsys@defobject{currentmarker}{\pgfqpoint{-0.048611in}{0.000000in}}{\pgfqpoint{0.000000in}{0.000000in}}{%
\pgfpathmoveto{\pgfqpoint{0.000000in}{0.000000in}}%
\pgfpathlineto{\pgfqpoint{-0.048611in}{0.000000in}}%
\pgfusepath{stroke,fill}%
}%
\begin{pgfscope}%
\pgfsys@transformshift{0.774444in}{2.170009in}%
\pgfsys@useobject{currentmarker}{}%
\end{pgfscope}%
\end{pgfscope}%
\begin{pgfscope}%
\definecolor{textcolor}{rgb}{0.000000,0.000000,1.000000}%
\pgfsetstrokecolor{textcolor}%
\pgfsetfillcolor{textcolor}%
\pgftext[x=0.339969in, y=2.117247in, left, base]{\color{textcolor}\sffamily\fontsize{10.000000}{12.000000}\selectfont −0.2}%
\end{pgfscope}%
\begin{pgfscope}%
\pgfsetbuttcap%
\pgfsetroundjoin%
\definecolor{currentfill}{rgb}{0.000000,0.000000,1.000000}%
\pgfsetfillcolor{currentfill}%
\pgfsetlinewidth{0.803000pt}%
\definecolor{currentstroke}{rgb}{0.000000,0.000000,1.000000}%
\pgfsetstrokecolor{currentstroke}%
\pgfsetdash{}{0pt}%
\pgfsys@defobject{currentmarker}{\pgfqpoint{-0.048611in}{0.000000in}}{\pgfqpoint{0.000000in}{0.000000in}}{%
\pgfpathmoveto{\pgfqpoint{0.000000in}{0.000000in}}%
\pgfpathlineto{\pgfqpoint{-0.048611in}{0.000000in}}%
\pgfusepath{stroke,fill}%
}%
\begin{pgfscope}%
\pgfsys@transformshift{0.774444in}{2.702810in}%
\pgfsys@useobject{currentmarker}{}%
\end{pgfscope}%
\end{pgfscope}%
\begin{pgfscope}%
\definecolor{textcolor}{rgb}{0.000000,0.000000,1.000000}%
\pgfsetstrokecolor{textcolor}%
\pgfsetfillcolor{textcolor}%
\pgftext[x=0.456342in, y=2.650049in, left, base]{\color{textcolor}\sffamily\fontsize{10.000000}{12.000000}\selectfont 0.0}%
\end{pgfscope}%
\begin{pgfscope}%
\definecolor{textcolor}{rgb}{0.000000,0.000000,1.000000}%
\pgfsetstrokecolor{textcolor}%
\pgfsetfillcolor{textcolor}%
\pgftext[x=0.284413in,y=1.637207in,,bottom,rotate=90.000000]{\color{textcolor}\sffamily\fontsize{10.000000}{12.000000}\selectfont MI Metrik}%
\end{pgfscope}%
\begin{pgfscope}%
\pgfpathrectangle{\pgfqpoint{0.774444in}{0.571604in}}{\pgfqpoint{1.567486in}{2.131206in}}%
\pgfusepath{clip}%
\pgfsetrectcap%
\pgfsetroundjoin%
\pgfsetlinewidth{1.505625pt}%
\definecolor{currentstroke}{rgb}{0.000000,0.000000,1.000000}%
\pgfsetstrokecolor{currentstroke}%
\pgfsetdash{}{0pt}%
\pgfpathmoveto{\pgfqpoint{0.774444in}{0.866113in}}%
\pgfpathlineto{\pgfqpoint{0.790119in}{0.914520in}}%
\pgfpathlineto{\pgfqpoint{0.805794in}{0.980375in}}%
\pgfpathlineto{\pgfqpoint{0.821469in}{1.055444in}}%
\pgfpathlineto{\pgfqpoint{0.837143in}{1.126008in}}%
\pgfpathlineto{\pgfqpoint{0.852818in}{1.191884in}}%
\pgfpathlineto{\pgfqpoint{0.868493in}{1.243728in}}%
\pgfpathlineto{\pgfqpoint{0.884168in}{1.288779in}}%
\pgfpathlineto{\pgfqpoint{0.899843in}{1.328989in}}%
\pgfpathlineto{\pgfqpoint{0.915518in}{1.365236in}}%
\pgfpathlineto{\pgfqpoint{0.931193in}{1.398706in}}%
\pgfpathlineto{\pgfqpoint{0.946867in}{1.428559in}}%
\pgfpathlineto{\pgfqpoint{0.962542in}{1.457160in}}%
\pgfpathlineto{\pgfqpoint{0.978217in}{1.483885in}}%
\pgfpathlineto{\pgfqpoint{0.993892in}{1.511426in}}%
\pgfpathlineto{\pgfqpoint{1.009567in}{1.536188in}}%
\pgfpathlineto{\pgfqpoint{1.025242in}{1.560651in}}%
\pgfpathlineto{\pgfqpoint{1.040917in}{1.584255in}}%
\pgfpathlineto{\pgfqpoint{1.056591in}{1.605700in}}%
\pgfpathlineto{\pgfqpoint{1.072266in}{1.627659in}}%
\pgfpathlineto{\pgfqpoint{1.087941in}{1.648138in}}%
\pgfpathlineto{\pgfqpoint{1.103616in}{1.668155in}}%
\pgfpathlineto{\pgfqpoint{1.119291in}{1.686723in}}%
\pgfpathlineto{\pgfqpoint{1.134966in}{1.705486in}}%
\pgfpathlineto{\pgfqpoint{1.150641in}{1.723353in}}%
\pgfpathlineto{\pgfqpoint{1.166315in}{1.741338in}}%
\pgfpathlineto{\pgfqpoint{1.181990in}{1.758515in}}%
\pgfpathlineto{\pgfqpoint{1.197665in}{1.775568in}}%
\pgfpathlineto{\pgfqpoint{1.213340in}{1.793123in}}%
\pgfpathlineto{\pgfqpoint{1.229015in}{1.809459in}}%
\pgfpathlineto{\pgfqpoint{1.244690in}{1.826807in}}%
\pgfpathlineto{\pgfqpoint{1.260365in}{1.845056in}}%
\pgfpathlineto{\pgfqpoint{1.276039in}{1.862500in}}%
\pgfpathlineto{\pgfqpoint{1.291714in}{1.880133in}}%
\pgfpathlineto{\pgfqpoint{1.307389in}{1.898112in}}%
\pgfpathlineto{\pgfqpoint{1.323064in}{1.914938in}}%
\pgfpathlineto{\pgfqpoint{1.338739in}{1.932520in}}%
\pgfpathlineto{\pgfqpoint{1.354414in}{1.949250in}}%
\pgfpathlineto{\pgfqpoint{1.370089in}{1.968018in}}%
\pgfpathlineto{\pgfqpoint{1.385763in}{1.986749in}}%
\pgfpathlineto{\pgfqpoint{1.401438in}{2.006159in}}%
\pgfpathlineto{\pgfqpoint{1.417113in}{2.027340in}}%
\pgfpathlineto{\pgfqpoint{1.432788in}{2.047014in}}%
\pgfpathlineto{\pgfqpoint{1.448463in}{2.067660in}}%
\pgfpathlineto{\pgfqpoint{1.464138in}{2.088634in}}%
\pgfpathlineto{\pgfqpoint{1.479813in}{2.109030in}}%
\pgfpathlineto{\pgfqpoint{1.495487in}{2.129961in}}%
\pgfpathlineto{\pgfqpoint{1.511162in}{2.149570in}}%
\pgfpathlineto{\pgfqpoint{1.526837in}{2.168735in}}%
\pgfpathlineto{\pgfqpoint{1.542512in}{2.189043in}}%
\pgfpathlineto{\pgfqpoint{1.558187in}{2.208869in}}%
\pgfpathlineto{\pgfqpoint{1.573862in}{2.229507in}}%
\pgfpathlineto{\pgfqpoint{1.589537in}{2.250198in}}%
\pgfpathlineto{\pgfqpoint{1.605211in}{2.272775in}}%
\pgfpathlineto{\pgfqpoint{1.620886in}{2.290568in}}%
\pgfpathlineto{\pgfqpoint{1.636561in}{2.309680in}}%
\pgfpathlineto{\pgfqpoint{1.652236in}{2.327904in}}%
\pgfpathlineto{\pgfqpoint{1.667911in}{2.345783in}}%
\pgfpathlineto{\pgfqpoint{1.683586in}{2.362899in}}%
\pgfpathlineto{\pgfqpoint{1.699261in}{2.380609in}}%
\pgfpathlineto{\pgfqpoint{1.714935in}{2.395701in}}%
\pgfpathlineto{\pgfqpoint{1.730610in}{2.412268in}}%
\pgfpathlineto{\pgfqpoint{1.746285in}{2.427493in}}%
\pgfpathlineto{\pgfqpoint{1.761960in}{2.442289in}}%
\pgfpathlineto{\pgfqpoint{1.777635in}{2.457206in}}%
\pgfpathlineto{\pgfqpoint{1.793310in}{2.470978in}}%
\pgfpathlineto{\pgfqpoint{1.808985in}{2.483887in}}%
\pgfpathlineto{\pgfqpoint{1.824659in}{2.498630in}}%
\pgfpathlineto{\pgfqpoint{1.840334in}{2.512551in}}%
\pgfpathlineto{\pgfqpoint{1.856009in}{2.526051in}}%
\pgfpathlineto{\pgfqpoint{1.871684in}{2.539727in}}%
\pgfpathlineto{\pgfqpoint{1.887359in}{2.554223in}}%
\pgfpathlineto{\pgfqpoint{1.903034in}{2.568380in}}%
\pgfpathlineto{\pgfqpoint{1.918709in}{2.583034in}}%
\pgfpathlineto{\pgfqpoint{1.934383in}{2.598037in}}%
\pgfpathlineto{\pgfqpoint{1.950058in}{2.611283in}}%
\pgfpathlineto{\pgfqpoint{1.965733in}{2.625376in}}%
\pgfpathlineto{\pgfqpoint{1.981408in}{2.636319in}}%
\pgfpathlineto{\pgfqpoint{1.997083in}{2.645677in}}%
\pgfpathlineto{\pgfqpoint{2.012758in}{2.652599in}}%
\pgfpathlineto{\pgfqpoint{2.028433in}{2.661813in}}%
\pgfpathlineto{\pgfqpoint{2.044107in}{2.664489in}}%
\pgfpathlineto{\pgfqpoint{2.059782in}{2.662622in}}%
\pgfpathlineto{\pgfqpoint{2.075457in}{2.663587in}}%
\pgfpathlineto{\pgfqpoint{2.091132in}{2.662423in}}%
\pgfpathlineto{\pgfqpoint{2.106807in}{2.660849in}}%
\pgfpathlineto{\pgfqpoint{2.122482in}{2.660869in}}%
\pgfpathlineto{\pgfqpoint{2.138157in}{2.660255in}}%
\pgfpathlineto{\pgfqpoint{2.153831in}{2.660448in}}%
\pgfpathlineto{\pgfqpoint{2.169506in}{2.656936in}}%
\pgfpathlineto{\pgfqpoint{2.185181in}{2.656214in}}%
\pgfpathlineto{\pgfqpoint{2.200856in}{2.657048in}}%
\pgfpathlineto{\pgfqpoint{2.216531in}{2.654544in}}%
\pgfpathlineto{\pgfqpoint{2.232206in}{2.652933in}}%
\pgfpathlineto{\pgfqpoint{2.247881in}{2.649622in}}%
\pgfpathlineto{\pgfqpoint{2.263555in}{2.646784in}}%
\pgfpathlineto{\pgfqpoint{2.279230in}{2.647504in}}%
\pgfpathlineto{\pgfqpoint{2.294905in}{2.645273in}}%
\pgfpathlineto{\pgfqpoint{2.310580in}{2.643881in}}%
\pgfpathlineto{\pgfqpoint{2.326255in}{2.641404in}}%
\pgfpathlineto{\pgfqpoint{2.341930in}{2.638472in}}%
\pgfusepath{stroke}%
\end{pgfscope}%
\begin{pgfscope}%
\pgfsetrectcap%
\pgfsetmiterjoin%
\pgfsetlinewidth{0.803000pt}%
\definecolor{currentstroke}{rgb}{0.000000,0.000000,0.000000}%
\pgfsetstrokecolor{currentstroke}%
\pgfsetdash{}{0pt}%
\pgfpathmoveto{\pgfqpoint{0.774444in}{0.571604in}}%
\pgfpathlineto{\pgfqpoint{0.774444in}{2.702810in}}%
\pgfusepath{stroke}%
\end{pgfscope}%
\begin{pgfscope}%
\pgfsetrectcap%
\pgfsetmiterjoin%
\pgfsetlinewidth{0.803000pt}%
\definecolor{currentstroke}{rgb}{0.000000,0.000000,0.000000}%
\pgfsetstrokecolor{currentstroke}%
\pgfsetdash{}{0pt}%
\pgfpathmoveto{\pgfqpoint{2.341930in}{0.571604in}}%
\pgfpathlineto{\pgfqpoint{2.341930in}{2.702810in}}%
\pgfusepath{stroke}%
\end{pgfscope}%
\begin{pgfscope}%
\pgfsetrectcap%
\pgfsetmiterjoin%
\pgfsetlinewidth{0.803000pt}%
\definecolor{currentstroke}{rgb}{0.000000,0.000000,0.000000}%
\pgfsetstrokecolor{currentstroke}%
\pgfsetdash{}{0pt}%
\pgfpathmoveto{\pgfqpoint{0.774444in}{0.571604in}}%
\pgfpathlineto{\pgfqpoint{2.341930in}{0.571604in}}%
\pgfusepath{stroke}%
\end{pgfscope}%
\begin{pgfscope}%
\pgfsetrectcap%
\pgfsetmiterjoin%
\pgfsetlinewidth{0.803000pt}%
\definecolor{currentstroke}{rgb}{0.000000,0.000000,0.000000}%
\pgfsetstrokecolor{currentstroke}%
\pgfsetdash{}{0pt}%
\pgfpathmoveto{\pgfqpoint{0.774444in}{2.702810in}}%
\pgfpathlineto{\pgfqpoint{2.341930in}{2.702810in}}%
\pgfusepath{stroke}%
\end{pgfscope}%
\begin{pgfscope}%
\pgfsetbuttcap%
\pgfsetroundjoin%
\definecolor{currentfill}{rgb}{1.000000,0.000000,0.000000}%
\pgfsetfillcolor{currentfill}%
\pgfsetlinewidth{0.803000pt}%
\definecolor{currentstroke}{rgb}{1.000000,0.000000,0.000000}%
\pgfsetstrokecolor{currentstroke}%
\pgfsetdash{}{0pt}%
\pgfsys@defobject{currentmarker}{\pgfqpoint{0.000000in}{0.000000in}}{\pgfqpoint{0.048611in}{0.000000in}}{%
\pgfpathmoveto{\pgfqpoint{0.000000in}{0.000000in}}%
\pgfpathlineto{\pgfqpoint{0.048611in}{0.000000in}}%
\pgfusepath{stroke,fill}%
}%
\begin{pgfscope}%
\pgfsys@transformshift{2.341930in}{0.571604in}%
\pgfsys@useobject{currentmarker}{}%
\end{pgfscope}%
\end{pgfscope}%
\begin{pgfscope}%
\definecolor{textcolor}{rgb}{1.000000,0.000000,0.000000}%
\pgfsetstrokecolor{textcolor}%
\pgfsetfillcolor{textcolor}%
\pgftext[x=2.439152in, y=0.518842in, left, base]{\color{textcolor}\sffamily\fontsize{10.000000}{12.000000}\selectfont 7.00}%
\end{pgfscope}%
\begin{pgfscope}%
\pgfsetbuttcap%
\pgfsetroundjoin%
\definecolor{currentfill}{rgb}{1.000000,0.000000,0.000000}%
\pgfsetfillcolor{currentfill}%
\pgfsetlinewidth{0.803000pt}%
\definecolor{currentstroke}{rgb}{1.000000,0.000000,0.000000}%
\pgfsetstrokecolor{currentstroke}%
\pgfsetdash{}{0pt}%
\pgfsys@defobject{currentmarker}{\pgfqpoint{0.000000in}{0.000000in}}{\pgfqpoint{0.048611in}{0.000000in}}{%
\pgfpathmoveto{\pgfqpoint{0.000000in}{0.000000in}}%
\pgfpathlineto{\pgfqpoint{0.048611in}{0.000000in}}%
\pgfusepath{stroke,fill}%
}%
\begin{pgfscope}%
\pgfsys@transformshift{2.341930in}{0.926805in}%
\pgfsys@useobject{currentmarker}{}%
\end{pgfscope}%
\end{pgfscope}%
\begin{pgfscope}%
\definecolor{textcolor}{rgb}{1.000000,0.000000,0.000000}%
\pgfsetstrokecolor{textcolor}%
\pgfsetfillcolor{textcolor}%
\pgftext[x=2.439152in, y=0.874043in, left, base]{\color{textcolor}\sffamily\fontsize{10.000000}{12.000000}\selectfont 7.25}%
\end{pgfscope}%
\begin{pgfscope}%
\pgfsetbuttcap%
\pgfsetroundjoin%
\definecolor{currentfill}{rgb}{1.000000,0.000000,0.000000}%
\pgfsetfillcolor{currentfill}%
\pgfsetlinewidth{0.803000pt}%
\definecolor{currentstroke}{rgb}{1.000000,0.000000,0.000000}%
\pgfsetstrokecolor{currentstroke}%
\pgfsetdash{}{0pt}%
\pgfsys@defobject{currentmarker}{\pgfqpoint{0.000000in}{0.000000in}}{\pgfqpoint{0.048611in}{0.000000in}}{%
\pgfpathmoveto{\pgfqpoint{0.000000in}{0.000000in}}%
\pgfpathlineto{\pgfqpoint{0.048611in}{0.000000in}}%
\pgfusepath{stroke,fill}%
}%
\begin{pgfscope}%
\pgfsys@transformshift{2.341930in}{1.282006in}%
\pgfsys@useobject{currentmarker}{}%
\end{pgfscope}%
\end{pgfscope}%
\begin{pgfscope}%
\definecolor{textcolor}{rgb}{1.000000,0.000000,0.000000}%
\pgfsetstrokecolor{textcolor}%
\pgfsetfillcolor{textcolor}%
\pgftext[x=2.439152in, y=1.229244in, left, base]{\color{textcolor}\sffamily\fontsize{10.000000}{12.000000}\selectfont 7.50}%
\end{pgfscope}%
\begin{pgfscope}%
\pgfsetbuttcap%
\pgfsetroundjoin%
\definecolor{currentfill}{rgb}{1.000000,0.000000,0.000000}%
\pgfsetfillcolor{currentfill}%
\pgfsetlinewidth{0.803000pt}%
\definecolor{currentstroke}{rgb}{1.000000,0.000000,0.000000}%
\pgfsetstrokecolor{currentstroke}%
\pgfsetdash{}{0pt}%
\pgfsys@defobject{currentmarker}{\pgfqpoint{0.000000in}{0.000000in}}{\pgfqpoint{0.048611in}{0.000000in}}{%
\pgfpathmoveto{\pgfqpoint{0.000000in}{0.000000in}}%
\pgfpathlineto{\pgfqpoint{0.048611in}{0.000000in}}%
\pgfusepath{stroke,fill}%
}%
\begin{pgfscope}%
\pgfsys@transformshift{2.341930in}{1.637207in}%
\pgfsys@useobject{currentmarker}{}%
\end{pgfscope}%
\end{pgfscope}%
\begin{pgfscope}%
\definecolor{textcolor}{rgb}{1.000000,0.000000,0.000000}%
\pgfsetstrokecolor{textcolor}%
\pgfsetfillcolor{textcolor}%
\pgftext[x=2.439152in, y=1.584446in, left, base]{\color{textcolor}\sffamily\fontsize{10.000000}{12.000000}\selectfont 7.75}%
\end{pgfscope}%
\begin{pgfscope}%
\pgfsetbuttcap%
\pgfsetroundjoin%
\definecolor{currentfill}{rgb}{1.000000,0.000000,0.000000}%
\pgfsetfillcolor{currentfill}%
\pgfsetlinewidth{0.803000pt}%
\definecolor{currentstroke}{rgb}{1.000000,0.000000,0.000000}%
\pgfsetstrokecolor{currentstroke}%
\pgfsetdash{}{0pt}%
\pgfsys@defobject{currentmarker}{\pgfqpoint{0.000000in}{0.000000in}}{\pgfqpoint{0.048611in}{0.000000in}}{%
\pgfpathmoveto{\pgfqpoint{0.000000in}{0.000000in}}%
\pgfpathlineto{\pgfqpoint{0.048611in}{0.000000in}}%
\pgfusepath{stroke,fill}%
}%
\begin{pgfscope}%
\pgfsys@transformshift{2.341930in}{1.992408in}%
\pgfsys@useobject{currentmarker}{}%
\end{pgfscope}%
\end{pgfscope}%
\begin{pgfscope}%
\definecolor{textcolor}{rgb}{1.000000,0.000000,0.000000}%
\pgfsetstrokecolor{textcolor}%
\pgfsetfillcolor{textcolor}%
\pgftext[x=2.439152in, y=1.939647in, left, base]{\color{textcolor}\sffamily\fontsize{10.000000}{12.000000}\selectfont 8.00}%
\end{pgfscope}%
\begin{pgfscope}%
\pgfsetbuttcap%
\pgfsetroundjoin%
\definecolor{currentfill}{rgb}{1.000000,0.000000,0.000000}%
\pgfsetfillcolor{currentfill}%
\pgfsetlinewidth{0.803000pt}%
\definecolor{currentstroke}{rgb}{1.000000,0.000000,0.000000}%
\pgfsetstrokecolor{currentstroke}%
\pgfsetdash{}{0pt}%
\pgfsys@defobject{currentmarker}{\pgfqpoint{0.000000in}{0.000000in}}{\pgfqpoint{0.048611in}{0.000000in}}{%
\pgfpathmoveto{\pgfqpoint{0.000000in}{0.000000in}}%
\pgfpathlineto{\pgfqpoint{0.048611in}{0.000000in}}%
\pgfusepath{stroke,fill}%
}%
\begin{pgfscope}%
\pgfsys@transformshift{2.341930in}{2.347609in}%
\pgfsys@useobject{currentmarker}{}%
\end{pgfscope}%
\end{pgfscope}%
\begin{pgfscope}%
\definecolor{textcolor}{rgb}{1.000000,0.000000,0.000000}%
\pgfsetstrokecolor{textcolor}%
\pgfsetfillcolor{textcolor}%
\pgftext[x=2.439152in, y=2.294848in, left, base]{\color{textcolor}\sffamily\fontsize{10.000000}{12.000000}\selectfont 8.25}%
\end{pgfscope}%
\begin{pgfscope}%
\pgfsetbuttcap%
\pgfsetroundjoin%
\definecolor{currentfill}{rgb}{1.000000,0.000000,0.000000}%
\pgfsetfillcolor{currentfill}%
\pgfsetlinewidth{0.803000pt}%
\definecolor{currentstroke}{rgb}{1.000000,0.000000,0.000000}%
\pgfsetstrokecolor{currentstroke}%
\pgfsetdash{}{0pt}%
\pgfsys@defobject{currentmarker}{\pgfqpoint{0.000000in}{0.000000in}}{\pgfqpoint{0.048611in}{0.000000in}}{%
\pgfpathmoveto{\pgfqpoint{0.000000in}{0.000000in}}%
\pgfpathlineto{\pgfqpoint{0.048611in}{0.000000in}}%
\pgfusepath{stroke,fill}%
}%
\begin{pgfscope}%
\pgfsys@transformshift{2.341930in}{2.702810in}%
\pgfsys@useobject{currentmarker}{}%
\end{pgfscope}%
\end{pgfscope}%
\begin{pgfscope}%
\definecolor{textcolor}{rgb}{1.000000,0.000000,0.000000}%
\pgfsetstrokecolor{textcolor}%
\pgfsetfillcolor{textcolor}%
\pgftext[x=2.439152in, y=2.650049in, left, base]{\color{textcolor}\sffamily\fontsize{10.000000}{12.000000}\selectfont 8.50}%
\end{pgfscope}%
\begin{pgfscope}%
\definecolor{textcolor}{rgb}{1.000000,0.000000,0.000000}%
\pgfsetstrokecolor{textcolor}%
\pgfsetfillcolor{textcolor}%
\pgftext[x=2.803952in,y=1.637207in,,top,rotate=90.000000]{\color{textcolor}\sffamily\fontsize{10.000000}{12.000000}\selectfont MSD Metrik}%
\end{pgfscope}%
\begin{pgfscope}%
\definecolor{textcolor}{rgb}{1.000000,0.000000,0.000000}%
\pgfsetstrokecolor{textcolor}%
\pgfsetfillcolor{textcolor}%
\pgftext[x=2.341930in,y=2.744477in,right,base]{\color{textcolor}\sffamily\fontsize{10.000000}{12.000000}\selectfont 1e5}%
\end{pgfscope}%
\begin{pgfscope}%
\pgfpathrectangle{\pgfqpoint{0.774444in}{0.571604in}}{\pgfqpoint{1.567486in}{2.131206in}}%
\pgfusepath{clip}%
\pgfsetrectcap%
\pgfsetroundjoin%
\pgfsetlinewidth{1.505625pt}%
\definecolor{currentstroke}{rgb}{1.000000,0.000000,0.000000}%
\pgfsetstrokecolor{currentstroke}%
\pgfsetdash{}{0pt}%
\pgfpathmoveto{\pgfqpoint{0.774444in}{2.624311in}}%
\pgfpathlineto{\pgfqpoint{0.790119in}{2.592968in}}%
\pgfpathlineto{\pgfqpoint{0.805794in}{2.558783in}}%
\pgfpathlineto{\pgfqpoint{0.821469in}{2.510575in}}%
\pgfpathlineto{\pgfqpoint{0.837143in}{2.464513in}}%
\pgfpathlineto{\pgfqpoint{0.852818in}{2.427345in}}%
\pgfpathlineto{\pgfqpoint{0.868493in}{2.381581in}}%
\pgfpathlineto{\pgfqpoint{0.884168in}{2.332293in}}%
\pgfpathlineto{\pgfqpoint{0.899843in}{2.278785in}}%
\pgfpathlineto{\pgfqpoint{0.915518in}{2.229270in}}%
\pgfpathlineto{\pgfqpoint{0.931193in}{2.184288in}}%
\pgfpathlineto{\pgfqpoint{0.946867in}{2.141394in}}%
\pgfpathlineto{\pgfqpoint{0.962542in}{2.096738in}}%
\pgfpathlineto{\pgfqpoint{0.978217in}{2.058859in}}%
\pgfpathlineto{\pgfqpoint{0.993892in}{2.024419in}}%
\pgfpathlineto{\pgfqpoint{1.009567in}{1.990717in}}%
\pgfpathlineto{\pgfqpoint{1.025242in}{1.956746in}}%
\pgfpathlineto{\pgfqpoint{1.040917in}{1.924011in}}%
\pgfpathlineto{\pgfqpoint{1.056591in}{1.891233in}}%
\pgfpathlineto{\pgfqpoint{1.072266in}{1.857759in}}%
\pgfpathlineto{\pgfqpoint{1.087941in}{1.822892in}}%
\pgfpathlineto{\pgfqpoint{1.103616in}{1.789958in}}%
\pgfpathlineto{\pgfqpoint{1.119291in}{1.756654in}}%
\pgfpathlineto{\pgfqpoint{1.134966in}{1.721148in}}%
\pgfpathlineto{\pgfqpoint{1.150641in}{1.685372in}}%
\pgfpathlineto{\pgfqpoint{1.166315in}{1.651131in}}%
\pgfpathlineto{\pgfqpoint{1.181990in}{1.616676in}}%
\pgfpathlineto{\pgfqpoint{1.197665in}{1.580020in}}%
\pgfpathlineto{\pgfqpoint{1.213340in}{1.544201in}}%
\pgfpathlineto{\pgfqpoint{1.229015in}{1.508283in}}%
\pgfpathlineto{\pgfqpoint{1.244690in}{1.467023in}}%
\pgfpathlineto{\pgfqpoint{1.260365in}{1.425976in}}%
\pgfpathlineto{\pgfqpoint{1.276039in}{1.385128in}}%
\pgfpathlineto{\pgfqpoint{1.291714in}{1.338625in}}%
\pgfpathlineto{\pgfqpoint{1.307389in}{1.292904in}}%
\pgfpathlineto{\pgfqpoint{1.323064in}{1.243460in}}%
\pgfpathlineto{\pgfqpoint{1.338739in}{1.190066in}}%
\pgfpathlineto{\pgfqpoint{1.354414in}{1.138647in}}%
\pgfpathlineto{\pgfqpoint{1.370089in}{1.087342in}}%
\pgfpathlineto{\pgfqpoint{1.385763in}{1.043311in}}%
\pgfpathlineto{\pgfqpoint{1.401438in}{1.005304in}}%
\pgfpathlineto{\pgfqpoint{1.417113in}{0.969713in}}%
\pgfpathlineto{\pgfqpoint{1.432788in}{0.936921in}}%
\pgfpathlineto{\pgfqpoint{1.448463in}{0.907340in}}%
\pgfpathlineto{\pgfqpoint{1.464138in}{0.881538in}}%
\pgfpathlineto{\pgfqpoint{1.479813in}{0.857583in}}%
\pgfpathlineto{\pgfqpoint{1.495487in}{0.835120in}}%
\pgfpathlineto{\pgfqpoint{1.511162in}{0.809603in}}%
\pgfpathlineto{\pgfqpoint{1.526837in}{0.783616in}}%
\pgfpathlineto{\pgfqpoint{1.542512in}{0.759662in}}%
\pgfpathlineto{\pgfqpoint{1.558187in}{0.737014in}}%
\pgfpathlineto{\pgfqpoint{1.573862in}{0.718430in}}%
\pgfpathlineto{\pgfqpoint{1.589537in}{0.702332in}}%
\pgfpathlineto{\pgfqpoint{1.605211in}{0.687698in}}%
\pgfpathlineto{\pgfqpoint{1.620886in}{0.675720in}}%
\pgfpathlineto{\pgfqpoint{1.636561in}{0.665448in}}%
\pgfpathlineto{\pgfqpoint{1.652236in}{0.656596in}}%
\pgfpathlineto{\pgfqpoint{1.667911in}{0.649805in}}%
\pgfpathlineto{\pgfqpoint{1.683586in}{0.643667in}}%
\pgfpathlineto{\pgfqpoint{1.699261in}{0.640712in}}%
\pgfpathlineto{\pgfqpoint{1.714935in}{0.637785in}}%
\pgfpathlineto{\pgfqpoint{1.730610in}{0.636663in}}%
\pgfpathlineto{\pgfqpoint{1.746285in}{0.637842in}}%
\pgfpathlineto{\pgfqpoint{1.761960in}{0.637188in}}%
\pgfpathlineto{\pgfqpoint{1.777635in}{0.637828in}}%
\pgfpathlineto{\pgfqpoint{1.793310in}{0.638254in}}%
\pgfpathlineto{\pgfqpoint{1.808985in}{0.638098in}}%
\pgfpathlineto{\pgfqpoint{1.824659in}{0.638410in}}%
\pgfpathlineto{\pgfqpoint{1.840334in}{0.638623in}}%
\pgfpathlineto{\pgfqpoint{1.856009in}{0.636705in}}%
\pgfpathlineto{\pgfqpoint{1.871684in}{0.631775in}}%
\pgfpathlineto{\pgfqpoint{1.887359in}{0.629402in}}%
\pgfpathlineto{\pgfqpoint{1.903034in}{0.627996in}}%
\pgfpathlineto{\pgfqpoint{1.918709in}{0.625495in}}%
\pgfpathlineto{\pgfqpoint{1.934383in}{0.619059in}}%
\pgfpathlineto{\pgfqpoint{1.950058in}{0.615549in}}%
\pgfpathlineto{\pgfqpoint{1.965733in}{0.614342in}}%
\pgfpathlineto{\pgfqpoint{1.981408in}{0.615038in}}%
\pgfpathlineto{\pgfqpoint{1.997083in}{0.624600in}}%
\pgfpathlineto{\pgfqpoint{2.012758in}{0.632642in}}%
\pgfpathlineto{\pgfqpoint{2.028433in}{0.640598in}}%
\pgfpathlineto{\pgfqpoint{2.044107in}{0.649720in}}%
\pgfpathlineto{\pgfqpoint{2.059782in}{0.659282in}}%
\pgfpathlineto{\pgfqpoint{2.075457in}{0.666002in}}%
\pgfpathlineto{\pgfqpoint{2.091132in}{0.672580in}}%
\pgfpathlineto{\pgfqpoint{2.106807in}{0.679926in}}%
\pgfpathlineto{\pgfqpoint{2.122482in}{0.687513in}}%
\pgfpathlineto{\pgfqpoint{2.138157in}{0.697075in}}%
\pgfpathlineto{\pgfqpoint{2.153831in}{0.702687in}}%
\pgfpathlineto{\pgfqpoint{2.169506in}{0.709294in}}%
\pgfpathlineto{\pgfqpoint{2.185181in}{0.721513in}}%
\pgfpathlineto{\pgfqpoint{2.200856in}{0.729967in}}%
\pgfpathlineto{\pgfqpoint{2.216531in}{0.735309in}}%
\pgfpathlineto{\pgfqpoint{2.232206in}{0.746093in}}%
\pgfpathlineto{\pgfqpoint{2.247881in}{0.761693in}}%
\pgfpathlineto{\pgfqpoint{2.263555in}{0.771454in}}%
\pgfpathlineto{\pgfqpoint{2.279230in}{0.780007in}}%
\pgfpathlineto{\pgfqpoint{2.294905in}{0.790919in}}%
\pgfpathlineto{\pgfqpoint{2.310580in}{0.798947in}}%
\pgfpathlineto{\pgfqpoint{2.326255in}{0.816707in}}%
\pgfpathlineto{\pgfqpoint{2.341930in}{0.822461in}}%
\pgfusepath{stroke}%
\end{pgfscope}%
\begin{pgfscope}%
\pgfsetrectcap%
\pgfsetmiterjoin%
\pgfsetlinewidth{0.803000pt}%
\definecolor{currentstroke}{rgb}{0.000000,0.000000,0.000000}%
\pgfsetstrokecolor{currentstroke}%
\pgfsetdash{}{0pt}%
\pgfpathmoveto{\pgfqpoint{0.774444in}{0.571604in}}%
\pgfpathlineto{\pgfqpoint{0.774444in}{2.702810in}}%
\pgfusepath{stroke}%
\end{pgfscope}%
\begin{pgfscope}%
\pgfsetrectcap%
\pgfsetmiterjoin%
\pgfsetlinewidth{0.803000pt}%
\definecolor{currentstroke}{rgb}{0.000000,0.000000,0.000000}%
\pgfsetstrokecolor{currentstroke}%
\pgfsetdash{}{0pt}%
\pgfpathmoveto{\pgfqpoint{2.341930in}{0.571604in}}%
\pgfpathlineto{\pgfqpoint{2.341930in}{2.702810in}}%
\pgfusepath{stroke}%
\end{pgfscope}%
\begin{pgfscope}%
\pgfsetrectcap%
\pgfsetmiterjoin%
\pgfsetlinewidth{0.803000pt}%
\definecolor{currentstroke}{rgb}{0.000000,0.000000,0.000000}%
\pgfsetstrokecolor{currentstroke}%
\pgfsetdash{}{0pt}%
\pgfpathmoveto{\pgfqpoint{0.774444in}{0.571604in}}%
\pgfpathlineto{\pgfqpoint{2.341930in}{0.571604in}}%
\pgfusepath{stroke}%
\end{pgfscope}%
\begin{pgfscope}%
\pgfsetrectcap%
\pgfsetmiterjoin%
\pgfsetlinewidth{0.803000pt}%
\definecolor{currentstroke}{rgb}{0.000000,0.000000,0.000000}%
\pgfsetstrokecolor{currentstroke}%
\pgfsetdash{}{0pt}%
\pgfpathmoveto{\pgfqpoint{0.774444in}{2.702810in}}%
\pgfpathlineto{\pgfqpoint{2.341930in}{2.702810in}}%
\pgfusepath{stroke}%
\end{pgfscope}%
\end{pgfpicture}%
\makeatother%
\endgroup%
}
  \hfill
  \resizebox{0.48\linewidth}{!}{%% Creator: Matplotlib, PGF backend
%%
%% To include the figure in your LaTeX document, write
%%   \input{<filename>.pgf}
%%
%% Make sure the required packages are loaded in your preamble
%%   \usepackage{pgf}
%%
%% and, on pdftex
%%   \usepackage[utf8]{inputenc}\DeclareUnicodeCharacter{2212}{-}
%%
%% or, on luatex and xetex
%%   \usepackage{unicode-math}
%%
%% Figures using additional raster images can only be included by \input if
%% they are in the same directory as the main LaTeX file. For loading figures
%% from other directories you can use the `import` package
%%   \usepackage{import}
%%
%% and then include the figures with
%%   \import{<path to file>}{<filename>.pgf}
%%
%% Matplotlib used the following preamble
%%   \usepackage{fontspec}
%%   \setmainfont{DejaVuSerif.ttf}[Path=/usr/share/matplotlib/mpl-data/fonts/ttf/]
%%   \setsansfont{DejaVuSans.ttf}[Path=/usr/share/matplotlib/mpl-data/fonts/ttf/]
%%   \setmonofont{DejaVuSansMono.ttf}[Path=/usr/share/matplotlib/mpl-data/fonts/ttf/]
%%
\begingroup%
\makeatletter%
\begin{pgfpicture}%
\pgfpathrectangle{\pgfpointorigin}{\pgfqpoint{3.000000in}{3.000000in}}%
\pgfusepath{use as bounding box, clip}%
\begin{pgfscope}%
\pgfsetbuttcap%
\pgfsetmiterjoin%
\definecolor{currentfill}{rgb}{1.000000,1.000000,1.000000}%
\pgfsetfillcolor{currentfill}%
\pgfsetlinewidth{0.000000pt}%
\definecolor{currentstroke}{rgb}{1.000000,1.000000,1.000000}%
\pgfsetstrokecolor{currentstroke}%
\pgfsetdash{}{0pt}%
\pgfpathmoveto{\pgfqpoint{0.000000in}{0.000000in}}%
\pgfpathlineto{\pgfqpoint{3.000000in}{0.000000in}}%
\pgfpathlineto{\pgfqpoint{3.000000in}{3.000000in}}%
\pgfpathlineto{\pgfqpoint{0.000000in}{3.000000in}}%
\pgfpathclose%
\pgfusepath{fill}%
\end{pgfscope}%
\begin{pgfscope}%
\pgfsetbuttcap%
\pgfsetmiterjoin%
\definecolor{currentfill}{rgb}{1.000000,1.000000,1.000000}%
\pgfsetfillcolor{currentfill}%
\pgfsetlinewidth{0.000000pt}%
\definecolor{currentstroke}{rgb}{0.000000,0.000000,0.000000}%
\pgfsetstrokecolor{currentstroke}%
\pgfsetstrokeopacity{0.000000}%
\pgfsetdash{}{0pt}%
\pgfpathmoveto{\pgfqpoint{1.039540in}{0.571604in}}%
\pgfpathlineto{\pgfqpoint{2.165199in}{0.571604in}}%
\pgfpathlineto{\pgfqpoint{2.165199in}{2.702810in}}%
\pgfpathlineto{\pgfqpoint{1.039540in}{2.702810in}}%
\pgfpathclose%
\pgfusepath{fill}%
\end{pgfscope}%
\begin{pgfscope}%
\pgfsetbuttcap%
\pgfsetroundjoin%
\definecolor{currentfill}{rgb}{0.000000,0.000000,0.000000}%
\pgfsetfillcolor{currentfill}%
\pgfsetlinewidth{0.803000pt}%
\definecolor{currentstroke}{rgb}{0.000000,0.000000,0.000000}%
\pgfsetstrokecolor{currentstroke}%
\pgfsetdash{}{0pt}%
\pgfsys@defobject{currentmarker}{\pgfqpoint{0.000000in}{-0.048611in}}{\pgfqpoint{0.000000in}{0.000000in}}{%
\pgfpathmoveto{\pgfqpoint{0.000000in}{0.000000in}}%
\pgfpathlineto{\pgfqpoint{0.000000in}{-0.048611in}}%
\pgfusepath{stroke,fill}%
}%
\begin{pgfscope}%
\pgfsys@transformshift{1.039540in}{0.571604in}%
\pgfsys@useobject{currentmarker}{}%
\end{pgfscope}%
\end{pgfscope}%
\begin{pgfscope}%
\definecolor{textcolor}{rgb}{0.000000,0.000000,0.000000}%
\pgfsetstrokecolor{textcolor}%
\pgfsetfillcolor{textcolor}%
\pgftext[x=1.039540in,y=0.474382in,,top]{\color{textcolor}\sffamily\fontsize{10.000000}{12.000000}\selectfont 0.0}%
\end{pgfscope}%
\begin{pgfscope}%
\pgfsetbuttcap%
\pgfsetroundjoin%
\definecolor{currentfill}{rgb}{0.000000,0.000000,0.000000}%
\pgfsetfillcolor{currentfill}%
\pgfsetlinewidth{0.803000pt}%
\definecolor{currentstroke}{rgb}{0.000000,0.000000,0.000000}%
\pgfsetstrokecolor{currentstroke}%
\pgfsetdash{}{0pt}%
\pgfsys@defobject{currentmarker}{\pgfqpoint{0.000000in}{-0.048611in}}{\pgfqpoint{0.000000in}{0.000000in}}{%
\pgfpathmoveto{\pgfqpoint{0.000000in}{0.000000in}}%
\pgfpathlineto{\pgfqpoint{0.000000in}{-0.048611in}}%
\pgfusepath{stroke,fill}%
}%
\begin{pgfscope}%
\pgfsys@transformshift{1.602370in}{0.571604in}%
\pgfsys@useobject{currentmarker}{}%
\end{pgfscope}%
\end{pgfscope}%
\begin{pgfscope}%
\definecolor{textcolor}{rgb}{0.000000,0.000000,0.000000}%
\pgfsetstrokecolor{textcolor}%
\pgfsetfillcolor{textcolor}%
\pgftext[x=1.602370in,y=0.474382in,,top]{\color{textcolor}\sffamily\fontsize{10.000000}{12.000000}\selectfont 0.5}%
\end{pgfscope}%
\begin{pgfscope}%
\pgfsetbuttcap%
\pgfsetroundjoin%
\definecolor{currentfill}{rgb}{0.000000,0.000000,0.000000}%
\pgfsetfillcolor{currentfill}%
\pgfsetlinewidth{0.803000pt}%
\definecolor{currentstroke}{rgb}{0.000000,0.000000,0.000000}%
\pgfsetstrokecolor{currentstroke}%
\pgfsetdash{}{0pt}%
\pgfsys@defobject{currentmarker}{\pgfqpoint{0.000000in}{-0.048611in}}{\pgfqpoint{0.000000in}{0.000000in}}{%
\pgfpathmoveto{\pgfqpoint{0.000000in}{0.000000in}}%
\pgfpathlineto{\pgfqpoint{0.000000in}{-0.048611in}}%
\pgfusepath{stroke,fill}%
}%
\begin{pgfscope}%
\pgfsys@transformshift{2.165199in}{0.571604in}%
\pgfsys@useobject{currentmarker}{}%
\end{pgfscope}%
\end{pgfscope}%
\begin{pgfscope}%
\definecolor{textcolor}{rgb}{0.000000,0.000000,0.000000}%
\pgfsetstrokecolor{textcolor}%
\pgfsetfillcolor{textcolor}%
\pgftext[x=2.165199in,y=0.474382in,,top]{\color{textcolor}\sffamily\fontsize{10.000000}{12.000000}\selectfont 1.0}%
\end{pgfscope}%
\begin{pgfscope}%
\definecolor{textcolor}{rgb}{0.000000,0.000000,0.000000}%
\pgfsetstrokecolor{textcolor}%
\pgfsetfillcolor{textcolor}%
\pgftext[x=1.602370in,y=0.284413in,,top]{\color{textcolor}\sffamily\fontsize{10.000000}{12.000000}\selectfont Shift von Paramter 3}%
\end{pgfscope}%
\begin{pgfscope}%
\pgfsetbuttcap%
\pgfsetroundjoin%
\definecolor{currentfill}{rgb}{0.000000,0.000000,1.000000}%
\pgfsetfillcolor{currentfill}%
\pgfsetlinewidth{0.803000pt}%
\definecolor{currentstroke}{rgb}{0.000000,0.000000,1.000000}%
\pgfsetstrokecolor{currentstroke}%
\pgfsetdash{}{0pt}%
\pgfsys@defobject{currentmarker}{\pgfqpoint{-0.048611in}{0.000000in}}{\pgfqpoint{0.000000in}{0.000000in}}{%
\pgfpathmoveto{\pgfqpoint{0.000000in}{0.000000in}}%
\pgfpathlineto{\pgfqpoint{-0.048611in}{0.000000in}}%
\pgfusepath{stroke,fill}%
}%
\begin{pgfscope}%
\pgfsys@transformshift{1.039540in}{0.823801in}%
\pgfsys@useobject{currentmarker}{}%
\end{pgfscope}%
\end{pgfscope}%
\begin{pgfscope}%
\definecolor{textcolor}{rgb}{0.000000,0.000000,1.000000}%
\pgfsetstrokecolor{textcolor}%
\pgfsetfillcolor{textcolor}%
\pgftext[x=0.428334in, y=0.771039in, left, base]{\color{textcolor}\sffamily\fontsize{10.000000}{12.000000}\selectfont −0.696}%
\end{pgfscope}%
\begin{pgfscope}%
\pgfsetbuttcap%
\pgfsetroundjoin%
\definecolor{currentfill}{rgb}{0.000000,0.000000,1.000000}%
\pgfsetfillcolor{currentfill}%
\pgfsetlinewidth{0.803000pt}%
\definecolor{currentstroke}{rgb}{0.000000,0.000000,1.000000}%
\pgfsetstrokecolor{currentstroke}%
\pgfsetdash{}{0pt}%
\pgfsys@defobject{currentmarker}{\pgfqpoint{-0.048611in}{0.000000in}}{\pgfqpoint{0.000000in}{0.000000in}}{%
\pgfpathmoveto{\pgfqpoint{0.000000in}{0.000000in}}%
\pgfpathlineto{\pgfqpoint{-0.048611in}{0.000000in}}%
\pgfusepath{stroke,fill}%
}%
\begin{pgfscope}%
\pgfsys@transformshift{1.039540in}{1.368797in}%
\pgfsys@useobject{currentmarker}{}%
\end{pgfscope}%
\end{pgfscope}%
\begin{pgfscope}%
\definecolor{textcolor}{rgb}{0.000000,0.000000,1.000000}%
\pgfsetstrokecolor{textcolor}%
\pgfsetfillcolor{textcolor}%
\pgftext[x=0.428334in, y=1.316035in, left, base]{\color{textcolor}\sffamily\fontsize{10.000000}{12.000000}\selectfont −0.695}%
\end{pgfscope}%
\begin{pgfscope}%
\pgfsetbuttcap%
\pgfsetroundjoin%
\definecolor{currentfill}{rgb}{0.000000,0.000000,1.000000}%
\pgfsetfillcolor{currentfill}%
\pgfsetlinewidth{0.803000pt}%
\definecolor{currentstroke}{rgb}{0.000000,0.000000,1.000000}%
\pgfsetstrokecolor{currentstroke}%
\pgfsetdash{}{0pt}%
\pgfsys@defobject{currentmarker}{\pgfqpoint{-0.048611in}{0.000000in}}{\pgfqpoint{0.000000in}{0.000000in}}{%
\pgfpathmoveto{\pgfqpoint{0.000000in}{0.000000in}}%
\pgfpathlineto{\pgfqpoint{-0.048611in}{0.000000in}}%
\pgfusepath{stroke,fill}%
}%
\begin{pgfscope}%
\pgfsys@transformshift{1.039540in}{1.913792in}%
\pgfsys@useobject{currentmarker}{}%
\end{pgfscope}%
\end{pgfscope}%
\begin{pgfscope}%
\definecolor{textcolor}{rgb}{0.000000,0.000000,1.000000}%
\pgfsetstrokecolor{textcolor}%
\pgfsetfillcolor{textcolor}%
\pgftext[x=0.428334in, y=1.861031in, left, base]{\color{textcolor}\sffamily\fontsize{10.000000}{12.000000}\selectfont −0.694}%
\end{pgfscope}%
\begin{pgfscope}%
\pgfsetbuttcap%
\pgfsetroundjoin%
\definecolor{currentfill}{rgb}{0.000000,0.000000,1.000000}%
\pgfsetfillcolor{currentfill}%
\pgfsetlinewidth{0.803000pt}%
\definecolor{currentstroke}{rgb}{0.000000,0.000000,1.000000}%
\pgfsetstrokecolor{currentstroke}%
\pgfsetdash{}{0pt}%
\pgfsys@defobject{currentmarker}{\pgfqpoint{-0.048611in}{0.000000in}}{\pgfqpoint{0.000000in}{0.000000in}}{%
\pgfpathmoveto{\pgfqpoint{0.000000in}{0.000000in}}%
\pgfpathlineto{\pgfqpoint{-0.048611in}{0.000000in}}%
\pgfusepath{stroke,fill}%
}%
\begin{pgfscope}%
\pgfsys@transformshift{1.039540in}{2.458788in}%
\pgfsys@useobject{currentmarker}{}%
\end{pgfscope}%
\end{pgfscope}%
\begin{pgfscope}%
\definecolor{textcolor}{rgb}{0.000000,0.000000,1.000000}%
\pgfsetstrokecolor{textcolor}%
\pgfsetfillcolor{textcolor}%
\pgftext[x=0.428334in, y=2.406027in, left, base]{\color{textcolor}\sffamily\fontsize{10.000000}{12.000000}\selectfont −0.693}%
\end{pgfscope}%
\begin{pgfscope}%
\definecolor{textcolor}{rgb}{0.000000,0.000000,1.000000}%
\pgfsetstrokecolor{textcolor}%
\pgfsetfillcolor{textcolor}%
\pgftext[x=0.372778in,y=1.637207in,,bottom,rotate=90.000000]{\color{textcolor}\sffamily\fontsize{10.000000}{12.000000}\selectfont MI Metrik}%
\end{pgfscope}%
\begin{pgfscope}%
\pgfpathrectangle{\pgfqpoint{1.039540in}{0.571604in}}{\pgfqpoint{1.125659in}{2.131206in}}%
\pgfusepath{clip}%
\pgfsetrectcap%
\pgfsetroundjoin%
\pgfsetlinewidth{1.505625pt}%
\definecolor{currentstroke}{rgb}{0.000000,0.000000,1.000000}%
\pgfsetstrokecolor{currentstroke}%
\pgfsetdash{}{0pt}%
\pgfpathmoveto{\pgfqpoint{1.039540in}{2.424999in}}%
\pgfpathlineto{\pgfqpoint{1.050797in}{2.354694in}}%
\pgfpathlineto{\pgfqpoint{1.062053in}{2.284935in}}%
\pgfpathlineto{\pgfqpoint{1.073310in}{2.217900in}}%
\pgfpathlineto{\pgfqpoint{1.084566in}{2.154136in}}%
\pgfpathlineto{\pgfqpoint{1.095823in}{2.092551in}}%
\pgfpathlineto{\pgfqpoint{1.107080in}{2.032057in}}%
\pgfpathlineto{\pgfqpoint{1.118336in}{1.973742in}}%
\pgfpathlineto{\pgfqpoint{1.129593in}{1.917062in}}%
\pgfpathlineto{\pgfqpoint{1.140849in}{1.861473in}}%
\pgfpathlineto{\pgfqpoint{1.152106in}{1.806973in}}%
\pgfpathlineto{\pgfqpoint{1.163362in}{1.753019in}}%
\pgfpathlineto{\pgfqpoint{1.174619in}{1.700154in}}%
\pgfpathlineto{\pgfqpoint{1.185876in}{1.648924in}}%
\pgfpathlineto{\pgfqpoint{1.197132in}{1.599330in}}%
\pgfpathlineto{\pgfqpoint{1.208389in}{1.550825in}}%
\pgfpathlineto{\pgfqpoint{1.219645in}{1.503956in}}%
\pgfpathlineto{\pgfqpoint{1.230902in}{1.458176in}}%
\pgfpathlineto{\pgfqpoint{1.242159in}{1.411306in}}%
\pgfpathlineto{\pgfqpoint{1.253415in}{1.365527in}}%
\pgfpathlineto{\pgfqpoint{1.264672in}{1.319747in}}%
\pgfpathlineto{\pgfqpoint{1.275928in}{1.275057in}}%
\pgfpathlineto{\pgfqpoint{1.287185in}{1.231458in}}%
\pgfpathlineto{\pgfqpoint{1.298442in}{1.189493in}}%
\pgfpathlineto{\pgfqpoint{1.309698in}{1.148073in}}%
\pgfpathlineto{\pgfqpoint{1.320955in}{1.108289in}}%
\pgfpathlineto{\pgfqpoint{1.332211in}{1.070139in}}%
\pgfpathlineto{\pgfqpoint{1.343468in}{1.034714in}}%
\pgfpathlineto{\pgfqpoint{1.354725in}{1.002014in}}%
\pgfpathlineto{\pgfqpoint{1.365981in}{0.971495in}}%
\pgfpathlineto{\pgfqpoint{1.377238in}{0.943700in}}%
\pgfpathlineto{\pgfqpoint{1.388494in}{0.917540in}}%
\pgfpathlineto{\pgfqpoint{1.399751in}{0.893560in}}%
\pgfpathlineto{\pgfqpoint{1.411007in}{0.871215in}}%
\pgfpathlineto{\pgfqpoint{1.422264in}{0.849961in}}%
\pgfpathlineto{\pgfqpoint{1.433521in}{0.829796in}}%
\pgfpathlineto{\pgfqpoint{1.444777in}{0.810176in}}%
\pgfpathlineto{\pgfqpoint{1.456034in}{0.791646in}}%
\pgfpathlineto{\pgfqpoint{1.467290in}{0.775296in}}%
\pgfpathlineto{\pgfqpoint{1.478547in}{0.760581in}}%
\pgfpathlineto{\pgfqpoint{1.489804in}{0.745321in}}%
\pgfpathlineto{\pgfqpoint{1.501060in}{0.732786in}}%
\pgfpathlineto{\pgfqpoint{1.512317in}{0.720797in}}%
\pgfpathlineto{\pgfqpoint{1.523573in}{0.709897in}}%
\pgfpathlineto{\pgfqpoint{1.534830in}{0.699542in}}%
\pgfpathlineto{\pgfqpoint{1.546087in}{0.689732in}}%
\pgfpathlineto{\pgfqpoint{1.557343in}{0.681557in}}%
\pgfpathlineto{\pgfqpoint{1.568600in}{0.675562in}}%
\pgfpathlineto{\pgfqpoint{1.579856in}{0.671202in}}%
\pgfpathlineto{\pgfqpoint{1.591113in}{0.669022in}}%
\pgfpathlineto{\pgfqpoint{1.602370in}{0.668477in}}%
\pgfpathlineto{\pgfqpoint{1.613626in}{0.671202in}}%
\pgfpathlineto{\pgfqpoint{1.624883in}{0.677197in}}%
\pgfpathlineto{\pgfqpoint{1.636139in}{0.687007in}}%
\pgfpathlineto{\pgfqpoint{1.647396in}{0.698997in}}%
\pgfpathlineto{\pgfqpoint{1.658652in}{0.711532in}}%
\pgfpathlineto{\pgfqpoint{1.669909in}{0.724611in}}%
\pgfpathlineto{\pgfqpoint{1.681166in}{0.738236in}}%
\pgfpathlineto{\pgfqpoint{1.692422in}{0.750771in}}%
\pgfpathlineto{\pgfqpoint{1.703679in}{0.762761in}}%
\pgfpathlineto{\pgfqpoint{1.714935in}{0.776386in}}%
\pgfpathlineto{\pgfqpoint{1.726192in}{0.793826in}}%
\pgfpathlineto{\pgfqpoint{1.737449in}{0.815081in}}%
\pgfpathlineto{\pgfqpoint{1.748705in}{0.837426in}}%
\pgfpathlineto{\pgfqpoint{1.759962in}{0.861950in}}%
\pgfpathlineto{\pgfqpoint{1.771218in}{0.888110in}}%
\pgfpathlineto{\pgfqpoint{1.782475in}{0.915905in}}%
\pgfpathlineto{\pgfqpoint{1.793732in}{0.944245in}}%
\pgfpathlineto{\pgfqpoint{1.804988in}{0.974220in}}%
\pgfpathlineto{\pgfqpoint{1.816245in}{1.003649in}}%
\pgfpathlineto{\pgfqpoint{1.827501in}{1.034169in}}%
\pgfpathlineto{\pgfqpoint{1.838758in}{1.065779in}}%
\pgfpathlineto{\pgfqpoint{1.850014in}{1.099569in}}%
\pgfpathlineto{\pgfqpoint{1.861271in}{1.134993in}}%
\pgfpathlineto{\pgfqpoint{1.872528in}{1.171508in}}%
\pgfpathlineto{\pgfqpoint{1.883784in}{1.209658in}}%
\pgfpathlineto{\pgfqpoint{1.895041in}{1.250533in}}%
\pgfpathlineto{\pgfqpoint{1.906297in}{1.293042in}}%
\pgfpathlineto{\pgfqpoint{1.917554in}{1.338277in}}%
\pgfpathlineto{\pgfqpoint{1.928811in}{1.385146in}}%
\pgfpathlineto{\pgfqpoint{1.940067in}{1.432561in}}%
\pgfpathlineto{\pgfqpoint{1.951324in}{1.479976in}}%
\pgfpathlineto{\pgfqpoint{1.962580in}{1.527390in}}%
\pgfpathlineto{\pgfqpoint{1.973837in}{1.576440in}}%
\pgfpathlineto{\pgfqpoint{1.985094in}{1.626035in}}%
\pgfpathlineto{\pgfqpoint{1.996350in}{1.678899in}}%
\pgfpathlineto{\pgfqpoint{2.007607in}{1.733399in}}%
\pgfpathlineto{\pgfqpoint{2.018863in}{1.788988in}}%
\pgfpathlineto{\pgfqpoint{2.030120in}{1.847303in}}%
\pgfpathlineto{\pgfqpoint{2.041377in}{1.906163in}}%
\pgfpathlineto{\pgfqpoint{2.052633in}{1.964477in}}%
\pgfpathlineto{\pgfqpoint{2.063890in}{2.023882in}}%
\pgfpathlineto{\pgfqpoint{2.075146in}{2.084921in}}%
\pgfpathlineto{\pgfqpoint{2.086403in}{2.147596in}}%
\pgfpathlineto{\pgfqpoint{2.097659in}{2.211360in}}%
\pgfpathlineto{\pgfqpoint{2.108916in}{2.274580in}}%
\pgfpathlineto{\pgfqpoint{2.120173in}{2.337254in}}%
\pgfpathlineto{\pgfqpoint{2.131429in}{2.401019in}}%
\pgfpathlineto{\pgfqpoint{2.142686in}{2.467508in}}%
\pgfpathlineto{\pgfqpoint{2.153942in}{2.536723in}}%
\pgfpathlineto{\pgfqpoint{2.165199in}{2.605937in}}%
\pgfusepath{stroke}%
\end{pgfscope}%
\begin{pgfscope}%
\pgfsetrectcap%
\pgfsetmiterjoin%
\pgfsetlinewidth{0.803000pt}%
\definecolor{currentstroke}{rgb}{0.000000,0.000000,0.000000}%
\pgfsetstrokecolor{currentstroke}%
\pgfsetdash{}{0pt}%
\pgfpathmoveto{\pgfqpoint{1.039540in}{0.571604in}}%
\pgfpathlineto{\pgfqpoint{1.039540in}{2.702810in}}%
\pgfusepath{stroke}%
\end{pgfscope}%
\begin{pgfscope}%
\pgfsetrectcap%
\pgfsetmiterjoin%
\pgfsetlinewidth{0.803000pt}%
\definecolor{currentstroke}{rgb}{0.000000,0.000000,0.000000}%
\pgfsetstrokecolor{currentstroke}%
\pgfsetdash{}{0pt}%
\pgfpathmoveto{\pgfqpoint{2.165199in}{0.571604in}}%
\pgfpathlineto{\pgfqpoint{2.165199in}{2.702810in}}%
\pgfusepath{stroke}%
\end{pgfscope}%
\begin{pgfscope}%
\pgfsetrectcap%
\pgfsetmiterjoin%
\pgfsetlinewidth{0.803000pt}%
\definecolor{currentstroke}{rgb}{0.000000,0.000000,0.000000}%
\pgfsetstrokecolor{currentstroke}%
\pgfsetdash{}{0pt}%
\pgfpathmoveto{\pgfqpoint{1.039540in}{0.571604in}}%
\pgfpathlineto{\pgfqpoint{2.165199in}{0.571604in}}%
\pgfusepath{stroke}%
\end{pgfscope}%
\begin{pgfscope}%
\pgfsetrectcap%
\pgfsetmiterjoin%
\pgfsetlinewidth{0.803000pt}%
\definecolor{currentstroke}{rgb}{0.000000,0.000000,0.000000}%
\pgfsetstrokecolor{currentstroke}%
\pgfsetdash{}{0pt}%
\pgfpathmoveto{\pgfqpoint{1.039540in}{2.702810in}}%
\pgfpathlineto{\pgfqpoint{2.165199in}{2.702810in}}%
\pgfusepath{stroke}%
\end{pgfscope}%
\begin{pgfscope}%
\pgfsetbuttcap%
\pgfsetroundjoin%
\definecolor{currentfill}{rgb}{1.000000,0.000000,0.000000}%
\pgfsetfillcolor{currentfill}%
\pgfsetlinewidth{0.803000pt}%
\definecolor{currentstroke}{rgb}{1.000000,0.000000,0.000000}%
\pgfsetstrokecolor{currentstroke}%
\pgfsetdash{}{0pt}%
\pgfsys@defobject{currentmarker}{\pgfqpoint{0.000000in}{0.000000in}}{\pgfqpoint{0.048611in}{0.000000in}}{%
\pgfpathmoveto{\pgfqpoint{0.000000in}{0.000000in}}%
\pgfpathlineto{\pgfqpoint{0.048611in}{0.000000in}}%
\pgfusepath{stroke,fill}%
}%
\begin{pgfscope}%
\pgfsys@transformshift{2.165199in}{0.914359in}%
\pgfsys@useobject{currentmarker}{}%
\end{pgfscope}%
\end{pgfscope}%
\begin{pgfscope}%
\definecolor{textcolor}{rgb}{1.000000,0.000000,0.000000}%
\pgfsetstrokecolor{textcolor}%
\pgfsetfillcolor{textcolor}%
\pgftext[x=2.262421in, y=0.861598in, left, base]{\color{textcolor}\sffamily\fontsize{10.000000}{12.000000}\selectfont 8.436}%
\end{pgfscope}%
\begin{pgfscope}%
\pgfsetbuttcap%
\pgfsetroundjoin%
\definecolor{currentfill}{rgb}{1.000000,0.000000,0.000000}%
\pgfsetfillcolor{currentfill}%
\pgfsetlinewidth{0.803000pt}%
\definecolor{currentstroke}{rgb}{1.000000,0.000000,0.000000}%
\pgfsetstrokecolor{currentstroke}%
\pgfsetdash{}{0pt}%
\pgfsys@defobject{currentmarker}{\pgfqpoint{0.000000in}{0.000000in}}{\pgfqpoint{0.048611in}{0.000000in}}{%
\pgfpathmoveto{\pgfqpoint{0.000000in}{0.000000in}}%
\pgfpathlineto{\pgfqpoint{0.048611in}{0.000000in}}%
\pgfusepath{stroke,fill}%
}%
\begin{pgfscope}%
\pgfsys@transformshift{2.165199in}{1.369696in}%
\pgfsys@useobject{currentmarker}{}%
\end{pgfscope}%
\end{pgfscope}%
\begin{pgfscope}%
\definecolor{textcolor}{rgb}{1.000000,0.000000,0.000000}%
\pgfsetstrokecolor{textcolor}%
\pgfsetfillcolor{textcolor}%
\pgftext[x=2.262421in, y=1.316935in, left, base]{\color{textcolor}\sffamily\fontsize{10.000000}{12.000000}\selectfont 8.438}%
\end{pgfscope}%
\begin{pgfscope}%
\pgfsetbuttcap%
\pgfsetroundjoin%
\definecolor{currentfill}{rgb}{1.000000,0.000000,0.000000}%
\pgfsetfillcolor{currentfill}%
\pgfsetlinewidth{0.803000pt}%
\definecolor{currentstroke}{rgb}{1.000000,0.000000,0.000000}%
\pgfsetstrokecolor{currentstroke}%
\pgfsetdash{}{0pt}%
\pgfsys@defobject{currentmarker}{\pgfqpoint{0.000000in}{0.000000in}}{\pgfqpoint{0.048611in}{0.000000in}}{%
\pgfpathmoveto{\pgfqpoint{0.000000in}{0.000000in}}%
\pgfpathlineto{\pgfqpoint{0.048611in}{0.000000in}}%
\pgfusepath{stroke,fill}%
}%
\begin{pgfscope}%
\pgfsys@transformshift{2.165199in}{1.825034in}%
\pgfsys@useobject{currentmarker}{}%
\end{pgfscope}%
\end{pgfscope}%
\begin{pgfscope}%
\definecolor{textcolor}{rgb}{1.000000,0.000000,0.000000}%
\pgfsetstrokecolor{textcolor}%
\pgfsetfillcolor{textcolor}%
\pgftext[x=2.262421in, y=1.772272in, left, base]{\color{textcolor}\sffamily\fontsize{10.000000}{12.000000}\selectfont 8.440}%
\end{pgfscope}%
\begin{pgfscope}%
\pgfsetbuttcap%
\pgfsetroundjoin%
\definecolor{currentfill}{rgb}{1.000000,0.000000,0.000000}%
\pgfsetfillcolor{currentfill}%
\pgfsetlinewidth{0.803000pt}%
\definecolor{currentstroke}{rgb}{1.000000,0.000000,0.000000}%
\pgfsetstrokecolor{currentstroke}%
\pgfsetdash{}{0pt}%
\pgfsys@defobject{currentmarker}{\pgfqpoint{0.000000in}{0.000000in}}{\pgfqpoint{0.048611in}{0.000000in}}{%
\pgfpathmoveto{\pgfqpoint{0.000000in}{0.000000in}}%
\pgfpathlineto{\pgfqpoint{0.048611in}{0.000000in}}%
\pgfusepath{stroke,fill}%
}%
\begin{pgfscope}%
\pgfsys@transformshift{2.165199in}{2.280371in}%
\pgfsys@useobject{currentmarker}{}%
\end{pgfscope}%
\end{pgfscope}%
\begin{pgfscope}%
\definecolor{textcolor}{rgb}{1.000000,0.000000,0.000000}%
\pgfsetstrokecolor{textcolor}%
\pgfsetfillcolor{textcolor}%
\pgftext[x=2.262421in, y=2.227610in, left, base]{\color{textcolor}\sffamily\fontsize{10.000000}{12.000000}\selectfont 8.442}%
\end{pgfscope}%
\begin{pgfscope}%
\definecolor{textcolor}{rgb}{1.000000,0.000000,0.000000}%
\pgfsetstrokecolor{textcolor}%
\pgfsetfillcolor{textcolor}%
\pgftext[x=2.715587in,y=1.637207in,,top,rotate=90.000000]{\color{textcolor}\sffamily\fontsize{10.000000}{12.000000}\selectfont MSD Metrik}%
\end{pgfscope}%
\begin{pgfscope}%
\definecolor{textcolor}{rgb}{1.000000,0.000000,0.000000}%
\pgfsetstrokecolor{textcolor}%
\pgfsetfillcolor{textcolor}%
\pgftext[x=2.165199in,y=2.744477in,right,base]{\color{textcolor}\sffamily\fontsize{10.000000}{12.000000}\selectfont 1e5}%
\end{pgfscope}%
\begin{pgfscope}%
\pgfpathrectangle{\pgfqpoint{1.039540in}{0.571604in}}{\pgfqpoint{1.125659in}{2.131206in}}%
\pgfusepath{clip}%
\pgfsetrectcap%
\pgfsetroundjoin%
\pgfsetlinewidth{1.505625pt}%
\definecolor{currentstroke}{rgb}{1.000000,0.000000,0.000000}%
\pgfsetstrokecolor{currentstroke}%
\pgfsetdash{}{0pt}%
\pgfpathmoveto{\pgfqpoint{1.039540in}{2.605937in}}%
\pgfpathlineto{\pgfqpoint{1.050797in}{2.592277in}}%
\pgfpathlineto{\pgfqpoint{1.062053in}{2.578617in}}%
\pgfpathlineto{\pgfqpoint{1.073310in}{2.567234in}}%
\pgfpathlineto{\pgfqpoint{1.084566in}{2.553573in}}%
\pgfpathlineto{\pgfqpoint{1.095823in}{2.539913in}}%
\pgfpathlineto{\pgfqpoint{1.107080in}{2.526253in}}%
\pgfpathlineto{\pgfqpoint{1.118336in}{2.512593in}}%
\pgfpathlineto{\pgfqpoint{1.129593in}{2.498933in}}%
\pgfpathlineto{\pgfqpoint{1.140849in}{2.485273in}}%
\pgfpathlineto{\pgfqpoint{1.152106in}{2.471613in}}%
\pgfpathlineto{\pgfqpoint{1.163362in}{2.457953in}}%
\pgfpathlineto{\pgfqpoint{1.174619in}{2.444293in}}%
\pgfpathlineto{\pgfqpoint{1.185876in}{2.430632in}}%
\pgfpathlineto{\pgfqpoint{1.197132in}{2.414696in}}%
\pgfpathlineto{\pgfqpoint{1.208389in}{2.401035in}}%
\pgfpathlineto{\pgfqpoint{1.219645in}{2.385099in}}%
\pgfpathlineto{\pgfqpoint{1.230902in}{2.371439in}}%
\pgfpathlineto{\pgfqpoint{1.242159in}{2.355502in}}%
\pgfpathlineto{\pgfqpoint{1.253415in}{2.339565in}}%
\pgfpathlineto{\pgfqpoint{1.264672in}{2.325905in}}%
\pgfpathlineto{\pgfqpoint{1.275928in}{2.309968in}}%
\pgfpathlineto{\pgfqpoint{1.287185in}{2.294031in}}%
\pgfpathlineto{\pgfqpoint{1.298442in}{2.278094in}}%
\pgfpathlineto{\pgfqpoint{1.309698in}{2.262158in}}%
\pgfpathlineto{\pgfqpoint{1.320955in}{2.246221in}}%
\pgfpathlineto{\pgfqpoint{1.332211in}{2.230284in}}%
\pgfpathlineto{\pgfqpoint{1.343468in}{2.212070in}}%
\pgfpathlineto{\pgfqpoint{1.354725in}{2.196134in}}%
\pgfpathlineto{\pgfqpoint{1.365981in}{2.180197in}}%
\pgfpathlineto{\pgfqpoint{1.377238in}{2.161983in}}%
\pgfpathlineto{\pgfqpoint{1.388494in}{2.146047in}}%
\pgfpathlineto{\pgfqpoint{1.399751in}{2.130110in}}%
\pgfpathlineto{\pgfqpoint{1.411007in}{2.111896in}}%
\pgfpathlineto{\pgfqpoint{1.422264in}{2.093683in}}%
\pgfpathlineto{\pgfqpoint{1.433521in}{2.077746in}}%
\pgfpathlineto{\pgfqpoint{1.444777in}{2.059532in}}%
\pgfpathlineto{\pgfqpoint{1.456034in}{2.041319in}}%
\pgfpathlineto{\pgfqpoint{1.467290in}{2.023105in}}%
\pgfpathlineto{\pgfqpoint{1.478547in}{2.007169in}}%
\pgfpathlineto{\pgfqpoint{1.489804in}{1.986678in}}%
\pgfpathlineto{\pgfqpoint{1.501060in}{1.970742in}}%
\pgfpathlineto{\pgfqpoint{1.512317in}{1.950252in}}%
\pgfpathlineto{\pgfqpoint{1.523573in}{1.932038in}}%
\pgfpathlineto{\pgfqpoint{1.534830in}{1.913825in}}%
\pgfpathlineto{\pgfqpoint{1.546087in}{1.895611in}}%
\pgfpathlineto{\pgfqpoint{1.557343in}{1.877398in}}%
\pgfpathlineto{\pgfqpoint{1.568600in}{1.856907in}}%
\pgfpathlineto{\pgfqpoint{1.579856in}{1.838694in}}%
\pgfpathlineto{\pgfqpoint{1.591113in}{1.818204in}}%
\pgfpathlineto{\pgfqpoint{1.602370in}{1.799990in}}%
\pgfpathlineto{\pgfqpoint{1.613626in}{1.779500in}}%
\pgfpathlineto{\pgfqpoint{1.624883in}{1.761287in}}%
\pgfpathlineto{\pgfqpoint{1.636139in}{1.740796in}}%
\pgfpathlineto{\pgfqpoint{1.647396in}{1.720306in}}%
\pgfpathlineto{\pgfqpoint{1.658652in}{1.699816in}}%
\pgfpathlineto{\pgfqpoint{1.669909in}{1.679326in}}%
\pgfpathlineto{\pgfqpoint{1.681166in}{1.661112in}}%
\pgfpathlineto{\pgfqpoint{1.692422in}{1.638345in}}%
\pgfpathlineto{\pgfqpoint{1.703679in}{1.620132in}}%
\pgfpathlineto{\pgfqpoint{1.714935in}{1.597365in}}%
\pgfpathlineto{\pgfqpoint{1.726192in}{1.576875in}}%
\pgfpathlineto{\pgfqpoint{1.737449in}{1.556385in}}%
\pgfpathlineto{\pgfqpoint{1.748705in}{1.535895in}}%
\pgfpathlineto{\pgfqpoint{1.759962in}{1.513128in}}%
\pgfpathlineto{\pgfqpoint{1.771218in}{1.492637in}}%
\pgfpathlineto{\pgfqpoint{1.782475in}{1.469871in}}%
\pgfpathlineto{\pgfqpoint{1.793732in}{1.449380in}}%
\pgfpathlineto{\pgfqpoint{1.804988in}{1.426614in}}%
\pgfpathlineto{\pgfqpoint{1.816245in}{1.406123in}}%
\pgfpathlineto{\pgfqpoint{1.827501in}{1.383357in}}%
\pgfpathlineto{\pgfqpoint{1.838758in}{1.362866in}}%
\pgfpathlineto{\pgfqpoint{1.850014in}{1.340099in}}%
\pgfpathlineto{\pgfqpoint{1.861271in}{1.317333in}}%
\pgfpathlineto{\pgfqpoint{1.872528in}{1.294566in}}%
\pgfpathlineto{\pgfqpoint{1.883784in}{1.271799in}}%
\pgfpathlineto{\pgfqpoint{1.895041in}{1.249032in}}%
\pgfpathlineto{\pgfqpoint{1.906297in}{1.226265in}}%
\pgfpathlineto{\pgfqpoint{1.917554in}{1.203498in}}%
\pgfpathlineto{\pgfqpoint{1.928811in}{1.180731in}}%
\pgfpathlineto{\pgfqpoint{1.940067in}{1.157965in}}%
\pgfpathlineto{\pgfqpoint{1.951324in}{1.135198in}}%
\pgfpathlineto{\pgfqpoint{1.962580in}{1.110154in}}%
\pgfpathlineto{\pgfqpoint{1.973837in}{1.087387in}}%
\pgfpathlineto{\pgfqpoint{1.985094in}{1.064620in}}%
\pgfpathlineto{\pgfqpoint{1.996350in}{1.039577in}}%
\pgfpathlineto{\pgfqpoint{2.007607in}{1.016810in}}%
\pgfpathlineto{\pgfqpoint{2.018863in}{0.991766in}}%
\pgfpathlineto{\pgfqpoint{2.030120in}{0.969000in}}%
\pgfpathlineto{\pgfqpoint{2.041377in}{0.943956in}}%
\pgfpathlineto{\pgfqpoint{2.052633in}{0.918912in}}%
\pgfpathlineto{\pgfqpoint{2.063890in}{0.893869in}}%
\pgfpathlineto{\pgfqpoint{2.075146in}{0.871102in}}%
\pgfpathlineto{\pgfqpoint{2.086403in}{0.846058in}}%
\pgfpathlineto{\pgfqpoint{2.097659in}{0.821015in}}%
\pgfpathlineto{\pgfqpoint{2.108916in}{0.795971in}}%
\pgfpathlineto{\pgfqpoint{2.120173in}{0.770928in}}%
\pgfpathlineto{\pgfqpoint{2.131429in}{0.743608in}}%
\pgfpathlineto{\pgfqpoint{2.142686in}{0.718564in}}%
\pgfpathlineto{\pgfqpoint{2.153942in}{0.693520in}}%
\pgfpathlineto{\pgfqpoint{2.165199in}{0.668477in}}%
\pgfusepath{stroke}%
\end{pgfscope}%
\begin{pgfscope}%
\pgfsetrectcap%
\pgfsetmiterjoin%
\pgfsetlinewidth{0.803000pt}%
\definecolor{currentstroke}{rgb}{0.000000,0.000000,0.000000}%
\pgfsetstrokecolor{currentstroke}%
\pgfsetdash{}{0pt}%
\pgfpathmoveto{\pgfqpoint{1.039540in}{0.571604in}}%
\pgfpathlineto{\pgfqpoint{1.039540in}{2.702810in}}%
\pgfusepath{stroke}%
\end{pgfscope}%
\begin{pgfscope}%
\pgfsetrectcap%
\pgfsetmiterjoin%
\pgfsetlinewidth{0.803000pt}%
\definecolor{currentstroke}{rgb}{0.000000,0.000000,0.000000}%
\pgfsetstrokecolor{currentstroke}%
\pgfsetdash{}{0pt}%
\pgfpathmoveto{\pgfqpoint{2.165199in}{0.571604in}}%
\pgfpathlineto{\pgfqpoint{2.165199in}{2.702810in}}%
\pgfusepath{stroke}%
\end{pgfscope}%
\begin{pgfscope}%
\pgfsetrectcap%
\pgfsetmiterjoin%
\pgfsetlinewidth{0.803000pt}%
\definecolor{currentstroke}{rgb}{0.000000,0.000000,0.000000}%
\pgfsetstrokecolor{currentstroke}%
\pgfsetdash{}{0pt}%
\pgfpathmoveto{\pgfqpoint{1.039540in}{0.571604in}}%
\pgfpathlineto{\pgfqpoint{2.165199in}{0.571604in}}%
\pgfusepath{stroke}%
\end{pgfscope}%
\begin{pgfscope}%
\pgfsetrectcap%
\pgfsetmiterjoin%
\pgfsetlinewidth{0.803000pt}%
\definecolor{currentstroke}{rgb}{0.000000,0.000000,0.000000}%
\pgfsetstrokecolor{currentstroke}%
\pgfsetdash{}{0pt}%
\pgfpathmoveto{\pgfqpoint{1.039540in}{2.702810in}}%
\pgfpathlineto{\pgfqpoint{2.165199in}{2.702810in}}%
\pgfusepath{stroke}%
\end{pgfscope}%
\end{pgfpicture}%
\makeatother%
\endgroup%
}
  \vspace{-10pt}
\end{figure}
In Abbildung \ref{fig:ct2mrt_param} ist das Verhalten der beiden Metriken für
einen Rotationsparameter und einen Translationsparameter dargestellt. Da CT und
MRT Scans unterschiedliche Kontrastverhältnisse haben, erwarten wir keine
sinnvollen Ergebnisse von der MSD Metrik (rot). Dies ist bei Parameter \num{0}
gut erkennbar, da der Metrikwert bei stärkerer Abweichung vom optimalen
Parameter abnimmt. Der Wert der MI Metrik (blau) steigt dagegen bei größerem
Shift, was bedeutet, dass die MI Metrik trotz unterschiedlicher Messmethoden
korrekt funktioniert.
Bei Parameter \num{3} handelt es sich um einen Translationsparameter, was
bedeutet, dass wir lediglich um einen Voxel insgesamt verschieben, was sehr
wenig ist. Die MI Metrik hat ihr Minimum bei \num{0.5} statt \num{0.0},
allerdings liegt der Unterschied des Metrikwertes bei etwa \num{0.003}, was
verglichen mit den Werten beim Translationsparameter sehr gering ist. Die MSD
Metrik kann bei so einer kleinen Skala nicht sinnvoll analysiert werden.
\vspace{12pt}
\begin{wrapfigure}{r}{0.5\linewidth}
  \vspace{-6pt}
  \vspace{-10pt}
  \caption{Variation der Binanzahl beim MI Algorithmus führt bei hoher Anzahl nur noch zu geringen Veränderungen der Metrik}
  \label{fig:ct2mrt_bins}
  \vspace{-10pt}
  \resizebox{\linewidth}{!}{%% Creator: Matplotlib, PGF backend
%%
%% To include the figure in your LaTeX document, write
%%   \input{<filename>.pgf}
%%
%% Make sure the required packages are loaded in your preamble
%%   \usepackage{pgf}
%%
%% Figures using additional raster images can only be included by \input if
%% they are in the same directory as the main LaTeX file. For loading figures
%% from other directories you can use the `import` package
%%   \usepackage{import}
%% and then include the figures with
%%   \import{<path to file>}{<filename>.pgf}
%%
%% Matplotlib used the following preamble
%%   \usepackage{fontspec}
%%   \setmainfont{DejaVuSerif.ttf}[Path=/usr/share/matplotlib/mpl-data/fonts/ttf/]
%%   \setsansfont{DejaVuSans.ttf}[Path=/usr/share/matplotlib/mpl-data/fonts/ttf/]
%%   \setmonofont{DejaVuSansMono.ttf}[Path=/usr/share/matplotlib/mpl-data/fonts/ttf/]
%%
\begingroup%
\makeatletter%
\begin{pgfpicture}%
\pgfpathrectangle{\pgfpointorigin}{\pgfqpoint{3.000000in}{3.000000in}}%
\pgfusepath{use as bounding box, clip}%
\begin{pgfscope}%
\pgfsetbuttcap%
\pgfsetmiterjoin%
\definecolor{currentfill}{rgb}{1.000000,1.000000,1.000000}%
\pgfsetfillcolor{currentfill}%
\pgfsetlinewidth{0.000000pt}%
\definecolor{currentstroke}{rgb}{1.000000,1.000000,1.000000}%
\pgfsetstrokecolor{currentstroke}%
\pgfsetdash{}{0pt}%
\pgfpathmoveto{\pgfqpoint{0.000000in}{0.000000in}}%
\pgfpathlineto{\pgfqpoint{3.000000in}{0.000000in}}%
\pgfpathlineto{\pgfqpoint{3.000000in}{3.000000in}}%
\pgfpathlineto{\pgfqpoint{0.000000in}{3.000000in}}%
\pgfpathclose%
\pgfusepath{fill}%
\end{pgfscope}%
\begin{pgfscope}%
\pgfsetbuttcap%
\pgfsetmiterjoin%
\definecolor{currentfill}{rgb}{1.000000,1.000000,1.000000}%
\pgfsetfillcolor{currentfill}%
\pgfsetlinewidth{0.000000pt}%
\definecolor{currentstroke}{rgb}{0.000000,0.000000,0.000000}%
\pgfsetstrokecolor{currentstroke}%
\pgfsetstrokeopacity{0.000000}%
\pgfsetdash{}{0pt}%
\pgfpathmoveto{\pgfqpoint{0.774444in}{0.571604in}}%
\pgfpathlineto{\pgfqpoint{2.739560in}{0.571604in}}%
\pgfpathlineto{\pgfqpoint{2.739560in}{2.797238in}}%
\pgfpathlineto{\pgfqpoint{0.774444in}{2.797238in}}%
\pgfpathclose%
\pgfusepath{fill}%
\end{pgfscope}%
\begin{pgfscope}%
\pgfpathrectangle{\pgfqpoint{0.774444in}{0.571604in}}{\pgfqpoint{1.965116in}{2.225635in}}%
\pgfusepath{clip}%
\pgfsetrectcap%
\pgfsetroundjoin%
\pgfsetlinewidth{0.803000pt}%
\definecolor{currentstroke}{rgb}{0.690196,0.690196,0.690196}%
\pgfsetstrokecolor{currentstroke}%
\pgfsetdash{}{0pt}%
\pgfpathmoveto{\pgfqpoint{0.774444in}{0.571604in}}%
\pgfpathlineto{\pgfqpoint{0.774444in}{2.797238in}}%
\pgfusepath{stroke}%
\end{pgfscope}%
\begin{pgfscope}%
\pgfsetbuttcap%
\pgfsetroundjoin%
\definecolor{currentfill}{rgb}{0.000000,0.000000,0.000000}%
\pgfsetfillcolor{currentfill}%
\pgfsetlinewidth{0.803000pt}%
\definecolor{currentstroke}{rgb}{0.000000,0.000000,0.000000}%
\pgfsetstrokecolor{currentstroke}%
\pgfsetdash{}{0pt}%
\pgfsys@defobject{currentmarker}{\pgfqpoint{0.000000in}{-0.048611in}}{\pgfqpoint{0.000000in}{0.000000in}}{%
\pgfpathmoveto{\pgfqpoint{0.000000in}{0.000000in}}%
\pgfpathlineto{\pgfqpoint{0.000000in}{-0.048611in}}%
\pgfusepath{stroke,fill}%
}%
\begin{pgfscope}%
\pgfsys@transformshift{0.774444in}{0.571604in}%
\pgfsys@useobject{currentmarker}{}%
\end{pgfscope}%
\end{pgfscope}%
\begin{pgfscope}%
\definecolor{textcolor}{rgb}{0.000000,0.000000,0.000000}%
\pgfsetstrokecolor{textcolor}%
\pgfsetfillcolor{textcolor}%
\pgftext[x=0.774444in,y=0.474382in,,top]{\color{textcolor}\sffamily\fontsize{10.000000}{12.000000}\selectfont 0.00}%
\end{pgfscope}%
\begin{pgfscope}%
\pgfpathrectangle{\pgfqpoint{0.774444in}{0.571604in}}{\pgfqpoint{1.965116in}{2.225635in}}%
\pgfusepath{clip}%
\pgfsetrectcap%
\pgfsetroundjoin%
\pgfsetlinewidth{0.803000pt}%
\definecolor{currentstroke}{rgb}{0.690196,0.690196,0.690196}%
\pgfsetstrokecolor{currentstroke}%
\pgfsetdash{}{0pt}%
\pgfpathmoveto{\pgfqpoint{1.265723in}{0.571604in}}%
\pgfpathlineto{\pgfqpoint{1.265723in}{2.797238in}}%
\pgfusepath{stroke}%
\end{pgfscope}%
\begin{pgfscope}%
\pgfsetbuttcap%
\pgfsetroundjoin%
\definecolor{currentfill}{rgb}{0.000000,0.000000,0.000000}%
\pgfsetfillcolor{currentfill}%
\pgfsetlinewidth{0.803000pt}%
\definecolor{currentstroke}{rgb}{0.000000,0.000000,0.000000}%
\pgfsetstrokecolor{currentstroke}%
\pgfsetdash{}{0pt}%
\pgfsys@defobject{currentmarker}{\pgfqpoint{0.000000in}{-0.048611in}}{\pgfqpoint{0.000000in}{0.000000in}}{%
\pgfpathmoveto{\pgfqpoint{0.000000in}{0.000000in}}%
\pgfpathlineto{\pgfqpoint{0.000000in}{-0.048611in}}%
\pgfusepath{stroke,fill}%
}%
\begin{pgfscope}%
\pgfsys@transformshift{1.265723in}{0.571604in}%
\pgfsys@useobject{currentmarker}{}%
\end{pgfscope}%
\end{pgfscope}%
\begin{pgfscope}%
\definecolor{textcolor}{rgb}{0.000000,0.000000,0.000000}%
\pgfsetstrokecolor{textcolor}%
\pgfsetfillcolor{textcolor}%
\pgftext[x=1.265723in,y=0.474382in,,top]{\color{textcolor}\sffamily\fontsize{10.000000}{12.000000}\selectfont 0.25}%
\end{pgfscope}%
\begin{pgfscope}%
\pgfpathrectangle{\pgfqpoint{0.774444in}{0.571604in}}{\pgfqpoint{1.965116in}{2.225635in}}%
\pgfusepath{clip}%
\pgfsetrectcap%
\pgfsetroundjoin%
\pgfsetlinewidth{0.803000pt}%
\definecolor{currentstroke}{rgb}{0.690196,0.690196,0.690196}%
\pgfsetstrokecolor{currentstroke}%
\pgfsetdash{}{0pt}%
\pgfpathmoveto{\pgfqpoint{1.757002in}{0.571604in}}%
\pgfpathlineto{\pgfqpoint{1.757002in}{2.797238in}}%
\pgfusepath{stroke}%
\end{pgfscope}%
\begin{pgfscope}%
\pgfsetbuttcap%
\pgfsetroundjoin%
\definecolor{currentfill}{rgb}{0.000000,0.000000,0.000000}%
\pgfsetfillcolor{currentfill}%
\pgfsetlinewidth{0.803000pt}%
\definecolor{currentstroke}{rgb}{0.000000,0.000000,0.000000}%
\pgfsetstrokecolor{currentstroke}%
\pgfsetdash{}{0pt}%
\pgfsys@defobject{currentmarker}{\pgfqpoint{0.000000in}{-0.048611in}}{\pgfqpoint{0.000000in}{0.000000in}}{%
\pgfpathmoveto{\pgfqpoint{0.000000in}{0.000000in}}%
\pgfpathlineto{\pgfqpoint{0.000000in}{-0.048611in}}%
\pgfusepath{stroke,fill}%
}%
\begin{pgfscope}%
\pgfsys@transformshift{1.757002in}{0.571604in}%
\pgfsys@useobject{currentmarker}{}%
\end{pgfscope}%
\end{pgfscope}%
\begin{pgfscope}%
\definecolor{textcolor}{rgb}{0.000000,0.000000,0.000000}%
\pgfsetstrokecolor{textcolor}%
\pgfsetfillcolor{textcolor}%
\pgftext[x=1.757002in,y=0.474382in,,top]{\color{textcolor}\sffamily\fontsize{10.000000}{12.000000}\selectfont 0.50}%
\end{pgfscope}%
\begin{pgfscope}%
\pgfpathrectangle{\pgfqpoint{0.774444in}{0.571604in}}{\pgfqpoint{1.965116in}{2.225635in}}%
\pgfusepath{clip}%
\pgfsetrectcap%
\pgfsetroundjoin%
\pgfsetlinewidth{0.803000pt}%
\definecolor{currentstroke}{rgb}{0.690196,0.690196,0.690196}%
\pgfsetstrokecolor{currentstroke}%
\pgfsetdash{}{0pt}%
\pgfpathmoveto{\pgfqpoint{2.248281in}{0.571604in}}%
\pgfpathlineto{\pgfqpoint{2.248281in}{2.797238in}}%
\pgfusepath{stroke}%
\end{pgfscope}%
\begin{pgfscope}%
\pgfsetbuttcap%
\pgfsetroundjoin%
\definecolor{currentfill}{rgb}{0.000000,0.000000,0.000000}%
\pgfsetfillcolor{currentfill}%
\pgfsetlinewidth{0.803000pt}%
\definecolor{currentstroke}{rgb}{0.000000,0.000000,0.000000}%
\pgfsetstrokecolor{currentstroke}%
\pgfsetdash{}{0pt}%
\pgfsys@defobject{currentmarker}{\pgfqpoint{0.000000in}{-0.048611in}}{\pgfqpoint{0.000000in}{0.000000in}}{%
\pgfpathmoveto{\pgfqpoint{0.000000in}{0.000000in}}%
\pgfpathlineto{\pgfqpoint{0.000000in}{-0.048611in}}%
\pgfusepath{stroke,fill}%
}%
\begin{pgfscope}%
\pgfsys@transformshift{2.248281in}{0.571604in}%
\pgfsys@useobject{currentmarker}{}%
\end{pgfscope}%
\end{pgfscope}%
\begin{pgfscope}%
\definecolor{textcolor}{rgb}{0.000000,0.000000,0.000000}%
\pgfsetstrokecolor{textcolor}%
\pgfsetfillcolor{textcolor}%
\pgftext[x=2.248281in,y=0.474382in,,top]{\color{textcolor}\sffamily\fontsize{10.000000}{12.000000}\selectfont 0.75}%
\end{pgfscope}%
\begin{pgfscope}%
\pgfpathrectangle{\pgfqpoint{0.774444in}{0.571604in}}{\pgfqpoint{1.965116in}{2.225635in}}%
\pgfusepath{clip}%
\pgfsetrectcap%
\pgfsetroundjoin%
\pgfsetlinewidth{0.803000pt}%
\definecolor{currentstroke}{rgb}{0.690196,0.690196,0.690196}%
\pgfsetstrokecolor{currentstroke}%
\pgfsetdash{}{0pt}%
\pgfpathmoveto{\pgfqpoint{2.739560in}{0.571604in}}%
\pgfpathlineto{\pgfqpoint{2.739560in}{2.797238in}}%
\pgfusepath{stroke}%
\end{pgfscope}%
\begin{pgfscope}%
\pgfsetbuttcap%
\pgfsetroundjoin%
\definecolor{currentfill}{rgb}{0.000000,0.000000,0.000000}%
\pgfsetfillcolor{currentfill}%
\pgfsetlinewidth{0.803000pt}%
\definecolor{currentstroke}{rgb}{0.000000,0.000000,0.000000}%
\pgfsetstrokecolor{currentstroke}%
\pgfsetdash{}{0pt}%
\pgfsys@defobject{currentmarker}{\pgfqpoint{0.000000in}{-0.048611in}}{\pgfqpoint{0.000000in}{0.000000in}}{%
\pgfpathmoveto{\pgfqpoint{0.000000in}{0.000000in}}%
\pgfpathlineto{\pgfqpoint{0.000000in}{-0.048611in}}%
\pgfusepath{stroke,fill}%
}%
\begin{pgfscope}%
\pgfsys@transformshift{2.739560in}{0.571604in}%
\pgfsys@useobject{currentmarker}{}%
\end{pgfscope}%
\end{pgfscope}%
\begin{pgfscope}%
\definecolor{textcolor}{rgb}{0.000000,0.000000,0.000000}%
\pgfsetstrokecolor{textcolor}%
\pgfsetfillcolor{textcolor}%
\pgftext[x=2.739560in,y=0.474382in,,top]{\color{textcolor}\sffamily\fontsize{10.000000}{12.000000}\selectfont 1.00}%
\end{pgfscope}%
\begin{pgfscope}%
\definecolor{textcolor}{rgb}{0.000000,0.000000,0.000000}%
\pgfsetstrokecolor{textcolor}%
\pgfsetfillcolor{textcolor}%
\pgftext[x=1.757002in,y=0.284413in,,top]{\color{textcolor}\sffamily\fontsize{10.000000}{12.000000}\selectfont Shift of parameter 0}%
\end{pgfscope}%
\begin{pgfscope}%
\pgfpathrectangle{\pgfqpoint{0.774444in}{0.571604in}}{\pgfqpoint{1.965116in}{2.225635in}}%
\pgfusepath{clip}%
\pgfsetrectcap%
\pgfsetroundjoin%
\pgfsetlinewidth{0.803000pt}%
\definecolor{currentstroke}{rgb}{0.690196,0.690196,0.690196}%
\pgfsetstrokecolor{currentstroke}%
\pgfsetdash{}{0pt}%
\pgfpathmoveto{\pgfqpoint{0.774444in}{0.571604in}}%
\pgfpathlineto{\pgfqpoint{2.739560in}{0.571604in}}%
\pgfusepath{stroke}%
\end{pgfscope}%
\begin{pgfscope}%
\pgfsetbuttcap%
\pgfsetroundjoin%
\definecolor{currentfill}{rgb}{0.000000,0.000000,0.000000}%
\pgfsetfillcolor{currentfill}%
\pgfsetlinewidth{0.803000pt}%
\definecolor{currentstroke}{rgb}{0.000000,0.000000,0.000000}%
\pgfsetstrokecolor{currentstroke}%
\pgfsetdash{}{0pt}%
\pgfsys@defobject{currentmarker}{\pgfqpoint{-0.048611in}{0.000000in}}{\pgfqpoint{0.000000in}{0.000000in}}{%
\pgfpathmoveto{\pgfqpoint{0.000000in}{0.000000in}}%
\pgfpathlineto{\pgfqpoint{-0.048611in}{0.000000in}}%
\pgfusepath{stroke,fill}%
}%
\begin{pgfscope}%
\pgfsys@transformshift{0.774444in}{0.571604in}%
\pgfsys@useobject{currentmarker}{}%
\end{pgfscope}%
\end{pgfscope}%
\begin{pgfscope}%
\definecolor{textcolor}{rgb}{0.000000,0.000000,0.000000}%
\pgfsetstrokecolor{textcolor}%
\pgfsetfillcolor{textcolor}%
\pgftext[x=0.339969in,y=0.518842in,left,base]{\color{textcolor}\sffamily\fontsize{10.000000}{12.000000}\selectfont −0.7}%
\end{pgfscope}%
\begin{pgfscope}%
\pgfpathrectangle{\pgfqpoint{0.774444in}{0.571604in}}{\pgfqpoint{1.965116in}{2.225635in}}%
\pgfusepath{clip}%
\pgfsetrectcap%
\pgfsetroundjoin%
\pgfsetlinewidth{0.803000pt}%
\definecolor{currentstroke}{rgb}{0.690196,0.690196,0.690196}%
\pgfsetstrokecolor{currentstroke}%
\pgfsetdash{}{0pt}%
\pgfpathmoveto{\pgfqpoint{0.774444in}{0.889552in}}%
\pgfpathlineto{\pgfqpoint{2.739560in}{0.889552in}}%
\pgfusepath{stroke}%
\end{pgfscope}%
\begin{pgfscope}%
\pgfsetbuttcap%
\pgfsetroundjoin%
\definecolor{currentfill}{rgb}{0.000000,0.000000,0.000000}%
\pgfsetfillcolor{currentfill}%
\pgfsetlinewidth{0.803000pt}%
\definecolor{currentstroke}{rgb}{0.000000,0.000000,0.000000}%
\pgfsetstrokecolor{currentstroke}%
\pgfsetdash{}{0pt}%
\pgfsys@defobject{currentmarker}{\pgfqpoint{-0.048611in}{0.000000in}}{\pgfqpoint{0.000000in}{0.000000in}}{%
\pgfpathmoveto{\pgfqpoint{0.000000in}{0.000000in}}%
\pgfpathlineto{\pgfqpoint{-0.048611in}{0.000000in}}%
\pgfusepath{stroke,fill}%
}%
\begin{pgfscope}%
\pgfsys@transformshift{0.774444in}{0.889552in}%
\pgfsys@useobject{currentmarker}{}%
\end{pgfscope}%
\end{pgfscope}%
\begin{pgfscope}%
\definecolor{textcolor}{rgb}{0.000000,0.000000,0.000000}%
\pgfsetstrokecolor{textcolor}%
\pgfsetfillcolor{textcolor}%
\pgftext[x=0.339969in,y=0.836790in,left,base]{\color{textcolor}\sffamily\fontsize{10.000000}{12.000000}\selectfont −0.6}%
\end{pgfscope}%
\begin{pgfscope}%
\pgfpathrectangle{\pgfqpoint{0.774444in}{0.571604in}}{\pgfqpoint{1.965116in}{2.225635in}}%
\pgfusepath{clip}%
\pgfsetrectcap%
\pgfsetroundjoin%
\pgfsetlinewidth{0.803000pt}%
\definecolor{currentstroke}{rgb}{0.690196,0.690196,0.690196}%
\pgfsetstrokecolor{currentstroke}%
\pgfsetdash{}{0pt}%
\pgfpathmoveto{\pgfqpoint{0.774444in}{1.207499in}}%
\pgfpathlineto{\pgfqpoint{2.739560in}{1.207499in}}%
\pgfusepath{stroke}%
\end{pgfscope}%
\begin{pgfscope}%
\pgfsetbuttcap%
\pgfsetroundjoin%
\definecolor{currentfill}{rgb}{0.000000,0.000000,0.000000}%
\pgfsetfillcolor{currentfill}%
\pgfsetlinewidth{0.803000pt}%
\definecolor{currentstroke}{rgb}{0.000000,0.000000,0.000000}%
\pgfsetstrokecolor{currentstroke}%
\pgfsetdash{}{0pt}%
\pgfsys@defobject{currentmarker}{\pgfqpoint{-0.048611in}{0.000000in}}{\pgfqpoint{0.000000in}{0.000000in}}{%
\pgfpathmoveto{\pgfqpoint{0.000000in}{0.000000in}}%
\pgfpathlineto{\pgfqpoint{-0.048611in}{0.000000in}}%
\pgfusepath{stroke,fill}%
}%
\begin{pgfscope}%
\pgfsys@transformshift{0.774444in}{1.207499in}%
\pgfsys@useobject{currentmarker}{}%
\end{pgfscope}%
\end{pgfscope}%
\begin{pgfscope}%
\definecolor{textcolor}{rgb}{0.000000,0.000000,0.000000}%
\pgfsetstrokecolor{textcolor}%
\pgfsetfillcolor{textcolor}%
\pgftext[x=0.339969in,y=1.154738in,left,base]{\color{textcolor}\sffamily\fontsize{10.000000}{12.000000}\selectfont −0.5}%
\end{pgfscope}%
\begin{pgfscope}%
\pgfpathrectangle{\pgfqpoint{0.774444in}{0.571604in}}{\pgfqpoint{1.965116in}{2.225635in}}%
\pgfusepath{clip}%
\pgfsetrectcap%
\pgfsetroundjoin%
\pgfsetlinewidth{0.803000pt}%
\definecolor{currentstroke}{rgb}{0.690196,0.690196,0.690196}%
\pgfsetstrokecolor{currentstroke}%
\pgfsetdash{}{0pt}%
\pgfpathmoveto{\pgfqpoint{0.774444in}{1.525447in}}%
\pgfpathlineto{\pgfqpoint{2.739560in}{1.525447in}}%
\pgfusepath{stroke}%
\end{pgfscope}%
\begin{pgfscope}%
\pgfsetbuttcap%
\pgfsetroundjoin%
\definecolor{currentfill}{rgb}{0.000000,0.000000,0.000000}%
\pgfsetfillcolor{currentfill}%
\pgfsetlinewidth{0.803000pt}%
\definecolor{currentstroke}{rgb}{0.000000,0.000000,0.000000}%
\pgfsetstrokecolor{currentstroke}%
\pgfsetdash{}{0pt}%
\pgfsys@defobject{currentmarker}{\pgfqpoint{-0.048611in}{0.000000in}}{\pgfqpoint{0.000000in}{0.000000in}}{%
\pgfpathmoveto{\pgfqpoint{0.000000in}{0.000000in}}%
\pgfpathlineto{\pgfqpoint{-0.048611in}{0.000000in}}%
\pgfusepath{stroke,fill}%
}%
\begin{pgfscope}%
\pgfsys@transformshift{0.774444in}{1.525447in}%
\pgfsys@useobject{currentmarker}{}%
\end{pgfscope}%
\end{pgfscope}%
\begin{pgfscope}%
\definecolor{textcolor}{rgb}{0.000000,0.000000,0.000000}%
\pgfsetstrokecolor{textcolor}%
\pgfsetfillcolor{textcolor}%
\pgftext[x=0.339969in,y=1.472686in,left,base]{\color{textcolor}\sffamily\fontsize{10.000000}{12.000000}\selectfont −0.4}%
\end{pgfscope}%
\begin{pgfscope}%
\pgfpathrectangle{\pgfqpoint{0.774444in}{0.571604in}}{\pgfqpoint{1.965116in}{2.225635in}}%
\pgfusepath{clip}%
\pgfsetrectcap%
\pgfsetroundjoin%
\pgfsetlinewidth{0.803000pt}%
\definecolor{currentstroke}{rgb}{0.690196,0.690196,0.690196}%
\pgfsetstrokecolor{currentstroke}%
\pgfsetdash{}{0pt}%
\pgfpathmoveto{\pgfqpoint{0.774444in}{1.843395in}}%
\pgfpathlineto{\pgfqpoint{2.739560in}{1.843395in}}%
\pgfusepath{stroke}%
\end{pgfscope}%
\begin{pgfscope}%
\pgfsetbuttcap%
\pgfsetroundjoin%
\definecolor{currentfill}{rgb}{0.000000,0.000000,0.000000}%
\pgfsetfillcolor{currentfill}%
\pgfsetlinewidth{0.803000pt}%
\definecolor{currentstroke}{rgb}{0.000000,0.000000,0.000000}%
\pgfsetstrokecolor{currentstroke}%
\pgfsetdash{}{0pt}%
\pgfsys@defobject{currentmarker}{\pgfqpoint{-0.048611in}{0.000000in}}{\pgfqpoint{0.000000in}{0.000000in}}{%
\pgfpathmoveto{\pgfqpoint{0.000000in}{0.000000in}}%
\pgfpathlineto{\pgfqpoint{-0.048611in}{0.000000in}}%
\pgfusepath{stroke,fill}%
}%
\begin{pgfscope}%
\pgfsys@transformshift{0.774444in}{1.843395in}%
\pgfsys@useobject{currentmarker}{}%
\end{pgfscope}%
\end{pgfscope}%
\begin{pgfscope}%
\definecolor{textcolor}{rgb}{0.000000,0.000000,0.000000}%
\pgfsetstrokecolor{textcolor}%
\pgfsetfillcolor{textcolor}%
\pgftext[x=0.339969in,y=1.790634in,left,base]{\color{textcolor}\sffamily\fontsize{10.000000}{12.000000}\selectfont −0.3}%
\end{pgfscope}%
\begin{pgfscope}%
\pgfpathrectangle{\pgfqpoint{0.774444in}{0.571604in}}{\pgfqpoint{1.965116in}{2.225635in}}%
\pgfusepath{clip}%
\pgfsetrectcap%
\pgfsetroundjoin%
\pgfsetlinewidth{0.803000pt}%
\definecolor{currentstroke}{rgb}{0.690196,0.690196,0.690196}%
\pgfsetstrokecolor{currentstroke}%
\pgfsetdash{}{0pt}%
\pgfpathmoveto{\pgfqpoint{0.774444in}{2.161343in}}%
\pgfpathlineto{\pgfqpoint{2.739560in}{2.161343in}}%
\pgfusepath{stroke}%
\end{pgfscope}%
\begin{pgfscope}%
\pgfsetbuttcap%
\pgfsetroundjoin%
\definecolor{currentfill}{rgb}{0.000000,0.000000,0.000000}%
\pgfsetfillcolor{currentfill}%
\pgfsetlinewidth{0.803000pt}%
\definecolor{currentstroke}{rgb}{0.000000,0.000000,0.000000}%
\pgfsetstrokecolor{currentstroke}%
\pgfsetdash{}{0pt}%
\pgfsys@defobject{currentmarker}{\pgfqpoint{-0.048611in}{0.000000in}}{\pgfqpoint{0.000000in}{0.000000in}}{%
\pgfpathmoveto{\pgfqpoint{0.000000in}{0.000000in}}%
\pgfpathlineto{\pgfqpoint{-0.048611in}{0.000000in}}%
\pgfusepath{stroke,fill}%
}%
\begin{pgfscope}%
\pgfsys@transformshift{0.774444in}{2.161343in}%
\pgfsys@useobject{currentmarker}{}%
\end{pgfscope}%
\end{pgfscope}%
\begin{pgfscope}%
\definecolor{textcolor}{rgb}{0.000000,0.000000,0.000000}%
\pgfsetstrokecolor{textcolor}%
\pgfsetfillcolor{textcolor}%
\pgftext[x=0.339969in,y=2.108581in,left,base]{\color{textcolor}\sffamily\fontsize{10.000000}{12.000000}\selectfont −0.2}%
\end{pgfscope}%
\begin{pgfscope}%
\pgfpathrectangle{\pgfqpoint{0.774444in}{0.571604in}}{\pgfqpoint{1.965116in}{2.225635in}}%
\pgfusepath{clip}%
\pgfsetrectcap%
\pgfsetroundjoin%
\pgfsetlinewidth{0.803000pt}%
\definecolor{currentstroke}{rgb}{0.690196,0.690196,0.690196}%
\pgfsetstrokecolor{currentstroke}%
\pgfsetdash{}{0pt}%
\pgfpathmoveto{\pgfqpoint{0.774444in}{2.479291in}}%
\pgfpathlineto{\pgfqpoint{2.739560in}{2.479291in}}%
\pgfusepath{stroke}%
\end{pgfscope}%
\begin{pgfscope}%
\pgfsetbuttcap%
\pgfsetroundjoin%
\definecolor{currentfill}{rgb}{0.000000,0.000000,0.000000}%
\pgfsetfillcolor{currentfill}%
\pgfsetlinewidth{0.803000pt}%
\definecolor{currentstroke}{rgb}{0.000000,0.000000,0.000000}%
\pgfsetstrokecolor{currentstroke}%
\pgfsetdash{}{0pt}%
\pgfsys@defobject{currentmarker}{\pgfqpoint{-0.048611in}{0.000000in}}{\pgfqpoint{0.000000in}{0.000000in}}{%
\pgfpathmoveto{\pgfqpoint{0.000000in}{0.000000in}}%
\pgfpathlineto{\pgfqpoint{-0.048611in}{0.000000in}}%
\pgfusepath{stroke,fill}%
}%
\begin{pgfscope}%
\pgfsys@transformshift{0.774444in}{2.479291in}%
\pgfsys@useobject{currentmarker}{}%
\end{pgfscope}%
\end{pgfscope}%
\begin{pgfscope}%
\definecolor{textcolor}{rgb}{0.000000,0.000000,0.000000}%
\pgfsetstrokecolor{textcolor}%
\pgfsetfillcolor{textcolor}%
\pgftext[x=0.339969in,y=2.426529in,left,base]{\color{textcolor}\sffamily\fontsize{10.000000}{12.000000}\selectfont −0.1}%
\end{pgfscope}%
\begin{pgfscope}%
\pgfpathrectangle{\pgfqpoint{0.774444in}{0.571604in}}{\pgfqpoint{1.965116in}{2.225635in}}%
\pgfusepath{clip}%
\pgfsetrectcap%
\pgfsetroundjoin%
\pgfsetlinewidth{0.803000pt}%
\definecolor{currentstroke}{rgb}{0.690196,0.690196,0.690196}%
\pgfsetstrokecolor{currentstroke}%
\pgfsetdash{}{0pt}%
\pgfpathmoveto{\pgfqpoint{0.774444in}{2.797238in}}%
\pgfpathlineto{\pgfqpoint{2.739560in}{2.797238in}}%
\pgfusepath{stroke}%
\end{pgfscope}%
\begin{pgfscope}%
\pgfsetbuttcap%
\pgfsetroundjoin%
\definecolor{currentfill}{rgb}{0.000000,0.000000,0.000000}%
\pgfsetfillcolor{currentfill}%
\pgfsetlinewidth{0.803000pt}%
\definecolor{currentstroke}{rgb}{0.000000,0.000000,0.000000}%
\pgfsetstrokecolor{currentstroke}%
\pgfsetdash{}{0pt}%
\pgfsys@defobject{currentmarker}{\pgfqpoint{-0.048611in}{0.000000in}}{\pgfqpoint{0.000000in}{0.000000in}}{%
\pgfpathmoveto{\pgfqpoint{0.000000in}{0.000000in}}%
\pgfpathlineto{\pgfqpoint{-0.048611in}{0.000000in}}%
\pgfusepath{stroke,fill}%
}%
\begin{pgfscope}%
\pgfsys@transformshift{0.774444in}{2.797238in}%
\pgfsys@useobject{currentmarker}{}%
\end{pgfscope}%
\end{pgfscope}%
\begin{pgfscope}%
\definecolor{textcolor}{rgb}{0.000000,0.000000,0.000000}%
\pgfsetstrokecolor{textcolor}%
\pgfsetfillcolor{textcolor}%
\pgftext[x=0.456342in,y=2.744477in,left,base]{\color{textcolor}\sffamily\fontsize{10.000000}{12.000000}\selectfont 0.0}%
\end{pgfscope}%
\begin{pgfscope}%
\definecolor{textcolor}{rgb}{0.000000,0.000000,0.000000}%
\pgfsetstrokecolor{textcolor}%
\pgfsetfillcolor{textcolor}%
\pgftext[x=0.284413in,y=1.684421in,,bottom,rotate=90.000000]{\color{textcolor}\sffamily\fontsize{10.000000}{12.000000}\selectfont MI value}%
\end{pgfscope}%
\begin{pgfscope}%
\pgfpathrectangle{\pgfqpoint{0.774444in}{0.571604in}}{\pgfqpoint{1.965116in}{2.225635in}}%
\pgfusepath{clip}%
\pgfsetrectcap%
\pgfsetroundjoin%
\pgfsetlinewidth{1.505625pt}%
\definecolor{currentstroke}{rgb}{0.121569,0.466667,0.705882}%
\pgfsetstrokecolor{currentstroke}%
\pgfsetdash{}{0pt}%
\pgfpathmoveto{\pgfqpoint{0.774444in}{1.671433in}}%
\pgfpathlineto{\pgfqpoint{0.794095in}{1.700090in}}%
\pgfpathlineto{\pgfqpoint{0.813746in}{1.746405in}}%
\pgfpathlineto{\pgfqpoint{0.833398in}{1.792838in}}%
\pgfpathlineto{\pgfqpoint{0.853049in}{1.830690in}}%
\pgfpathlineto{\pgfqpoint{0.872700in}{1.864764in}}%
\pgfpathlineto{\pgfqpoint{0.892351in}{1.892528in}}%
\pgfpathlineto{\pgfqpoint{0.912002in}{1.913671in}}%
\pgfpathlineto{\pgfqpoint{0.931653in}{1.932850in}}%
\pgfpathlineto{\pgfqpoint{0.951304in}{1.953256in}}%
\pgfpathlineto{\pgfqpoint{0.970956in}{1.972714in}}%
\pgfpathlineto{\pgfqpoint{0.990607in}{1.991228in}}%
\pgfpathlineto{\pgfqpoint{1.010258in}{2.009679in}}%
\pgfpathlineto{\pgfqpoint{1.029909in}{2.027719in}}%
\pgfpathlineto{\pgfqpoint{1.049560in}{2.045708in}}%
\pgfpathlineto{\pgfqpoint{1.069211in}{2.063599in}}%
\pgfpathlineto{\pgfqpoint{1.088863in}{2.081191in}}%
\pgfpathlineto{\pgfqpoint{1.108514in}{2.098647in}}%
\pgfpathlineto{\pgfqpoint{1.128165in}{2.115050in}}%
\pgfpathlineto{\pgfqpoint{1.147816in}{2.131100in}}%
\pgfpathlineto{\pgfqpoint{1.167467in}{2.146577in}}%
\pgfpathlineto{\pgfqpoint{1.187118in}{2.161524in}}%
\pgfpathlineto{\pgfqpoint{1.206770in}{2.175940in}}%
\pgfpathlineto{\pgfqpoint{1.226421in}{2.190906in}}%
\pgfpathlineto{\pgfqpoint{1.246072in}{2.205735in}}%
\pgfpathlineto{\pgfqpoint{1.265723in}{2.220090in}}%
\pgfpathlineto{\pgfqpoint{1.285374in}{2.234992in}}%
\pgfpathlineto{\pgfqpoint{1.305025in}{2.249586in}}%
\pgfpathlineto{\pgfqpoint{1.324677in}{2.263868in}}%
\pgfpathlineto{\pgfqpoint{1.344328in}{2.278373in}}%
\pgfpathlineto{\pgfqpoint{1.363979in}{2.292852in}}%
\pgfpathlineto{\pgfqpoint{1.383630in}{2.307383in}}%
\pgfpathlineto{\pgfqpoint{1.403281in}{2.322336in}}%
\pgfpathlineto{\pgfqpoint{1.422932in}{2.337146in}}%
\pgfpathlineto{\pgfqpoint{1.442584in}{2.352118in}}%
\pgfpathlineto{\pgfqpoint{1.462235in}{2.366925in}}%
\pgfpathlineto{\pgfqpoint{1.481886in}{2.381894in}}%
\pgfpathlineto{\pgfqpoint{1.501537in}{2.397352in}}%
\pgfpathlineto{\pgfqpoint{1.521188in}{2.413431in}}%
\pgfpathlineto{\pgfqpoint{1.540839in}{2.429990in}}%
\pgfpathlineto{\pgfqpoint{1.560491in}{2.446679in}}%
\pgfpathlineto{\pgfqpoint{1.580142in}{2.463842in}}%
\pgfpathlineto{\pgfqpoint{1.599793in}{2.480924in}}%
\pgfpathlineto{\pgfqpoint{1.619444in}{2.498133in}}%
\pgfpathlineto{\pgfqpoint{1.639095in}{2.514959in}}%
\pgfpathlineto{\pgfqpoint{1.658746in}{2.531949in}}%
\pgfpathlineto{\pgfqpoint{1.678398in}{2.548582in}}%
\pgfpathlineto{\pgfqpoint{1.698049in}{2.564975in}}%
\pgfpathlineto{\pgfqpoint{1.717700in}{2.580872in}}%
\pgfpathlineto{\pgfqpoint{1.737351in}{2.596219in}}%
\pgfpathlineto{\pgfqpoint{1.757002in}{2.611008in}}%
\pgfpathlineto{\pgfqpoint{1.776653in}{2.625103in}}%
\pgfpathlineto{\pgfqpoint{1.796304in}{2.638683in}}%
\pgfpathlineto{\pgfqpoint{1.815956in}{2.651532in}}%
\pgfpathlineto{\pgfqpoint{1.835607in}{2.663964in}}%
\pgfpathlineto{\pgfqpoint{1.855258in}{2.675878in}}%
\pgfpathlineto{\pgfqpoint{1.874909in}{2.687180in}}%
\pgfpathlineto{\pgfqpoint{1.894560in}{2.697977in}}%
\pgfpathlineto{\pgfqpoint{1.914211in}{2.708137in}}%
\pgfpathlineto{\pgfqpoint{1.933863in}{2.717706in}}%
\pgfpathlineto{\pgfqpoint{1.953514in}{2.726400in}}%
\pgfpathlineto{\pgfqpoint{1.973165in}{2.734466in}}%
\pgfpathlineto{\pgfqpoint{1.992816in}{2.741935in}}%
\pgfpathlineto{\pgfqpoint{2.012467in}{2.748985in}}%
\pgfpathlineto{\pgfqpoint{2.032118in}{2.755625in}}%
\pgfpathlineto{\pgfqpoint{2.051770in}{2.762059in}}%
\pgfpathlineto{\pgfqpoint{2.071421in}{2.768517in}}%
\pgfpathlineto{\pgfqpoint{2.091072in}{2.774742in}}%
\pgfpathlineto{\pgfqpoint{2.110723in}{2.780598in}}%
\pgfpathlineto{\pgfqpoint{2.130374in}{2.785770in}}%
\pgfpathlineto{\pgfqpoint{2.150025in}{2.789907in}}%
\pgfpathlineto{\pgfqpoint{2.169677in}{2.792955in}}%
\pgfpathlineto{\pgfqpoint{2.189328in}{2.794868in}}%
\pgfpathlineto{\pgfqpoint{2.208979in}{2.795803in}}%
\pgfpathlineto{\pgfqpoint{2.228630in}{2.796344in}}%
\pgfpathlineto{\pgfqpoint{2.248281in}{2.796692in}}%
\pgfpathlineto{\pgfqpoint{2.267932in}{2.796902in}}%
\pgfpathlineto{\pgfqpoint{2.287584in}{2.797026in}}%
\pgfpathlineto{\pgfqpoint{2.307235in}{2.797121in}}%
\pgfpathlineto{\pgfqpoint{2.326886in}{2.797178in}}%
\pgfpathlineto{\pgfqpoint{2.346537in}{2.797198in}}%
\pgfpathlineto{\pgfqpoint{2.366188in}{2.797205in}}%
\pgfpathlineto{\pgfqpoint{2.385839in}{2.797209in}}%
\pgfpathlineto{\pgfqpoint{2.405491in}{2.797211in}}%
\pgfpathlineto{\pgfqpoint{2.425142in}{2.797211in}}%
\pgfpathlineto{\pgfqpoint{2.444793in}{2.797215in}}%
\pgfpathlineto{\pgfqpoint{2.464444in}{2.797219in}}%
\pgfpathlineto{\pgfqpoint{2.484095in}{2.797219in}}%
\pgfpathlineto{\pgfqpoint{2.503746in}{2.797214in}}%
\pgfpathlineto{\pgfqpoint{2.523397in}{2.797218in}}%
\pgfpathlineto{\pgfqpoint{2.543049in}{2.797222in}}%
\pgfpathlineto{\pgfqpoint{2.562700in}{2.797222in}}%
\pgfpathlineto{\pgfqpoint{2.582351in}{2.797214in}}%
\pgfpathlineto{\pgfqpoint{2.602002in}{2.797209in}}%
\pgfpathlineto{\pgfqpoint{2.621653in}{2.797211in}}%
\pgfpathlineto{\pgfqpoint{2.641304in}{2.797208in}}%
\pgfpathlineto{\pgfqpoint{2.660956in}{2.797202in}}%
\pgfpathlineto{\pgfqpoint{2.680607in}{2.797210in}}%
\pgfpathlineto{\pgfqpoint{2.700258in}{2.797210in}}%
\pgfpathlineto{\pgfqpoint{2.719909in}{2.797206in}}%
\pgfpathlineto{\pgfqpoint{2.739560in}{2.797207in}}%
\pgfusepath{stroke}%
\end{pgfscope}%
\begin{pgfscope}%
\pgfpathrectangle{\pgfqpoint{0.774444in}{0.571604in}}{\pgfqpoint{1.965116in}{2.225635in}}%
\pgfusepath{clip}%
\pgfsetrectcap%
\pgfsetroundjoin%
\pgfsetlinewidth{1.505625pt}%
\definecolor{currentstroke}{rgb}{1.000000,0.498039,0.054902}%
\pgfsetstrokecolor{currentstroke}%
\pgfsetdash{}{0pt}%
\pgfpathmoveto{\pgfqpoint{0.774444in}{1.172624in}}%
\pgfpathlineto{\pgfqpoint{0.794095in}{1.215041in}}%
\pgfpathlineto{\pgfqpoint{0.813746in}{1.284659in}}%
\pgfpathlineto{\pgfqpoint{0.833398in}{1.357049in}}%
\pgfpathlineto{\pgfqpoint{0.853049in}{1.422776in}}%
\pgfpathlineto{\pgfqpoint{0.872700in}{1.479793in}}%
\pgfpathlineto{\pgfqpoint{0.892351in}{1.524993in}}%
\pgfpathlineto{\pgfqpoint{0.912002in}{1.560759in}}%
\pgfpathlineto{\pgfqpoint{0.931653in}{1.591501in}}%
\pgfpathlineto{\pgfqpoint{0.951304in}{1.621852in}}%
\pgfpathlineto{\pgfqpoint{0.970956in}{1.648442in}}%
\pgfpathlineto{\pgfqpoint{0.990607in}{1.673884in}}%
\pgfpathlineto{\pgfqpoint{1.010258in}{1.698529in}}%
\pgfpathlineto{\pgfqpoint{1.029909in}{1.722127in}}%
\pgfpathlineto{\pgfqpoint{1.049560in}{1.745550in}}%
\pgfpathlineto{\pgfqpoint{1.069211in}{1.768925in}}%
\pgfpathlineto{\pgfqpoint{1.088863in}{1.791903in}}%
\pgfpathlineto{\pgfqpoint{1.108514in}{1.813699in}}%
\pgfpathlineto{\pgfqpoint{1.128165in}{1.834594in}}%
\pgfpathlineto{\pgfqpoint{1.147816in}{1.855429in}}%
\pgfpathlineto{\pgfqpoint{1.167467in}{1.874684in}}%
\pgfpathlineto{\pgfqpoint{1.187118in}{1.893812in}}%
\pgfpathlineto{\pgfqpoint{1.206770in}{1.912367in}}%
\pgfpathlineto{\pgfqpoint{1.226421in}{1.931196in}}%
\pgfpathlineto{\pgfqpoint{1.246072in}{1.949478in}}%
\pgfpathlineto{\pgfqpoint{1.265723in}{1.967910in}}%
\pgfpathlineto{\pgfqpoint{1.285374in}{1.986087in}}%
\pgfpathlineto{\pgfqpoint{1.305025in}{2.004277in}}%
\pgfpathlineto{\pgfqpoint{1.324677in}{2.021853in}}%
\pgfpathlineto{\pgfqpoint{1.344328in}{2.039216in}}%
\pgfpathlineto{\pgfqpoint{1.363979in}{2.056884in}}%
\pgfpathlineto{\pgfqpoint{1.383630in}{2.075224in}}%
\pgfpathlineto{\pgfqpoint{1.403281in}{2.093585in}}%
\pgfpathlineto{\pgfqpoint{1.422932in}{2.111228in}}%
\pgfpathlineto{\pgfqpoint{1.442584in}{2.129291in}}%
\pgfpathlineto{\pgfqpoint{1.462235in}{2.146813in}}%
\pgfpathlineto{\pgfqpoint{1.481886in}{2.164274in}}%
\pgfpathlineto{\pgfqpoint{1.501537in}{2.182633in}}%
\pgfpathlineto{\pgfqpoint{1.521188in}{2.202002in}}%
\pgfpathlineto{\pgfqpoint{1.540839in}{2.223295in}}%
\pgfpathlineto{\pgfqpoint{1.560491in}{2.244998in}}%
\pgfpathlineto{\pgfqpoint{1.580142in}{2.267922in}}%
\pgfpathlineto{\pgfqpoint{1.599793in}{2.290799in}}%
\pgfpathlineto{\pgfqpoint{1.619444in}{2.313859in}}%
\pgfpathlineto{\pgfqpoint{1.639095in}{2.336538in}}%
\pgfpathlineto{\pgfqpoint{1.658746in}{2.358776in}}%
\pgfpathlineto{\pgfqpoint{1.678398in}{2.380883in}}%
\pgfpathlineto{\pgfqpoint{1.698049in}{2.402068in}}%
\pgfpathlineto{\pgfqpoint{1.717700in}{2.422448in}}%
\pgfpathlineto{\pgfqpoint{1.737351in}{2.443467in}}%
\pgfpathlineto{\pgfqpoint{1.757002in}{2.464573in}}%
\pgfpathlineto{\pgfqpoint{1.776653in}{2.485437in}}%
\pgfpathlineto{\pgfqpoint{1.796304in}{2.506056in}}%
\pgfpathlineto{\pgfqpoint{1.815956in}{2.525867in}}%
\pgfpathlineto{\pgfqpoint{1.835607in}{2.544756in}}%
\pgfpathlineto{\pgfqpoint{1.855258in}{2.562348in}}%
\pgfpathlineto{\pgfqpoint{1.874909in}{2.579001in}}%
\pgfpathlineto{\pgfqpoint{1.894560in}{2.594354in}}%
\pgfpathlineto{\pgfqpoint{1.914211in}{2.609061in}}%
\pgfpathlineto{\pgfqpoint{1.933863in}{2.623028in}}%
\pgfpathlineto{\pgfqpoint{1.953514in}{2.635947in}}%
\pgfpathlineto{\pgfqpoint{1.973165in}{2.647955in}}%
\pgfpathlineto{\pgfqpoint{1.992816in}{2.659448in}}%
\pgfpathlineto{\pgfqpoint{2.012467in}{2.670835in}}%
\pgfpathlineto{\pgfqpoint{2.032118in}{2.682374in}}%
\pgfpathlineto{\pgfqpoint{2.051770in}{2.693898in}}%
\pgfpathlineto{\pgfqpoint{2.071421in}{2.705236in}}%
\pgfpathlineto{\pgfqpoint{2.091072in}{2.717161in}}%
\pgfpathlineto{\pgfqpoint{2.110723in}{2.729490in}}%
\pgfpathlineto{\pgfqpoint{2.130374in}{2.741308in}}%
\pgfpathlineto{\pgfqpoint{2.150025in}{2.751802in}}%
\pgfpathlineto{\pgfqpoint{2.169677in}{2.761770in}}%
\pgfpathlineto{\pgfqpoint{2.189328in}{2.771618in}}%
\pgfpathlineto{\pgfqpoint{2.208979in}{2.780065in}}%
\pgfpathlineto{\pgfqpoint{2.228630in}{2.786233in}}%
\pgfpathlineto{\pgfqpoint{2.248281in}{2.790492in}}%
\pgfpathlineto{\pgfqpoint{2.267932in}{2.792975in}}%
\pgfpathlineto{\pgfqpoint{2.287584in}{2.794556in}}%
\pgfpathlineto{\pgfqpoint{2.307235in}{2.795871in}}%
\pgfpathlineto{\pgfqpoint{2.326886in}{2.796653in}}%
\pgfpathlineto{\pgfqpoint{2.346537in}{2.796916in}}%
\pgfpathlineto{\pgfqpoint{2.366188in}{2.796982in}}%
\pgfpathlineto{\pgfqpoint{2.385839in}{2.796994in}}%
\pgfpathlineto{\pgfqpoint{2.405491in}{2.796969in}}%
\pgfpathlineto{\pgfqpoint{2.425142in}{2.796935in}}%
\pgfpathlineto{\pgfqpoint{2.444793in}{2.796924in}}%
\pgfpathlineto{\pgfqpoint{2.464444in}{2.796919in}}%
\pgfpathlineto{\pgfqpoint{2.484095in}{2.796901in}}%
\pgfpathlineto{\pgfqpoint{2.503746in}{2.796869in}}%
\pgfpathlineto{\pgfqpoint{2.523397in}{2.796859in}}%
\pgfpathlineto{\pgfqpoint{2.543049in}{2.796879in}}%
\pgfpathlineto{\pgfqpoint{2.562700in}{2.796862in}}%
\pgfpathlineto{\pgfqpoint{2.582351in}{2.796790in}}%
\pgfpathlineto{\pgfqpoint{2.602002in}{2.796764in}}%
\pgfpathlineto{\pgfqpoint{2.621653in}{2.796769in}}%
\pgfpathlineto{\pgfqpoint{2.641304in}{2.796712in}}%
\pgfpathlineto{\pgfqpoint{2.660956in}{2.796628in}}%
\pgfpathlineto{\pgfqpoint{2.680607in}{2.796664in}}%
\pgfpathlineto{\pgfqpoint{2.700258in}{2.796621in}}%
\pgfpathlineto{\pgfqpoint{2.719909in}{2.796572in}}%
\pgfpathlineto{\pgfqpoint{2.739560in}{2.796637in}}%
\pgfusepath{stroke}%
\end{pgfscope}%
\begin{pgfscope}%
\pgfpathrectangle{\pgfqpoint{0.774444in}{0.571604in}}{\pgfqpoint{1.965116in}{2.225635in}}%
\pgfusepath{clip}%
\pgfsetrectcap%
\pgfsetroundjoin%
\pgfsetlinewidth{1.505625pt}%
\definecolor{currentstroke}{rgb}{0.172549,0.627451,0.172549}%
\pgfsetstrokecolor{currentstroke}%
\pgfsetdash{}{0pt}%
\pgfpathmoveto{\pgfqpoint{0.774444in}{0.800765in}}%
\pgfpathlineto{\pgfqpoint{0.794095in}{0.857572in}}%
\pgfpathlineto{\pgfqpoint{0.813746in}{0.940633in}}%
\pgfpathlineto{\pgfqpoint{0.833398in}{1.031293in}}%
\pgfpathlineto{\pgfqpoint{0.853049in}{1.115841in}}%
\pgfpathlineto{\pgfqpoint{0.872700in}{1.190518in}}%
\pgfpathlineto{\pgfqpoint{0.892351in}{1.248744in}}%
\pgfpathlineto{\pgfqpoint{0.912002in}{1.296875in}}%
\pgfpathlineto{\pgfqpoint{0.931653in}{1.338182in}}%
\pgfpathlineto{\pgfqpoint{0.951304in}{1.376972in}}%
\pgfpathlineto{\pgfqpoint{0.970956in}{1.410986in}}%
\pgfpathlineto{\pgfqpoint{0.990607in}{1.443299in}}%
\pgfpathlineto{\pgfqpoint{1.010258in}{1.474102in}}%
\pgfpathlineto{\pgfqpoint{1.029909in}{1.501385in}}%
\pgfpathlineto{\pgfqpoint{1.049560in}{1.529600in}}%
\pgfpathlineto{\pgfqpoint{1.069211in}{1.555598in}}%
\pgfpathlineto{\pgfqpoint{1.088863in}{1.581635in}}%
\pgfpathlineto{\pgfqpoint{1.108514in}{1.606928in}}%
\pgfpathlineto{\pgfqpoint{1.128165in}{1.630850in}}%
\pgfpathlineto{\pgfqpoint{1.147816in}{1.654407in}}%
\pgfpathlineto{\pgfqpoint{1.167467in}{1.676450in}}%
\pgfpathlineto{\pgfqpoint{1.187118in}{1.698220in}}%
\pgfpathlineto{\pgfqpoint{1.206770in}{1.719465in}}%
\pgfpathlineto{\pgfqpoint{1.226421in}{1.740046in}}%
\pgfpathlineto{\pgfqpoint{1.246072in}{1.760315in}}%
\pgfpathlineto{\pgfqpoint{1.265723in}{1.780763in}}%
\pgfpathlineto{\pgfqpoint{1.285374in}{1.799528in}}%
\pgfpathlineto{\pgfqpoint{1.305025in}{1.818643in}}%
\pgfpathlineto{\pgfqpoint{1.324677in}{1.837392in}}%
\pgfpathlineto{\pgfqpoint{1.344328in}{1.855099in}}%
\pgfpathlineto{\pgfqpoint{1.363979in}{1.873285in}}%
\pgfpathlineto{\pgfqpoint{1.383630in}{1.893583in}}%
\pgfpathlineto{\pgfqpoint{1.403281in}{1.912650in}}%
\pgfpathlineto{\pgfqpoint{1.422932in}{1.931279in}}%
\pgfpathlineto{\pgfqpoint{1.442584in}{1.950782in}}%
\pgfpathlineto{\pgfqpoint{1.462235in}{1.969013in}}%
\pgfpathlineto{\pgfqpoint{1.481886in}{1.987473in}}%
\pgfpathlineto{\pgfqpoint{1.501537in}{2.006604in}}%
\pgfpathlineto{\pgfqpoint{1.521188in}{2.027477in}}%
\pgfpathlineto{\pgfqpoint{1.540839in}{2.049028in}}%
\pgfpathlineto{\pgfqpoint{1.560491in}{2.072597in}}%
\pgfpathlineto{\pgfqpoint{1.580142in}{2.096838in}}%
\pgfpathlineto{\pgfqpoint{1.599793in}{2.122146in}}%
\pgfpathlineto{\pgfqpoint{1.619444in}{2.147964in}}%
\pgfpathlineto{\pgfqpoint{1.639095in}{2.173139in}}%
\pgfpathlineto{\pgfqpoint{1.658746in}{2.198231in}}%
\pgfpathlineto{\pgfqpoint{1.678398in}{2.222023in}}%
\pgfpathlineto{\pgfqpoint{1.698049in}{2.244801in}}%
\pgfpathlineto{\pgfqpoint{1.717700in}{2.266580in}}%
\pgfpathlineto{\pgfqpoint{1.737351in}{2.289931in}}%
\pgfpathlineto{\pgfqpoint{1.757002in}{2.313764in}}%
\pgfpathlineto{\pgfqpoint{1.776653in}{2.338545in}}%
\pgfpathlineto{\pgfqpoint{1.796304in}{2.363265in}}%
\pgfpathlineto{\pgfqpoint{1.815956in}{2.387932in}}%
\pgfpathlineto{\pgfqpoint{1.835607in}{2.410487in}}%
\pgfpathlineto{\pgfqpoint{1.855258in}{2.432864in}}%
\pgfpathlineto{\pgfqpoint{1.874909in}{2.454208in}}%
\pgfpathlineto{\pgfqpoint{1.894560in}{2.474890in}}%
\pgfpathlineto{\pgfqpoint{1.914211in}{2.494436in}}%
\pgfpathlineto{\pgfqpoint{1.933863in}{2.513213in}}%
\pgfpathlineto{\pgfqpoint{1.953514in}{2.530664in}}%
\pgfpathlineto{\pgfqpoint{1.973165in}{2.548568in}}%
\pgfpathlineto{\pgfqpoint{1.992816in}{2.564990in}}%
\pgfpathlineto{\pgfqpoint{2.012467in}{2.581454in}}%
\pgfpathlineto{\pgfqpoint{2.032118in}{2.596751in}}%
\pgfpathlineto{\pgfqpoint{2.051770in}{2.611810in}}%
\pgfpathlineto{\pgfqpoint{2.071421in}{2.626581in}}%
\pgfpathlineto{\pgfqpoint{2.091072in}{2.641069in}}%
\pgfpathlineto{\pgfqpoint{2.110723in}{2.655598in}}%
\pgfpathlineto{\pgfqpoint{2.130374in}{2.669228in}}%
\pgfpathlineto{\pgfqpoint{2.150025in}{2.681519in}}%
\pgfpathlineto{\pgfqpoint{2.169677in}{2.693902in}}%
\pgfpathlineto{\pgfqpoint{2.189328in}{2.707166in}}%
\pgfpathlineto{\pgfqpoint{2.208979in}{2.722732in}}%
\pgfpathlineto{\pgfqpoint{2.228630in}{2.737396in}}%
\pgfpathlineto{\pgfqpoint{2.248281in}{2.752325in}}%
\pgfpathlineto{\pgfqpoint{2.267932in}{2.764366in}}%
\pgfpathlineto{\pgfqpoint{2.287584in}{2.773043in}}%
\pgfpathlineto{\pgfqpoint{2.307235in}{2.779908in}}%
\pgfpathlineto{\pgfqpoint{2.326886in}{2.786507in}}%
\pgfpathlineto{\pgfqpoint{2.346537in}{2.790668in}}%
\pgfpathlineto{\pgfqpoint{2.366188in}{2.791526in}}%
\pgfpathlineto{\pgfqpoint{2.385839in}{2.791440in}}%
\pgfpathlineto{\pgfqpoint{2.405491in}{2.791377in}}%
\pgfpathlineto{\pgfqpoint{2.425142in}{2.791167in}}%
\pgfpathlineto{\pgfqpoint{2.444793in}{2.791131in}}%
\pgfpathlineto{\pgfqpoint{2.464444in}{2.791153in}}%
\pgfpathlineto{\pgfqpoint{2.484095in}{2.791225in}}%
\pgfpathlineto{\pgfqpoint{2.503746in}{2.791043in}}%
\pgfpathlineto{\pgfqpoint{2.523397in}{2.790573in}}%
\pgfpathlineto{\pgfqpoint{2.543049in}{2.790622in}}%
\pgfpathlineto{\pgfqpoint{2.562700in}{2.790529in}}%
\pgfpathlineto{\pgfqpoint{2.582351in}{2.789930in}}%
\pgfpathlineto{\pgfqpoint{2.602002in}{2.789633in}}%
\pgfpathlineto{\pgfqpoint{2.621653in}{2.789225in}}%
\pgfpathlineto{\pgfqpoint{2.641304in}{2.789052in}}%
\pgfpathlineto{\pgfqpoint{2.660956in}{2.788788in}}%
\pgfpathlineto{\pgfqpoint{2.680607in}{2.788601in}}%
\pgfpathlineto{\pgfqpoint{2.700258in}{2.788603in}}%
\pgfpathlineto{\pgfqpoint{2.719909in}{2.787798in}}%
\pgfpathlineto{\pgfqpoint{2.739560in}{2.787581in}}%
\pgfusepath{stroke}%
\end{pgfscope}%
\begin{pgfscope}%
\pgfpathrectangle{\pgfqpoint{0.774444in}{0.571604in}}{\pgfqpoint{1.965116in}{2.225635in}}%
\pgfusepath{clip}%
\pgfsetrectcap%
\pgfsetroundjoin%
\pgfsetlinewidth{1.505625pt}%
\definecolor{currentstroke}{rgb}{0.839216,0.152941,0.156863}%
\pgfsetstrokecolor{currentstroke}%
\pgfsetdash{}{0pt}%
\pgfpathmoveto{\pgfqpoint{0.774444in}{0.698764in}}%
\pgfpathlineto{\pgfqpoint{0.794095in}{0.752682in}}%
\pgfpathlineto{\pgfqpoint{0.813746in}{0.834734in}}%
\pgfpathlineto{\pgfqpoint{0.833398in}{0.927839in}}%
\pgfpathlineto{\pgfqpoint{0.853049in}{1.014642in}}%
\pgfpathlineto{\pgfqpoint{0.872700in}{1.093417in}}%
\pgfpathlineto{\pgfqpoint{0.892351in}{1.154122in}}%
\pgfpathlineto{\pgfqpoint{0.912002in}{1.205198in}}%
\pgfpathlineto{\pgfqpoint{0.931653in}{1.248992in}}%
\pgfpathlineto{\pgfqpoint{0.951304in}{1.288681in}}%
\pgfpathlineto{\pgfqpoint{0.970956in}{1.324800in}}%
\pgfpathlineto{\pgfqpoint{0.990607in}{1.358556in}}%
\pgfpathlineto{\pgfqpoint{1.010258in}{1.390602in}}%
\pgfpathlineto{\pgfqpoint{1.029909in}{1.419952in}}%
\pgfpathlineto{\pgfqpoint{1.049560in}{1.450160in}}%
\pgfpathlineto{\pgfqpoint{1.069211in}{1.477876in}}%
\pgfpathlineto{\pgfqpoint{1.088863in}{1.505283in}}%
\pgfpathlineto{\pgfqpoint{1.108514in}{1.532143in}}%
\pgfpathlineto{\pgfqpoint{1.128165in}{1.556969in}}%
\pgfpathlineto{\pgfqpoint{1.147816in}{1.581721in}}%
\pgfpathlineto{\pgfqpoint{1.167467in}{1.604998in}}%
\pgfpathlineto{\pgfqpoint{1.187118in}{1.627842in}}%
\pgfpathlineto{\pgfqpoint{1.206770in}{1.649123in}}%
\pgfpathlineto{\pgfqpoint{1.226421in}{1.670530in}}%
\pgfpathlineto{\pgfqpoint{1.246072in}{1.691499in}}%
\pgfpathlineto{\pgfqpoint{1.265723in}{1.712534in}}%
\pgfpathlineto{\pgfqpoint{1.285374in}{1.731919in}}%
\pgfpathlineto{\pgfqpoint{1.305025in}{1.751693in}}%
\pgfpathlineto{\pgfqpoint{1.324677in}{1.771227in}}%
\pgfpathlineto{\pgfqpoint{1.344328in}{1.789306in}}%
\pgfpathlineto{\pgfqpoint{1.363979in}{1.808786in}}%
\pgfpathlineto{\pgfqpoint{1.383630in}{1.829548in}}%
\pgfpathlineto{\pgfqpoint{1.403281in}{1.849239in}}%
\pgfpathlineto{\pgfqpoint{1.422932in}{1.868821in}}%
\pgfpathlineto{\pgfqpoint{1.442584in}{1.889087in}}%
\pgfpathlineto{\pgfqpoint{1.462235in}{1.907605in}}%
\pgfpathlineto{\pgfqpoint{1.481886in}{1.927092in}}%
\pgfpathlineto{\pgfqpoint{1.501537in}{1.946114in}}%
\pgfpathlineto{\pgfqpoint{1.521188in}{1.967344in}}%
\pgfpathlineto{\pgfqpoint{1.540839in}{1.988735in}}%
\pgfpathlineto{\pgfqpoint{1.560491in}{2.012006in}}%
\pgfpathlineto{\pgfqpoint{1.580142in}{2.036946in}}%
\pgfpathlineto{\pgfqpoint{1.599793in}{2.061755in}}%
\pgfpathlineto{\pgfqpoint{1.619444in}{2.087442in}}%
\pgfpathlineto{\pgfqpoint{1.639095in}{2.113126in}}%
\pgfpathlineto{\pgfqpoint{1.658746in}{2.138540in}}%
\pgfpathlineto{\pgfqpoint{1.678398in}{2.163623in}}%
\pgfpathlineto{\pgfqpoint{1.698049in}{2.187488in}}%
\pgfpathlineto{\pgfqpoint{1.717700in}{2.210946in}}%
\pgfpathlineto{\pgfqpoint{1.737351in}{2.235600in}}%
\pgfpathlineto{\pgfqpoint{1.757002in}{2.260180in}}%
\pgfpathlineto{\pgfqpoint{1.776653in}{2.285603in}}%
\pgfpathlineto{\pgfqpoint{1.796304in}{2.311608in}}%
\pgfpathlineto{\pgfqpoint{1.815956in}{2.338373in}}%
\pgfpathlineto{\pgfqpoint{1.835607in}{2.360849in}}%
\pgfpathlineto{\pgfqpoint{1.855258in}{2.383862in}}%
\pgfpathlineto{\pgfqpoint{1.874909in}{2.406312in}}%
\pgfpathlineto{\pgfqpoint{1.894560in}{2.428848in}}%
\pgfpathlineto{\pgfqpoint{1.914211in}{2.450192in}}%
\pgfpathlineto{\pgfqpoint{1.933863in}{2.470913in}}%
\pgfpathlineto{\pgfqpoint{1.953514in}{2.489670in}}%
\pgfpathlineto{\pgfqpoint{1.973165in}{2.509584in}}%
\pgfpathlineto{\pgfqpoint{1.992816in}{2.527449in}}%
\pgfpathlineto{\pgfqpoint{2.012467in}{2.544549in}}%
\pgfpathlineto{\pgfqpoint{2.032118in}{2.560866in}}%
\pgfpathlineto{\pgfqpoint{2.051770in}{2.575584in}}%
\pgfpathlineto{\pgfqpoint{2.071421in}{2.589584in}}%
\pgfpathlineto{\pgfqpoint{2.091072in}{2.604853in}}%
\pgfpathlineto{\pgfqpoint{2.110723in}{2.620213in}}%
\pgfpathlineto{\pgfqpoint{2.130374in}{2.634040in}}%
\pgfpathlineto{\pgfqpoint{2.150025in}{2.646267in}}%
\pgfpathlineto{\pgfqpoint{2.169677in}{2.660178in}}%
\pgfpathlineto{\pgfqpoint{2.189328in}{2.673448in}}%
\pgfpathlineto{\pgfqpoint{2.208979in}{2.688756in}}%
\pgfpathlineto{\pgfqpoint{2.228630in}{2.704679in}}%
\pgfpathlineto{\pgfqpoint{2.248281in}{2.720331in}}%
\pgfpathlineto{\pgfqpoint{2.267932in}{2.735631in}}%
\pgfpathlineto{\pgfqpoint{2.287584in}{2.747787in}}%
\pgfpathlineto{\pgfqpoint{2.307235in}{2.757103in}}%
\pgfpathlineto{\pgfqpoint{2.326886in}{2.765443in}}%
\pgfpathlineto{\pgfqpoint{2.346537in}{2.773279in}}%
\pgfpathlineto{\pgfqpoint{2.366188in}{2.775699in}}%
\pgfpathlineto{\pgfqpoint{2.385839in}{2.774933in}}%
\pgfpathlineto{\pgfqpoint{2.405491in}{2.775623in}}%
\pgfpathlineto{\pgfqpoint{2.425142in}{2.775154in}}%
\pgfpathlineto{\pgfqpoint{2.444793in}{2.774415in}}%
\pgfpathlineto{\pgfqpoint{2.464444in}{2.774569in}}%
\pgfpathlineto{\pgfqpoint{2.484095in}{2.774705in}}%
\pgfpathlineto{\pgfqpoint{2.503746in}{2.774513in}}%
\pgfpathlineto{\pgfqpoint{2.523397in}{2.772611in}}%
\pgfpathlineto{\pgfqpoint{2.543049in}{2.772353in}}%
\pgfpathlineto{\pgfqpoint{2.562700in}{2.772600in}}%
\pgfpathlineto{\pgfqpoint{2.582351in}{2.770936in}}%
\pgfpathlineto{\pgfqpoint{2.602002in}{2.770355in}}%
\pgfpathlineto{\pgfqpoint{2.621653in}{2.768842in}}%
\pgfpathlineto{\pgfqpoint{2.641304in}{2.768043in}}%
\pgfpathlineto{\pgfqpoint{2.660956in}{2.768070in}}%
\pgfpathlineto{\pgfqpoint{2.680607in}{2.767428in}}%
\pgfpathlineto{\pgfqpoint{2.700258in}{2.767187in}}%
\pgfpathlineto{\pgfqpoint{2.719909in}{2.764761in}}%
\pgfpathlineto{\pgfqpoint{2.739560in}{2.763128in}}%
\pgfusepath{stroke}%
\end{pgfscope}%
\begin{pgfscope}%
\pgfpathrectangle{\pgfqpoint{0.774444in}{0.571604in}}{\pgfqpoint{1.965116in}{2.225635in}}%
\pgfusepath{clip}%
\pgfsetrectcap%
\pgfsetroundjoin%
\pgfsetlinewidth{1.505625pt}%
\definecolor{currentstroke}{rgb}{0.580392,0.403922,0.741176}%
\pgfsetstrokecolor{currentstroke}%
\pgfsetdash{}{0pt}%
\pgfpathmoveto{\pgfqpoint{0.774444in}{0.605151in}}%
\pgfpathlineto{\pgfqpoint{0.794095in}{0.662925in}}%
\pgfpathlineto{\pgfqpoint{0.813746in}{0.741522in}}%
\pgfpathlineto{\pgfqpoint{0.833398in}{0.831116in}}%
\pgfpathlineto{\pgfqpoint{0.853049in}{0.915334in}}%
\pgfpathlineto{\pgfqpoint{0.872700in}{0.993956in}}%
\pgfpathlineto{\pgfqpoint{0.892351in}{1.055832in}}%
\pgfpathlineto{\pgfqpoint{0.912002in}{1.109600in}}%
\pgfpathlineto{\pgfqpoint{0.931653in}{1.157591in}}%
\pgfpathlineto{\pgfqpoint{0.951304in}{1.200851in}}%
\pgfpathlineto{\pgfqpoint{0.970956in}{1.240798in}}%
\pgfpathlineto{\pgfqpoint{0.990607in}{1.276427in}}%
\pgfpathlineto{\pgfqpoint{1.010258in}{1.310562in}}%
\pgfpathlineto{\pgfqpoint{1.029909in}{1.342459in}}%
\pgfpathlineto{\pgfqpoint{1.049560in}{1.375328in}}%
\pgfpathlineto{\pgfqpoint{1.069211in}{1.404881in}}%
\pgfpathlineto{\pgfqpoint{1.088863in}{1.434079in}}%
\pgfpathlineto{\pgfqpoint{1.108514in}{1.462249in}}%
\pgfpathlineto{\pgfqpoint{1.128165in}{1.487844in}}%
\pgfpathlineto{\pgfqpoint{1.147816in}{1.514052in}}%
\pgfpathlineto{\pgfqpoint{1.167467in}{1.538493in}}%
\pgfpathlineto{\pgfqpoint{1.187118in}{1.562383in}}%
\pgfpathlineto{\pgfqpoint{1.206770in}{1.584544in}}%
\pgfpathlineto{\pgfqpoint{1.226421in}{1.606937in}}%
\pgfpathlineto{\pgfqpoint{1.246072in}{1.628262in}}%
\pgfpathlineto{\pgfqpoint{1.265723in}{1.649727in}}%
\pgfpathlineto{\pgfqpoint{1.285374in}{1.670228in}}%
\pgfpathlineto{\pgfqpoint{1.305025in}{1.690580in}}%
\pgfpathlineto{\pgfqpoint{1.324677in}{1.711533in}}%
\pgfpathlineto{\pgfqpoint{1.344328in}{1.731029in}}%
\pgfpathlineto{\pgfqpoint{1.363979in}{1.751734in}}%
\pgfpathlineto{\pgfqpoint{1.383630in}{1.773513in}}%
\pgfpathlineto{\pgfqpoint{1.403281in}{1.794333in}}%
\pgfpathlineto{\pgfqpoint{1.422932in}{1.815378in}}%
\pgfpathlineto{\pgfqpoint{1.442584in}{1.836836in}}%
\pgfpathlineto{\pgfqpoint{1.462235in}{1.856917in}}%
\pgfpathlineto{\pgfqpoint{1.481886in}{1.877902in}}%
\pgfpathlineto{\pgfqpoint{1.501537in}{1.897869in}}%
\pgfpathlineto{\pgfqpoint{1.521188in}{1.920268in}}%
\pgfpathlineto{\pgfqpoint{1.540839in}{1.942623in}}%
\pgfpathlineto{\pgfqpoint{1.560491in}{1.965789in}}%
\pgfpathlineto{\pgfqpoint{1.580142in}{1.991069in}}%
\pgfpathlineto{\pgfqpoint{1.599793in}{2.014550in}}%
\pgfpathlineto{\pgfqpoint{1.619444in}{2.039191in}}%
\pgfpathlineto{\pgfqpoint{1.639095in}{2.064223in}}%
\pgfpathlineto{\pgfqpoint{1.658746in}{2.088565in}}%
\pgfpathlineto{\pgfqpoint{1.678398in}{2.113546in}}%
\pgfpathlineto{\pgfqpoint{1.698049in}{2.136950in}}%
\pgfpathlineto{\pgfqpoint{1.717700in}{2.159823in}}%
\pgfpathlineto{\pgfqpoint{1.737351in}{2.184060in}}%
\pgfpathlineto{\pgfqpoint{1.757002in}{2.207722in}}%
\pgfpathlineto{\pgfqpoint{1.776653in}{2.232353in}}%
\pgfpathlineto{\pgfqpoint{1.796304in}{2.257048in}}%
\pgfpathlineto{\pgfqpoint{1.815956in}{2.283994in}}%
\pgfpathlineto{\pgfqpoint{1.835607in}{2.305230in}}%
\pgfpathlineto{\pgfqpoint{1.855258in}{2.328040in}}%
\pgfpathlineto{\pgfqpoint{1.874909in}{2.349791in}}%
\pgfpathlineto{\pgfqpoint{1.894560in}{2.371128in}}%
\pgfpathlineto{\pgfqpoint{1.914211in}{2.391556in}}%
\pgfpathlineto{\pgfqpoint{1.933863in}{2.412693in}}%
\pgfpathlineto{\pgfqpoint{1.953514in}{2.430705in}}%
\pgfpathlineto{\pgfqpoint{1.973165in}{2.450478in}}%
\pgfpathlineto{\pgfqpoint{1.992816in}{2.468649in}}%
\pgfpathlineto{\pgfqpoint{2.012467in}{2.486308in}}%
\pgfpathlineto{\pgfqpoint{2.032118in}{2.504111in}}%
\pgfpathlineto{\pgfqpoint{2.051770in}{2.520548in}}%
\pgfpathlineto{\pgfqpoint{2.071421in}{2.535955in}}%
\pgfpathlineto{\pgfqpoint{2.091072in}{2.553551in}}%
\pgfpathlineto{\pgfqpoint{2.110723in}{2.570165in}}%
\pgfpathlineto{\pgfqpoint{2.130374in}{2.586277in}}%
\pgfpathlineto{\pgfqpoint{2.150025in}{2.602599in}}%
\pgfpathlineto{\pgfqpoint{2.169677in}{2.619900in}}%
\pgfpathlineto{\pgfqpoint{2.189328in}{2.636796in}}%
\pgfpathlineto{\pgfqpoint{2.208979in}{2.654286in}}%
\pgfpathlineto{\pgfqpoint{2.228630in}{2.672192in}}%
\pgfpathlineto{\pgfqpoint{2.248281in}{2.688002in}}%
\pgfpathlineto{\pgfqpoint{2.267932in}{2.704821in}}%
\pgfpathlineto{\pgfqpoint{2.287584in}{2.717881in}}%
\pgfpathlineto{\pgfqpoint{2.307235in}{2.729050in}}%
\pgfpathlineto{\pgfqpoint{2.326886in}{2.737311in}}%
\pgfpathlineto{\pgfqpoint{2.346537in}{2.748308in}}%
\pgfpathlineto{\pgfqpoint{2.366188in}{2.751502in}}%
\pgfpathlineto{\pgfqpoint{2.385839in}{2.749274in}}%
\pgfpathlineto{\pgfqpoint{2.405491in}{2.750426in}}%
\pgfpathlineto{\pgfqpoint{2.425142in}{2.749037in}}%
\pgfpathlineto{\pgfqpoint{2.444793in}{2.747158in}}%
\pgfpathlineto{\pgfqpoint{2.464444in}{2.747181in}}%
\pgfpathlineto{\pgfqpoint{2.484095in}{2.746449in}}%
\pgfpathlineto{\pgfqpoint{2.503746in}{2.746680in}}%
\pgfpathlineto{\pgfqpoint{2.523397in}{2.742488in}}%
\pgfpathlineto{\pgfqpoint{2.543049in}{2.741626in}}%
\pgfpathlineto{\pgfqpoint{2.562700in}{2.742622in}}%
\pgfpathlineto{\pgfqpoint{2.582351in}{2.739633in}}%
\pgfpathlineto{\pgfqpoint{2.602002in}{2.737710in}}%
\pgfpathlineto{\pgfqpoint{2.621653in}{2.733758in}}%
\pgfpathlineto{\pgfqpoint{2.641304in}{2.730371in}}%
\pgfpathlineto{\pgfqpoint{2.660956in}{2.731231in}}%
\pgfpathlineto{\pgfqpoint{2.680607in}{2.728567in}}%
\pgfpathlineto{\pgfqpoint{2.700258in}{2.726906in}}%
\pgfpathlineto{\pgfqpoint{2.719909in}{2.723951in}}%
\pgfpathlineto{\pgfqpoint{2.739560in}{2.720451in}}%
\pgfusepath{stroke}%
\end{pgfscope}%
\begin{pgfscope}%
\pgfsetrectcap%
\pgfsetmiterjoin%
\pgfsetlinewidth{0.803000pt}%
\definecolor{currentstroke}{rgb}{0.000000,0.000000,0.000000}%
\pgfsetstrokecolor{currentstroke}%
\pgfsetdash{}{0pt}%
\pgfpathmoveto{\pgfqpoint{0.774444in}{0.571604in}}%
\pgfpathlineto{\pgfqpoint{0.774444in}{2.797238in}}%
\pgfusepath{stroke}%
\end{pgfscope}%
\begin{pgfscope}%
\pgfsetrectcap%
\pgfsetmiterjoin%
\pgfsetlinewidth{0.803000pt}%
\definecolor{currentstroke}{rgb}{0.000000,0.000000,0.000000}%
\pgfsetstrokecolor{currentstroke}%
\pgfsetdash{}{0pt}%
\pgfpathmoveto{\pgfqpoint{2.739560in}{0.571604in}}%
\pgfpathlineto{\pgfqpoint{2.739560in}{2.797238in}}%
\pgfusepath{stroke}%
\end{pgfscope}%
\begin{pgfscope}%
\pgfsetrectcap%
\pgfsetmiterjoin%
\pgfsetlinewidth{0.803000pt}%
\definecolor{currentstroke}{rgb}{0.000000,0.000000,0.000000}%
\pgfsetstrokecolor{currentstroke}%
\pgfsetdash{}{0pt}%
\pgfpathmoveto{\pgfqpoint{0.774444in}{0.571604in}}%
\pgfpathlineto{\pgfqpoint{2.739560in}{0.571604in}}%
\pgfusepath{stroke}%
\end{pgfscope}%
\begin{pgfscope}%
\pgfsetrectcap%
\pgfsetmiterjoin%
\pgfsetlinewidth{0.803000pt}%
\definecolor{currentstroke}{rgb}{0.000000,0.000000,0.000000}%
\pgfsetstrokecolor{currentstroke}%
\pgfsetdash{}{0pt}%
\pgfpathmoveto{\pgfqpoint{0.774444in}{2.797238in}}%
\pgfpathlineto{\pgfqpoint{2.739560in}{2.797238in}}%
\pgfusepath{stroke}%
\end{pgfscope}%
\begin{pgfscope}%
\pgfsetbuttcap%
\pgfsetmiterjoin%
\definecolor{currentfill}{rgb}{1.000000,1.000000,1.000000}%
\pgfsetfillcolor{currentfill}%
\pgfsetfillopacity{0.800000}%
\pgfsetlinewidth{1.003750pt}%
\definecolor{currentstroke}{rgb}{0.800000,0.800000,0.800000}%
\pgfsetstrokecolor{currentstroke}%
\pgfsetstrokeopacity{0.800000}%
\pgfsetdash{}{0pt}%
\pgfpathmoveto{\pgfqpoint{1.601512in}{0.641048in}}%
\pgfpathlineto{\pgfqpoint{2.642338in}{0.641048in}}%
\pgfpathquadraticcurveto{\pgfqpoint{2.670116in}{0.641048in}}{\pgfqpoint{2.670116in}{0.668826in}}%
\pgfpathlineto{\pgfqpoint{2.670116in}{1.674225in}}%
\pgfpathquadraticcurveto{\pgfqpoint{2.670116in}{1.702002in}}{\pgfqpoint{2.642338in}{1.702002in}}%
\pgfpathlineto{\pgfqpoint{1.601512in}{1.702002in}}%
\pgfpathquadraticcurveto{\pgfqpoint{1.573735in}{1.702002in}}{\pgfqpoint{1.573735in}{1.674225in}}%
\pgfpathlineto{\pgfqpoint{1.573735in}{0.668826in}}%
\pgfpathquadraticcurveto{\pgfqpoint{1.573735in}{0.641048in}}{\pgfqpoint{1.601512in}{0.641048in}}%
\pgfpathclose%
\pgfusepath{stroke,fill}%
\end{pgfscope}%
\begin{pgfscope}%
\pgfsetrectcap%
\pgfsetroundjoin%
\pgfsetlinewidth{1.505625pt}%
\definecolor{currentstroke}{rgb}{0.121569,0.466667,0.705882}%
\pgfsetstrokecolor{currentstroke}%
\pgfsetdash{}{0pt}%
\pgfpathmoveto{\pgfqpoint{1.629290in}{1.589535in}}%
\pgfpathlineto{\pgfqpoint{1.907068in}{1.589535in}}%
\pgfusepath{stroke}%
\end{pgfscope}%
\begin{pgfscope}%
\definecolor{textcolor}{rgb}{0.000000,0.000000,0.000000}%
\pgfsetstrokecolor{textcolor}%
\pgfsetfillcolor{textcolor}%
\pgftext[x=2.018179in,y=1.540924in,left,base]{\color{textcolor}\sffamily\fontsize{10.000000}{12.000000}\selectfont 10 bins}%
\end{pgfscope}%
\begin{pgfscope}%
\pgfsetrectcap%
\pgfsetroundjoin%
\pgfsetlinewidth{1.505625pt}%
\definecolor{currentstroke}{rgb}{1.000000,0.498039,0.054902}%
\pgfsetstrokecolor{currentstroke}%
\pgfsetdash{}{0pt}%
\pgfpathmoveto{\pgfqpoint{1.629290in}{1.385677in}}%
\pgfpathlineto{\pgfqpoint{1.907068in}{1.385677in}}%
\pgfusepath{stroke}%
\end{pgfscope}%
\begin{pgfscope}%
\definecolor{textcolor}{rgb}{0.000000,0.000000,0.000000}%
\pgfsetstrokecolor{textcolor}%
\pgfsetfillcolor{textcolor}%
\pgftext[x=2.018179in,y=1.337066in,left,base]{\color{textcolor}\sffamily\fontsize{10.000000}{12.000000}\selectfont 20 bins}%
\end{pgfscope}%
\begin{pgfscope}%
\pgfsetrectcap%
\pgfsetroundjoin%
\pgfsetlinewidth{1.505625pt}%
\definecolor{currentstroke}{rgb}{0.172549,0.627451,0.172549}%
\pgfsetstrokecolor{currentstroke}%
\pgfsetdash{}{0pt}%
\pgfpathmoveto{\pgfqpoint{1.629290in}{1.181820in}}%
\pgfpathlineto{\pgfqpoint{1.907068in}{1.181820in}}%
\pgfusepath{stroke}%
\end{pgfscope}%
\begin{pgfscope}%
\definecolor{textcolor}{rgb}{0.000000,0.000000,0.000000}%
\pgfsetstrokecolor{textcolor}%
\pgfsetfillcolor{textcolor}%
\pgftext[x=2.018179in,y=1.133209in,left,base]{\color{textcolor}\sffamily\fontsize{10.000000}{12.000000}\selectfont 50 bins}%
\end{pgfscope}%
\begin{pgfscope}%
\pgfsetrectcap%
\pgfsetroundjoin%
\pgfsetlinewidth{1.505625pt}%
\definecolor{currentstroke}{rgb}{0.839216,0.152941,0.156863}%
\pgfsetstrokecolor{currentstroke}%
\pgfsetdash{}{0pt}%
\pgfpathmoveto{\pgfqpoint{1.629290in}{0.977962in}}%
\pgfpathlineto{\pgfqpoint{1.907068in}{0.977962in}}%
\pgfusepath{stroke}%
\end{pgfscope}%
\begin{pgfscope}%
\definecolor{textcolor}{rgb}{0.000000,0.000000,0.000000}%
\pgfsetstrokecolor{textcolor}%
\pgfsetfillcolor{textcolor}%
\pgftext[x=2.018179in,y=0.929351in,left,base]{\color{textcolor}\sffamily\fontsize{10.000000}{12.000000}\selectfont 100 bins}%
\end{pgfscope}%
\begin{pgfscope}%
\pgfsetrectcap%
\pgfsetroundjoin%
\pgfsetlinewidth{1.505625pt}%
\definecolor{currentstroke}{rgb}{0.580392,0.403922,0.741176}%
\pgfsetstrokecolor{currentstroke}%
\pgfsetdash{}{0pt}%
\pgfpathmoveto{\pgfqpoint{1.629290in}{0.774105in}}%
\pgfpathlineto{\pgfqpoint{1.907068in}{0.774105in}}%
\pgfusepath{stroke}%
\end{pgfscope}%
\begin{pgfscope}%
\definecolor{textcolor}{rgb}{0.000000,0.000000,0.000000}%
\pgfsetstrokecolor{textcolor}%
\pgfsetfillcolor{textcolor}%
\pgftext[x=2.018179in,y=0.725494in,left,base]{\color{textcolor}\sffamily\fontsize{10.000000}{12.000000}\selectfont 256 bins}%
\end{pgfscope}%
\end{pgfpicture}%
\makeatother%
\endgroup%
}
  \vspace{-30pt}
\end{wrapfigure}
Verändert man die Anzahl der Bins der MI Metrik, das heißt man verfeinert das
Histogram zur Bestimmung der Wahrscheinlichkeit für Gleichung \ref{eq:mi_h}.
Wie man an Abbildung \ref{fig:ct2mrt_bins} erkennt, verschiebt sich vor allem
das Minimum nahe des optimalen Wertes bei \num{0.0}, außerdem erkennt man, dass
die Kurven mit hoher Anzahl an Bins etwas mehr Konturen in der Nähe des Wertes
\num{0.75} aufweisen. Es ist anzumerken, dass bei einer Anzahl von \num{5} Bins
die Metrik keine Ergebnisse mehr geliefert hat.\\
Verwendet man mehr Bins, so kann der Algorithmus genauer zwischen den
Intensitätswerten unterscheiden, so dass die Metrik besser minimiert werden
kann. Je höher die Anzahl der Bins, desto näher kann man an das Optimum
gelangen, daher verbessert sich das Ergebniss bei hoher Binzahl nur noch
geringfügig.

\subsection{Rigide 3D-Bildregistrierung von medizinischen Bilddaten}
\begin{wrapfigure}{r}{0.5\linewidth}
  \vspace{-6pt}
  \vspace{-10pt}
  \caption{Die Berechnungsdauer steigt quasi linear mit der Samplingrate des Algorithmus}
  \label{fig:A3}
  \vspace{-10pt}
  \resizebox{\linewidth}{!}{%% Creator: Matplotlib, PGF backend
%%
%% To include the figure in your LaTeX document, write
%%   \input{<filename>.pgf}
%%
%% Make sure the required packages are loaded in your preamble
%%   \usepackage{pgf}
%%
%% Figures using additional raster images can only be included by \input if
%% they are in the same directory as the main LaTeX file. For loading figures
%% from other directories you can use the `import` package
%%   \usepackage{import}
%% and then include the figures with
%%   \import{<path to file>}{<filename>.pgf}
%%
%% Matplotlib used the following preamble
%%   \usepackage{fontspec}
%%   \setmainfont{DejaVuSerif.ttf}[Path=/usr/share/matplotlib/mpl-data/fonts/ttf/]
%%   \setsansfont{DejaVuSans.ttf}[Path=/usr/share/matplotlib/mpl-data/fonts/ttf/]
%%   \setmonofont{DejaVuSansMono.ttf}[Path=/usr/share/matplotlib/mpl-data/fonts/ttf/]
%%
\begingroup%
\makeatletter%
\begin{pgfpicture}%
\pgfpathrectangle{\pgfpointorigin}{\pgfqpoint{3.000000in}{3.000000in}}%
\pgfusepath{use as bounding box, clip}%
\begin{pgfscope}%
\pgfsetbuttcap%
\pgfsetmiterjoin%
\definecolor{currentfill}{rgb}{1.000000,1.000000,1.000000}%
\pgfsetfillcolor{currentfill}%
\pgfsetlinewidth{0.000000pt}%
\definecolor{currentstroke}{rgb}{1.000000,1.000000,1.000000}%
\pgfsetstrokecolor{currentstroke}%
\pgfsetdash{}{0pt}%
\pgfpathmoveto{\pgfqpoint{0.000000in}{0.000000in}}%
\pgfpathlineto{\pgfqpoint{3.000000in}{0.000000in}}%
\pgfpathlineto{\pgfqpoint{3.000000in}{3.000000in}}%
\pgfpathlineto{\pgfqpoint{0.000000in}{3.000000in}}%
\pgfpathclose%
\pgfusepath{fill}%
\end{pgfscope}%
\begin{pgfscope}%
\pgfsetbuttcap%
\pgfsetmiterjoin%
\definecolor{currentfill}{rgb}{1.000000,1.000000,1.000000}%
\pgfsetfillcolor{currentfill}%
\pgfsetlinewidth{0.000000pt}%
\definecolor{currentstroke}{rgb}{0.000000,0.000000,0.000000}%
\pgfsetstrokecolor{currentstroke}%
\pgfsetstrokeopacity{0.000000}%
\pgfsetdash{}{0pt}%
\pgfpathmoveto{\pgfqpoint{0.613922in}{0.571604in}}%
\pgfpathlineto{\pgfqpoint{2.806092in}{0.571604in}}%
\pgfpathlineto{\pgfqpoint{2.806092in}{2.850000in}}%
\pgfpathlineto{\pgfqpoint{0.613922in}{2.850000in}}%
\pgfpathclose%
\pgfusepath{fill}%
\end{pgfscope}%
\begin{pgfscope}%
\pgfpathrectangle{\pgfqpoint{0.613922in}{0.571604in}}{\pgfqpoint{2.192170in}{2.278396in}}%
\pgfusepath{clip}%
\pgfsetrectcap%
\pgfsetroundjoin%
\pgfsetlinewidth{0.803000pt}%
\definecolor{currentstroke}{rgb}{0.690196,0.690196,0.690196}%
\pgfsetstrokecolor{currentstroke}%
\pgfsetdash{}{0pt}%
\pgfpathmoveto{\pgfqpoint{0.613922in}{0.571604in}}%
\pgfpathlineto{\pgfqpoint{0.613922in}{2.850000in}}%
\pgfusepath{stroke}%
\end{pgfscope}%
\begin{pgfscope}%
\pgfsetbuttcap%
\pgfsetroundjoin%
\definecolor{currentfill}{rgb}{0.000000,0.000000,0.000000}%
\pgfsetfillcolor{currentfill}%
\pgfsetlinewidth{0.803000pt}%
\definecolor{currentstroke}{rgb}{0.000000,0.000000,0.000000}%
\pgfsetstrokecolor{currentstroke}%
\pgfsetdash{}{0pt}%
\pgfsys@defobject{currentmarker}{\pgfqpoint{0.000000in}{-0.048611in}}{\pgfqpoint{0.000000in}{0.000000in}}{%
\pgfpathmoveto{\pgfqpoint{0.000000in}{0.000000in}}%
\pgfpathlineto{\pgfqpoint{0.000000in}{-0.048611in}}%
\pgfusepath{stroke,fill}%
}%
\begin{pgfscope}%
\pgfsys@transformshift{0.613922in}{0.571604in}%
\pgfsys@useobject{currentmarker}{}%
\end{pgfscope}%
\end{pgfscope}%
\begin{pgfscope}%
\definecolor{textcolor}{rgb}{0.000000,0.000000,0.000000}%
\pgfsetstrokecolor{textcolor}%
\pgfsetfillcolor{textcolor}%
\pgftext[x=0.613922in,y=0.474382in,,top]{\color{textcolor}\sffamily\fontsize{10.000000}{12.000000}\selectfont 0.00}%
\end{pgfscope}%
\begin{pgfscope}%
\pgfpathrectangle{\pgfqpoint{0.613922in}{0.571604in}}{\pgfqpoint{2.192170in}{2.278396in}}%
\pgfusepath{clip}%
\pgfsetrectcap%
\pgfsetroundjoin%
\pgfsetlinewidth{0.803000pt}%
\definecolor{currentstroke}{rgb}{0.690196,0.690196,0.690196}%
\pgfsetstrokecolor{currentstroke}%
\pgfsetdash{}{0pt}%
\pgfpathmoveto{\pgfqpoint{1.135867in}{0.571604in}}%
\pgfpathlineto{\pgfqpoint{1.135867in}{2.850000in}}%
\pgfusepath{stroke}%
\end{pgfscope}%
\begin{pgfscope}%
\pgfsetbuttcap%
\pgfsetroundjoin%
\definecolor{currentfill}{rgb}{0.000000,0.000000,0.000000}%
\pgfsetfillcolor{currentfill}%
\pgfsetlinewidth{0.803000pt}%
\definecolor{currentstroke}{rgb}{0.000000,0.000000,0.000000}%
\pgfsetstrokecolor{currentstroke}%
\pgfsetdash{}{0pt}%
\pgfsys@defobject{currentmarker}{\pgfqpoint{0.000000in}{-0.048611in}}{\pgfqpoint{0.000000in}{0.000000in}}{%
\pgfpathmoveto{\pgfqpoint{0.000000in}{0.000000in}}%
\pgfpathlineto{\pgfqpoint{0.000000in}{-0.048611in}}%
\pgfusepath{stroke,fill}%
}%
\begin{pgfscope}%
\pgfsys@transformshift{1.135867in}{0.571604in}%
\pgfsys@useobject{currentmarker}{}%
\end{pgfscope}%
\end{pgfscope}%
\begin{pgfscope}%
\definecolor{textcolor}{rgb}{0.000000,0.000000,0.000000}%
\pgfsetstrokecolor{textcolor}%
\pgfsetfillcolor{textcolor}%
\pgftext[x=1.135867in,y=0.474382in,,top]{\color{textcolor}\sffamily\fontsize{10.000000}{12.000000}\selectfont 0.25}%
\end{pgfscope}%
\begin{pgfscope}%
\pgfpathrectangle{\pgfqpoint{0.613922in}{0.571604in}}{\pgfqpoint{2.192170in}{2.278396in}}%
\pgfusepath{clip}%
\pgfsetrectcap%
\pgfsetroundjoin%
\pgfsetlinewidth{0.803000pt}%
\definecolor{currentstroke}{rgb}{0.690196,0.690196,0.690196}%
\pgfsetstrokecolor{currentstroke}%
\pgfsetdash{}{0pt}%
\pgfpathmoveto{\pgfqpoint{1.657812in}{0.571604in}}%
\pgfpathlineto{\pgfqpoint{1.657812in}{2.850000in}}%
\pgfusepath{stroke}%
\end{pgfscope}%
\begin{pgfscope}%
\pgfsetbuttcap%
\pgfsetroundjoin%
\definecolor{currentfill}{rgb}{0.000000,0.000000,0.000000}%
\pgfsetfillcolor{currentfill}%
\pgfsetlinewidth{0.803000pt}%
\definecolor{currentstroke}{rgb}{0.000000,0.000000,0.000000}%
\pgfsetstrokecolor{currentstroke}%
\pgfsetdash{}{0pt}%
\pgfsys@defobject{currentmarker}{\pgfqpoint{0.000000in}{-0.048611in}}{\pgfqpoint{0.000000in}{0.000000in}}{%
\pgfpathmoveto{\pgfqpoint{0.000000in}{0.000000in}}%
\pgfpathlineto{\pgfqpoint{0.000000in}{-0.048611in}}%
\pgfusepath{stroke,fill}%
}%
\begin{pgfscope}%
\pgfsys@transformshift{1.657812in}{0.571604in}%
\pgfsys@useobject{currentmarker}{}%
\end{pgfscope}%
\end{pgfscope}%
\begin{pgfscope}%
\definecolor{textcolor}{rgb}{0.000000,0.000000,0.000000}%
\pgfsetstrokecolor{textcolor}%
\pgfsetfillcolor{textcolor}%
\pgftext[x=1.657812in,y=0.474382in,,top]{\color{textcolor}\sffamily\fontsize{10.000000}{12.000000}\selectfont 0.50}%
\end{pgfscope}%
\begin{pgfscope}%
\pgfpathrectangle{\pgfqpoint{0.613922in}{0.571604in}}{\pgfqpoint{2.192170in}{2.278396in}}%
\pgfusepath{clip}%
\pgfsetrectcap%
\pgfsetroundjoin%
\pgfsetlinewidth{0.803000pt}%
\definecolor{currentstroke}{rgb}{0.690196,0.690196,0.690196}%
\pgfsetstrokecolor{currentstroke}%
\pgfsetdash{}{0pt}%
\pgfpathmoveto{\pgfqpoint{2.179757in}{0.571604in}}%
\pgfpathlineto{\pgfqpoint{2.179757in}{2.850000in}}%
\pgfusepath{stroke}%
\end{pgfscope}%
\begin{pgfscope}%
\pgfsetbuttcap%
\pgfsetroundjoin%
\definecolor{currentfill}{rgb}{0.000000,0.000000,0.000000}%
\pgfsetfillcolor{currentfill}%
\pgfsetlinewidth{0.803000pt}%
\definecolor{currentstroke}{rgb}{0.000000,0.000000,0.000000}%
\pgfsetstrokecolor{currentstroke}%
\pgfsetdash{}{0pt}%
\pgfsys@defobject{currentmarker}{\pgfqpoint{0.000000in}{-0.048611in}}{\pgfqpoint{0.000000in}{0.000000in}}{%
\pgfpathmoveto{\pgfqpoint{0.000000in}{0.000000in}}%
\pgfpathlineto{\pgfqpoint{0.000000in}{-0.048611in}}%
\pgfusepath{stroke,fill}%
}%
\begin{pgfscope}%
\pgfsys@transformshift{2.179757in}{0.571604in}%
\pgfsys@useobject{currentmarker}{}%
\end{pgfscope}%
\end{pgfscope}%
\begin{pgfscope}%
\definecolor{textcolor}{rgb}{0.000000,0.000000,0.000000}%
\pgfsetstrokecolor{textcolor}%
\pgfsetfillcolor{textcolor}%
\pgftext[x=2.179757in,y=0.474382in,,top]{\color{textcolor}\sffamily\fontsize{10.000000}{12.000000}\selectfont 0.75}%
\end{pgfscope}%
\begin{pgfscope}%
\pgfpathrectangle{\pgfqpoint{0.613922in}{0.571604in}}{\pgfqpoint{2.192170in}{2.278396in}}%
\pgfusepath{clip}%
\pgfsetrectcap%
\pgfsetroundjoin%
\pgfsetlinewidth{0.803000pt}%
\definecolor{currentstroke}{rgb}{0.690196,0.690196,0.690196}%
\pgfsetstrokecolor{currentstroke}%
\pgfsetdash{}{0pt}%
\pgfpathmoveto{\pgfqpoint{2.701703in}{0.571604in}}%
\pgfpathlineto{\pgfqpoint{2.701703in}{2.850000in}}%
\pgfusepath{stroke}%
\end{pgfscope}%
\begin{pgfscope}%
\pgfsetbuttcap%
\pgfsetroundjoin%
\definecolor{currentfill}{rgb}{0.000000,0.000000,0.000000}%
\pgfsetfillcolor{currentfill}%
\pgfsetlinewidth{0.803000pt}%
\definecolor{currentstroke}{rgb}{0.000000,0.000000,0.000000}%
\pgfsetstrokecolor{currentstroke}%
\pgfsetdash{}{0pt}%
\pgfsys@defobject{currentmarker}{\pgfqpoint{0.000000in}{-0.048611in}}{\pgfqpoint{0.000000in}{0.000000in}}{%
\pgfpathmoveto{\pgfqpoint{0.000000in}{0.000000in}}%
\pgfpathlineto{\pgfqpoint{0.000000in}{-0.048611in}}%
\pgfusepath{stroke,fill}%
}%
\begin{pgfscope}%
\pgfsys@transformshift{2.701703in}{0.571604in}%
\pgfsys@useobject{currentmarker}{}%
\end{pgfscope}%
\end{pgfscope}%
\begin{pgfscope}%
\definecolor{textcolor}{rgb}{0.000000,0.000000,0.000000}%
\pgfsetstrokecolor{textcolor}%
\pgfsetfillcolor{textcolor}%
\pgftext[x=2.701703in,y=0.474382in,,top]{\color{textcolor}\sffamily\fontsize{10.000000}{12.000000}\selectfont 1.00}%
\end{pgfscope}%
\begin{pgfscope}%
\definecolor{textcolor}{rgb}{0.000000,0.000000,0.000000}%
\pgfsetstrokecolor{textcolor}%
\pgfsetfillcolor{textcolor}%
\pgftext[x=1.710007in,y=0.284413in,,top]{\color{textcolor}\sffamily\fontsize{10.000000}{12.000000}\selectfont Sampling rate}%
\end{pgfscope}%
\begin{pgfscope}%
\pgfpathrectangle{\pgfqpoint{0.613922in}{0.571604in}}{\pgfqpoint{2.192170in}{2.278396in}}%
\pgfusepath{clip}%
\pgfsetrectcap%
\pgfsetroundjoin%
\pgfsetlinewidth{0.803000pt}%
\definecolor{currentstroke}{rgb}{0.690196,0.690196,0.690196}%
\pgfsetstrokecolor{currentstroke}%
\pgfsetdash{}{0pt}%
\pgfpathmoveto{\pgfqpoint{0.613922in}{0.571604in}}%
\pgfpathlineto{\pgfqpoint{2.806092in}{0.571604in}}%
\pgfusepath{stroke}%
\end{pgfscope}%
\begin{pgfscope}%
\pgfsetbuttcap%
\pgfsetroundjoin%
\definecolor{currentfill}{rgb}{0.000000,0.000000,0.000000}%
\pgfsetfillcolor{currentfill}%
\pgfsetlinewidth{0.803000pt}%
\definecolor{currentstroke}{rgb}{0.000000,0.000000,0.000000}%
\pgfsetstrokecolor{currentstroke}%
\pgfsetdash{}{0pt}%
\pgfsys@defobject{currentmarker}{\pgfqpoint{-0.048611in}{0.000000in}}{\pgfqpoint{0.000000in}{0.000000in}}{%
\pgfpathmoveto{\pgfqpoint{0.000000in}{0.000000in}}%
\pgfpathlineto{\pgfqpoint{-0.048611in}{0.000000in}}%
\pgfusepath{stroke,fill}%
}%
\begin{pgfscope}%
\pgfsys@transformshift{0.613922in}{0.571604in}%
\pgfsys@useobject{currentmarker}{}%
\end{pgfscope}%
\end{pgfscope}%
\begin{pgfscope}%
\definecolor{textcolor}{rgb}{0.000000,0.000000,0.000000}%
\pgfsetstrokecolor{textcolor}%
\pgfsetfillcolor{textcolor}%
\pgftext[x=0.428334in,y=0.518842in,left,base]{\color{textcolor}\sffamily\fontsize{10.000000}{12.000000}\selectfont 0}%
\end{pgfscope}%
\begin{pgfscope}%
\pgfpathrectangle{\pgfqpoint{0.613922in}{0.571604in}}{\pgfqpoint{2.192170in}{2.278396in}}%
\pgfusepath{clip}%
\pgfsetrectcap%
\pgfsetroundjoin%
\pgfsetlinewidth{0.803000pt}%
\definecolor{currentstroke}{rgb}{0.690196,0.690196,0.690196}%
\pgfsetstrokecolor{currentstroke}%
\pgfsetdash{}{0pt}%
\pgfpathmoveto{\pgfqpoint{0.613922in}{1.005584in}}%
\pgfpathlineto{\pgfqpoint{2.806092in}{1.005584in}}%
\pgfusepath{stroke}%
\end{pgfscope}%
\begin{pgfscope}%
\pgfsetbuttcap%
\pgfsetroundjoin%
\definecolor{currentfill}{rgb}{0.000000,0.000000,0.000000}%
\pgfsetfillcolor{currentfill}%
\pgfsetlinewidth{0.803000pt}%
\definecolor{currentstroke}{rgb}{0.000000,0.000000,0.000000}%
\pgfsetstrokecolor{currentstroke}%
\pgfsetdash{}{0pt}%
\pgfsys@defobject{currentmarker}{\pgfqpoint{-0.048611in}{0.000000in}}{\pgfqpoint{0.000000in}{0.000000in}}{%
\pgfpathmoveto{\pgfqpoint{0.000000in}{0.000000in}}%
\pgfpathlineto{\pgfqpoint{-0.048611in}{0.000000in}}%
\pgfusepath{stroke,fill}%
}%
\begin{pgfscope}%
\pgfsys@transformshift{0.613922in}{1.005584in}%
\pgfsys@useobject{currentmarker}{}%
\end{pgfscope}%
\end{pgfscope}%
\begin{pgfscope}%
\definecolor{textcolor}{rgb}{0.000000,0.000000,0.000000}%
\pgfsetstrokecolor{textcolor}%
\pgfsetfillcolor{textcolor}%
\pgftext[x=0.339969in,y=0.952823in,left,base]{\color{textcolor}\sffamily\fontsize{10.000000}{12.000000}\selectfont 10}%
\end{pgfscope}%
\begin{pgfscope}%
\pgfpathrectangle{\pgfqpoint{0.613922in}{0.571604in}}{\pgfqpoint{2.192170in}{2.278396in}}%
\pgfusepath{clip}%
\pgfsetrectcap%
\pgfsetroundjoin%
\pgfsetlinewidth{0.803000pt}%
\definecolor{currentstroke}{rgb}{0.690196,0.690196,0.690196}%
\pgfsetstrokecolor{currentstroke}%
\pgfsetdash{}{0pt}%
\pgfpathmoveto{\pgfqpoint{0.613922in}{1.439564in}}%
\pgfpathlineto{\pgfqpoint{2.806092in}{1.439564in}}%
\pgfusepath{stroke}%
\end{pgfscope}%
\begin{pgfscope}%
\pgfsetbuttcap%
\pgfsetroundjoin%
\definecolor{currentfill}{rgb}{0.000000,0.000000,0.000000}%
\pgfsetfillcolor{currentfill}%
\pgfsetlinewidth{0.803000pt}%
\definecolor{currentstroke}{rgb}{0.000000,0.000000,0.000000}%
\pgfsetstrokecolor{currentstroke}%
\pgfsetdash{}{0pt}%
\pgfsys@defobject{currentmarker}{\pgfqpoint{-0.048611in}{0.000000in}}{\pgfqpoint{0.000000in}{0.000000in}}{%
\pgfpathmoveto{\pgfqpoint{0.000000in}{0.000000in}}%
\pgfpathlineto{\pgfqpoint{-0.048611in}{0.000000in}}%
\pgfusepath{stroke,fill}%
}%
\begin{pgfscope}%
\pgfsys@transformshift{0.613922in}{1.439564in}%
\pgfsys@useobject{currentmarker}{}%
\end{pgfscope}%
\end{pgfscope}%
\begin{pgfscope}%
\definecolor{textcolor}{rgb}{0.000000,0.000000,0.000000}%
\pgfsetstrokecolor{textcolor}%
\pgfsetfillcolor{textcolor}%
\pgftext[x=0.339969in,y=1.386803in,left,base]{\color{textcolor}\sffamily\fontsize{10.000000}{12.000000}\selectfont 20}%
\end{pgfscope}%
\begin{pgfscope}%
\pgfpathrectangle{\pgfqpoint{0.613922in}{0.571604in}}{\pgfqpoint{2.192170in}{2.278396in}}%
\pgfusepath{clip}%
\pgfsetrectcap%
\pgfsetroundjoin%
\pgfsetlinewidth{0.803000pt}%
\definecolor{currentstroke}{rgb}{0.690196,0.690196,0.690196}%
\pgfsetstrokecolor{currentstroke}%
\pgfsetdash{}{0pt}%
\pgfpathmoveto{\pgfqpoint{0.613922in}{1.873545in}}%
\pgfpathlineto{\pgfqpoint{2.806092in}{1.873545in}}%
\pgfusepath{stroke}%
\end{pgfscope}%
\begin{pgfscope}%
\pgfsetbuttcap%
\pgfsetroundjoin%
\definecolor{currentfill}{rgb}{0.000000,0.000000,0.000000}%
\pgfsetfillcolor{currentfill}%
\pgfsetlinewidth{0.803000pt}%
\definecolor{currentstroke}{rgb}{0.000000,0.000000,0.000000}%
\pgfsetstrokecolor{currentstroke}%
\pgfsetdash{}{0pt}%
\pgfsys@defobject{currentmarker}{\pgfqpoint{-0.048611in}{0.000000in}}{\pgfqpoint{0.000000in}{0.000000in}}{%
\pgfpathmoveto{\pgfqpoint{0.000000in}{0.000000in}}%
\pgfpathlineto{\pgfqpoint{-0.048611in}{0.000000in}}%
\pgfusepath{stroke,fill}%
}%
\begin{pgfscope}%
\pgfsys@transformshift{0.613922in}{1.873545in}%
\pgfsys@useobject{currentmarker}{}%
\end{pgfscope}%
\end{pgfscope}%
\begin{pgfscope}%
\definecolor{textcolor}{rgb}{0.000000,0.000000,0.000000}%
\pgfsetstrokecolor{textcolor}%
\pgfsetfillcolor{textcolor}%
\pgftext[x=0.339969in,y=1.820783in,left,base]{\color{textcolor}\sffamily\fontsize{10.000000}{12.000000}\selectfont 30}%
\end{pgfscope}%
\begin{pgfscope}%
\pgfpathrectangle{\pgfqpoint{0.613922in}{0.571604in}}{\pgfqpoint{2.192170in}{2.278396in}}%
\pgfusepath{clip}%
\pgfsetrectcap%
\pgfsetroundjoin%
\pgfsetlinewidth{0.803000pt}%
\definecolor{currentstroke}{rgb}{0.690196,0.690196,0.690196}%
\pgfsetstrokecolor{currentstroke}%
\pgfsetdash{}{0pt}%
\pgfpathmoveto{\pgfqpoint{0.613922in}{2.307525in}}%
\pgfpathlineto{\pgfqpoint{2.806092in}{2.307525in}}%
\pgfusepath{stroke}%
\end{pgfscope}%
\begin{pgfscope}%
\pgfsetbuttcap%
\pgfsetroundjoin%
\definecolor{currentfill}{rgb}{0.000000,0.000000,0.000000}%
\pgfsetfillcolor{currentfill}%
\pgfsetlinewidth{0.803000pt}%
\definecolor{currentstroke}{rgb}{0.000000,0.000000,0.000000}%
\pgfsetstrokecolor{currentstroke}%
\pgfsetdash{}{0pt}%
\pgfsys@defobject{currentmarker}{\pgfqpoint{-0.048611in}{0.000000in}}{\pgfqpoint{0.000000in}{0.000000in}}{%
\pgfpathmoveto{\pgfqpoint{0.000000in}{0.000000in}}%
\pgfpathlineto{\pgfqpoint{-0.048611in}{0.000000in}}%
\pgfusepath{stroke,fill}%
}%
\begin{pgfscope}%
\pgfsys@transformshift{0.613922in}{2.307525in}%
\pgfsys@useobject{currentmarker}{}%
\end{pgfscope}%
\end{pgfscope}%
\begin{pgfscope}%
\definecolor{textcolor}{rgb}{0.000000,0.000000,0.000000}%
\pgfsetstrokecolor{textcolor}%
\pgfsetfillcolor{textcolor}%
\pgftext[x=0.339969in,y=2.254763in,left,base]{\color{textcolor}\sffamily\fontsize{10.000000}{12.000000}\selectfont 40}%
\end{pgfscope}%
\begin{pgfscope}%
\pgfpathrectangle{\pgfqpoint{0.613922in}{0.571604in}}{\pgfqpoint{2.192170in}{2.278396in}}%
\pgfusepath{clip}%
\pgfsetrectcap%
\pgfsetroundjoin%
\pgfsetlinewidth{0.803000pt}%
\definecolor{currentstroke}{rgb}{0.690196,0.690196,0.690196}%
\pgfsetstrokecolor{currentstroke}%
\pgfsetdash{}{0pt}%
\pgfpathmoveto{\pgfqpoint{0.613922in}{2.741505in}}%
\pgfpathlineto{\pgfqpoint{2.806092in}{2.741505in}}%
\pgfusepath{stroke}%
\end{pgfscope}%
\begin{pgfscope}%
\pgfsetbuttcap%
\pgfsetroundjoin%
\definecolor{currentfill}{rgb}{0.000000,0.000000,0.000000}%
\pgfsetfillcolor{currentfill}%
\pgfsetlinewidth{0.803000pt}%
\definecolor{currentstroke}{rgb}{0.000000,0.000000,0.000000}%
\pgfsetstrokecolor{currentstroke}%
\pgfsetdash{}{0pt}%
\pgfsys@defobject{currentmarker}{\pgfqpoint{-0.048611in}{0.000000in}}{\pgfqpoint{0.000000in}{0.000000in}}{%
\pgfpathmoveto{\pgfqpoint{0.000000in}{0.000000in}}%
\pgfpathlineto{\pgfqpoint{-0.048611in}{0.000000in}}%
\pgfusepath{stroke,fill}%
}%
\begin{pgfscope}%
\pgfsys@transformshift{0.613922in}{2.741505in}%
\pgfsys@useobject{currentmarker}{}%
\end{pgfscope}%
\end{pgfscope}%
\begin{pgfscope}%
\definecolor{textcolor}{rgb}{0.000000,0.000000,0.000000}%
\pgfsetstrokecolor{textcolor}%
\pgfsetfillcolor{textcolor}%
\pgftext[x=0.339969in,y=2.688743in,left,base]{\color{textcolor}\sffamily\fontsize{10.000000}{12.000000}\selectfont 50}%
\end{pgfscope}%
\begin{pgfscope}%
\definecolor{textcolor}{rgb}{0.000000,0.000000,0.000000}%
\pgfsetstrokecolor{textcolor}%
\pgfsetfillcolor{textcolor}%
\pgftext[x=0.284413in,y=1.710802in,,bottom,rotate=90.000000]{\color{textcolor}\sffamily\fontsize{10.000000}{12.000000}\selectfont Computation time [s]}%
\end{pgfscope}%
\begin{pgfscope}%
\pgfpathrectangle{\pgfqpoint{0.613922in}{0.571604in}}{\pgfqpoint{2.192170in}{2.278396in}}%
\pgfusepath{clip}%
\pgfsetbuttcap%
\pgfsetroundjoin%
\pgfsetlinewidth{1.505625pt}%
\definecolor{currentstroke}{rgb}{0.121569,0.466667,0.705882}%
\pgfsetstrokecolor{currentstroke}%
\pgfsetdash{}{0pt}%
\pgfpathmoveto{\pgfqpoint{2.701703in}{2.554471in}}%
\pgfpathlineto{\pgfqpoint{2.701703in}{2.723122in}}%
\pgfusepath{stroke}%
\end{pgfscope}%
\begin{pgfscope}%
\pgfpathrectangle{\pgfqpoint{0.613922in}{0.571604in}}{\pgfqpoint{2.192170in}{2.278396in}}%
\pgfusepath{clip}%
\pgfsetbuttcap%
\pgfsetroundjoin%
\pgfsetlinewidth{1.505625pt}%
\definecolor{currentstroke}{rgb}{0.121569,0.466667,0.705882}%
\pgfsetstrokecolor{currentstroke}%
\pgfsetdash{}{0pt}%
\pgfpathmoveto{\pgfqpoint{2.284147in}{1.684026in}}%
\pgfpathlineto{\pgfqpoint{2.284147in}{1.979161in}}%
\pgfusepath{stroke}%
\end{pgfscope}%
\begin{pgfscope}%
\pgfpathrectangle{\pgfqpoint{0.613922in}{0.571604in}}{\pgfqpoint{2.192170in}{2.278396in}}%
\pgfusepath{clip}%
\pgfsetbuttcap%
\pgfsetroundjoin%
\pgfsetlinewidth{1.505625pt}%
\definecolor{currentstroke}{rgb}{0.121569,0.466667,0.705882}%
\pgfsetstrokecolor{currentstroke}%
\pgfsetdash{}{0pt}%
\pgfpathmoveto{\pgfqpoint{1.866590in}{1.687748in}}%
\pgfpathlineto{\pgfqpoint{1.866590in}{1.784487in}}%
\pgfusepath{stroke}%
\end{pgfscope}%
\begin{pgfscope}%
\pgfpathrectangle{\pgfqpoint{0.613922in}{0.571604in}}{\pgfqpoint{2.192170in}{2.278396in}}%
\pgfusepath{clip}%
\pgfsetbuttcap%
\pgfsetroundjoin%
\pgfsetlinewidth{1.505625pt}%
\definecolor{currentstroke}{rgb}{0.121569,0.466667,0.705882}%
\pgfsetstrokecolor{currentstroke}%
\pgfsetdash{}{0pt}%
\pgfpathmoveto{\pgfqpoint{1.449034in}{1.385263in}}%
\pgfpathlineto{\pgfqpoint{1.449034in}{1.453361in}}%
\pgfusepath{stroke}%
\end{pgfscope}%
\begin{pgfscope}%
\pgfpathrectangle{\pgfqpoint{0.613922in}{0.571604in}}{\pgfqpoint{2.192170in}{2.278396in}}%
\pgfusepath{clip}%
\pgfsetbuttcap%
\pgfsetroundjoin%
\pgfsetlinewidth{1.505625pt}%
\definecolor{currentstroke}{rgb}{0.121569,0.466667,0.705882}%
\pgfsetstrokecolor{currentstroke}%
\pgfsetdash{}{0pt}%
\pgfpathmoveto{\pgfqpoint{1.031478in}{1.006703in}}%
\pgfpathlineto{\pgfqpoint{1.031478in}{1.166485in}}%
\pgfusepath{stroke}%
\end{pgfscope}%
\begin{pgfscope}%
\pgfpathrectangle{\pgfqpoint{0.613922in}{0.571604in}}{\pgfqpoint{2.192170in}{2.278396in}}%
\pgfusepath{clip}%
\pgfsetrectcap%
\pgfsetroundjoin%
\pgfsetlinewidth{1.505625pt}%
\definecolor{currentstroke}{rgb}{1.000000,0.498039,0.054902}%
\pgfsetstrokecolor{currentstroke}%
\pgfsetdash{}{0pt}%
\pgfpathmoveto{\pgfqpoint{0.613922in}{0.571604in}}%
\pgfpathlineto{\pgfqpoint{0.658660in}{0.612669in}}%
\pgfpathlineto{\pgfqpoint{0.703398in}{0.653733in}}%
\pgfpathlineto{\pgfqpoint{0.748136in}{0.694798in}}%
\pgfpathlineto{\pgfqpoint{0.792874in}{0.735863in}}%
\pgfpathlineto{\pgfqpoint{0.837612in}{0.776927in}}%
\pgfpathlineto{\pgfqpoint{0.882351in}{0.817992in}}%
\pgfpathlineto{\pgfqpoint{0.927089in}{0.859057in}}%
\pgfpathlineto{\pgfqpoint{0.971827in}{0.900121in}}%
\pgfpathlineto{\pgfqpoint{1.016565in}{0.941186in}}%
\pgfpathlineto{\pgfqpoint{1.061303in}{0.982251in}}%
\pgfpathlineto{\pgfqpoint{1.106041in}{1.023315in}}%
\pgfpathlineto{\pgfqpoint{1.150780in}{1.064380in}}%
\pgfpathlineto{\pgfqpoint{1.195518in}{1.105445in}}%
\pgfpathlineto{\pgfqpoint{1.240256in}{1.146509in}}%
\pgfpathlineto{\pgfqpoint{1.284994in}{1.187574in}}%
\pgfpathlineto{\pgfqpoint{1.329732in}{1.228639in}}%
\pgfpathlineto{\pgfqpoint{1.374470in}{1.269703in}}%
\pgfpathlineto{\pgfqpoint{1.419209in}{1.310768in}}%
\pgfpathlineto{\pgfqpoint{1.463947in}{1.351833in}}%
\pgfpathlineto{\pgfqpoint{1.508685in}{1.392897in}}%
\pgfpathlineto{\pgfqpoint{1.553423in}{1.433962in}}%
\pgfpathlineto{\pgfqpoint{1.598161in}{1.475027in}}%
\pgfpathlineto{\pgfqpoint{1.642899in}{1.516091in}}%
\pgfpathlineto{\pgfqpoint{1.687638in}{1.557156in}}%
\pgfpathlineto{\pgfqpoint{1.732376in}{1.598221in}}%
\pgfpathlineto{\pgfqpoint{1.777114in}{1.639285in}}%
\pgfpathlineto{\pgfqpoint{1.821852in}{1.680350in}}%
\pgfpathlineto{\pgfqpoint{1.866590in}{1.721415in}}%
\pgfpathlineto{\pgfqpoint{1.911328in}{1.762479in}}%
\pgfpathlineto{\pgfqpoint{1.956067in}{1.803544in}}%
\pgfpathlineto{\pgfqpoint{2.000805in}{1.844609in}}%
\pgfpathlineto{\pgfqpoint{2.045543in}{1.885673in}}%
\pgfpathlineto{\pgfqpoint{2.090281in}{1.926738in}}%
\pgfpathlineto{\pgfqpoint{2.135019in}{1.967803in}}%
\pgfpathlineto{\pgfqpoint{2.179757in}{2.008867in}}%
\pgfpathlineto{\pgfqpoint{2.224496in}{2.049932in}}%
\pgfpathlineto{\pgfqpoint{2.269234in}{2.090997in}}%
\pgfpathlineto{\pgfqpoint{2.313972in}{2.132061in}}%
\pgfpathlineto{\pgfqpoint{2.358710in}{2.173126in}}%
\pgfpathlineto{\pgfqpoint{2.403448in}{2.214191in}}%
\pgfpathlineto{\pgfqpoint{2.448186in}{2.255255in}}%
\pgfpathlineto{\pgfqpoint{2.492925in}{2.296320in}}%
\pgfpathlineto{\pgfqpoint{2.537663in}{2.337385in}}%
\pgfpathlineto{\pgfqpoint{2.582401in}{2.378450in}}%
\pgfpathlineto{\pgfqpoint{2.627139in}{2.419514in}}%
\pgfpathlineto{\pgfqpoint{2.671877in}{2.460579in}}%
\pgfpathlineto{\pgfqpoint{2.716616in}{2.501644in}}%
\pgfpathlineto{\pgfqpoint{2.761354in}{2.542708in}}%
\pgfpathlineto{\pgfqpoint{2.806092in}{2.583773in}}%
\pgfusepath{stroke}%
\end{pgfscope}%
\begin{pgfscope}%
\pgfpathrectangle{\pgfqpoint{0.613922in}{0.571604in}}{\pgfqpoint{2.192170in}{2.278396in}}%
\pgfusepath{clip}%
\pgfsetbuttcap%
\pgfsetroundjoin%
\definecolor{currentfill}{rgb}{0.121569,0.466667,0.705882}%
\pgfsetfillcolor{currentfill}%
\pgfsetlinewidth{1.003750pt}%
\definecolor{currentstroke}{rgb}{0.121569,0.466667,0.705882}%
\pgfsetstrokecolor{currentstroke}%
\pgfsetdash{}{0pt}%
\pgfsys@defobject{currentmarker}{\pgfqpoint{-0.020833in}{-0.020833in}}{\pgfqpoint{0.020833in}{0.020833in}}{%
\pgfpathmoveto{\pgfqpoint{0.000000in}{-0.020833in}}%
\pgfpathcurveto{\pgfqpoint{0.005525in}{-0.020833in}}{\pgfqpoint{0.010825in}{-0.018638in}}{\pgfqpoint{0.014731in}{-0.014731in}}%
\pgfpathcurveto{\pgfqpoint{0.018638in}{-0.010825in}}{\pgfqpoint{0.020833in}{-0.005525in}}{\pgfqpoint{0.020833in}{0.000000in}}%
\pgfpathcurveto{\pgfqpoint{0.020833in}{0.005525in}}{\pgfqpoint{0.018638in}{0.010825in}}{\pgfqpoint{0.014731in}{0.014731in}}%
\pgfpathcurveto{\pgfqpoint{0.010825in}{0.018638in}}{\pgfqpoint{0.005525in}{0.020833in}}{\pgfqpoint{0.000000in}{0.020833in}}%
\pgfpathcurveto{\pgfqpoint{-0.005525in}{0.020833in}}{\pgfqpoint{-0.010825in}{0.018638in}}{\pgfqpoint{-0.014731in}{0.014731in}}%
\pgfpathcurveto{\pgfqpoint{-0.018638in}{0.010825in}}{\pgfqpoint{-0.020833in}{0.005525in}}{\pgfqpoint{-0.020833in}{0.000000in}}%
\pgfpathcurveto{\pgfqpoint{-0.020833in}{-0.005525in}}{\pgfqpoint{-0.018638in}{-0.010825in}}{\pgfqpoint{-0.014731in}{-0.014731in}}%
\pgfpathcurveto{\pgfqpoint{-0.010825in}{-0.018638in}}{\pgfqpoint{-0.005525in}{-0.020833in}}{\pgfqpoint{0.000000in}{-0.020833in}}%
\pgfpathclose%
\pgfusepath{stroke,fill}%
}%
\begin{pgfscope}%
\pgfsys@transformshift{2.701703in}{2.638796in}%
\pgfsys@useobject{currentmarker}{}%
\end{pgfscope}%
\begin{pgfscope}%
\pgfsys@transformshift{2.284147in}{1.831593in}%
\pgfsys@useobject{currentmarker}{}%
\end{pgfscope}%
\begin{pgfscope}%
\pgfsys@transformshift{1.866590in}{1.736117in}%
\pgfsys@useobject{currentmarker}{}%
\end{pgfscope}%
\begin{pgfscope}%
\pgfsys@transformshift{1.449034in}{1.419312in}%
\pgfsys@useobject{currentmarker}{}%
\end{pgfscope}%
\begin{pgfscope}%
\pgfsys@transformshift{1.031478in}{1.086594in}%
\pgfsys@useobject{currentmarker}{}%
\end{pgfscope}%
\end{pgfscope}%
\begin{pgfscope}%
\pgfsetrectcap%
\pgfsetmiterjoin%
\pgfsetlinewidth{0.803000pt}%
\definecolor{currentstroke}{rgb}{0.000000,0.000000,0.000000}%
\pgfsetstrokecolor{currentstroke}%
\pgfsetdash{}{0pt}%
\pgfpathmoveto{\pgfqpoint{0.613922in}{0.571604in}}%
\pgfpathlineto{\pgfqpoint{0.613922in}{2.850000in}}%
\pgfusepath{stroke}%
\end{pgfscope}%
\begin{pgfscope}%
\pgfsetrectcap%
\pgfsetmiterjoin%
\pgfsetlinewidth{0.803000pt}%
\definecolor{currentstroke}{rgb}{0.000000,0.000000,0.000000}%
\pgfsetstrokecolor{currentstroke}%
\pgfsetdash{}{0pt}%
\pgfpathmoveto{\pgfqpoint{2.806092in}{0.571604in}}%
\pgfpathlineto{\pgfqpoint{2.806092in}{2.850000in}}%
\pgfusepath{stroke}%
\end{pgfscope}%
\begin{pgfscope}%
\pgfsetrectcap%
\pgfsetmiterjoin%
\pgfsetlinewidth{0.803000pt}%
\definecolor{currentstroke}{rgb}{0.000000,0.000000,0.000000}%
\pgfsetstrokecolor{currentstroke}%
\pgfsetdash{}{0pt}%
\pgfpathmoveto{\pgfqpoint{0.613922in}{0.571604in}}%
\pgfpathlineto{\pgfqpoint{2.806092in}{0.571604in}}%
\pgfusepath{stroke}%
\end{pgfscope}%
\begin{pgfscope}%
\pgfsetrectcap%
\pgfsetmiterjoin%
\pgfsetlinewidth{0.803000pt}%
\definecolor{currentstroke}{rgb}{0.000000,0.000000,0.000000}%
\pgfsetstrokecolor{currentstroke}%
\pgfsetdash{}{0pt}%
\pgfpathmoveto{\pgfqpoint{0.613922in}{2.850000in}}%
\pgfpathlineto{\pgfqpoint{2.806092in}{2.850000in}}%
\pgfusepath{stroke}%
\end{pgfscope}%
\begin{pgfscope}%
\pgfsetbuttcap%
\pgfsetmiterjoin%
\definecolor{currentfill}{rgb}{1.000000,1.000000,1.000000}%
\pgfsetfillcolor{currentfill}%
\pgfsetfillopacity{0.800000}%
\pgfsetlinewidth{1.003750pt}%
\definecolor{currentstroke}{rgb}{0.800000,0.800000,0.800000}%
\pgfsetstrokecolor{currentstroke}%
\pgfsetstrokeopacity{0.800000}%
\pgfsetdash{}{0pt}%
\pgfpathmoveto{\pgfqpoint{0.711144in}{2.331174in}}%
\pgfpathlineto{\pgfqpoint{2.196916in}{2.331174in}}%
\pgfpathquadraticcurveto{\pgfqpoint{2.224694in}{2.331174in}}{\pgfqpoint{2.224694in}{2.358952in}}%
\pgfpathlineto{\pgfqpoint{2.224694in}{2.752778in}}%
\pgfpathquadraticcurveto{\pgfqpoint{2.224694in}{2.780556in}}{\pgfqpoint{2.196916in}{2.780556in}}%
\pgfpathlineto{\pgfqpoint{0.711144in}{2.780556in}}%
\pgfpathquadraticcurveto{\pgfqpoint{0.683366in}{2.780556in}}{\pgfqpoint{0.683366in}{2.752778in}}%
\pgfpathlineto{\pgfqpoint{0.683366in}{2.358952in}}%
\pgfpathquadraticcurveto{\pgfqpoint{0.683366in}{2.331174in}}{\pgfqpoint{0.711144in}{2.331174in}}%
\pgfpathclose%
\pgfusepath{stroke,fill}%
\end{pgfscope}%
\begin{pgfscope}%
\pgfsetrectcap%
\pgfsetroundjoin%
\pgfsetlinewidth{1.505625pt}%
\definecolor{currentstroke}{rgb}{1.000000,0.498039,0.054902}%
\pgfsetstrokecolor{currentstroke}%
\pgfsetdash{}{0pt}%
\pgfpathmoveto{\pgfqpoint{0.738922in}{2.668088in}}%
\pgfpathlineto{\pgfqpoint{1.016699in}{2.668088in}}%
\pgfusepath{stroke}%
\end{pgfscope}%
\begin{pgfscope}%
\definecolor{textcolor}{rgb}{0.000000,0.000000,0.000000}%
\pgfsetstrokecolor{textcolor}%
\pgfsetfillcolor{textcolor}%
\pgftext[x=1.127810in,y=2.619477in,left,base]{\color{textcolor}\sffamily\fontsize{10.000000}{12.000000}\selectfont fit}%
\end{pgfscope}%
\begin{pgfscope}%
\pgfsetbuttcap%
\pgfsetroundjoin%
\pgfsetlinewidth{1.505625pt}%
\definecolor{currentstroke}{rgb}{0.121569,0.466667,0.705882}%
\pgfsetstrokecolor{currentstroke}%
\pgfsetdash{}{0pt}%
\pgfpathmoveto{\pgfqpoint{0.877810in}{2.394786in}}%
\pgfpathlineto{\pgfqpoint{0.877810in}{2.533675in}}%
\pgfusepath{stroke}%
\end{pgfscope}%
\begin{pgfscope}%
\pgfsetbuttcap%
\pgfsetroundjoin%
\definecolor{currentfill}{rgb}{0.121569,0.466667,0.705882}%
\pgfsetfillcolor{currentfill}%
\pgfsetlinewidth{1.003750pt}%
\definecolor{currentstroke}{rgb}{0.121569,0.466667,0.705882}%
\pgfsetstrokecolor{currentstroke}%
\pgfsetdash{}{0pt}%
\pgfsys@defobject{currentmarker}{\pgfqpoint{-0.020833in}{-0.020833in}}{\pgfqpoint{0.020833in}{0.020833in}}{%
\pgfpathmoveto{\pgfqpoint{0.000000in}{-0.020833in}}%
\pgfpathcurveto{\pgfqpoint{0.005525in}{-0.020833in}}{\pgfqpoint{0.010825in}{-0.018638in}}{\pgfqpoint{0.014731in}{-0.014731in}}%
\pgfpathcurveto{\pgfqpoint{0.018638in}{-0.010825in}}{\pgfqpoint{0.020833in}{-0.005525in}}{\pgfqpoint{0.020833in}{0.000000in}}%
\pgfpathcurveto{\pgfqpoint{0.020833in}{0.005525in}}{\pgfqpoint{0.018638in}{0.010825in}}{\pgfqpoint{0.014731in}{0.014731in}}%
\pgfpathcurveto{\pgfqpoint{0.010825in}{0.018638in}}{\pgfqpoint{0.005525in}{0.020833in}}{\pgfqpoint{0.000000in}{0.020833in}}%
\pgfpathcurveto{\pgfqpoint{-0.005525in}{0.020833in}}{\pgfqpoint{-0.010825in}{0.018638in}}{\pgfqpoint{-0.014731in}{0.014731in}}%
\pgfpathcurveto{\pgfqpoint{-0.018638in}{0.010825in}}{\pgfqpoint{-0.020833in}{0.005525in}}{\pgfqpoint{-0.020833in}{0.000000in}}%
\pgfpathcurveto{\pgfqpoint{-0.020833in}{-0.005525in}}{\pgfqpoint{-0.018638in}{-0.010825in}}{\pgfqpoint{-0.014731in}{-0.014731in}}%
\pgfpathcurveto{\pgfqpoint{-0.010825in}{-0.018638in}}{\pgfqpoint{-0.005525in}{-0.020833in}}{\pgfqpoint{0.000000in}{-0.020833in}}%
\pgfpathclose%
\pgfusepath{stroke,fill}%
}%
\begin{pgfscope}%
\pgfsys@transformshift{0.877810in}{2.464231in}%
\pgfsys@useobject{currentmarker}{}%
\end{pgfscope}%
\end{pgfscope}%
\begin{pgfscope}%
\definecolor{textcolor}{rgb}{0.000000,0.000000,0.000000}%
\pgfsetstrokecolor{textcolor}%
\pgfsetfillcolor{textcolor}%
\pgftext[x=1.127810in,y=2.415619in,left,base]{\color{textcolor}\sffamily\fontsize{10.000000}{12.000000}\selectfont measurements}%
\end{pgfscope}%
\end{pgfpicture}%
\makeatother%
\endgroup%
}
  \vspace{-30pt}
\end{wrapfigure}
In diesem Teil haben wir MRT Scans auf CT Scans transformiert, und dabei die
Qualität der Transformation und die benötigte Rechenzeit beobachtet.\\
Wir fanden gute Ergebnisse bei drei Registrierungsstufen mit Shrinkfaktoren von
\num{4}, \num{2} \& \num{1}, so wie einem Glättungsfaktor von \num{2} für alle
Stufen bei einer Sampling-Rate von \SI{50}{\percent}.\\
In Abbildung \ref{pic:A3} sind Überlagerungen mit der genannten guten Strategie,
so wie einer Berechnung ohne zusätzliche Registrierungsstufen bei einer vollen
Samplingrate zu sehen, die deutlich länger zum Berechnen benötigt hat. Bei
genauen Betrachten erkennt man Qualitätsunterschiede, etwa im Fenster links
oben im unteren Bereich des Gehirns, oder im Fenster oben rechts an den Augen.
Es stellt sich also heraus, dass die Qualität nicht durch die Samplingrate
bestimmt wird, obwohl diese einen großen Einfluss auf die Rechenzeit hat, denn
an Abbildung \ref{fig:A3} erkennt man, dass die Zeit zur Berechnung in etwa
linear zur Samplingrate (also zur Anzahl der Bildpunkte) ist.
\newpage

\subsection{Deformierbare Registrierung}
\begin{wrapfigure}{r}{0.5\linewidth}
  \vspace{-6pt}
  \vspace{-10pt}
  \caption{Berechnungsdauer der deformierbaren Registrierung in Abhängigkeit der Anzahl der Gitterpunkte pro Dimension}
  \label{fig:A4}
  \vspace{-10pt}
  \resizebox{\linewidth}{!}{%% Creator: Matplotlib, PGF backend
%%
%% To include the figure in your LaTeX document, write
%%   \input{<filename>.pgf}
%%
%% Make sure the required packages are loaded in your preamble
%%   \usepackage{pgf}
%%
%% and, on pdftex
%%   \usepackage[utf8]{inputenc}\DeclareUnicodeCharacter{2212}{-}
%%
%% or, on luatex and xetex
%%   \usepackage{unicode-math}
%%
%% Figures using additional raster images can only be included by \input if
%% they are in the same directory as the main LaTeX file. For loading figures
%% from other directories you can use the `import` package
%%   \usepackage{import}
%%
%% and then include the figures with
%%   \import{<path to file>}{<filename>.pgf}
%%
%% Matplotlib used the following preamble
%%   \usepackage{fontspec}
%%   \setmainfont{DejaVuSerif.ttf}[Path=/usr/share/matplotlib/mpl-data/fonts/ttf/]
%%   \setsansfont{DejaVuSans.ttf}[Path=/usr/share/matplotlib/mpl-data/fonts/ttf/]
%%   \setmonofont{DejaVuSansMono.ttf}[Path=/usr/share/matplotlib/mpl-data/fonts/ttf/]
%%
\begingroup%
\makeatletter%
\begin{pgfpicture}%
\pgfpathrectangle{\pgfpointorigin}{\pgfqpoint{3.000000in}{3.000000in}}%
\pgfusepath{use as bounding box, clip}%
\begin{pgfscope}%
\pgfsetbuttcap%
\pgfsetmiterjoin%
\definecolor{currentfill}{rgb}{1.000000,1.000000,1.000000}%
\pgfsetfillcolor{currentfill}%
\pgfsetlinewidth{0.000000pt}%
\definecolor{currentstroke}{rgb}{1.000000,1.000000,1.000000}%
\pgfsetstrokecolor{currentstroke}%
\pgfsetdash{}{0pt}%
\pgfpathmoveto{\pgfqpoint{0.000000in}{0.000000in}}%
\pgfpathlineto{\pgfqpoint{3.000000in}{0.000000in}}%
\pgfpathlineto{\pgfqpoint{3.000000in}{3.000000in}}%
\pgfpathlineto{\pgfqpoint{0.000000in}{3.000000in}}%
\pgfpathclose%
\pgfusepath{fill}%
\end{pgfscope}%
\begin{pgfscope}%
\pgfsetbuttcap%
\pgfsetmiterjoin%
\definecolor{currentfill}{rgb}{1.000000,1.000000,1.000000}%
\pgfsetfillcolor{currentfill}%
\pgfsetlinewidth{0.000000pt}%
\definecolor{currentstroke}{rgb}{0.000000,0.000000,0.000000}%
\pgfsetstrokecolor{currentstroke}%
\pgfsetstrokeopacity{0.000000}%
\pgfsetdash{}{0pt}%
\pgfpathmoveto{\pgfqpoint{0.613922in}{0.571604in}}%
\pgfpathlineto{\pgfqpoint{2.785170in}{0.571604in}}%
\pgfpathlineto{\pgfqpoint{2.785170in}{2.797238in}}%
\pgfpathlineto{\pgfqpoint{0.613922in}{2.797238in}}%
\pgfpathclose%
\pgfusepath{fill}%
\end{pgfscope}%
\begin{pgfscope}%
\pgfpathrectangle{\pgfqpoint{0.613922in}{0.571604in}}{\pgfqpoint{2.171249in}{2.225635in}}%
\pgfusepath{clip}%
\pgfsetrectcap%
\pgfsetroundjoin%
\pgfsetlinewidth{0.803000pt}%
\definecolor{currentstroke}{rgb}{0.690196,0.690196,0.690196}%
\pgfsetstrokecolor{currentstroke}%
\pgfsetdash{}{0pt}%
\pgfpathmoveto{\pgfqpoint{0.613922in}{0.571604in}}%
\pgfpathlineto{\pgfqpoint{0.613922in}{2.797238in}}%
\pgfusepath{stroke}%
\end{pgfscope}%
\begin{pgfscope}%
\pgfsetbuttcap%
\pgfsetroundjoin%
\definecolor{currentfill}{rgb}{0.000000,0.000000,0.000000}%
\pgfsetfillcolor{currentfill}%
\pgfsetlinewidth{0.803000pt}%
\definecolor{currentstroke}{rgb}{0.000000,0.000000,0.000000}%
\pgfsetstrokecolor{currentstroke}%
\pgfsetdash{}{0pt}%
\pgfsys@defobject{currentmarker}{\pgfqpoint{0.000000in}{-0.048611in}}{\pgfqpoint{0.000000in}{0.000000in}}{%
\pgfpathmoveto{\pgfqpoint{0.000000in}{0.000000in}}%
\pgfpathlineto{\pgfqpoint{0.000000in}{-0.048611in}}%
\pgfusepath{stroke,fill}%
}%
\begin{pgfscope}%
\pgfsys@transformshift{0.613922in}{0.571604in}%
\pgfsys@useobject{currentmarker}{}%
\end{pgfscope}%
\end{pgfscope}%
\begin{pgfscope}%
\definecolor{textcolor}{rgb}{0.000000,0.000000,0.000000}%
\pgfsetstrokecolor{textcolor}%
\pgfsetfillcolor{textcolor}%
\pgftext[x=0.613922in,y=0.474382in,,top]{\color{textcolor}\sffamily\fontsize{10.000000}{12.000000}\selectfont 0}%
\end{pgfscope}%
\begin{pgfscope}%
\pgfpathrectangle{\pgfqpoint{0.613922in}{0.571604in}}{\pgfqpoint{2.171249in}{2.225635in}}%
\pgfusepath{clip}%
\pgfsetrectcap%
\pgfsetroundjoin%
\pgfsetlinewidth{0.803000pt}%
\definecolor{currentstroke}{rgb}{0.690196,0.690196,0.690196}%
\pgfsetstrokecolor{currentstroke}%
\pgfsetdash{}{0pt}%
\pgfpathmoveto{\pgfqpoint{1.140924in}{0.571604in}}%
\pgfpathlineto{\pgfqpoint{1.140924in}{2.797238in}}%
\pgfusepath{stroke}%
\end{pgfscope}%
\begin{pgfscope}%
\pgfsetbuttcap%
\pgfsetroundjoin%
\definecolor{currentfill}{rgb}{0.000000,0.000000,0.000000}%
\pgfsetfillcolor{currentfill}%
\pgfsetlinewidth{0.803000pt}%
\definecolor{currentstroke}{rgb}{0.000000,0.000000,0.000000}%
\pgfsetstrokecolor{currentstroke}%
\pgfsetdash{}{0pt}%
\pgfsys@defobject{currentmarker}{\pgfqpoint{0.000000in}{-0.048611in}}{\pgfqpoint{0.000000in}{0.000000in}}{%
\pgfpathmoveto{\pgfqpoint{0.000000in}{0.000000in}}%
\pgfpathlineto{\pgfqpoint{0.000000in}{-0.048611in}}%
\pgfusepath{stroke,fill}%
}%
\begin{pgfscope}%
\pgfsys@transformshift{1.140924in}{0.571604in}%
\pgfsys@useobject{currentmarker}{}%
\end{pgfscope}%
\end{pgfscope}%
\begin{pgfscope}%
\definecolor{textcolor}{rgb}{0.000000,0.000000,0.000000}%
\pgfsetstrokecolor{textcolor}%
\pgfsetfillcolor{textcolor}%
\pgftext[x=1.140924in,y=0.474382in,,top]{\color{textcolor}\sffamily\fontsize{10.000000}{12.000000}\selectfont 25}%
\end{pgfscope}%
\begin{pgfscope}%
\pgfpathrectangle{\pgfqpoint{0.613922in}{0.571604in}}{\pgfqpoint{2.171249in}{2.225635in}}%
\pgfusepath{clip}%
\pgfsetrectcap%
\pgfsetroundjoin%
\pgfsetlinewidth{0.803000pt}%
\definecolor{currentstroke}{rgb}{0.690196,0.690196,0.690196}%
\pgfsetstrokecolor{currentstroke}%
\pgfsetdash{}{0pt}%
\pgfpathmoveto{\pgfqpoint{1.667926in}{0.571604in}}%
\pgfpathlineto{\pgfqpoint{1.667926in}{2.797238in}}%
\pgfusepath{stroke}%
\end{pgfscope}%
\begin{pgfscope}%
\pgfsetbuttcap%
\pgfsetroundjoin%
\definecolor{currentfill}{rgb}{0.000000,0.000000,0.000000}%
\pgfsetfillcolor{currentfill}%
\pgfsetlinewidth{0.803000pt}%
\definecolor{currentstroke}{rgb}{0.000000,0.000000,0.000000}%
\pgfsetstrokecolor{currentstroke}%
\pgfsetdash{}{0pt}%
\pgfsys@defobject{currentmarker}{\pgfqpoint{0.000000in}{-0.048611in}}{\pgfqpoint{0.000000in}{0.000000in}}{%
\pgfpathmoveto{\pgfqpoint{0.000000in}{0.000000in}}%
\pgfpathlineto{\pgfqpoint{0.000000in}{-0.048611in}}%
\pgfusepath{stroke,fill}%
}%
\begin{pgfscope}%
\pgfsys@transformshift{1.667926in}{0.571604in}%
\pgfsys@useobject{currentmarker}{}%
\end{pgfscope}%
\end{pgfscope}%
\begin{pgfscope}%
\definecolor{textcolor}{rgb}{0.000000,0.000000,0.000000}%
\pgfsetstrokecolor{textcolor}%
\pgfsetfillcolor{textcolor}%
\pgftext[x=1.667926in,y=0.474382in,,top]{\color{textcolor}\sffamily\fontsize{10.000000}{12.000000}\selectfont 50}%
\end{pgfscope}%
\begin{pgfscope}%
\pgfpathrectangle{\pgfqpoint{0.613922in}{0.571604in}}{\pgfqpoint{2.171249in}{2.225635in}}%
\pgfusepath{clip}%
\pgfsetrectcap%
\pgfsetroundjoin%
\pgfsetlinewidth{0.803000pt}%
\definecolor{currentstroke}{rgb}{0.690196,0.690196,0.690196}%
\pgfsetstrokecolor{currentstroke}%
\pgfsetdash{}{0pt}%
\pgfpathmoveto{\pgfqpoint{2.194928in}{0.571604in}}%
\pgfpathlineto{\pgfqpoint{2.194928in}{2.797238in}}%
\pgfusepath{stroke}%
\end{pgfscope}%
\begin{pgfscope}%
\pgfsetbuttcap%
\pgfsetroundjoin%
\definecolor{currentfill}{rgb}{0.000000,0.000000,0.000000}%
\pgfsetfillcolor{currentfill}%
\pgfsetlinewidth{0.803000pt}%
\definecolor{currentstroke}{rgb}{0.000000,0.000000,0.000000}%
\pgfsetstrokecolor{currentstroke}%
\pgfsetdash{}{0pt}%
\pgfsys@defobject{currentmarker}{\pgfqpoint{0.000000in}{-0.048611in}}{\pgfqpoint{0.000000in}{0.000000in}}{%
\pgfpathmoveto{\pgfqpoint{0.000000in}{0.000000in}}%
\pgfpathlineto{\pgfqpoint{0.000000in}{-0.048611in}}%
\pgfusepath{stroke,fill}%
}%
\begin{pgfscope}%
\pgfsys@transformshift{2.194928in}{0.571604in}%
\pgfsys@useobject{currentmarker}{}%
\end{pgfscope}%
\end{pgfscope}%
\begin{pgfscope}%
\definecolor{textcolor}{rgb}{0.000000,0.000000,0.000000}%
\pgfsetstrokecolor{textcolor}%
\pgfsetfillcolor{textcolor}%
\pgftext[x=2.194928in,y=0.474382in,,top]{\color{textcolor}\sffamily\fontsize{10.000000}{12.000000}\selectfont 75}%
\end{pgfscope}%
\begin{pgfscope}%
\pgfpathrectangle{\pgfqpoint{0.613922in}{0.571604in}}{\pgfqpoint{2.171249in}{2.225635in}}%
\pgfusepath{clip}%
\pgfsetrectcap%
\pgfsetroundjoin%
\pgfsetlinewidth{0.803000pt}%
\definecolor{currentstroke}{rgb}{0.690196,0.690196,0.690196}%
\pgfsetstrokecolor{currentstroke}%
\pgfsetdash{}{0pt}%
\pgfpathmoveto{\pgfqpoint{2.721930in}{0.571604in}}%
\pgfpathlineto{\pgfqpoint{2.721930in}{2.797238in}}%
\pgfusepath{stroke}%
\end{pgfscope}%
\begin{pgfscope}%
\pgfsetbuttcap%
\pgfsetroundjoin%
\definecolor{currentfill}{rgb}{0.000000,0.000000,0.000000}%
\pgfsetfillcolor{currentfill}%
\pgfsetlinewidth{0.803000pt}%
\definecolor{currentstroke}{rgb}{0.000000,0.000000,0.000000}%
\pgfsetstrokecolor{currentstroke}%
\pgfsetdash{}{0pt}%
\pgfsys@defobject{currentmarker}{\pgfqpoint{0.000000in}{-0.048611in}}{\pgfqpoint{0.000000in}{0.000000in}}{%
\pgfpathmoveto{\pgfqpoint{0.000000in}{0.000000in}}%
\pgfpathlineto{\pgfqpoint{0.000000in}{-0.048611in}}%
\pgfusepath{stroke,fill}%
}%
\begin{pgfscope}%
\pgfsys@transformshift{2.721930in}{0.571604in}%
\pgfsys@useobject{currentmarker}{}%
\end{pgfscope}%
\end{pgfscope}%
\begin{pgfscope}%
\definecolor{textcolor}{rgb}{0.000000,0.000000,0.000000}%
\pgfsetstrokecolor{textcolor}%
\pgfsetfillcolor{textcolor}%
\pgftext[x=2.721930in,y=0.474382in,,top]{\color{textcolor}\sffamily\fontsize{10.000000}{12.000000}\selectfont 100}%
\end{pgfscope}%
\begin{pgfscope}%
\definecolor{textcolor}{rgb}{0.000000,0.000000,0.000000}%
\pgfsetstrokecolor{textcolor}%
\pgfsetfillcolor{textcolor}%
\pgftext[x=1.699546in,y=0.284413in,,top]{\color{textcolor}\sffamily\fontsize{10.000000}{12.000000}\selectfont Gitterpunkte pro Dimension}%
\end{pgfscope}%
\begin{pgfscope}%
\pgfpathrectangle{\pgfqpoint{0.613922in}{0.571604in}}{\pgfqpoint{2.171249in}{2.225635in}}%
\pgfusepath{clip}%
\pgfsetrectcap%
\pgfsetroundjoin%
\pgfsetlinewidth{0.803000pt}%
\definecolor{currentstroke}{rgb}{0.690196,0.690196,0.690196}%
\pgfsetstrokecolor{currentstroke}%
\pgfsetdash{}{0pt}%
\pgfpathmoveto{\pgfqpoint{0.613922in}{0.633427in}}%
\pgfpathlineto{\pgfqpoint{2.785170in}{0.633427in}}%
\pgfusepath{stroke}%
\end{pgfscope}%
\begin{pgfscope}%
\pgfsetbuttcap%
\pgfsetroundjoin%
\definecolor{currentfill}{rgb}{0.000000,0.000000,0.000000}%
\pgfsetfillcolor{currentfill}%
\pgfsetlinewidth{0.803000pt}%
\definecolor{currentstroke}{rgb}{0.000000,0.000000,0.000000}%
\pgfsetstrokecolor{currentstroke}%
\pgfsetdash{}{0pt}%
\pgfsys@defobject{currentmarker}{\pgfqpoint{-0.048611in}{0.000000in}}{\pgfqpoint{0.000000in}{0.000000in}}{%
\pgfpathmoveto{\pgfqpoint{0.000000in}{0.000000in}}%
\pgfpathlineto{\pgfqpoint{-0.048611in}{0.000000in}}%
\pgfusepath{stroke,fill}%
}%
\begin{pgfscope}%
\pgfsys@transformshift{0.613922in}{0.633427in}%
\pgfsys@useobject{currentmarker}{}%
\end{pgfscope}%
\end{pgfscope}%
\begin{pgfscope}%
\definecolor{textcolor}{rgb}{0.000000,0.000000,0.000000}%
\pgfsetstrokecolor{textcolor}%
\pgfsetfillcolor{textcolor}%
\pgftext[x=0.428334in, y=0.580666in, left, base]{\color{textcolor}\sffamily\fontsize{10.000000}{12.000000}\selectfont 0}%
\end{pgfscope}%
\begin{pgfscope}%
\pgfpathrectangle{\pgfqpoint{0.613922in}{0.571604in}}{\pgfqpoint{2.171249in}{2.225635in}}%
\pgfusepath{clip}%
\pgfsetrectcap%
\pgfsetroundjoin%
\pgfsetlinewidth{0.803000pt}%
\definecolor{currentstroke}{rgb}{0.690196,0.690196,0.690196}%
\pgfsetstrokecolor{currentstroke}%
\pgfsetdash{}{0pt}%
\pgfpathmoveto{\pgfqpoint{0.613922in}{0.942543in}}%
\pgfpathlineto{\pgfqpoint{2.785170in}{0.942543in}}%
\pgfusepath{stroke}%
\end{pgfscope}%
\begin{pgfscope}%
\pgfsetbuttcap%
\pgfsetroundjoin%
\definecolor{currentfill}{rgb}{0.000000,0.000000,0.000000}%
\pgfsetfillcolor{currentfill}%
\pgfsetlinewidth{0.803000pt}%
\definecolor{currentstroke}{rgb}{0.000000,0.000000,0.000000}%
\pgfsetstrokecolor{currentstroke}%
\pgfsetdash{}{0pt}%
\pgfsys@defobject{currentmarker}{\pgfqpoint{-0.048611in}{0.000000in}}{\pgfqpoint{0.000000in}{0.000000in}}{%
\pgfpathmoveto{\pgfqpoint{0.000000in}{0.000000in}}%
\pgfpathlineto{\pgfqpoint{-0.048611in}{0.000000in}}%
\pgfusepath{stroke,fill}%
}%
\begin{pgfscope}%
\pgfsys@transformshift{0.613922in}{0.942543in}%
\pgfsys@useobject{currentmarker}{}%
\end{pgfscope}%
\end{pgfscope}%
\begin{pgfscope}%
\definecolor{textcolor}{rgb}{0.000000,0.000000,0.000000}%
\pgfsetstrokecolor{textcolor}%
\pgfsetfillcolor{textcolor}%
\pgftext[x=0.428334in, y=0.889781in, left, base]{\color{textcolor}\sffamily\fontsize{10.000000}{12.000000}\selectfont 5}%
\end{pgfscope}%
\begin{pgfscope}%
\pgfpathrectangle{\pgfqpoint{0.613922in}{0.571604in}}{\pgfqpoint{2.171249in}{2.225635in}}%
\pgfusepath{clip}%
\pgfsetrectcap%
\pgfsetroundjoin%
\pgfsetlinewidth{0.803000pt}%
\definecolor{currentstroke}{rgb}{0.690196,0.690196,0.690196}%
\pgfsetstrokecolor{currentstroke}%
\pgfsetdash{}{0pt}%
\pgfpathmoveto{\pgfqpoint{0.613922in}{1.251659in}}%
\pgfpathlineto{\pgfqpoint{2.785170in}{1.251659in}}%
\pgfusepath{stroke}%
\end{pgfscope}%
\begin{pgfscope}%
\pgfsetbuttcap%
\pgfsetroundjoin%
\definecolor{currentfill}{rgb}{0.000000,0.000000,0.000000}%
\pgfsetfillcolor{currentfill}%
\pgfsetlinewidth{0.803000pt}%
\definecolor{currentstroke}{rgb}{0.000000,0.000000,0.000000}%
\pgfsetstrokecolor{currentstroke}%
\pgfsetdash{}{0pt}%
\pgfsys@defobject{currentmarker}{\pgfqpoint{-0.048611in}{0.000000in}}{\pgfqpoint{0.000000in}{0.000000in}}{%
\pgfpathmoveto{\pgfqpoint{0.000000in}{0.000000in}}%
\pgfpathlineto{\pgfqpoint{-0.048611in}{0.000000in}}%
\pgfusepath{stroke,fill}%
}%
\begin{pgfscope}%
\pgfsys@transformshift{0.613922in}{1.251659in}%
\pgfsys@useobject{currentmarker}{}%
\end{pgfscope}%
\end{pgfscope}%
\begin{pgfscope}%
\definecolor{textcolor}{rgb}{0.000000,0.000000,0.000000}%
\pgfsetstrokecolor{textcolor}%
\pgfsetfillcolor{textcolor}%
\pgftext[x=0.339969in, y=1.198897in, left, base]{\color{textcolor}\sffamily\fontsize{10.000000}{12.000000}\selectfont 10}%
\end{pgfscope}%
\begin{pgfscope}%
\pgfpathrectangle{\pgfqpoint{0.613922in}{0.571604in}}{\pgfqpoint{2.171249in}{2.225635in}}%
\pgfusepath{clip}%
\pgfsetrectcap%
\pgfsetroundjoin%
\pgfsetlinewidth{0.803000pt}%
\definecolor{currentstroke}{rgb}{0.690196,0.690196,0.690196}%
\pgfsetstrokecolor{currentstroke}%
\pgfsetdash{}{0pt}%
\pgfpathmoveto{\pgfqpoint{0.613922in}{1.560775in}}%
\pgfpathlineto{\pgfqpoint{2.785170in}{1.560775in}}%
\pgfusepath{stroke}%
\end{pgfscope}%
\begin{pgfscope}%
\pgfsetbuttcap%
\pgfsetroundjoin%
\definecolor{currentfill}{rgb}{0.000000,0.000000,0.000000}%
\pgfsetfillcolor{currentfill}%
\pgfsetlinewidth{0.803000pt}%
\definecolor{currentstroke}{rgb}{0.000000,0.000000,0.000000}%
\pgfsetstrokecolor{currentstroke}%
\pgfsetdash{}{0pt}%
\pgfsys@defobject{currentmarker}{\pgfqpoint{-0.048611in}{0.000000in}}{\pgfqpoint{0.000000in}{0.000000in}}{%
\pgfpathmoveto{\pgfqpoint{0.000000in}{0.000000in}}%
\pgfpathlineto{\pgfqpoint{-0.048611in}{0.000000in}}%
\pgfusepath{stroke,fill}%
}%
\begin{pgfscope}%
\pgfsys@transformshift{0.613922in}{1.560775in}%
\pgfsys@useobject{currentmarker}{}%
\end{pgfscope}%
\end{pgfscope}%
\begin{pgfscope}%
\definecolor{textcolor}{rgb}{0.000000,0.000000,0.000000}%
\pgfsetstrokecolor{textcolor}%
\pgfsetfillcolor{textcolor}%
\pgftext[x=0.339969in, y=1.508013in, left, base]{\color{textcolor}\sffamily\fontsize{10.000000}{12.000000}\selectfont 15}%
\end{pgfscope}%
\begin{pgfscope}%
\pgfpathrectangle{\pgfqpoint{0.613922in}{0.571604in}}{\pgfqpoint{2.171249in}{2.225635in}}%
\pgfusepath{clip}%
\pgfsetrectcap%
\pgfsetroundjoin%
\pgfsetlinewidth{0.803000pt}%
\definecolor{currentstroke}{rgb}{0.690196,0.690196,0.690196}%
\pgfsetstrokecolor{currentstroke}%
\pgfsetdash{}{0pt}%
\pgfpathmoveto{\pgfqpoint{0.613922in}{1.869891in}}%
\pgfpathlineto{\pgfqpoint{2.785170in}{1.869891in}}%
\pgfusepath{stroke}%
\end{pgfscope}%
\begin{pgfscope}%
\pgfsetbuttcap%
\pgfsetroundjoin%
\definecolor{currentfill}{rgb}{0.000000,0.000000,0.000000}%
\pgfsetfillcolor{currentfill}%
\pgfsetlinewidth{0.803000pt}%
\definecolor{currentstroke}{rgb}{0.000000,0.000000,0.000000}%
\pgfsetstrokecolor{currentstroke}%
\pgfsetdash{}{0pt}%
\pgfsys@defobject{currentmarker}{\pgfqpoint{-0.048611in}{0.000000in}}{\pgfqpoint{0.000000in}{0.000000in}}{%
\pgfpathmoveto{\pgfqpoint{0.000000in}{0.000000in}}%
\pgfpathlineto{\pgfqpoint{-0.048611in}{0.000000in}}%
\pgfusepath{stroke,fill}%
}%
\begin{pgfscope}%
\pgfsys@transformshift{0.613922in}{1.869891in}%
\pgfsys@useobject{currentmarker}{}%
\end{pgfscope}%
\end{pgfscope}%
\begin{pgfscope}%
\definecolor{textcolor}{rgb}{0.000000,0.000000,0.000000}%
\pgfsetstrokecolor{textcolor}%
\pgfsetfillcolor{textcolor}%
\pgftext[x=0.339969in, y=1.817129in, left, base]{\color{textcolor}\sffamily\fontsize{10.000000}{12.000000}\selectfont 20}%
\end{pgfscope}%
\begin{pgfscope}%
\pgfpathrectangle{\pgfqpoint{0.613922in}{0.571604in}}{\pgfqpoint{2.171249in}{2.225635in}}%
\pgfusepath{clip}%
\pgfsetrectcap%
\pgfsetroundjoin%
\pgfsetlinewidth{0.803000pt}%
\definecolor{currentstroke}{rgb}{0.690196,0.690196,0.690196}%
\pgfsetstrokecolor{currentstroke}%
\pgfsetdash{}{0pt}%
\pgfpathmoveto{\pgfqpoint{0.613922in}{2.179007in}}%
\pgfpathlineto{\pgfqpoint{2.785170in}{2.179007in}}%
\pgfusepath{stroke}%
\end{pgfscope}%
\begin{pgfscope}%
\pgfsetbuttcap%
\pgfsetroundjoin%
\definecolor{currentfill}{rgb}{0.000000,0.000000,0.000000}%
\pgfsetfillcolor{currentfill}%
\pgfsetlinewidth{0.803000pt}%
\definecolor{currentstroke}{rgb}{0.000000,0.000000,0.000000}%
\pgfsetstrokecolor{currentstroke}%
\pgfsetdash{}{0pt}%
\pgfsys@defobject{currentmarker}{\pgfqpoint{-0.048611in}{0.000000in}}{\pgfqpoint{0.000000in}{0.000000in}}{%
\pgfpathmoveto{\pgfqpoint{0.000000in}{0.000000in}}%
\pgfpathlineto{\pgfqpoint{-0.048611in}{0.000000in}}%
\pgfusepath{stroke,fill}%
}%
\begin{pgfscope}%
\pgfsys@transformshift{0.613922in}{2.179007in}%
\pgfsys@useobject{currentmarker}{}%
\end{pgfscope}%
\end{pgfscope}%
\begin{pgfscope}%
\definecolor{textcolor}{rgb}{0.000000,0.000000,0.000000}%
\pgfsetstrokecolor{textcolor}%
\pgfsetfillcolor{textcolor}%
\pgftext[x=0.339969in, y=2.126245in, left, base]{\color{textcolor}\sffamily\fontsize{10.000000}{12.000000}\selectfont 25}%
\end{pgfscope}%
\begin{pgfscope}%
\pgfpathrectangle{\pgfqpoint{0.613922in}{0.571604in}}{\pgfqpoint{2.171249in}{2.225635in}}%
\pgfusepath{clip}%
\pgfsetrectcap%
\pgfsetroundjoin%
\pgfsetlinewidth{0.803000pt}%
\definecolor{currentstroke}{rgb}{0.690196,0.690196,0.690196}%
\pgfsetstrokecolor{currentstroke}%
\pgfsetdash{}{0pt}%
\pgfpathmoveto{\pgfqpoint{0.613922in}{2.488123in}}%
\pgfpathlineto{\pgfqpoint{2.785170in}{2.488123in}}%
\pgfusepath{stroke}%
\end{pgfscope}%
\begin{pgfscope}%
\pgfsetbuttcap%
\pgfsetroundjoin%
\definecolor{currentfill}{rgb}{0.000000,0.000000,0.000000}%
\pgfsetfillcolor{currentfill}%
\pgfsetlinewidth{0.803000pt}%
\definecolor{currentstroke}{rgb}{0.000000,0.000000,0.000000}%
\pgfsetstrokecolor{currentstroke}%
\pgfsetdash{}{0pt}%
\pgfsys@defobject{currentmarker}{\pgfqpoint{-0.048611in}{0.000000in}}{\pgfqpoint{0.000000in}{0.000000in}}{%
\pgfpathmoveto{\pgfqpoint{0.000000in}{0.000000in}}%
\pgfpathlineto{\pgfqpoint{-0.048611in}{0.000000in}}%
\pgfusepath{stroke,fill}%
}%
\begin{pgfscope}%
\pgfsys@transformshift{0.613922in}{2.488123in}%
\pgfsys@useobject{currentmarker}{}%
\end{pgfscope}%
\end{pgfscope}%
\begin{pgfscope}%
\definecolor{textcolor}{rgb}{0.000000,0.000000,0.000000}%
\pgfsetstrokecolor{textcolor}%
\pgfsetfillcolor{textcolor}%
\pgftext[x=0.339969in, y=2.435361in, left, base]{\color{textcolor}\sffamily\fontsize{10.000000}{12.000000}\selectfont 30}%
\end{pgfscope}%
\begin{pgfscope}%
\pgfpathrectangle{\pgfqpoint{0.613922in}{0.571604in}}{\pgfqpoint{2.171249in}{2.225635in}}%
\pgfusepath{clip}%
\pgfsetrectcap%
\pgfsetroundjoin%
\pgfsetlinewidth{0.803000pt}%
\definecolor{currentstroke}{rgb}{0.690196,0.690196,0.690196}%
\pgfsetstrokecolor{currentstroke}%
\pgfsetdash{}{0pt}%
\pgfpathmoveto{\pgfqpoint{0.613922in}{2.797238in}}%
\pgfpathlineto{\pgfqpoint{2.785170in}{2.797238in}}%
\pgfusepath{stroke}%
\end{pgfscope}%
\begin{pgfscope}%
\pgfsetbuttcap%
\pgfsetroundjoin%
\definecolor{currentfill}{rgb}{0.000000,0.000000,0.000000}%
\pgfsetfillcolor{currentfill}%
\pgfsetlinewidth{0.803000pt}%
\definecolor{currentstroke}{rgb}{0.000000,0.000000,0.000000}%
\pgfsetstrokecolor{currentstroke}%
\pgfsetdash{}{0pt}%
\pgfsys@defobject{currentmarker}{\pgfqpoint{-0.048611in}{0.000000in}}{\pgfqpoint{0.000000in}{0.000000in}}{%
\pgfpathmoveto{\pgfqpoint{0.000000in}{0.000000in}}%
\pgfpathlineto{\pgfqpoint{-0.048611in}{0.000000in}}%
\pgfusepath{stroke,fill}%
}%
\begin{pgfscope}%
\pgfsys@transformshift{0.613922in}{2.797238in}%
\pgfsys@useobject{currentmarker}{}%
\end{pgfscope}%
\end{pgfscope}%
\begin{pgfscope}%
\definecolor{textcolor}{rgb}{0.000000,0.000000,0.000000}%
\pgfsetstrokecolor{textcolor}%
\pgfsetfillcolor{textcolor}%
\pgftext[x=0.339969in, y=2.744477in, left, base]{\color{textcolor}\sffamily\fontsize{10.000000}{12.000000}\selectfont 35}%
\end{pgfscope}%
\begin{pgfscope}%
\definecolor{textcolor}{rgb}{0.000000,0.000000,0.000000}%
\pgfsetstrokecolor{textcolor}%
\pgfsetfillcolor{textcolor}%
\pgftext[x=0.284413in,y=1.684421in,,bottom,rotate=90.000000]{\color{textcolor}\sffamily\fontsize{10.000000}{12.000000}\selectfont Berechnungsdauer [s]}%
\end{pgfscope}%
\begin{pgfscope}%
\pgfpathrectangle{\pgfqpoint{0.613922in}{0.571604in}}{\pgfqpoint{2.171249in}{2.225635in}}%
\pgfusepath{clip}%
\pgfsetbuttcap%
\pgfsetroundjoin%
\pgfsetlinewidth{1.505625pt}%
\definecolor{currentstroke}{rgb}{0.121569,0.466667,0.705882}%
\pgfsetstrokecolor{currentstroke}%
\pgfsetdash{}{0pt}%
\pgfpathmoveto{\pgfqpoint{0.719322in}{0.640846in}}%
\pgfpathlineto{\pgfqpoint{0.719322in}{0.640846in}}%
\pgfusepath{stroke}%
\end{pgfscope}%
\begin{pgfscope}%
\pgfpathrectangle{\pgfqpoint{0.613922in}{0.571604in}}{\pgfqpoint{2.171249in}{2.225635in}}%
\pgfusepath{clip}%
\pgfsetbuttcap%
\pgfsetroundjoin%
\pgfsetlinewidth{1.505625pt}%
\definecolor{currentstroke}{rgb}{0.121569,0.466667,0.705882}%
\pgfsetstrokecolor{currentstroke}%
\pgfsetdash{}{0pt}%
\pgfpathmoveto{\pgfqpoint{0.824722in}{0.663720in}}%
\pgfpathlineto{\pgfqpoint{0.824722in}{0.663720in}}%
\pgfusepath{stroke}%
\end{pgfscope}%
\begin{pgfscope}%
\pgfpathrectangle{\pgfqpoint{0.613922in}{0.571604in}}{\pgfqpoint{2.171249in}{2.225635in}}%
\pgfusepath{clip}%
\pgfsetbuttcap%
\pgfsetroundjoin%
\pgfsetlinewidth{1.505625pt}%
\definecolor{currentstroke}{rgb}{0.121569,0.466667,0.705882}%
\pgfsetstrokecolor{currentstroke}%
\pgfsetdash{}{0pt}%
\pgfpathmoveto{\pgfqpoint{1.035523in}{0.757692in}}%
\pgfpathlineto{\pgfqpoint{1.035523in}{0.757692in}}%
\pgfusepath{stroke}%
\end{pgfscope}%
\begin{pgfscope}%
\pgfpathrectangle{\pgfqpoint{0.613922in}{0.571604in}}{\pgfqpoint{2.171249in}{2.225635in}}%
\pgfusepath{clip}%
\pgfsetbuttcap%
\pgfsetroundjoin%
\pgfsetlinewidth{1.505625pt}%
\definecolor{currentstroke}{rgb}{0.121569,0.466667,0.705882}%
\pgfsetstrokecolor{currentstroke}%
\pgfsetdash{}{0pt}%
\pgfpathmoveto{\pgfqpoint{1.246324in}{0.883802in}}%
\pgfpathlineto{\pgfqpoint{1.246324in}{0.888354in}}%
\pgfusepath{stroke}%
\end{pgfscope}%
\begin{pgfscope}%
\pgfpathrectangle{\pgfqpoint{0.613922in}{0.571604in}}{\pgfqpoint{2.171249in}{2.225635in}}%
\pgfusepath{clip}%
\pgfsetbuttcap%
\pgfsetroundjoin%
\pgfsetlinewidth{1.505625pt}%
\definecolor{currentstroke}{rgb}{0.121569,0.466667,0.705882}%
\pgfsetstrokecolor{currentstroke}%
\pgfsetdash{}{0pt}%
\pgfpathmoveto{\pgfqpoint{1.667926in}{1.013248in}}%
\pgfpathlineto{\pgfqpoint{1.667926in}{1.015267in}}%
\pgfusepath{stroke}%
\end{pgfscope}%
\begin{pgfscope}%
\pgfpathrectangle{\pgfqpoint{0.613922in}{0.571604in}}{\pgfqpoint{2.171249in}{2.225635in}}%
\pgfusepath{clip}%
\pgfsetbuttcap%
\pgfsetroundjoin%
\pgfsetlinewidth{1.505625pt}%
\definecolor{currentstroke}{rgb}{0.121569,0.466667,0.705882}%
\pgfsetstrokecolor{currentstroke}%
\pgfsetdash{}{0pt}%
\pgfpathmoveto{\pgfqpoint{2.721930in}{2.571058in}}%
\pgfpathlineto{\pgfqpoint{2.721930in}{2.667317in}}%
\pgfusepath{stroke}%
\end{pgfscope}%
\begin{pgfscope}%
\pgfpathrectangle{\pgfqpoint{0.613922in}{0.571604in}}{\pgfqpoint{2.171249in}{2.225635in}}%
\pgfusepath{clip}%
\pgfsetrectcap%
\pgfsetroundjoin%
\pgfsetlinewidth{1.505625pt}%
\definecolor{currentstroke}{rgb}{1.000000,0.498039,0.054902}%
\pgfsetstrokecolor{currentstroke}%
\pgfsetdash{}{0pt}%
\pgfpathmoveto{\pgfqpoint{0.613922in}{0.633427in}}%
\pgfpathlineto{\pgfqpoint{0.658233in}{0.634296in}}%
\pgfpathlineto{\pgfqpoint{0.702544in}{0.636903in}}%
\pgfpathlineto{\pgfqpoint{0.746855in}{0.641248in}}%
\pgfpathlineto{\pgfqpoint{0.791166in}{0.647332in}}%
\pgfpathlineto{\pgfqpoint{0.835478in}{0.655153in}}%
\pgfpathlineto{\pgfqpoint{0.879789in}{0.664713in}}%
\pgfpathlineto{\pgfqpoint{0.924100in}{0.676010in}}%
\pgfpathlineto{\pgfqpoint{0.968411in}{0.689046in}}%
\pgfpathlineto{\pgfqpoint{1.012722in}{0.703819in}}%
\pgfpathlineto{\pgfqpoint{1.057034in}{0.720331in}}%
\pgfpathlineto{\pgfqpoint{1.101345in}{0.738581in}}%
\pgfpathlineto{\pgfqpoint{1.145656in}{0.758569in}}%
\pgfpathlineto{\pgfqpoint{1.189967in}{0.780295in}}%
\pgfpathlineto{\pgfqpoint{1.234278in}{0.803759in}}%
\pgfpathlineto{\pgfqpoint{1.278590in}{0.828961in}}%
\pgfpathlineto{\pgfqpoint{1.322901in}{0.855901in}}%
\pgfpathlineto{\pgfqpoint{1.367212in}{0.884580in}}%
\pgfpathlineto{\pgfqpoint{1.411523in}{0.914996in}}%
\pgfpathlineto{\pgfqpoint{1.455834in}{0.947151in}}%
\pgfpathlineto{\pgfqpoint{1.500146in}{0.981043in}}%
\pgfpathlineto{\pgfqpoint{1.544457in}{1.016674in}}%
\pgfpathlineto{\pgfqpoint{1.588768in}{1.054042in}}%
\pgfpathlineto{\pgfqpoint{1.633079in}{1.093149in}}%
\pgfpathlineto{\pgfqpoint{1.677390in}{1.133994in}}%
\pgfpathlineto{\pgfqpoint{1.721702in}{1.176577in}}%
\pgfpathlineto{\pgfqpoint{1.766013in}{1.220898in}}%
\pgfpathlineto{\pgfqpoint{1.810324in}{1.266957in}}%
\pgfpathlineto{\pgfqpoint{1.854635in}{1.314755in}}%
\pgfpathlineto{\pgfqpoint{1.898946in}{1.364290in}}%
\pgfpathlineto{\pgfqpoint{1.943258in}{1.415563in}}%
\pgfpathlineto{\pgfqpoint{1.987569in}{1.468575in}}%
\pgfpathlineto{\pgfqpoint{2.031880in}{1.523324in}}%
\pgfpathlineto{\pgfqpoint{2.076191in}{1.579812in}}%
\pgfpathlineto{\pgfqpoint{2.120502in}{1.638037in}}%
\pgfpathlineto{\pgfqpoint{2.164814in}{1.698001in}}%
\pgfpathlineto{\pgfqpoint{2.209125in}{1.759703in}}%
\pgfpathlineto{\pgfqpoint{2.253436in}{1.823143in}}%
\pgfpathlineto{\pgfqpoint{2.297747in}{1.888321in}}%
\pgfpathlineto{\pgfqpoint{2.342058in}{1.955237in}}%
\pgfpathlineto{\pgfqpoint{2.386370in}{2.023891in}}%
\pgfpathlineto{\pgfqpoint{2.430681in}{2.094283in}}%
\pgfpathlineto{\pgfqpoint{2.474992in}{2.166414in}}%
\pgfpathlineto{\pgfqpoint{2.519303in}{2.240282in}}%
\pgfpathlineto{\pgfqpoint{2.563614in}{2.315889in}}%
\pgfpathlineto{\pgfqpoint{2.607926in}{2.393233in}}%
\pgfpathlineto{\pgfqpoint{2.652237in}{2.472316in}}%
\pgfpathlineto{\pgfqpoint{2.696548in}{2.553137in}}%
\pgfpathlineto{\pgfqpoint{2.740859in}{2.635695in}}%
\pgfpathlineto{\pgfqpoint{2.785170in}{2.719992in}}%
\pgfusepath{stroke}%
\end{pgfscope}%
\begin{pgfscope}%
\pgfpathrectangle{\pgfqpoint{0.613922in}{0.571604in}}{\pgfqpoint{2.171249in}{2.225635in}}%
\pgfusepath{clip}%
\pgfsetbuttcap%
\pgfsetroundjoin%
\definecolor{currentfill}{rgb}{0.121569,0.466667,0.705882}%
\pgfsetfillcolor{currentfill}%
\pgfsetlinewidth{1.003750pt}%
\definecolor{currentstroke}{rgb}{0.121569,0.466667,0.705882}%
\pgfsetstrokecolor{currentstroke}%
\pgfsetdash{}{0pt}%
\pgfsys@defobject{currentmarker}{\pgfqpoint{-0.020833in}{-0.020833in}}{\pgfqpoint{0.020833in}{0.020833in}}{%
\pgfpathmoveto{\pgfqpoint{0.000000in}{-0.020833in}}%
\pgfpathcurveto{\pgfqpoint{0.005525in}{-0.020833in}}{\pgfqpoint{0.010825in}{-0.018638in}}{\pgfqpoint{0.014731in}{-0.014731in}}%
\pgfpathcurveto{\pgfqpoint{0.018638in}{-0.010825in}}{\pgfqpoint{0.020833in}{-0.005525in}}{\pgfqpoint{0.020833in}{0.000000in}}%
\pgfpathcurveto{\pgfqpoint{0.020833in}{0.005525in}}{\pgfqpoint{0.018638in}{0.010825in}}{\pgfqpoint{0.014731in}{0.014731in}}%
\pgfpathcurveto{\pgfqpoint{0.010825in}{0.018638in}}{\pgfqpoint{0.005525in}{0.020833in}}{\pgfqpoint{0.000000in}{0.020833in}}%
\pgfpathcurveto{\pgfqpoint{-0.005525in}{0.020833in}}{\pgfqpoint{-0.010825in}{0.018638in}}{\pgfqpoint{-0.014731in}{0.014731in}}%
\pgfpathcurveto{\pgfqpoint{-0.018638in}{0.010825in}}{\pgfqpoint{-0.020833in}{0.005525in}}{\pgfqpoint{-0.020833in}{0.000000in}}%
\pgfpathcurveto{\pgfqpoint{-0.020833in}{-0.005525in}}{\pgfqpoint{-0.018638in}{-0.010825in}}{\pgfqpoint{-0.014731in}{-0.014731in}}%
\pgfpathcurveto{\pgfqpoint{-0.010825in}{-0.018638in}}{\pgfqpoint{-0.005525in}{-0.020833in}}{\pgfqpoint{0.000000in}{-0.020833in}}%
\pgfpathclose%
\pgfusepath{stroke,fill}%
}%
\begin{pgfscope}%
\pgfsys@transformshift{0.719322in}{0.640846in}%
\pgfsys@useobject{currentmarker}{}%
\end{pgfscope}%
\begin{pgfscope}%
\pgfsys@transformshift{0.824722in}{0.663720in}%
\pgfsys@useobject{currentmarker}{}%
\end{pgfscope}%
\begin{pgfscope}%
\pgfsys@transformshift{1.035523in}{0.757692in}%
\pgfsys@useobject{currentmarker}{}%
\end{pgfscope}%
\begin{pgfscope}%
\pgfsys@transformshift{1.246324in}{0.886078in}%
\pgfsys@useobject{currentmarker}{}%
\end{pgfscope}%
\begin{pgfscope}%
\pgfsys@transformshift{1.667926in}{1.014258in}%
\pgfsys@useobject{currentmarker}{}%
\end{pgfscope}%
\begin{pgfscope}%
\pgfsys@transformshift{2.721930in}{2.619188in}%
\pgfsys@useobject{currentmarker}{}%
\end{pgfscope}%
\end{pgfscope}%
\begin{pgfscope}%
\pgfsetrectcap%
\pgfsetmiterjoin%
\pgfsetlinewidth{0.803000pt}%
\definecolor{currentstroke}{rgb}{0.000000,0.000000,0.000000}%
\pgfsetstrokecolor{currentstroke}%
\pgfsetdash{}{0pt}%
\pgfpathmoveto{\pgfqpoint{0.613922in}{0.571604in}}%
\pgfpathlineto{\pgfqpoint{0.613922in}{2.797238in}}%
\pgfusepath{stroke}%
\end{pgfscope}%
\begin{pgfscope}%
\pgfsetrectcap%
\pgfsetmiterjoin%
\pgfsetlinewidth{0.803000pt}%
\definecolor{currentstroke}{rgb}{0.000000,0.000000,0.000000}%
\pgfsetstrokecolor{currentstroke}%
\pgfsetdash{}{0pt}%
\pgfpathmoveto{\pgfqpoint{2.785170in}{0.571604in}}%
\pgfpathlineto{\pgfqpoint{2.785170in}{2.797238in}}%
\pgfusepath{stroke}%
\end{pgfscope}%
\begin{pgfscope}%
\pgfsetrectcap%
\pgfsetmiterjoin%
\pgfsetlinewidth{0.803000pt}%
\definecolor{currentstroke}{rgb}{0.000000,0.000000,0.000000}%
\pgfsetstrokecolor{currentstroke}%
\pgfsetdash{}{0pt}%
\pgfpathmoveto{\pgfqpoint{0.613922in}{0.571604in}}%
\pgfpathlineto{\pgfqpoint{2.785170in}{0.571604in}}%
\pgfusepath{stroke}%
\end{pgfscope}%
\begin{pgfscope}%
\pgfsetrectcap%
\pgfsetmiterjoin%
\pgfsetlinewidth{0.803000pt}%
\definecolor{currentstroke}{rgb}{0.000000,0.000000,0.000000}%
\pgfsetstrokecolor{currentstroke}%
\pgfsetdash{}{0pt}%
\pgfpathmoveto{\pgfqpoint{0.613922in}{2.797238in}}%
\pgfpathlineto{\pgfqpoint{2.785170in}{2.797238in}}%
\pgfusepath{stroke}%
\end{pgfscope}%
\begin{pgfscope}%
\pgfsetbuttcap%
\pgfsetmiterjoin%
\definecolor{currentfill}{rgb}{1.000000,1.000000,1.000000}%
\pgfsetfillcolor{currentfill}%
\pgfsetfillopacity{0.800000}%
\pgfsetlinewidth{1.003750pt}%
\definecolor{currentstroke}{rgb}{0.800000,0.800000,0.800000}%
\pgfsetstrokecolor{currentstroke}%
\pgfsetstrokeopacity{0.800000}%
\pgfsetdash{}{0pt}%
\pgfpathmoveto{\pgfqpoint{0.711144in}{2.278412in}}%
\pgfpathlineto{\pgfqpoint{1.901641in}{2.278412in}}%
\pgfpathquadraticcurveto{\pgfqpoint{1.929419in}{2.278412in}}{\pgfqpoint{1.929419in}{2.306190in}}%
\pgfpathlineto{\pgfqpoint{1.929419in}{2.700016in}}%
\pgfpathquadraticcurveto{\pgfqpoint{1.929419in}{2.727794in}}{\pgfqpoint{1.901641in}{2.727794in}}%
\pgfpathlineto{\pgfqpoint{0.711144in}{2.727794in}}%
\pgfpathquadraticcurveto{\pgfqpoint{0.683366in}{2.727794in}}{\pgfqpoint{0.683366in}{2.700016in}}%
\pgfpathlineto{\pgfqpoint{0.683366in}{2.306190in}}%
\pgfpathquadraticcurveto{\pgfqpoint{0.683366in}{2.278412in}}{\pgfqpoint{0.711144in}{2.278412in}}%
\pgfpathclose%
\pgfusepath{stroke,fill}%
\end{pgfscope}%
\begin{pgfscope}%
\pgfsetrectcap%
\pgfsetroundjoin%
\pgfsetlinewidth{1.505625pt}%
\definecolor{currentstroke}{rgb}{1.000000,0.498039,0.054902}%
\pgfsetstrokecolor{currentstroke}%
\pgfsetdash{}{0pt}%
\pgfpathmoveto{\pgfqpoint{0.738922in}{2.615327in}}%
\pgfpathlineto{\pgfqpoint{1.016699in}{2.615327in}}%
\pgfusepath{stroke}%
\end{pgfscope}%
\begin{pgfscope}%
\definecolor{textcolor}{rgb}{0.000000,0.000000,0.000000}%
\pgfsetstrokecolor{textcolor}%
\pgfsetfillcolor{textcolor}%
\pgftext[x=1.127810in,y=2.566715in,left,base]{\color{textcolor}\sffamily\fontsize{10.000000}{12.000000}\selectfont Fit}%
\end{pgfscope}%
\begin{pgfscope}%
\pgfsetbuttcap%
\pgfsetroundjoin%
\pgfsetlinewidth{1.505625pt}%
\definecolor{currentstroke}{rgb}{0.121569,0.466667,0.705882}%
\pgfsetstrokecolor{currentstroke}%
\pgfsetdash{}{0pt}%
\pgfpathmoveto{\pgfqpoint{0.877810in}{2.342025in}}%
\pgfpathlineto{\pgfqpoint{0.877810in}{2.480913in}}%
\pgfusepath{stroke}%
\end{pgfscope}%
\begin{pgfscope}%
\pgfsetbuttcap%
\pgfsetroundjoin%
\definecolor{currentfill}{rgb}{0.121569,0.466667,0.705882}%
\pgfsetfillcolor{currentfill}%
\pgfsetlinewidth{1.003750pt}%
\definecolor{currentstroke}{rgb}{0.121569,0.466667,0.705882}%
\pgfsetstrokecolor{currentstroke}%
\pgfsetdash{}{0pt}%
\pgfsys@defobject{currentmarker}{\pgfqpoint{-0.020833in}{-0.020833in}}{\pgfqpoint{0.020833in}{0.020833in}}{%
\pgfpathmoveto{\pgfqpoint{0.000000in}{-0.020833in}}%
\pgfpathcurveto{\pgfqpoint{0.005525in}{-0.020833in}}{\pgfqpoint{0.010825in}{-0.018638in}}{\pgfqpoint{0.014731in}{-0.014731in}}%
\pgfpathcurveto{\pgfqpoint{0.018638in}{-0.010825in}}{\pgfqpoint{0.020833in}{-0.005525in}}{\pgfqpoint{0.020833in}{0.000000in}}%
\pgfpathcurveto{\pgfqpoint{0.020833in}{0.005525in}}{\pgfqpoint{0.018638in}{0.010825in}}{\pgfqpoint{0.014731in}{0.014731in}}%
\pgfpathcurveto{\pgfqpoint{0.010825in}{0.018638in}}{\pgfqpoint{0.005525in}{0.020833in}}{\pgfqpoint{0.000000in}{0.020833in}}%
\pgfpathcurveto{\pgfqpoint{-0.005525in}{0.020833in}}{\pgfqpoint{-0.010825in}{0.018638in}}{\pgfqpoint{-0.014731in}{0.014731in}}%
\pgfpathcurveto{\pgfqpoint{-0.018638in}{0.010825in}}{\pgfqpoint{-0.020833in}{0.005525in}}{\pgfqpoint{-0.020833in}{0.000000in}}%
\pgfpathcurveto{\pgfqpoint{-0.020833in}{-0.005525in}}{\pgfqpoint{-0.018638in}{-0.010825in}}{\pgfqpoint{-0.014731in}{-0.014731in}}%
\pgfpathcurveto{\pgfqpoint{-0.010825in}{-0.018638in}}{\pgfqpoint{-0.005525in}{-0.020833in}}{\pgfqpoint{0.000000in}{-0.020833in}}%
\pgfpathclose%
\pgfusepath{stroke,fill}%
}%
\begin{pgfscope}%
\pgfsys@transformshift{0.877810in}{2.411469in}%
\pgfsys@useobject{currentmarker}{}%
\end{pgfscope}%
\end{pgfscope}%
\begin{pgfscope}%
\definecolor{textcolor}{rgb}{0.000000,0.000000,0.000000}%
\pgfsetstrokecolor{textcolor}%
\pgfsetfillcolor{textcolor}%
\pgftext[x=1.127810in,y=2.362858in,left,base]{\color{textcolor}\sffamily\fontsize{10.000000}{12.000000}\selectfont Messwerte}%
\end{pgfscope}%
\end{pgfpicture}%
\makeatother%
\endgroup%
}
  \vspace{-30pt}
\end{wrapfigure}
In Abbildung \ref{pic:deform} (Appendix) sind das bewegte Bild und das fixierte
Zielbild abgebildet. Das optimale Registrierungsverfahren um das Quadrat auf
den Kreis zu transformieren lag bei uns bei \SI{30}{gppd} (grid points per
dimension, Gitterpunkte pro Dimension), wobei die gleichen Registrierungsstufen
wie zuvor verwendet wurden. Es ist erkennbar, dass bei niedrigen Auflösungen
der Kreis nicht rund ist, und am Rand graue Artefakte zu sehen sind. In diesen
Bereichen kann der Algorithmus keine Werte finden. Im Fall von \SI{5}{gppd}
kann man sich etwa vorstellen, dass in den Ecken nur ein einzelner Gitterpunkt
im schwarzen Bereich ist und dieser in Richtung Mitte gezogen wird, so dass im
transformierten Bild am Rand keine Informationen verfügbar sind. Ist die Anzahl
der Gitterpunkte zu hoch, so ergibt sich ein anderes Problem, die Gitterpunkte
werden zu nah aneinander transformiert. Gut erkennbar ist das bei der
Transformation mit \SI{100}{gppd}, also wo jeder Pixel auch ein Gitterpunkt
ist.\\
An den Ecken des Quadrats müssen mehrere Pixel auf die Ränder des Kreises
transformiert werden, was zu Informationsverlust an diesen Stellen führt,
weshalb die grauen Bereiche um den transformierten Kreis entstehen.\\
Um die Transformation besser sichtbar zu machen, haben wir die Transformation
mit den optimalen Parametern auf ein Schachbrett angewendet. Man erkennt gut,
dass die äußeren und inneren Quadrate von der Transformation nicht beeinflusst
werden. Da unser Schachbrett \num{10} Quadrate pro Dimension hat, kommen auf
jedes Quadrat \num{9} Gitterpunkte. An den Quadraten in den Ecken, die noch
transformiert wurden, erkennt man gut, dass zwischen den Gitterpunkten
interpoliert wird.\\
Wir haben auch die benötigte Rechenzeit mit der Anzahl der Gitterpunkte pro
Dimension verglichen, die Messergebnisse sind in Abbildung \ref{fig:A4}
dargestellt. Wie nach dem vorhergien Teil zu erwarten, ist die benötigte
Rechenzeit des Algorithmus linear zur Anzahl der Gitterpunkte. Bei einem
2-dimensionalen Bild ist der Verlauf also quadratisch zur Anzahl der
Gitterpunkte pro Dimension.
