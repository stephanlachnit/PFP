\section{Diskussion}
Im ersten Versuchteil war sehr anschaulich zu sehen, wie die Bildregistrierung
einer einfachen Translation funktionieren. Wir haben herausgefunden, dass die
Registrierung bei kleinen Translationen am besten funktioniert, da bei größeren
Abständen aufgrund des räumlichen Effekts die Ähnlichkeit schlechter
festgestellt werden kann.\\
Das Verschiebungsverhältnis zwischen fixiertem und bewegten Bild ist konstant
und lag bei uns bei \SI{3.39(1)}{pixel\per\centi\meter}.\\
Im zweiten Teil haben wir die verschiedenen Metriken untersucht. Im monomodalen
Fall konnten wir zwischen Rotations- und Translationsparametern ahnand der
Periodizität unterscheiden. Bei den Rotationsparametern ist zu sehen, dass die
Metriken um Vielfache von \num{2\pi} ihr Minimum haben und dazwischen der
Algorithmus keine Ähnlichkeit mehr festellen kann.\\
Im multimodalen Fall konnten wir am Rotationsparameter gut erkennen, dass die
MSD Metrik wie zu erwarten keine sinnvollen Ergebnisse mehr liefert, während die
MI Metrik noch einen sinnvollen Verlauf hat.\\
Beim Translationsparameter haben wir leider nur einen sehr kleinen Shift
berechnet, dadurch konnten wir dies dort nicht wirklich erkennen. Allerdings
kann man am Verlauf der MI Metrik feststellen, dass der vorgegebene optimale
Wert um etwa einen halben Voxel von dem Metrikminimum abweicht. Das liegt daran,
dass die optimalen Parameter von Hand bestimmt wurden.\\
Bei der Variation der Anzahl der Bins der MI Metrik, ist zu erkennen, dass eine
größere Anzahl positive Auswirkungen auf die Genauigkeit des Algorithmus hat,
wobei der Effekt mit steigender Anzahl abnimmt.\\
Im Versuchsteil über die Rigiden 3D-Registrierungen war das Finden der
optimalen Registrierungsstrategie die Hauptaufgabe. Aufgrund der begrenzten
Zeit, die wir hatten um uns in die Programmbibliothek einzulesen, bestand
dieser Teil hauptsächlich aus Ausprobieren. Hatte man einmal eine gute
Strategie, hat diese auch bei unterschiedlichen Patientendaten gute Ergebnisse
geliefert.\\
Im letzten Teil wurde die deformierbare Transformation genauer betrachtet.
Zunächst haben wir eine gute Auflösung des Transformationsfeldes gesucht, und
dies anhand einer Transformation eines Rechtecks auf einen Kreis untersucht.
Dabei haben wir festgestellt, dass zu viele oder zu wenige Gitterpunkte bei der
Transformationen zu unerwünschten Ergebnissen führen. Nachdem wir einen guten
Wert gefunden haben, haben wir diese Transformation auf ein Schachbrettmuster
angewandt, um sie besser zu verdeutlichen.\\
Als Fazit können wir festhalten, dass der Versuch gut durchzuführen war und gut
aufgezeigt, wie Physik in anderen Bereichen, wie in diesem Fall der Medizin,
Anwendungen findet. Besonders realitätsnah war hierbei die Verwendung von
medizinischen Patientenbildern. Der Teil mit der deformierbaren Transformation
hat besonders anschaulich gezeigt, dass das Themengebiet aber auch darüber
hinausgehen kann.
