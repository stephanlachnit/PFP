\section{Grundlagen}
\subsection{Bildmodalitäten}
In der medizinischen Bildgebung gibt es verschiedene Bildgebungstechniken,
welche je nach Technik unterschiedliche Informationen besser darstellen.
Besonders wichtig sind die Magnetresonanztomographie (MRT) und
Computertomographie (CT) zur Diagnostizierung, welche wir im folgenden genauer
betrachten wollen.

\subsubsection{MRT-Bildgebung}
Für die Magnetresonanztomographie sind der Spin, der Drehimpuls und das
magnetisches Moment des Protons grundlegend. Wenn ein Magnetfeld angelegt wird,
können durch diese Eigenschaften die Lage der Rotationsachse des Protons, der
induzierte Magnetfeldvektor sowie dessen Änderungen beobachtet werden.\\
Die Ausrichtung der Spins wird mit der Larmorfrequenz $\omega_0$ beschrieben.
Diese ist die Frequenz, welche die Präzessionsbewegung eines Protons in einem
äußeren Magnetfeld $B_0$ beschreibt: $\omega_0 = \gamma_0 \cdot B_0$, wobei
$\gamma_0$ das gyromagnetische Verhältnis beschreibt.\\
Mit einer elektromagnetischen Welle mit Larmorfrequenz kann dem System Energie
hinzugefügt werden. Da in einem äußeren Magnetfeld in $z$-Richtung fast alle
Spins parallel in $z$-Richtung ausgerichtet sind, kippen diese durch solch eine
Anregung aus der $z$-Achse. Das MR-Signal, auf welchem die MRT-Bildgebung
beruht, entsteht durch einen \ang{90}-Puls. Dabei kippen die Spins in die
$x$-$y$-Ebene und induzieren eine Wechselspannung mit der Larmorfrequenz in der
Empfangsspule.\\
Da der Ausgangszustand (parallele Ausrichtung in $z$-Richtung) stabiler ist,
fallen die Spins in diesen mit der Zeit zurück. Dies geschieht durch die
Spin-Gitter- und die Spin-Spin-Wechselwirkung oder auch T1- beziehungsweise
T2-Relaxation genannt.\\
Die T1-Relaxation wird auch longitudinale Relaxation genannt, da durch die
Ausrichtung der Spins entlang des Magnetfeldes die longitudinale Magnetisierung
zunimmt. Die zugehörige Zeitkonstante T1 ist abhängig von der Magnetfeldstärke
und vom Material.  Die Spin-Spin Wechselwirkung mit der zugehörigen
Zeitkonstante T2 trägt ebenso zur Abnahme der transversalen Magnetisierung bei.
Hierbei entsteht eine Dephasierung des Spins durch einen Energieaustausch der
Spins untereinander.  Auch Inhomogenitäten des Magnetfeldes verstärken den
Zerfall der Phasenkohärenz. Diese entstehen einerseits durch das Gerät
andererseits durch den Patienten im Magnetfeld. Mit Hilfe von spezielle
Sequenzen kann dieser Effekt unterdrückt werden.\\
Die Anzahl anregbarer Spins pro Volumeneinheit, die T1-Zeit (Zeit bis sich die
Spins nach einer Anregung wieder entlang des Magnetfeldes ausgerichtet sind)
und die T2-Zeit (Zeit bis das Signal nach einer Anregung abklingt aufgrund der
aus der Phase laufenden Spins) bestimmen die Intensität des Gewebes auf einem
MRT-Bild. Somit entsteht je nach Parameter eine andere Eigenschaft auf dem
MRT-Bild. Für die Aufnahme eines dreidimensionalen Bildes wird den MR-Signalen
noch Koordinaten zugeordnet: Man benötigt eine zusätzliche Magnetspule für
einen Gradienten entlang der $z$-Richtung um eine Auflösung der $z$-Koordinate
zu gewährleisten. Da die Larmorfrequenz abhängig von dem angelegten Magnetfeld
ist, sind entlang dieses Gradienten auch die Larmorfrequenzen unterschiedlich.
Je nach Frequenz wird nun nur eine Schicht des Körpers angeregt und somit
können je nach Stärke des Gradienten feinere oder gröbere Schichten aufgelöst
werden. Auch in der $x$- und $y$-Richtung wird ein zusätzlicher Gradient mit
einer Magnetspule erzeugt. In $y$-Richtung kreisen durch die unterschiedlichen
Larmorfrequenzen die Spins unterschiedlich schnell und erzeugen einen
Phasengradienten. Wenn nun kurz dannach dieser Gradient ausgeschaltet wird,
bewegen sich die Spins gleich schnell, allerdings in unterschiedlicher Phase
und können somit ihrer Position in $y$-Richtung zugeordnet werden. Durch einen
Gradienten in $x$-Richtung besteht das MR-Signal nicht mehr aus nur aus einer
Frequenz sondern einem Frequenzspektrum, da die Spins unterschiedlich schnell
präzedieren.

\subsubsection{CT-Bildgebung}
Der Computertomograph (CT) besteht aus einer Röntgenröhre mit Blenden, welche
die Röntgenstrahlung kontrolliert zum Patienten leiten und einem Detektor.
Dieser detektiert die Energie der Strahlung, welche durch das Gewebe
abgeschwächt wurde. Nun werden viele Bilder aus unterschiedlichen Perspektiven
aufgenommen durch Drehungen der Röntgenröhre und des Detektors. Die
unterschiedlichen Absorptionscharakteristiken der Gewebeschichten resultiert in
Kontrasten auf den Bildern.

\subsection{Bildregistrierung}
Die Informationen dieser zwei verschiedenen Bildgebungssysteme werden
miteinander kombiniert um eine Diagnose zu stellen oder medizinische Eingriffe
zu planen. Um diese Daten geometrisch zu verknüpfen benötigt es die sogenannte
Bildregistrierung. Die Bildregistrierung beruht darauf, dass ein Datensatz als
Referenz verwendet wird und für den zweiten Datensatz eine Transformation
gesucht wird, die diesen mit dem ersten optimal überlagert. In dieser
Transformation wird für jeden Bildpunkt berechnet wohin dieser im
transformierten Bild liegt. Daneben benötigt man ein Interpolationsverfahren
um die Intensitätswerte an den Pixel-Positionen im transformierten Bild, da
ein Bild aufgrund der Pixel diskretisiert ist.\\
Da es sich hier um ein inverses Problem handelt, weil die Transformation nicht
bekannt ist, sondern nur das Referenzbild und das transformierte Bild (bewegtes
Bild), kann dies nur mit Hilfe eines Optimierungsverfahren gelöst werden. 
Der Algorithmus der Bildregistrierung basiert auf einer Metrik, welche die
Ähnlichkeit zweier Bilddatensätze beschreibt und einer Transformation, die die
Ähnlichkeit zwischen bewegtem Bild und Referenzbild erhöht. Hinzu kommen das
Interpolationsverfahren und ein Optimierungsalgorithmus.

\subsubsection{Ähnlichkeitsmaße}
Zwar gibt es neben dem intensitätsbasierten auch das landmarkenbasierte Maß,
jedoch wollen wir uns hier auf Ersteres konzentrieren. Hierbei werden die
Intensitätswerte der Bilder genutzt um die Ähnlichkeit zu bestimmen. Es wird
nun zwischen einer Metrik, welche die Bildintensitäten vergleicht, und einer,
welche nur statistische Informationen der Bilder nutzt, unterschieden. Es muss
jedoch beachtet werden, dass die erstere Metrik nur im monomodalen Fall genutzt
werden kann da nur dort garantiert werden kann, dass gleiche Informationen
durch gleiche Intensitätenskontraste dargestellt werden.\\
Mit Hilfe der mittleren quadratischen Abweichung (Mean Squared Difference, MSD)
kann man die Bildintensitäten direkt vergleichen:
\begin{equation}
  \label{eq:msd}
  M_{MSD}=\frac{1}{N} \sum_{i=1}^N (R(x_i)-M^\mathcal{T}(x_i))^2
\end{equation}
$N$ ist die Anzahl betrachteter Punkte, $R$ das Referenzbild, $x_i$ die
Voxelkoordinaten und $M^\mathcal{T}$ das bewegte Bild, welches mit der
Transformation $\mathcal{T}$ in das Koordinatensystem des Referenzbildes
transformiert wurde.\\
Ein anderes Ähnlichkeitsmaß ist Mutual Information (MI), welches genutzt wird
falls CT- und MRT-Aufnahmen miteinander verglichen werden sollen, da hier die
Kontraste der Bildintensitäten nicht direkt verglichen werden können. Bei
diesem Ähnlichkeitsmaß werden nur statistische Informationen der
Bildintensitäten verwendet:
\begin{equation}
  \label{eq:mi}
  M_{MI}=H(I_A)+H(I_B)-H(I_A,I_B)
\end{equation}
wobei die marginale Entropie $H(I)$ der Bilddaten $I$ wie folgt definiert ist:
\begin{equation}
  \label{eq:mi_h}
  H(I)=-K \sum_a p_I(a) \cdot \log{p_I(a)}
\end{equation}
mit der Wahrscheinlichkeit $p(a)$ der Intensität $a$ und einer positiven
Konstante $K$, die abhängig von der gewählten Basis des Logarithmus ist. Die
gemeinsame Entropie $H(I_A,I_B)$ ist analog definiert, allerdings ist die
Wahrscheinlichkeit $p_{I_A,I_B}(a,b)$ nun entsprechend zweidimensional.\\
Die Wahrscheinlichkeit wird mit Hilfe von einem (zweidimensionalen)
Histogramms berechnet, auf dem die Häufigkeiten der Intensitäten $a$ in
gruppierte Bins eingetragen werden.

\subsubsection{Transformationen}
Man unterscheidet zwischen der Transformation der Bildintensitäten und der
geometrischen Transformation, welche wir genauer betrachten wollen. Hier wird
zusätzlich differenziert zwischen einer lokalen oder globalen Transformation
der Bilddaten. Die globale Transformation basiert darauf, dass sie für alle
Bildpunkte gleich ist, die lokale Transformation ist für jeden Punkt
verschieden. Ein einfaches Modell der globalen Transformation ist eine
Translation mit einer Rotation. Hierbei gibt es 3 Freiheitsgrade für die
Translation und 3 für die Rotation. Da sich hier der Abstand zwischen zwei
Bildpunkten nicht ändert, bezeichnet man eine solche Transformation als rigide.
Erweitert man diese Transformation um eine mögliche Änderung der Skalierungen
und Scherungen, so erhöht sich die Zahl der zu optimierenden Parameter auf 12
und man spricht von einer affinen Transformation. Für die transformierten
Koordinaten $(x',y',z')$ eines Punktes $(x,y,z)$ gilt nun folgende Darstellung:
\begin{equation}
  \begin{pmatrix}
    x' \\ y' \\ z' \\ 1
  \end{pmatrix}
  =
  \begin{pmatrix}
    p_1 & p_2 & p_3 & p_4 \\
    p_5 & p_6 & p_7 & p_8 \\
    p_9 & p_{10} & p_{11} & p_{12} \\
    p_{13} & p_{14} & p_{15} & p_{16}
  \end{pmatrix}
  \cdot
  \begin{pmatrix}
    x \\ y \\ z \\ 1
  \end{pmatrix}
\end{equation}
Eine deformierbare Transformation wird hingegen beschrieben mittels eines
Transformationsfeldes. Hier ist die Transformation eines Voxels abhängig von
dessen Position im Bildvolumen. Um nicht für jedes Voxel eigene
Transformationsparameter zu wählen, definiert man ein grobes Gitter. An den
Gitterpunkten wird die Transformation berechnet und anschließend werden für die
dazwischenliegenden Punkte die Parameter interpoliert, da die Transformation
kontinuierlich sein soll. Um eine solche deformierbare Transformation zu
beschleunigen, wird zunächst ein grobes Gitter gewählt und später diese
Gitterpunkte genutzt um auf einem feineren Gitter zu optimieren.
