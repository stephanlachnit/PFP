\section{Versuchsdurchführung}
\subsection{Ermittlung einer Kamerabewegung}
Im ersten Versuchsteil soll mit Hilfe einer Bildfolge die Translationsbewegung
einer Kamera ermittelt werden. 
\subsubsection{Aufbau}
Der Versuchsaufbau besteht aus einer USB-Kamera, welche auf einer Schiene für
die Kameraführung befestigt ist. Die Kamera ist an einem Computer
angeschlossen, wie auch im Versuchsprotokoll zu erkennen. An diesem soll nun
mit Hilfe von Matlab die Bilder ausgewertet werden. Die Kamerafunktionalität
wird anschließend getestet indem das Bild als Video angezeigt wird und die
Kamera auf der Schiene bewegt wird. Wenn dies alles ohne Probleme funktioniert
kann mit dem ersten Versuchsteil begonnen werden:

\subsubsection{Bildaufnahme mit der Kamera}
Die Kamera wird auf eine Wand mit möglichst markanten Gegenständen
ausgerichtet. Von dieser Wand werden mehrere Bilder mit verschiedenen
Positionen der Kamera auf der Schiene aufgenommen. Diese Abstände werden im
Versuchsprotokoll notiert.

\subsubsection{Paarweise Registrierung der Aufnahmen}
Diese Bilder werden nun in Matlab mit Hilfe eines Algorithmus für eine reine
Translation paarweise zueinander registriert und die resultierenden Bilder
gespeichert. Zusätzlich sollen die durch Matlab ermittelten
Translationparameter notiert werden.

\subsection{Ähnlichkeit bei monomodalen und multimodalen Bilddaten}
Nun sollen Ähnlichkeitsmaße genauer betrachtet werden. Mit der MSD- und
MI-Metrik wird untersucht wie sich die Ähnlichkeit von der verwendeten
Transformation ändert. Dafür werden 3D-Bilddatensätze aus CT und MRT Aufnahmen
des Kopfes genutzt. Mit Hilfe des Trainingsdatensatzes wurden bereits die
optimalen Transformationsparameter bestimmt. Diese Parameter sollen nun sowohl
im monomodalen als auch im multimodalen Fall betrachtet werden:

\subsubsection{Aufnahme der Ähnlichkeit im monomodalen Fall}
Zunächst ist im monomodalen Fall das Referenzvolumen und das zu registrierende
Volumen dasselbe, in unserem Fall ist das zu untersuchende Volumen das
CT-Volumen.

\subsubsection{Aufnahme der Ähnlichkeit im multimodalen Fall}
Nun soll als zu registrierendes Volumen das MR-T2-Volumen gewählt werden. Neben
der Variation der Parameter soll zusätzlich die Anzahl der Histogramm-Bins
verändert werden.

\subsection{Rigide 3D-Registrierung von medizinischen Bilddaten}
Mit einem Algorithmus sollen in diesem Versuchsteil CT- und MR-T2 Daten
registriert werden um anschließend eine optimale Transformation zu ermitteln.
\subsubsection{Implementierung des Algorithmus}
In diesem Algorithmus wird zur Registrierung Mutual Information zusammen mit
einem Quasi-Newton-Optimierer verwendet. 
\subsubsection{Anwendung auf die Testdaten}
Der Registrierungsalgorithmus mit den optimalen Parameter der Trainingsdaten
soll nun auf die Testdaten angewandt werden.

\subsection{Deformierbare Registrierung}
Nachdem die ersten drei Versuchsteile sich nur mit rigiden Transformationen
beschäftigt haben, soll nun die deformierbare Registrierung anhand zweier
Beispiele demonstriert werden.\\
Zuerst soll ein weißes Rechteck auf schwarzem Hintergrund zu einem weißen Kreis
auf schwarzem Hintergrund deformiert werden. Besonders soll bei diesem Teil der
Einfluss der Anzahl der Gitterpunkte des Transformationsgitters auf das
Registrierungsergebnis und die Laufzeit der Registrierung untersucht werden.
Danach soll das optimale Transformationsfeld auf ein Schachbrettmuster
angewandt werden.
